\documentclass[10pt, fleqn, label-section=none]{bxjsarticle}

%\usepackage[driver=dvipdfm,hmargin=25truemm,vmargin=25truemm]{geometry}

\setpagelayout{driver=dvipdfm,hmargin=25truemm,vmargin=20truemm}


\usepackage{amsmath}
\usepackage{amssymb}
\usepackage{amsfonts}
\usepackage{amsthm}
\usepackage{mathtools}
\usepackage{mleftright}

\usepackage{ascmac}




\usepackage{otf}

\theoremstyle{definition}
\newtheorem{dfn}{定義}[section]
\newtheorem{ex}[dfn]{例}
\newtheorem{lem}[dfn]{補題}
\newtheorem{prop}[dfn]{命題}
\newtheorem{thm}[dfn]{定理}
\newtheorem{cor}[dfn]{系}
\newtheorem*{pf*}{証明}
\newtheorem{problem}[dfn]{問題}
\newtheorem*{problem*}{問題}
\newtheorem{remark}[dfn]{注意}
\newtheorem*{claim*}{\underline{claim}}



\newtheorem*{solution*}{解答}

%箇条書きの様式
\renewcommand{\labelenumi}{(\arabic{enumi})}


%

\newcommand{\forany}{\rm{for} \ {}^{\forall}}
\newcommand{\foranyeps}{
\rm{for} \ {}^{\forall}\varepsilon >0}
\newcommand{\foranyk}{
\rm{for} \ {}^{\forall}k}


\newcommand{\any}{{}^{\forall}}
\newcommand{\suchthat}{\, \rm{s.t.} \, \it{}}




\newcommand{\veps}{\varepsilon}
\newcommand{\paren}[1]{\mleft( #1\mright )}
\newcommand{\cbra}[1]{\mleft\{#1\mright\}}
\newcommand{\sbra}[1]{\mleft\lbrack#1\mright\rbrack}
\newcommand{\tbra}[1]{\mleft\langle#1\mright\rangle}
\newcommand{\abs}[1]{\left|#1\right|}
\newcommand{\norm}[1]{\left\|#1\right\|}
\newcommand{\lopen}[1]{\mleft(#1\mright\rbrack}
\newcommand{\ropen}[1]{\mleft\lbrack #1 \mright)}



%
\newcommand{\Rn}{\mathbb{R}^n}
\newcommand{\Cn}{\mathbb{C}^n}

\newcommand{\Rm}{\mathbb{R}^m}
\newcommand{\Cm}{\mathbb{C}^m}


\newcommand{\projs}[2]{\it{p}_{#1,\ldots,#2}}
\newcommand{\projproj}[2]{\it{p}_{#1,#2}}

\newcommand{\proj}[1]{p_{#1}}

%可測空間
\newcommand{\stdProbSp}{\paren{\Omega, \mathcal{F}, P}}

%微分作用素
\newcommand{\ddt}{\frac{d}{dt}}
\newcommand{\ddx}{\frac{d}{dx}}
\newcommand{\ddy}{\frac{d}{dy}}

\newcommand{\delt}{\frac{\partial}{\partial t}}
\newcommand{\delx}{\frac{\partial}{\partial x}}

%ハイフン
\newcommand{\hyphen}{\text{-}}

%displaystyle
\newcommand{\dstyle}{\displaystyle}

%⇔, ⇒, \UTF{21D0}%
\newcommand{\LR}{\Leftrightarrow}
\newcommand{\naraba}{\Rightarrow}
\newcommand{\gyaku}{\Leftarrow}

%理由
\newcommand{\naze}[1]{\paren{\because {\mathop{ #1 }}}}

%
\newcommand{\sankaku}{\hfill $\triangle$}

%
\newcommand{\push}{_{\#}}

%手抜き
\newcommand{\textif}{\textrm{if}\,\,\,}
\newcommand{\Ric}{\textrm{Ric}}
\newcommand{\tr}{\textrm{tr}}
\newcommand{\vol}{\textrm{vol}}
\newcommand{\diam}{\textrm{diam}}
\newcommand{\supp}{\textrm{supp}}
\newcommand{\Med}{\textrm{Med}}
\newcommand{\Leb}{\textrm{Leb}}
\newcommand{\Const}{\textrm{Const}}
\newcommand{\Avg}{\textrm{Avg}}
\newcommand{\id}{\textrm{id}}
\newcommand{\Ker}{\textrm{Ker}}
\newcommand{\im}{\textrm{Im}}




\renewcommand{\;}{\, ; \,}
\renewcommand{\d}{\, {d}}

\newcommand{\gyouretsu}[1]{\begin{pmatrix} #1 \end{pmatrix} }

%%図式

\usepackage[dvipdfm,all]{xy}


\newenvironment{claim}[1]{\par\noindent\underline{Step:}\space#1}{}
\newenvironment{claimproof}[1]{\par\noindent{($\because$)}\space#1}{\hfill $\blacktriangle $}


\newcommand{\pprime}{{\prime \prime}}


%%


\title{双線形写像}
\date{}


\author{付け焼き刃コース}


\begin{document}


\maketitle


$V, W$ でベクトル空間, $V^*, W^*$ でそれぞれの双対空間を表す. 係数体は$\mathbb R$ としておく. 

\section{双線形写像}

\begin{dfn}双線形写像
\begin{align*} b: V \times W \rightarrow \mathbb R \end{align*}
は, 
\begin{align*} &b(v,w) = 0 \,\,\,(\any w \in W) \naraba v = 0 \\&  b(v,w) = 0 \,\,\, (\any v \in V) \naraba w = 0,\end{align*}
を満たす時に, 非退化であるという. 
\end{dfn}

\begin{remark}
この条件はすなわち, 任意の$w \in W$ に対して$b(\cdot , w)$ が単射であり, かつ任意の$v \in V$ に対して $b(v, \cdot)$が単射であることと同じである. 
\end{remark}

\begin{prop}
$V, W$ を有限次元とする. 双線形形式$b: V \times W \rightarrow \mathbb R$ で非退化なものが存在するならば, 
\begin{align*} V \simeq W^* , \quad W \simeq V^* \end{align*}
\end{prop}
\begin{pf*}

\begin{claim}
\begin{align*} \dim W = \dim V^* = \dim V = \dim W^* \end{align*}
\end{claim}
\begin{claimproof}
線型写像を
\begin{align*} \iota^b :W \rightarrow V^* ; w \mapsto b(\cdot, w) \end{align*}
により定める. 
\begin{align*} \iota^b(w) = 0 \naraba b(v, w) = 0\,\,\,(\any v \in V) \naraba w = 0\end{align*}
であるので, $\iota^b$ は単射であるので, $\dim W \leq \dim V^*$ が成り立つ. 全く同様にして, $\dim W \leq \dim V^*$ も成り立つ.
有限次元であることから, 
\begin{align*} \dim V = \dim V^* , \quad \dim W = \dim W^*  \end{align*}
であるので, 
\begin{align*} \dim W \leq \dim V^* = \dim V \leq \dim W^* = \dim W \end{align*}
である. 
\end{claimproof}

従って, $\iota^b$ は同じ次元のベクトル空間の間の単射線型写像であるので, 同型写像である. 
\qed
\end{pf*}

\begin{ex}
無限次元の場合には, 非退化双線形写像$b$ の定める線型写像$\iota^b $ は単射であっても全射であるとは限らない. $\mathbb R$ の中の有界閉区間$[a,b]$上の滑らかな関数全体$C^\infty ([a,b])$に
\begin{align*} b(f, g) \coloneqq_{[a,b]} \int f(x) g(x) dx \end{align*}
により双線形写像を定めると, よく知られた結果としてこれは非退化である. 一方で, ディラック測度が定める双対空間の元を考えると, 全射でないことがわかる. 
\end{ex}


\begin{dfn}
対称双線形写像$b: V \times V \rightarrow \mathbb R$は
\begin{align*} b(v,v) > 0 \,\,\, (\any v (\neq 0)\in V )\end{align*}
が成り立つ時に, 正定値であるという. 
\end{dfn}

\begin{prop}
正定値双線形形式は非退化である. 
\end{prop}
\begin{pf*}
$b(v, v_0) = 0\quad (v \in V)$ とすると, $b(v_0, v_0) = 0$ である. $v_0 \neq 0 $ とすると$b(v_0, v_0) > 0$ となるので, $v_0 = 0$
\qed
\end{pf*}

\begin{dfn}
対称双線形写像$b: V \times V \rightarrow \mathbb R$は
\begin{align*} b(v,v) \geq 0 \,\,\, (\any v (\neq 0)\in V )\end{align*}
が成り立つ時に, 半正定値であるという. 
\end{dfn}

\begin{ex}
例えば, 恒等的に$0$を与える双線形写像$b = 0$ は半正定値であるが, 明らかに非退化でない. 
\end{ex}

\begin{prop}$V$ を有限次元ベクトル空間とし, 基底を$\cbra{e_i}$ とする. 双線形写像$b: V \times V \rightarrow \mathbb R$ に対して
\begin{align*} b(v, w) = v^i b_{ij} w^j \quad (v = v^i e_i, w = w^i e_i \in V)\end{align*}
を満たす行列$B = (b_{ij})$ が存在する. 
\end{prop}
\begin{pf*}
\begin{align*} b_{ij} \coloneqq b(e_i, e_j)\end{align*}
により定義すれば, 
\begin{align*} b(v,w) = b(v^ie_i, w^je_j) = v^i b(e_i, e_j) w^j \end{align*}
が成り立つ. 
\qed
\end{pf*}












\end{document}