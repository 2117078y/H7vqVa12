\documentclass[10pt, fleqn, label-section=none]{bxjsarticle}

%\usepackage[driver=dvipdfm,hmargin=25truemm,vmargin=25truemm]{geometry}

\setpagelayout{driver=dvipdfm,hmargin=25truemm,vmargin=20truemm}


\usepackage{amsmath}
\usepackage{amssymb}
\usepackage{amsfonts}
\usepackage{amsthm}
\usepackage{mathtools}
\usepackage{mleftright}

\usepackage{ascmac}




\usepackage{otf}

\theoremstyle{definition}
\newtheorem{dfn}{定義}[section]
\newtheorem{ex}[dfn]{例}
\newtheorem{lem}[dfn]{補題}
\newtheorem{prop}[dfn]{命題}
\newtheorem{thm}[dfn]{定理}
\newtheorem{setting}[dfn]{設定}
\newtheorem{cor}[dfn]{系}
\newtheorem*{pf*}{証明}
\newtheorem{problem}[dfn]{問題}
\newtheorem*{problem*}{問題}
\newtheorem{remark}[dfn]{注意}
\newtheorem*{claim*}{\underline{claim}}



\newtheorem*{solution*}{解答}

%箇条書きの様式
\renewcommand{\labelenumi}{(\arabic{enumi})}


%

\newcommand{\forany}{\rm{for} \ {}^{\forall}}
\newcommand{\foranyeps}{
\rm{for} \ {}^{\forall}\varepsilon >0}
\newcommand{\foranyk}{
\rm{for} \ {}^{\forall}k}


\newcommand{\any}{{}^{\forall}}
\newcommand{\suchthat}{\, \rm{s.t.} \, \it{}}




\newcommand{\veps}{\varepsilon}
\newcommand{\paren}[1]{\mleft( #1\mright )}
\newcommand{\cbra}[1]{\mleft\{#1\mright\}}
\newcommand{\sbra}[1]{\mleft\lbrack#1\mright\rbrack}
\newcommand{\tbra}[1]{\mleft\langle#1\mright\rangle}
\newcommand{\abs}[1]{\left|#1\right|}
\newcommand{\norm}[1]{\left\|#1\right\|}
\newcommand{\lopen}[1]{\mleft(#1\mright\rbrack}
\newcommand{\ropen}[1]{\mleft\lbrack #1 \mright)}



%
\newcommand{\Rn}{\mathbb{R}^n}
\newcommand{\Cn}{\mathbb{C}^n}

\newcommand{\Rm}{\mathbb{R}^m}
\newcommand{\Cm}{\mathbb{C}^m}


\newcommand{\projs}[2]{\it{p}_{#1,\ldots,#2}}
\newcommand{\projproj}[2]{\it{p}_{#1,#2}}

\newcommand{\proj}[1]{p_{#1}}

%可測空間
\newcommand{\stdProbSp}{\paren{\Omega, \mathcal{F}, P}}

%微分作用素
\newcommand{\ddt}{\frac{d}{dt}}
\newcommand{\ddx}{\frac{d}{dx}}
\newcommand{\ddy}{\frac{d}{dy}}

\newcommand{\delt}{\frac{\partial}{\partial t}}
\newcommand{\delx}{\frac{\partial}{\partial x}}

%ハイフン
\newcommand{\hyphen}{\text{-}}

%displaystyle
\newcommand{\dstyle}{\displaystyle}

%⇔, ⇒, \UTF{21D0}%
\newcommand{\LR}{\Leftrightarrow}
\newcommand{\naraba}{\Rightarrow}
\newcommand{\gyaku}{\Leftarrow}

%理由
\newcommand{\naze}[1]{\paren{\because {\mathop{ #1 }}}}

%
\newcommand{\sankaku}{\hfill $\triangle$}

%
\newcommand{\push}{_{\#}}

%手抜き
\newcommand{\textif}{\textrm{if}\,\,\,}
\newcommand{\Ric}{\textrm{Ric}}
\newcommand{\tr}{\textrm{tr}}
\newcommand{\vol}{\textrm{vol}}
\newcommand{\diam}{\textrm{diam}}
\newcommand{\supp}{\textrm{supp}}
\newcommand{\Med}{\textrm{Med}}
\newcommand{\Leb}{\textrm{Leb}}
\newcommand{\Const}{\textrm{Const}}
\newcommand{\Avg}{\textrm{Avg}}
\newcommand{\id}{\textrm{id}}
\newcommand{\Ker}{\textrm{Ker}}
\newcommand{\im}{\textrm{Im}}




\renewcommand{\;}{\, ; \,}
\renewcommand{\d}{\, {d}}

\newcommand{\gyouretsu}[1]{\begin{pmatrix} #1 \end{pmatrix} }

%%図式

\usepackage[dvipdfm,all]{xy}


\newenvironment{claim}[1]{\par\noindent\underline{step:}\space#1}{}
\newenvironment{claimproof}[1]{\par\noindent{($\because$)}\space#1}{\hfill $\blacktriangle $}


\newcommand{\pprime}{{\prime \prime}}





%%


\title{臨界値を含まない水位上昇}
\date{}


\author{}


\begin{document}


\maketitle

\section{}


\begin{setting} $M$ を多様体, $f: M \rightarrow \mathbb R$ とする. 区間$ I \in \mathbb R$ に対して
\begin{align*}  M_{I}  \coloneqq  \cbra{p \in M \mid f(p) \in  I}   \end{align*}
という記号を導入する. 
\end{setting}



\begin{dfn}(上向きベクトル場). $M$ を閉多様体, $f: M \rightarrow \mathbb R$ をモース関数とする. $X$ を$M$ の滑らかなベクトル場とする. $X$ は \\
(1) $p \in M$ が$f$ の臨界点ではないならば, $X_p f > 0$ \\
(2) $p \in M $ が$f$ の指数$\lambda$ の臨界点であるならば, $p$ のまわりの局所座標で, $f, X$ をそれぞれ \\  
\begin{align*} \quad f = -x_1 ^2 - \cdots - x^2_{\lambda} + x^2_{\lambda + 1} + \cdots + x^2_m \\
\quad X =  - 2x_1 \partial_1 - \cdots - 2 x_{\lambda } \partial_\lambda + 2 x_{\lambda + 1} \partial_{\lambda + 1} + \cdots + 2x_m \partial_m \end{align*}
と局所表示できるようなものがとれる. 
$f$ に適合した上向きベクトル場 という. 
\end{dfn}

\begin{prop}(上向きベクトル場の存在). $M$ をコンパクト多様体, $f: M \rightarrow \mathbb R$ をモース関数とする. このとき, $f$ に適合した上向きベクトル場$X$ が存在する.

\end{prop}
\begin{pf*}
有限個の, コンパクト集合と座標近傍
\qed
\end{pf*}



\begin{prop}
$M$ を連結な閉多様体, $f: M \rightarrow \mathbb R$ をモース関数とする. $[a,b]$ の中に$f$ の臨界値を含まなければ, $M_{[a,b]}$ は
\begin{align*} [f = a] \times [0,1]  \end{align*}
と微分同相である. 
\end{prop}
\begin{pf*}

\qed
\end{pf*}




\begin{prop} $M$ を連結な閉多様体, $f: M \rightarrow \mathbb R$ をモース関数とする. $a, A$ をそれぞれ$f$ の最小値と最大値とする. $a < b < c < A$ なる実数$b, c$ に対して, $M_{[b,c]}$ が$f$ の臨界値を含まないならば, $M_{( - \infty, b]}$ と$M_{( - \infty, c]} $ は微分同相である.   

\end{prop}
\begin{pf*}

\qed
\end{pf*}







\end{document}