\documentclass[10pt, fleqn, label-section=none]{bxjsarticle}

%\usepackage[driver=dvipdfm,hmargin=25truemm,vmargin=25truemm]{geometry}

\setpagelayout{driver=dvipdfm,hmargin=25truemm,vmargin=20truemm}


\usepackage{amsmath}
\usepackage{amssymb}
\usepackage{amsfonts}
\usepackage{amsthm}
\usepackage{mathtools}
\usepackage{mleftright}

\usepackage{ascmac}




\usepackage{otf}

\theoremstyle{definition}
\newtheorem{dfn}{定義}[section]
\newtheorem{ex}[dfn]{例}
\newtheorem{lem}[dfn]{補題}
\newtheorem{prop}[dfn]{命題}
\newtheorem{thm}[dfn]{定理}
\newtheorem{setting}[dfn]{設定}
\newtheorem{cor}[dfn]{系}
\newtheorem*{pf*}{証明}
\newtheorem{problem}[dfn]{問題}
\newtheorem*{problem*}{問題}
\newtheorem{remark}[dfn]{注意}
\newtheorem*{claim*}{\underline{claim}}



\newtheorem*{solution*}{解答}

%箇条書きの様式
\renewcommand{\labelenumi}{(\arabic{enumi})}


%

\newcommand{\forany}{\rm{for} \ {}^{\forall}}
\newcommand{\foranyeps}{
\rm{for} \ {}^{\forall}\varepsilon >0}
\newcommand{\foranyk}{
\rm{for} \ {}^{\forall}k}


\newcommand{\any}{{}^{\forall}}
\newcommand{\suchthat}{\, \rm{s.t.} \, \it{}}




\newcommand{\veps}{\varepsilon}
\newcommand{\paren}[1]{\mleft( #1\mright )}
\newcommand{\cbra}[1]{\mleft\{#1\mright\}}
\newcommand{\sbra}[1]{\mleft\lbrack#1\mright\rbrack}
\newcommand{\tbra}[1]{\mleft\langle#1\mright\rangle}
\newcommand{\abs}[1]{\left|#1\right|}
\newcommand{\norm}[1]{\left\|#1\right\|}
\newcommand{\lopen}[1]{\mleft(#1\mright\rbrack}
\newcommand{\ropen}[1]{\mleft\lbrack #1 \mright)}



%
\newcommand{\Rn}{\mathbb{R}^n}
\newcommand{\Cn}{\mathbb{C}^n}

\newcommand{\Rm}{\mathbb{R}^m}
\newcommand{\Cm}{\mathbb{C}^m}


\newcommand{\projs}[2]{\it{p}_{#1,\ldots,#2}}
\newcommand{\projproj}[2]{\it{p}_{#1,#2}}

\newcommand{\proj}[1]{p_{#1}}

%可測空間
\newcommand{\stdProbSp}{\paren{\Omega, \mathcal{F}, P}}

%微分作用素
\newcommand{\ddt}{\frac{d}{dt}}
\newcommand{\ddx}{\frac{d}{dx}}
\newcommand{\ddy}{\frac{d}{dy}}

\newcommand{\delt}{\frac{\partial}{\partial t}}
\newcommand{\delx}{\frac{\partial}{\partial x}}

%ハイフン
\newcommand{\hyphen}{\text{-}}

%displaystyle
\newcommand{\dstyle}{\displaystyle}

%⇔, ⇒, \UTF{21D0}%
\newcommand{\LR}{\Leftrightarrow}
\newcommand{\naraba}{\Rightarrow}
\newcommand{\gyaku}{\Leftarrow}

%理由
\newcommand{\naze}[1]{\paren{\because {\mathop{ #1 }}}}

%
\newcommand{\sankaku}{\hfill $\triangle$}

%
\newcommand{\push}{_{\#}}

%手抜き
\newcommand{\textif}{\textrm{if}\,\,\,}
\newcommand{\Ric}{\textrm{Ric}}
\newcommand{\tr}{\textrm{tr}}
\newcommand{\vol}{\textrm{vol}}
\newcommand{\diam}{\textrm{diam}}
\newcommand{\supp}{\textrm{supp}}
\newcommand{\Med}{\textrm{Med}}
\newcommand{\Leb}{\textrm{Leb}}
\newcommand{\Const}{\textrm{Const}}
\newcommand{\Avg}{\textrm{Avg}}
\newcommand{\id}{\textrm{id}}
\newcommand{\Ker}{\textrm{Ker}}
\newcommand{\im}{\textrm{Im}}
\newcommand{\dil}{\textrm{dil}}
\newcommand{\Ch}{\textrm{Ch}}
\newcommand{\Lip}{\textrm{Lip}}
\newcommand{\Ent}{\textrm{Ent}}
\newcommand{\dom}{\textrm{dom}}



\renewcommand{\;}{\, ; \,}
\renewcommand{\d}{\, {d}}

\newcommand{\gyouretsu}[1]{\begin{pmatrix} #1 \end{pmatrix} }

%%図式

\usepackage[dvipdfm,all]{xy}


\newenvironment{claim}[1]{\par\noindent\underline{step:}\space#1}{}
\newenvironment{claimproof}[1]{\par\noindent{($\because$)}\space#1}{\hfill $\blacktriangle $}


\newcommand{\pprime}{{\prime \prime}}





%%


\title{接続と平行移動}
\date{}


\author{}


\begin{document}


\maketitle

\section{}

\subsection{接続と平行移動}

\begin{dfn}滑らかな曲線$c$に沿ったベクトル場$X \in \Gamma(c^* TM)$ は, 任意の$t \in \dom c$ に対して
\begin{align*} (\nabla^{c^*}_t X )(t) = 0 \end{align*}
が成り立つときに, 曲線$c$と平行であるという. 
\end{dfn}

\begin{dfn}滑らかな曲線$c$ と, ベクトル場$X_s \in T_{c(s)} M$ に対して同型写像$P^s_t : T_{c(s)} M \rightarrow T_{c(t)} M$ を
$X_s$ を微分方程式
\begin{align*} (\nabla^{c^*}_t V )(t) = 0, \quad V_s = X_s \end{align*}
の解の時刻$t$ における値$V_t$ で定め, 
\begin{align*} P^s_t X_s \coloneqq X_t\end{align*}
により定める. 
\end{dfn}

\begin{remark}
同型写像であることは, 微分方程式の解の一意性から従う. また, 規約表記と同じ見方ができるように, $P^s_t$ の右上と$X_s$の右下が同じ時刻になるように記号を定めている. 
\end{remark}


\begin{prop}$X \in \Gamma(c^* TM)$ を滑らかな曲線$c$に平行なベクトル場とし, $s \in \dom c$ とする. $P^s_t X_s \in \Gamma(c^* TM)$ に対して
\begin{align*} \nabla_t P^{s}_t X_{s} = 0  \end{align*}
が成り立つ. 
\end{prop}
\begin{pf*}
そうなるように定義しているので明らか. 
\qed
\end{pf*}


\begin{prop}
$X$ を滑らかな曲線$c$ に沿った平行とは限らないベクトル場とする. 
\begin{align*} (\nabla^{c^*}_t X ) _0 =  \lim_{t \rightarrow 0} \frac{P^0_t X_0 - X_0}{ t } \end{align*}
が成り立つ. 
\end{prop}
\begin{pf*}
$c_{0}$ において正規直交基底$\cbra{e_{i0} }$ をとり, $X_t = X^i_t P^{0}_t e_{i0}  $ と表示しておく.  
\begin{align*} &(\nabla^{c^*}_t X ) _0 = (\partial_t X^i)_0 e_{i0} +  X^i_t (\nabla^{c^*}_t P^{0}_t e_{i0})_0  =  (\partial_t X^i)_0 e_{i0}. \\ 
&\lim_{t \rightarrow 0} \frac{P^t_0 X_t - X_0}{ t } = \lim_{t \rightarrow 0} \frac{P^t_0 X^i_t P^{0}_t e_{i0} - X^i_0 e_{i0}}{ t } =  \lim_{t \rightarrow 0} \frac{X^i_t P^t_0 P^{0}_t e_{i0} - X^i_0 e_{i0}}{ t } = \lim_{t \rightarrow 0} \frac{X^i_t - X^i_0}{t} e_{i0} = (\partial_t X^i)_0\end{align*}
\qed
\end{pf*}


\subsection{接空間の標準的な同一視について}

\begin{setting}
多様体$M$ の$p$ における接空間を, $v:C^\infty(M ; \mathbb R) \rightarrow \mathbb R$ でライプニッツ則をみたすもの全体として定義する. 
\end{setting}

\begin{setting}
ユークリッド空間$\mathbb R^n$ の座標は恒等写像によりとる. 
\end{setting}

\begin{setting}
接空間$T_pM$ 上の滑らかな関数に対する, 同相写像で定めた座標に対応する偏微分作用素を$(\delta_i)v$ で表す. 
\end{setting}

\begin{dfn}(ユークリッド空間の接空間と同一視). 線型写像$\iota_p : \mathbb R^n \rightarrow T_p \mathbb R^n$ を
\begin{align*} \iota_p (e_i) \coloneqq (\partial_i )_p  \quad (i = 1, \ldots , n)\end{align*}
とし, これを線型に拡張することで定める. これをユークリッド空間と標準的な同一視という. 
\end{dfn}

\begin{dfn}(接空間の接空間と同一視). 線型写像$\iota_v :  T_p M \rightarrow T_v(T_p \mathbb M)$ を
\begin{align*} \iota_v ((\partial_i)_p) \coloneqq (\delta_i )_v  \quad (i = 1, \ldots , n)\end{align*}
とし, これを線型に拡張することで定める. これを接空間と2次接空間の標準的な同一視という. 
\end{dfn}

\begin{prop}$T_pM$上の曲線$\omega(t) \coloneqq t\xi$ に対して
\begin{align*} \dot \omega (0) = \iota_o \xi \end{align*}
が成り立つ. 
\end{prop}
\begin{pf*}
$\xi = \xi^i (\partial_i)_p$ と表す. $\iota_o (\xi) = \iota_o (\xi^i (\partial_i)_p) = \xi^i (\delta_i)_o$ である. 滑らかな関数$f: T_pM \rightarrow \mathbb R$ に対して
\begin{align*} \partial_t |_{t = 0} f(\omega (t) ) = \nabla f (o) \cdot \xi = \xi^i ((\delta_i )_o f) = \iota_o (\xi) f  \end{align*}
より主張が従う. 
\qed
\end{pf*}

\begin{prop}$T_pM$上の曲線$\omega(s) \coloneqq t \xi + ts \zeta$ に対して
\begin{align*} \dot \omega (0) = \iota_{t \xi}  (t \zeta) \end{align*}
が成り立つ. 
\end{prop}
\begin{pf*}
$\zeta = \zeta^i (\partial_i)_p$ と表す. $\iota_{t\xi} (\zeta) = \iota_{t\xi} (\zeta^i (\partial_i)_p) = \zeta^i (\delta_i)_{t\xi}$ である. 滑らかな関数$f: T_pM \rightarrow \mathbb R$ に対して
\begin{align*} \partial_s |_{s= 0} f(\omega (s) ) = \nabla f (t\xi) \cdot (t \zeta) = t \zeta^i ((\delta_i )_{t\xi} f) = \iota_{t\xi} (t\zeta) f  \end{align*}
より主張が従う. 
\qed
\end{pf*}


\begin{prop}
\begin{align*} d(\exp_p)_{o} \circ \iota_o = \id_{T_p M} \end{align*}
が成り立つ. 
\end{prop}
\begin{pf*}任意に$\xi \in T_pM$ をとる. 
$T_p M$ 上の曲線$\omega (t) \coloneqq t \xi$ を考える. 
\begin{align*} d(\exp_p)_{o} \circ \iota_o (\xi) =  d(\exp_p)_{o} (\dot \omega (0)) = (\partial_t )|_{t = 0} (\exp_p \circ \omega ) (t) = (\partial_t )|_{t = 0} (\exp _p (t \xi)) = \xi  \end{align*}
\qed
\end{pf*}



\subsection{ヤコビ場と指数写像}

\begin{setting}
指数写像による座標近傍をとる. 
\end{setting}

\begin{prop}$J_0 = 0, (\nabla_t J)_0 = \eta \in T_pM$ を満たす$\exp_p (t\xi)$ に沿ったヤコビ場に対して
\begin{align*} d(\exp_p)_{t\xi} \iota_{ t \xi} \eta = \frac{1}{t} J_t \end{align*} 
が成り立つ. 
\end{prop}
\begin{pf*}このヤコビ場は$\exp_p(t\xi + ts \eta)$ の変分ベクトル場として与えられるので, 
\begin{align*} J_t = (\partial_s)|_{s = 0} \exp_p (t \xi + ts \eta) = (d\exp_p)_{t \xi} (\iota_{t\xi} t \eta) =  t (d\exp_p)_{t \xi} (\iota_{t\xi}  \eta) \end{align*}
が成り立つ. 
\qed
\end{pf*}

\begin{prop}$J_0 = 0, (\nabla_t J)_0 = (\partial_i)_p \in T_p M $ を満たす$\exp_p (t\xi)$ に沿ったヤコビ場に対して
\begin{align*} d(\exp_p)_{t\xi} \iota_{ t \xi} (\partial_i)_p = \frac{1}{t} J_t, \quad  (\partial_i)_{\exp_p t\xi} = \frac{1}{t} J_t \end{align*} 
が成り立つ. 
\end{prop}
\begin{pf*}一つ目は前述の命題から直ちに従う. 二つ目は, 
\begin{align*} d(\exp_p)_{t\xi} \iota_{t \xi} (\partial_i)_p f = \partial_s|_{s = 0} \exp_p (t \xi + ts (\partial_i)p) f = \partial_ i (f \circ \exp_p) (\exp_p t\xi) = \partial_i |_{\exp_p (t\xi)} f  \end{align*}
が成り立つ. 
\qed
\end{pf*}





\end{document}