\documentclass[10pt, fleqn, label-section=none]{bxjsarticle}

%\usepackage[driver=dvipdfm,hmargin=25truemm,vmargin=25truemm]{geometry}

\setpagelayout{driver=dvipdfm,hmargin=25truemm,vmargin=20truemm}


\usepackage{amsmath}
\usepackage{amssymb}
\usepackage{amsfonts}
\usepackage{amsthm}
\usepackage{mathtools}
\usepackage{mleftright}

\usepackage{ascmac}




\usepackage{otf}

\theoremstyle{definition}
\newtheorem{dfn}{定義}[section]
\newtheorem{ex}[dfn]{例}
\newtheorem{lem}[dfn]{補題}
\newtheorem{prop}[dfn]{命題}
\newtheorem{thm}[dfn]{定理}
\newtheorem{setting}[dfn]{設定}
\newtheorem{notation}[dfn]{記号}
\newtheorem{cor}[dfn]{系}
\newtheorem*{pf*}{証明}
\newtheorem{problem}[dfn]{問題}
\newtheorem*{problem*}{問題}
\newtheorem{remark}[dfn]{注意}
\newtheorem*{claim*}{\underline{claim}}



\newtheorem*{solution*}{解答}

%箇条書きの様式
\renewcommand{\labelenumi}{(\arabic{enumi})}


%

\newcommand{\forany}{\rm{for} \ {}^{\forall}}
\newcommand{\foranyeps}{
\rm{for} \ {}^{\forall}\varepsilon >0}
\newcommand{\foranyk}{
\rm{for} \ {}^{\forall}k}


\newcommand{\any}{{}^{\forall}}
\newcommand{\suchthat}{\, \rm{s.t.} \, \it{}}




\newcommand{\veps}{\varepsilon}
\newcommand{\paren}[1]{\mleft( #1\mright )}
\newcommand{\cbra}[1]{\mleft\{#1\mright\}}
\newcommand{\sbra}[1]{\mleft\lbrack#1\mright\rbrack}
\newcommand{\tbra}[1]{\mleft\langle#1\mright\rangle}
\newcommand{\abs}[1]{\left|#1\right|}
\newcommand{\norm}[1]{\left\|#1\right\|}
\newcommand{\lopen}[1]{\mleft(#1\mright\rbrack}
\newcommand{\ropen}[1]{\mleft\lbrack #1 \mright)}



%
\newcommand{\Rn}{\mathbb{R}^n}
\newcommand{\Cn}{\mathbb{C}^n}

\newcommand{\Rm}{\mathbb{R}^m}
\newcommand{\Cm}{\mathbb{C}^m}


\newcommand{\projs}[2]{\it{p}_{#1,\ldots,#2}}
\newcommand{\projproj}[2]{\it{p}_{#1,#2}}

\newcommand{\proj}[1]{p_{#1}}

%可測空間
\newcommand{\stdProbSp}{\paren{\Omega, \mathcal{F}, P}}

%微分作用素
\newcommand{\ddt}{\frac{d}{dt}}
\newcommand{\ddx}{\frac{d}{dx}}
\newcommand{\ddy}{\frac{d}{dy}}

\newcommand{\delt}{\frac{\partial}{\partial t}}
\newcommand{\delx}{\frac{\partial}{\partial x}}

%ハイフン
\newcommand{\hyphen}{\text{-}}

%displaystyle
\newcommand{\dstyle}{\displaystyle}

%⇔, ⇒, \UTF{21D0}%
\newcommand{\LR}{\Leftrightarrow}
\newcommand{\naraba}{\Rightarrow}
\newcommand{\gyaku}{\Leftarrow}

%理由
\newcommand{\naze}[1]{\paren{\because {\mathop{ #1 }}}}

%
\newcommand{\sankaku}{\hfill $\triangle$}

%
\newcommand{\push}{_{\#}}

%手抜き
\newcommand{\textif}{\textrm{if}\,\,\,}
\newcommand{\Ric}{\textrm{Ric}}
\newcommand{\tr}{\textrm{tr}}
\newcommand{\vol}{\textrm{vol}}
\newcommand{\diam}{\textrm{diam}}
\newcommand{\supp}{\textrm{supp}}
\newcommand{\Med}{\textrm{Med}}
\newcommand{\Leb}{\textrm{Leb}}
\newcommand{\Const}{\textrm{Const}}
\newcommand{\Avg}{\textrm{Avg}}
\newcommand{\id}{\textrm{id}}
\newcommand{\Ker}{\textrm{Ker}}
\newcommand{\im}{\textrm{Im}}
\newcommand{\dil}{\textrm{dil}}
\newcommand{\Ch}{\textrm{Ch}}
\newcommand{\Lip}{\textrm{Lip}}
\newcommand{\Ent}{\textrm{Ent}}
\newcommand{\grad}{\textrm{grad}}
\newcommand{\dom}{\textrm{dom}}
\newcommand{\diag}{\textrm{diag}}

\renewcommand{\;}{\, ; \,}
\renewcommand{\d}{\, {d}}

\newcommand{\gyouretsu}[1]{\begin{pmatrix} #1 \end{pmatrix} }

\renewcommand{\div}{\textrm{div}}


%%図式

\usepackage[dvipdfm,all]{xy}


\newenvironment{claim}[1]{\par\noindent\underline{step:}\space#1}{}
\newenvironment{claimproof}[1]{\par\noindent{($\because$)}\space#1}{\hfill $\blacktriangle $}


\newcommand{\pprime}{{\prime \prime}}

%%マグニチュード


\newcommand{\Mag}{\textrm{Mag}}

\usepackage{mathrsfs}


%%6.13
\def\chint#1{\mathchoice
{\XXint\displaystyle\textstyle{#1}}%
{\XXint\textstyle\scriptstyle{#1}}%
{\XXint\scriptstyle\scriptscriptstyle{#1}}%
{\XXint\scriptscriptstyle\scriptscriptstyle{#1}}%
\!\int}
\def\XXint#1#2#3{{\setbox0=\hbox{$#1{#2#3}{\int}$ }
\vcenter{\hbox{$#2#3$ }}\kern-.6\wd0}}
\def\ddashint{\chint=}
\def\dashint{\chint-}

%%7.13

\usepackage{here}

%7.15
\newcommand{\Span}{\textrm{Span}}


\title{直交凸体上の単調かつ平行移動不変な付値は連続}
\date{}


\author{}


\begin{document}


\maketitle

\section{}

\begin{dfn}(直交凸体). $\mathbb R^n$ の直交凸かつコンパクトな集合を直交凸体といい, その全体を$\mathcal K$ で表す. 

\end{dfn}

\begin{remark}$\mathcal K$ にはハウスドルフ距離を備えておく. 

\end{remark}


\begin{notation}$n$ 次元単位立方体を
\begin{align*} C_n \coloneqq [-\frac{1}{2}, \frac{1}{2}]^n  \end{align*}
で表す. 
\end{notation}

\begin{notation}$R \geq 0$ とする. $\mathcal K_n [R]$ で, $RC_n$ に含まれる直交凸集合を表す. 

\end{notation}

\begin{notation}$\nu_i: \mathbb R \rightarrow \mathbb R^n; x \mapsto (0, \ldots, x, \ldots, 0 )$ で第$i$ 成分への入射を表すことにする. 
\end{notation}



\begin{prop}単調増大な付値は, 非負である. 
\end{prop}
\begin{pf*}$\varphi$ を付値とする. 
$\varphi (\varnothing) = 0$ であるという付値の定義から明らか. 
\qed
\end{pf*}


\begin{prop}(区間との和に関する連続性). $\varphi : \mathcal K \rightarrow \mathbb R$ を, 単調かつ平行移動不変な付値とする. このとき, 任意の$K \in \mathcal K$ に対して
\begin{align*} \lim_{\delta \rightarrow 0} \varphi(K+ \nu_i [-\delta/2, \delta/2] ) = \varphi(K)   \end{align*}
が任意の$i$ に対して成り立つ. 
\end{prop}
\begin{pf*}任意に$K \in \mathcal K$ をとる. 
\begin{align*} F: [0, \infty) \rightarrow \mathbb R; \delta \mapsto \varphi(K+ \nu_i [-\delta/2, \delta/2] ) - \varphi(K) \end{align*}
と定める. 付値の定義から包除原理が成り立つことと, 平行移動不変であることに注意すると, 
\begin{align*} &F(\delta_1 + \delta_2) =  \varphi(K+ \nu_i [-\delta_1 , \delta _2] ) - \varphi(K) \\&= \varphi(K+ \nu_i [-\delta_1 , 0] ) + \varphi(K+ \nu_i [0 , \delta _2] ) - \varphi(K) - \varphi(K) = F(\delta_1) + F(\delta_2)  \end{align*}
が成り立つので, 加法的である. $\varphi$ が単調であること合わせると, $F$ は単調かつ加法的な関数であるので, 
\begin{align*} F(\delta) = \delta F(1) \end{align*}
が成り立つ. $\varphi$ が単調増大であるときには, 十分大きい$R \geq 0$ をとると, 

\begin{align*} &0 \leq F(1) = \varphi(K+ \nu_i [- 1/2 , 1/2 ] ) - \varphi (K) \\& \leq \varphi(K+ \nu_i [- 1/2 , 1/2 ] )  \leq \varphi(RC_n) \end{align*}
が成り立つ. 従って, 
\begin{align*} 0 \leq F(\delta ) =  \delta F(1) \leq  \delta \varphi(RC_n) \end{align*}
が成り立つので, $\delta \rightarrow 0$ と極限をとればよい. $\varphi$ が単調減少のときも同様にしてできる. 
\qed
\end{pf*}

\begin{prop}(キューブとの和に関する連続性). $\varphi : \mathcal K \rightarrow \mathbb R$ を, 単調かつ平行移動不変な付値とする. このとき, 任意の$K \in \mathcal K$ に対して
\begin{align*} \lim_{\delta \rightarrow 0} \varphi(K+ \delta C_n ) =  \varphi(K)   \end{align*}
が成り立つ. 

\end{prop}
\begin{pf*} $R \geq 0$ は十分大きくとっておく. 前述の命題の証明と全く同様に
\begin{align*} 0 \leq F(\delta ) \leq  \delta \varphi(RC_n) \end{align*}
が成り立つ. すなわち
\begin{align*} 0  \leq \varphi(K + \nu_i [- \delta /2 , \delta/2]) \leq \varphi(K) + \delta \varphi(R C_n) \end{align*}
\begin{align*} C_n = \nu_1 [-1/2, 1/2] + \nu_2 [- 1/2, 1/2] + \cdots \nu_n [-1/2, 1/2] \end{align*}
であるので, 任意の$\delta \geq  0$ に対して, 
\begin{align*} &\varphi (K + \delta C_n) \\&= \varphi(K + \nu_1 [-1/2, 1/2] + \nu_2 [- 1/2, 1/2] + \cdots \nu_n [-1/2, 1/2] ) \\&= \varphi(K + \nu_1 [-1/2, 1/2] + \nu_2 [- 1/2, 1/2] +\cdots \nu_{n-1} [-1/2, 1/2] ) +  \delta \varphi(RC_n) 
\\& = \varphi(K + \nu_1 [-1/2, 1/2] + \nu_2 [- 1/2, 1/2] +\cdots \nu_{n-2} [-1/2, 1/2] ) + 2 \delta \varphi(RC_n) 
\\& \cdots 
\\& = \varphi(K) + n  \delta \varphi(RC_n)
 \end{align*}
 が成り立つ. 
 故に, 
 \begin{align*} 0 \leq \varphi(K + \delta C_n) - \varphi(K) \leq n\delta (RC_n)  \end{align*}
 が成り立つので, 極限をとればよい. 
\qed
\end{pf*}

\begin{remark}
適当に証明を読めば, 一様収束であることもわかる. 
\end{remark}


\begin{prop}$\varphi : \mathcal K \rightarrow \mathbb R$ を, 単調かつ平行移動不変な付値とする. 
このとき, $\varphi$ は連続である. 
\end{prop}
\begin{pf*}$X \in \mathcal K$ に対して十分大きな$R \geq 2$ で, $X \subset (R - 2)C_n$ を満たすものをとる. 任意に$\veps > 0$ をとる. 前述の命題より$\eta \geq 0$ で
\begin{align*} \varphi(K + \eta C_n) \leq \varphi(K) + \veps \quad ( \any K \in \mathcal K) \end{align*}

\qed
\end{pf*}






\end{document}