\documentclass[twocolumn, landscape, a4paper , 8pt, fleqn, titlepage ]{jsarticle}
\usepackage[driver=dvipdfm,hmargin=20truemm,vmargin=25truemm]{geometry}

\usepackage{amsmath}
\usepackage{amssymb}
\usepackage{amsfonts}
\usepackage{amsthm}
\usepackage{mathtools}
\usepackage{mleftright}

%box
\usepackage{ascmac}

%%
\usepackage{xcolor} 
\usepackage[dvipdfmx]{hyperref}
\usepackage{pxjahyper}
\hypersetup{
setpagesize=false,
 bookmarksnumbered=true,
 bookmarksopen=true,
 colorlinks=true,
 linkcolor=teal,
 citecolor=black,
}
%
%

%


%%図式

\usepackage[dvipdfm,all]{xy}


%%



\usepackage{otf}

\theoremstyle{definition}
\newtheorem{dfn}{定義}[section]
\newtheorem{ex}[dfn]{例}
\newtheorem{lem}[dfn]{補題}
\newtheorem{prop}[dfn]{命題}
\newtheorem{thm}[dfn]{定理}
\newtheorem{cor}[dfn]{系}
\newtheorem*{pf*}{証明}
\newtheorem{problem}[dfn]{問題}
\newtheorem*{problem*}{問題}
\newtheorem{remark}[dfn]{注意}

\newtheorem*{solution*}{解答}

%箇条書きの様式
\renewcommand{\labelenumi}{(\arabic{enumi})}


%

\newcommand{\forany}{\rm{for} \ {}^{\forall}}
\newcommand{\foranyeps}{
\rm{for} \ {}^{\forall}\varepsilon >0}
\newcommand{\foranyk}{
\rm{for} \ {}^{\forall}k}


\newcommand{\any}{{}^{\forall}}
\newcommand{\suchthat}{\, \textrm{s.t.} \, }




\newcommand{\veps}{\varepsilon}
\newcommand{\paren}[1]{\mleft( #1\mright )}
\newcommand{\cbra}[1]{\mleft\{#1\mright\}}
\newcommand{\sbra}[1]{\mleft\lbrack#1\mright\rbrack}
\newcommand{\tbra}[1]{\mleft\langle#1\mright\rangle}

\newcommand{\ntbra}[1]{\langle#1\rangle}

\newcommand{\abs}[1]{\left|#1\right|}
\newcommand{\norm}[1]{\left\|#1\right\|}
\newcommand{\lopen}[1]{\mleft(#1\mright\rbrack}
\newcommand{\ropen}[1]{\mleft\lbrack #1 \mright)}



%
\newcommand{\Rn}{\mathbb{R}^n}
\newcommand{\Cn}{\mathbb{C}^n}

\newcommand{\Rm}{\mathbb{R}^m}
\newcommand{\Cm}{\mathbb{C}^m}


\newcommand{\supp}{\textrm{supp}\,} 

\newcommand{\ifufu}{\,\textrm {iff} \, \it}


\newcommand{\proj}[1]{\it{p}_{#1}}
\newcommand{\projs}[2]{\it{p}_{#1,\ldots,#2}}
\newcommand{\projproj}[2]{\it{p}_{#1,#2}}

\newcommand{\push}{_{\#}}

%可測空間
\newcommand{\stdProbSp}{\paren{\Omega, \mathcal{F}, P}}

%微分作用素
\newcommand{\ddt}{\frac{d}{dt}}
\newcommand{\ddx}{\frac{d}{dx}}
\newcommand{\ddy}{\frac{d}{dy}}

\newcommand{\delt}{\frac{\partial}{\partial t}}
\newcommand{\delx}{\frac{\partial}{\partial x}}

%ハイフン
\newcommand{\hyphen}{\text{-}}

%displaystyle
\newcommand{\dstyle}{\displaystyle}

%⇔, ⇒, \UTF{21D0}%
\newcommand{\LR}{\Leftrightarrow}
\newcommand{\naraba}{\Rightarrow}
\newcommand{\gyaku}{\Leftarrow}

%理由
\newcommand{\naze}[1]{\paren{\because {\mathop{ #1 }}}}

%ベクトル解析
\newcommand{\grad}{\textrm{grad}}
\renewcommand{\div}{\textrm{div}}

%手抜き
\newcommand{\textif}{\textrm{if}\,\,\,}
\newcommand{\Sgn}{\textrm{Sgn}}
\newcommand{\Ric}{\textrm{Ric}}
\newcommand{\Sec}{\textrm{Sec}}
\newcommand{\Scal}{\textrm{Scal}}
\newcommand{\tr}{\textrm{tr}}
\newcommand{\vol}{\textrm{vol}}
\newcommand{\diam}{\textrm{diam}}
\newcommand{\Med}{\textrm{Med}}
\newcommand{\Leb}{\textrm{Leb}}
\newcommand{\Const}{\textrm{Const}}
\newcommand{\Avg}{\textrm{Avg}}
\renewcommand{\d}{\, d}
\newcommand{\length}{\textrm{length}}
\newcommand{\Func}{\textrm{Func}}
\newcommand{\Ker}{\textrm{Ker}}
\newcommand{\Cone}{\textrm{Cone}}
\newcommand{\hess}{\textrm{hess}}
\newcommand{\esssup}{\textrm{ess}\,\textrm{sup}}

\newcommand{\sub}{\textrm{sub}}
\newcommand{\Par}{\textrm{Par}}


\newcommand{\perpperp}{{\perp \perp}}

\newcommand{\sgyouretsu}[1]{\paren{\begin{smallmatrix} #1 \end{smallmatrix} }}

\renewcommand{\ni}{\hspace{2pt} \textrm{I} \hspace{-5pt} \textrm{I} \hspace{2pt}}





%↓本体↓

\title{等長写像が接続を保存することの証明}

\author{}
\date{}

\begin{document}

\maketitle
\scriptsize 

\section{}

\begin{prop}
$f: M \rightarrow N$ を微分同相とする. 
\begin{align*} df[X,Y] = [dfX, dfY] \quad (X,Y \in \Gamma(TM))\end{align*}
が成り立つ. 
\end{prop}
\begin{pf*}
\begin{align*} ((dfY)F)\circ f = ((dfY)_{f(\cdot)})F = (df_{(\cdot)} Y) F = Y_{(\cdot)}(F\circ f) \end{align*}
であることに注意すると, 
\begin{align*} &(df[X,Y]) F = [X,Y] (F \circ f) = X (Y(F \circ f)) - Y(X(F \circ f)) \\& \quad = X(((dfY) F) \circ f) - Y (((dfX) F) \circ f) = (dfX)((dfY)F) - (dfY)((dfX)F) = [dfX, dfY] \end{align*}
\qed
\end{pf*}


\begin{prop}
$f: (M,g) \rightarrow (\tilde M, \tilde g)$ を等長写像, $\nabla, \tilde \nabla$ をそれぞれ$(M, g), (\tilde M, \tilde g)$ のレビチビタ接続とする. このとき, 
\begin{align*} &(1)df(\nabla_X Y ) = \tilde \nabla _{dfX} dfY \quad (X, Y \in \Gamma (TM)) \\&(2) df(R(X,Y)Z) = R(dfX,dfY)dfZ \quad (X,Y,Z \in \Gamma(TM) ) \end{align*}
が成り立つ. 
\end{prop}
\begin{pf*}
\begin{align*} \check \nabla _{\tilde X} \tilde Y \coloneqq df(\nabla_X Y) \quad (\tilde X, \tilde Y \in \Gamma(\tilde M) ) \end{align*}
により$\tilde M$ に新しい接続を定める(ただし, $X, Y \coloneqq df^{-1} (X), df^{-1} (Y) $ ).  $p \in M$ に対して$\dot c_0 = X_p , c_0 = p$ なる曲線$c$ をとる. 
\begin{align*} &\check{\nabla}_{\tilde X} (\tilde g (\tilde Y, \tilde Z)) = \tilde X (\tilde g (\tilde Y, \tilde Z)) = (dfX) \tilde g (dfY, dfZ) \\
& \quad = \partial_t|_0 \paren{\tilde g (dfY, dfZ) \circ (f(c(t))) } = \partial_t | _0 (f^* \tilde g (Y,Z) _{c(t)}) = \partial_t |_0 (g(Y,Z)) \\
&\quad = X(g(Y,Z)) = g(\nabla_X Y, Z ) + g(Y, \nabla_X Z) = f^* \tilde g (\nabla_X Y, Z)+ f^* \tilde g (Y, \nabla_X Z) \\& \quad = \tilde g (\check{\nabla}_{\tilde X} \tilde Y, \tilde Z) + \tilde g (\tilde Y, \check{\nabla}_{\tilde X} \tilde Z)  \end{align*}
であることと, 
\begin{align*} \check{\nabla} _{\tilde X} \tilde Y -  \check{\nabla} _{\tilde Y} \tilde X -  [\tilde X, \tilde Y] = df(\nabla_X Y - \nabla_Y X) - df[X,Y] = df[X,Y] - df[X,Y] = 0\end{align*}
であることから, $\check \nabla$ は$(\tilde M , \tilde g)$ のレビチビタ接続である. 従って, レビチビタ接続の一意性から, $\check \nabla = \tilde \nabla$ が成り立つ. すなわち, 
\begin{align*} \tilde{\nabla}_{dfX} dfY =  \tilde{\nabla}_{\tilde X} \tilde Y = \check{\nabla}_{\tilde X} \tilde Y = df(\nabla_X Y) \end{align*}
である. (2)は$R(X,Y)Z = \nabla_X \nabla_Y Z - \nabla_Y \nabla_X Z - \nabla_{[X,Y]}Z$ であるので, (1)と合わせると容易に従う. 
\qed
\end{pf*}


\begin{prop}
$f: (M,g) \rightarrow (\tilde M, \tilde g)$ を等長写像とする. このとき, $v \in T_pM, df_p v \in T_{f(p)} \tilde M$ を始方向とする測地線$\gamma, \tilde \gamma$ に対して
\begin{align*} &(1) \tilde \gamma (t) = f \circ \gamma (t) \\&(2)df\circ P_{\gamma}^{s,t} = P_{\tilde \gamma}^{s,t}\circ df \end{align*}
が成り立つ.
\end{prop}
\begin{pf*}
($1$)$\tilde \nabla_{df \dot \gamma_t} (df \dot \gamma_t) = df (\nabla_{\dot \gamma_t} \dot \gamma_t) = 0$ なので, $f\circ \gamma (t) $ は測地線である. また, 始点が$f \circ \gamma (0) = f(p)$ であり, 始方向が$df(\dot \gamma_0) = df_p (v)$ であるので, $f\circ \gamma = \tilde \gamma$ が成り立つ. ($2$)$w \in T_{\gamma_s} M$ に対して, $\tilde W _t \coloneqq df(P_{\gamma}^{s,t}(w))$ は$\tilde \nabla_{\dot{\tilde \gamma}_t} df(P_{\gamma}^{s,t}(w)) = df(\nabla_{\dot \gamma_t} (P_{\gamma}^{s,t} w)) = 0$ より, $\tilde W_t$ は$\tilde \gamma_t$ に沿って平行なベクトル場であるので, 
\begin{align*}df(P_{\gamma}^{s,t}(w)) =  \tilde W_t = P_{\tilde \gamma}^{s,t} \tilde W_s  = P_{\tilde \gamma}^{s,t} df(P_{\gamma}^{s,s}(w)) = P_{\tilde \gamma}^{s,t} dfw \end{align*} 
\qed
\end{pf*}



















\end{document}