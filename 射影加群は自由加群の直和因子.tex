\documentclass[10pt, fleqn, label-section=none]{bxjsarticle}

%\usepackage[driver=dvipdfm,hmargin=25truemm,vmargin=25truemm]{geometry}

\setpagelayout{driver=dvipdfm,hmargin=25truemm,vmargin=20truemm}


\usepackage{amsmath}
\usepackage{amssymb}
\usepackage{amsfonts}
\usepackage{amsthm}
\usepackage{mathtools}
\usepackage{mleftright}

\usepackage{ascmac}




\usepackage{otf}

\theoremstyle{definition}
\newtheorem{dfn}{定義}[section]
\newtheorem{ex}[dfn]{例}
\newtheorem{lem}[dfn]{補題}
\newtheorem{prop}[dfn]{命題}
\newtheorem{thm}[dfn]{定理}
\newtheorem{cor}[dfn]{系}
\newtheorem*{pf*}{証明}
\newtheorem{problem}[dfn]{問題}
\newtheorem*{problem*}{問題}
\newtheorem{remark}[dfn]{注意}
\newtheorem*{claim*}{\underline{claim}}



\newtheorem*{solution*}{解答}

%箇条書きの様式
\renewcommand{\labelenumi}{(\arabic{enumi})}


%

\newcommand{\forany}{\rm{for} \ {}^{\forall}}
\newcommand{\foranyeps}{
\rm{for} \ {}^{\forall}\varepsilon >0}
\newcommand{\foranyk}{
\rm{for} \ {}^{\forall}k}


\newcommand{\any}{{}^{\forall}}
\newcommand{\suchthat}{\, \rm{s.t.} \, \it{}}




\newcommand{\veps}{\varepsilon}
\newcommand{\paren}[1]{\mleft( #1\mright )}
\newcommand{\cbra}[1]{\mleft\{#1\mright\}}
\newcommand{\sbra}[1]{\mleft\lbrack#1\mright\rbrack}
\newcommand{\tbra}[1]{\mleft\langle#1\mright\rangle}
\newcommand{\abs}[1]{\left|#1\right|}
\newcommand{\norm}[1]{\left\|#1\right\|}
\newcommand{\lopen}[1]{\mleft(#1\mright\rbrack}
\newcommand{\ropen}[1]{\mleft\lbrack #1 \mright)}



%
\newcommand{\Rn}{\mathbb{R}^n}
\newcommand{\Cn}{\mathbb{C}^n}

\newcommand{\Rm}{\mathbb{R}^m}
\newcommand{\Cm}{\mathbb{C}^m}


\newcommand{\projs}[2]{\it{p}_{#1,\ldots,#2}}
\newcommand{\projproj}[2]{\it{p}_{#1,#2}}

\newcommand{\proj}[1]{p_{#1}}

%可測空間
\newcommand{\stdProbSp}{\paren{\Omega, \mathcal{F}, P}}

%微分作用素
\newcommand{\ddt}{\frac{d}{dt}}
\newcommand{\ddx}{\frac{d}{dx}}
\newcommand{\ddy}{\frac{d}{dy}}

\newcommand{\delt}{\frac{\partial}{\partial t}}
\newcommand{\delx}{\frac{\partial}{\partial x}}

%ハイフン
\newcommand{\hyphen}{\text{-}}

%displaystyle
\newcommand{\dstyle}{\displaystyle}

%⇔, ⇒, \UTF{21D0}%
\newcommand{\LR}{\Leftrightarrow}
\newcommand{\naraba}{\Rightarrow}
\newcommand{\gyaku}{\Leftarrow}

%理由
\newcommand{\naze}[1]{\paren{\because {\mathop{ #1 }}}}

%
\newcommand{\sankaku}{\hfill $\triangle$}

%
\newcommand{\push}{_{\#}}

%手抜き
\newcommand{\textif}{\textrm{if}\,\,\,}
\newcommand{\Ric}{\textrm{Ric}}
\newcommand{\tr}{\textrm{tr}}
\newcommand{\vol}{\textrm{vol}}
\newcommand{\diam}{\textrm{diam}}
\newcommand{\supp}{\textrm{supp}}
\newcommand{\Med}{\textrm{Med}}
\newcommand{\Leb}{\textrm{Leb}}
\newcommand{\Const}{\textrm{Const}}
\newcommand{\Avg}{\textrm{Avg}}
\newcommand{\id}{\textrm{id}}
\newcommand{\Ker}{\textrm{Ker}}
\newcommand{\im}{\textrm{Im}}




\renewcommand{\;}{\, ; \,}
\renewcommand{\d}{\, {d}}

\newcommand{\gyouretsu}[1]{\begin{pmatrix} #1 \end{pmatrix} }

%%図式

\usepackage[dvipdfm,all]{xy}



\title{射影加群は自由加群の直和因子}
\date{}


\author{}


\begin{document}


\maketitle



\section{}

\subsection{}

\begin{remark}
本文中の加群は環$R$ 上のものとする. 
\end{remark}


\begin{dfn}(短完全系列の分裂). 加群の短完全系列
\begin{align*}  \xymatrix@C=13pt{
& 0 \ar[r] &A \ar[r]^f &B \ar[r]^g &C \ar[r] & 0
} \end{align*}
は,  $\textrm{Im} f = \textrm{Ker} g$ が$B$ の直和因子であるときに分裂するという. 
\end{dfn}


\begin{prop}加群の短完全系列
\begin{align*}  \xymatrix@C=13pt{
& 0 \ar[r] &A \ar[r] ^i &B  \ar[r] &C \ar[r] & 0
} \end{align*}
は, 準同型$j: B\rightarrow A$で$j \circ i =  \textrm{id}$ を満たすものが存在するならば, 分裂する. 

\end{prop}
\begin{pf*}
実際, $B = \im i \oplus \Ker j$ であることを確かめる. 任意の$b \in B$ に対して
$i\circ j (b)  \in \im i, b- i \circ j (b) \in \Ker j$ ととれば, $b = i \circ j (b) + b- i \circ j (b)$ であるので, $B$ は$\im i, \Ker j$ の和空間である. また, $i(a) \in \im i, b \in \Ker j$ に対して
\begin{align*} i (a) + b = 0\end{align*}
であるならば, $0 = j \circ i (a) + j(b) = a + 0 = a$ より, $i(a) = 0$ であることがわかり, それによって$b = 0$ であることもわかる. 従って, 直和である. 
\qed
\end{pf*}

\begin{remark}
前述の証明をなぞると, \xymatrix@C=13pt{B \ar[r] ^i &A  \ar[r]^j &B } は$j \circ i = \textrm{id}_B$ であれば, $B = \im i \oplus \Ker j$ 
\end{remark}


\begin{dfn}(射影加群).
加群$P$ は, 任意の加群$B, A$と全射準同型$g:B \rightarrow A$と準同型$f: P \rightarrow A$ に対して準同型$h: P \rightarrow B$ で$g \circ h = f$ をみたすものが存在する時に, 射影(的)加群という.
\end{dfn}

\begin{prop}
射影加群$P$は自由加群の直和因子である. 
\end{prop}
\begin{pf*}
$P$ を集合と見做したときの形式的有限和$\displaystyle \sum_{\textrm{有限}} r_i a_i \quad (r_i \in R, a_i \in P)$ 全体により自然に定まる自由加群を$F(P)$とする. 明らかに$F(P)$ から$P$ への全射準同型が存在するのでそれを$g$ で表すことにする. 
$P$ は射影的加群なので, $g \circ h = \textrm{id}_P $ を満たす準同型$h: P \rightarrow F(P)$ が存在する. 従って, 図式
\begin{align*}  \xymatrix@C=13pt{
  &P \ar[r]^h &F(P)  \ar[r]^g &P  
} \end{align*}
は$g \circ h = \textrm{id}_P$ を満たすので, $F(P) = \im h \oplus \Ker g  \simeq P \oplus \Ker g$
\qed
\end{pf*}















\end{document}