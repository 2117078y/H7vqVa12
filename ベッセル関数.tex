\documentclass[10pt, fleqn, label-section=none]{bxjsarticle}

%\usepackage[driver=dvipdfm,hmargin=25truemm,vmargin=25truemm]{geometry}

\setpagelayout{driver=dvipdfm,hmargin=25truemm,vmargin=20truemm}


\usepackage{amsmath}
\usepackage{amssymb}
\usepackage{amsfonts}
\usepackage{amsthm}
\usepackage{mathtools}
\usepackage{mleftright}

\usepackage{ascmac}




\usepackage{otf}

\theoremstyle{definition}
\newtheorem{dfn}{定義}[section]
\newtheorem{ex}[dfn]{例}
\newtheorem{lem}[dfn]{補題}
\newtheorem{prop}[dfn]{命題}
\newtheorem{thm}[dfn]{定理}
\newtheorem{setting}[dfn]{設定}
\newtheorem{notation}[dfn]{記号}
\newtheorem{cor}[dfn]{系}
\newtheorem*{pf*}{証明}
\newtheorem{problem}[dfn]{問題}
\newtheorem*{problem*}{問題}
\newtheorem{remark}[dfn]{注意}
\newtheorem*{claim*}{\underline{claim}}



\newtheorem*{solution*}{解答}

%箇条書きの様式
\renewcommand{\labelenumi}{(\arabic{enumi})}


%

\newcommand{\forany}{\rm{for} \ {}^{\forall}}
\newcommand{\foranyeps}{
\rm{for} \ {}^{\forall}\varepsilon >0}
\newcommand{\foranyk}{
\rm{for} \ {}^{\forall}k}


\newcommand{\any}{{}^{\forall}}
\newcommand{\suchthat}{\, \rm{s.t.} \, \it{}}




\newcommand{\veps}{\varepsilon}
\newcommand{\paren}[1]{\mleft( #1\mright )}
\newcommand{\cbra}[1]{\mleft\{#1\mright\}}
\newcommand{\sbra}[1]{\mleft\lbrack#1\mright\rbrack}
\newcommand{\tbra}[1]{\mleft\langle#1\mright\rangle}
\newcommand{\abs}[1]{\left|#1\right|}
\newcommand{\norm}[1]{\left\|#1\right\|}
\newcommand{\lopen}[1]{\mleft(#1\mright\rbrack}
\newcommand{\ropen}[1]{\mleft\lbrack #1 \mright)}



%
\newcommand{\Rn}{\mathbb{R}^n}
\newcommand{\Cn}{\mathbb{C}^n}

\newcommand{\Rm}{\mathbb{R}^m}
\newcommand{\Cm}{\mathbb{C}^m}


\newcommand{\projs}[2]{\it{p}_{#1,\ldots,#2}}
\newcommand{\projproj}[2]{\it{p}_{#1,#2}}

\newcommand{\proj}[1]{p_{#1}}

%可測空間
\newcommand{\stdProbSp}{\paren{\Omega, \mathcal{F}, P}}

%微分作用素
\newcommand{\ddt}{\frac{d}{dt}}
\newcommand{\ddx}{\frac{d}{dx}}
\newcommand{\ddy}{\frac{d}{dy}}

\newcommand{\delt}{\frac{\partial}{\partial t}}
\newcommand{\delx}{\frac{\partial}{\partial x}}

%ハイフン
\newcommand{\hyphen}{\text{-}}

%displaystyle
\newcommand{\dstyle}{\displaystyle}

%⇔, ⇒, \UTF{21D0}%
\newcommand{\LR}{\Leftrightarrow}
\newcommand{\naraba}{\Rightarrow}
\newcommand{\gyaku}{\Leftarrow}

%理由
\newcommand{\naze}[1]{\paren{\because {\mathop{ #1 }}}}

%
\newcommand{\sankaku}{\hfill $\triangle$}

%
\newcommand{\push}{_{\#}}

%手抜き
\newcommand{\textif}{\textrm{if}\,\,\,}
\newcommand{\Ric}{\textrm{Ric}}
\newcommand{\tr}{\textrm{tr}}
\newcommand{\vol}{\textrm{vol}}
\newcommand{\diam}{\textrm{diam}}
\newcommand{\supp}{\textrm{supp}}
\newcommand{\Med}{\textrm{Med}}
\newcommand{\Leb}{\textrm{Leb}}
\newcommand{\Const}{\textrm{Const}}
\newcommand{\Avg}{\textrm{Avg}}
\newcommand{\id}{\textrm{id}}
\newcommand{\Ker}{\textrm{Ker}}
\newcommand{\im}{\textrm{Im}}
\newcommand{\dil}{\textrm{dil}}
\newcommand{\Ch}{\textrm{Ch}}
\newcommand{\Lip}{\textrm{Lip}}
\newcommand{\Ent}{\textrm{Ent}}
\newcommand{\grad}{\textrm{grad}}
\newcommand{\dom}{\textrm{dom}}
\newcommand{\diag}{\textrm{diag}}

\renewcommand{\;}{\, ; \,}
\renewcommand{\d}{\, {d}}

\newcommand{\gyouretsu}[1]{\begin{pmatrix} #1 \end{pmatrix} }

\renewcommand{\div}{\textrm{div}}


%%図式

\usepackage[dvipdfm,all]{xy}


\newenvironment{claim}[1]{\par\noindent\underline{step:}\space#1}{}
\newenvironment{claimproof}[1]{\par\noindent{($\because$)}\space#1}{\hfill $\blacktriangle $}


\newcommand{\pprime}{{\prime \prime}}

%%マグニチュード


\newcommand{\Mag}{\textrm{Mag}}

\usepackage{mathrsfs}


\title{ベッセル関数}
\date{}


\author{}


\begin{document}


\maketitle

\section{フロベニウスの方法}

\begin{dfn}(確定特異点). $2$階線形微分方程式
\begin{align*} \partial_x ^2 u(x) + b(x) \partial_x u(x) + c(x)u(x) = 0 \end{align*}
において, $x_0$ は, $(x-x_0)b(x), (x - x_0)^2 c(x)$ が$x_0$ において実解析であるとき, 確定特異点という. 
\end{dfn}

\begin{remark}
\begin{align*} \partial_x ^2 u(x) + b(x) \partial_x u(x) + c(x)u(x) = 0 \end{align*}
は, たとえば$0$ を確定特異点にもつとき, \\
$xb(x), x^2c(x)$ を冪級数で表示し, 
\begin{align*} u = x^\alpha \sum_{n =0} ^\infty u_n x^n \end{align*}
の形の解を探すことにすれば, 解けることが知られている. 
\end{remark}



\section{ベッセル関数}

\begin{dfn}(ケプラーの方程式). 
\begin{align*} \frac{2 \pi}{ T} t = \varphi(t) - e \sin (\varphi (t))  \end{align*}
をケプラー方程式という. 
\end{dfn}

\begin{prop}$\varphi$ をケプラー方程式の滑らかな解とすると, $\varphi ^\prime $ は周期$T$ の偶関数である. 
\end{prop}
\begin{pf*}$t$ が$0$ から$T$ まで動くと, $\varphi(t)$ は$0$ から$2 \pi$ まで動くことをみとめると, 
\begin{align*} \frac{2\pi}{T} = \paren{ \varphi^\prime (t) - e \cos (\varphi(t) ) } \varphi^\prime(t)  \end{align*}
より, わかる. 
\qed
\end{pf*}

\begin{prop}
\begin{align*} \frac{2}{T} \int_{0}^T \varphi^\prime (t) \cos\paren{\frac{2 \pi j t }{ T} } dt = \frac{2}{T} \int_0 ^ {2\pi} \cos (j(\varphi - e \sin \varphi)) d\varphi  \end{align*}
\end{prop}
\begin{pf*}
ただの変数変換. 
\qed
\end{pf*}

\begin{dfn}(積分表示ベッセル関数). $j \geq 1$ に対して,  
\begin{align*} J_j (s) \coloneqq \frac{1}{2\pi} \int_0^{2\pi} \cos(j \varphi - s \sin \varphi) d \varphi \end{align*}
と定め, これを積分表示ベッセル関数という.
\end{dfn}

\begin{prop}
\begin{align*} J_j(x) = \frac{1}{2\pi} \int_{-\pi}^{\pi} e^{i(k \varphi - s \sin \varphi )} d\varphi  \end{align*}
\end{prop}
\begin{pf*}
計算. 
\qed
\end{pf*}

\begin{prop}
\begin{align*} J_j (s) = \sum_{m = 0} ^ \infty \frac{(-1)^m}{m! (m + k) !} (\frac{s}{2})^{2m + k}  \end{align*}
\end{prop}
\begin{pf*}

\qed
\end{pf*}

\begin{dfn}(ベッセル微分方程式). $\nu \geq 0$ をパラメータとする微分方程式
\begin{align*} x^2 \partial_x ^2 u  + x \partial_x u +  ( 1 - \frac{\nu^2}{x^2}  ) u  = 0   \end{align*}
をベッセル微分方程式という. 
\end{dfn}

\begin{prop}
ベッセル微分方程式において, $x = 0$ は確定特異点である.  
\end{prop}
\begin{pf*}
明らか. 
\qed
\end{pf*}


\begin{dfn}(第$1$種ベッセル関数). 
\begin{align*} J_\nu(x) \coloneqq \sum_{m = 0}^\infty \frac{(-1)^m}{m! \Gamma(1 + \nu + m)} \paren{\frac{x^{2m + \nu}}{2^{2m + \nu}} }  \end{align*}
\end{dfn}







\end{document}