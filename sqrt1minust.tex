\documentclass[10pt, fleqn, label-section=none]{bxjsarticle}

%\usepackage[driver=dvipdfm,hmargin=25truemm,vmargin=25truemm]{geometry}

\setpagelayout{driver=dvipdfm,hmargin=25truemm,vmargin=20truemm}


\usepackage{amsmath}
\usepackage{amssymb}
\usepackage{amsfonts}
\usepackage{amsthm}
\usepackage{mathtools}
\usepackage{mleftright}

\usepackage{ascmac}




\usepackage{otf}

\theoremstyle{definition}
\newtheorem{dfn}{定義}[section]
\newtheorem{ex}[dfn]{例}
\newtheorem{lem}[dfn]{補題}
\newtheorem{prop}[dfn]{命題}
\newtheorem{thm}[dfn]{定理}
\newtheorem{cor}[dfn]{系}
\newtheorem*{pf*}{証明}
\newtheorem{problem}[dfn]{問題}
\newtheorem*{problem*}{問題}
\newtheorem{remark}[dfn]{注意}
\newtheorem*{claim*}{\underline{claim}}



\newtheorem*{solution*}{解答}

%箇条書きの様式
\renewcommand{\labelenumi}{(\arabic{enumi})}


%

\newcommand{\forany}{\rm{for} \ {}^{\forall}}
\newcommand{\foranyeps}{
\rm{for} \ {}^{\forall}\varepsilon >0}
\newcommand{\foranyk}{
\rm{for} \ {}^{\forall}k}


\newcommand{\any}{{}^{\forall}}
\newcommand{\suchthat}{\, \rm{s.t.} \, \it{}}




\newcommand{\veps}{\varepsilon}
\newcommand{\paren}[1]{\mleft( #1\mright )}
\newcommand{\cbra}[1]{\mleft\{#1\mright\}}
\newcommand{\sbra}[1]{\mleft\lbrack#1\mright\rbrack}
\newcommand{\tbra}[1]{\mleft\langle#1\mright\rangle}
\newcommand{\abs}[1]{\left|#1\right|}
\newcommand{\norm}[1]{\left\|#1\right\|}
\newcommand{\lopen}[1]{\mleft(#1\mright\rbrack}
\newcommand{\ropen}[1]{\mleft\lbrack #1 \mright)}



%
\newcommand{\Rn}{\mathbb{R}^n}
\newcommand{\Cn}{\mathbb{C}^n}

\newcommand{\Rm}{\mathbb{R}^m}
\newcommand{\Cm}{\mathbb{C}^m}


\newcommand{\projs}[2]{\it{p}_{#1,\ldots,#2}}
\newcommand{\projproj}[2]{\it{p}_{#1,#2}}

\newcommand{\proj}[1]{p_{#1}}

%可測空間
\newcommand{\stdProbSp}{\paren{\Omega, \mathcal{F}, P}}

%微分作用素
\newcommand{\ddt}{\frac{d}{dt}}
\newcommand{\ddx}{\frac{d}{dx}}
\newcommand{\ddy}{\frac{d}{dy}}

\newcommand{\delt}{\frac{\partial}{\partial t}}
\newcommand{\delx}{\frac{\partial}{\partial x}}

%ハイフン
\newcommand{\hyphen}{\text{-}}

%displaystyle
\newcommand{\dstyle}{\displaystyle}

%⇔, ⇒, \UTF{21D0}%
\newcommand{\LR}{\Leftrightarrow}
\newcommand{\naraba}{\Rightarrow}
\newcommand{\gyaku}{\Leftarrow}

%理由
\newcommand{\naze}[1]{\paren{\because {\mathop{ #1 }}}}

%
\newcommand{\sankaku}{\hfill $\triangle$}

%
\newcommand{\push}{_{\#}}

%手抜き
\newcommand{\textif}{\textrm{if}\,\,\,}
\newcommand{\Ric}{\textrm{Ric}}
\newcommand{\tr}{\textrm{tr}}
\newcommand{\vol}{\textrm{vol}}
\newcommand{\diam}{\textrm{diam}}
\newcommand{\supp}{\textrm{supp}}
\newcommand{\Med}{\textrm{Med}}
\newcommand{\Leb}{\textrm{Leb}}
\newcommand{\Const}{\textrm{Const}}
\newcommand{\Avg}{\textrm{Avg}}



\renewcommand{\;}{\, ; \,}
\renewcommand{\d}{\, {d}}

\newcommand{\gyouretsu}[1]{\begin{pmatrix} #1 \end{pmatrix} }



\title{$\sqrt{1-t}$}
\date{}


\author{}


\begin{document}


\maketitle



\section{}

\subsection{}

\begin{ex}
$\sqrt{1-t}$ と$0$ の近傍で一致する冪級数を探す. 候補としては
\begin{align*} 1 + \frac{1}{1!} \frac{1}{2} (1-t)^{-\frac{1}{2}} (-1) |_{t = 0} t +  \frac{1}{2!} \frac{1}{2} \frac{-1}{2} (1-t)^{-\frac{3}{2}} (-1)^2  |_{t = 0}  t^2 + \frac{1}{3!} \frac{1}{2} \frac{-1}{2} \frac{-3}{2}(1-t)^{-\frac{5}{2}} (-1)^3  |_{t = 0}  t^3 + \cdots   \end{align*}
である. $n$次の係数を$a_n$ とすると, 適当に眺めると
\begin{align*} \abs{\frac{a_{n+1}}{a_n}} =  \abs{-1} \frac{1}{n+1} \frac{2(n+1) -3}{2} \rightarrow 1\end{align*}
であることから, $(-1, 1)$ でこの級数は各点収束する.  この級数を$f(t)$ で表すことにする. 適当に眺めると$f(t) = -\frac{1}{2 (1-t)} f^\prime(t)$ が成り立つので, $f(0) = 1$ を初期条件としてこの微分方程式を解くと
\begin{align*} f(t) = \sqrt{1-t}\end{align*}
が解であることがわかる. 微分方程式の解の一意性から$(-1,1)$ で$\sum a_n t^n = \sqrt{1-t}$ であることがわかる. $[0,1)$ における状況を観察することにする. 地道に$\cbra{\sum_{n=0}^{N} a_n \cdot 1}_N$ を評価すると, 上に有界な単調増大列であることがわかるので, $\sum_{n=0}^\infty a_n$ は存在するので, $f$ は$\cbra{1}$ を定義域に含み, $f(1) = \sum_{n=0}^\infty a_n$ である. 従って, $[0, 1)$ 上で冪級数$f(t)$は一様収束し, $f(t) = \sqrt{1-t}$ である. 
\end{ex}

\subsection{参考文献}

Matt Young, "The Stone-Weierstrass Theorem", https://mast.queensu.ca/~speicher/Section14.pdf, 2006.













\end{document}