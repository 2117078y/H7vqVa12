\documentclass[twocolumn, landscape, a4paper , 8pt, fleqn, titlepage ]{jsarticle}
\usepackage[driver=dvipdfm,hmargin=20truemm,vmargin=25truemm]{geometry}

\usepackage{amsmath}
\usepackage{amssymb}
\usepackage{amsfonts}
\usepackage{amsthm}
\usepackage{mathtools}
\usepackage{mleftright}

%box
\usepackage{ascmac}

%%
\usepackage{xcolor} 
\usepackage[dvipdfmx]{hyperref}
\usepackage{pxjahyper}
\hypersetup{
setpagesize=false,
 bookmarksnumbered=true,
 bookmarksopen=true,
 colorlinks=true,
 linkcolor=teal,
 citecolor=black,
}
%
%

%


%%図式

\usepackage[dvipdfm,all]{xy}


%%



\usepackage{otf}

\theoremstyle{definition}
\newtheorem{dfn}{定義}[section]
\newtheorem{ex}[dfn]{例}
\newtheorem{lem}[dfn]{補題}
\newtheorem{prop}[dfn]{命題}
\newtheorem{thm}[dfn]{定理}
\newtheorem{cor}[dfn]{系}
\newtheorem*{pf*}{証明}
\newtheorem{problem}[dfn]{問題}
\newtheorem*{problem*}{問題}
\newtheorem{remark}[dfn]{注意}

\newtheorem*{solution*}{解答}

%箇条書きの様式
\renewcommand{\labelenumi}{(\arabic{enumi})}


%

\newcommand{\forany}{\rm{for} \ {}^{\forall}}
\newcommand{\foranyeps}{
\rm{for} \ {}^{\forall}\varepsilon >0}
\newcommand{\foranyk}{
\rm{for} \ {}^{\forall}k}


\newcommand{\any}{{}^{\forall}}
\newcommand{\suchthat}{\, \textrm{s.t.} \, }




\newcommand{\veps}{\varepsilon}
\newcommand{\paren}[1]{\mleft( #1\mright )}
\newcommand{\cbra}[1]{\mleft\{#1\mright\}}
\newcommand{\sbra}[1]{\mleft\lbrack#1\mright\rbrack}
\newcommand{\tbra}[1]{\mleft\langle#1\mright\rangle}

\newcommand{\ntbra}[1]{\langle#1\rangle}

\newcommand{\abs}[1]{\left|#1\right|}
\newcommand{\norm}[1]{\left\|#1\right\|}
\newcommand{\lopen}[1]{\mleft(#1\mright\rbrack}
\newcommand{\ropen}[1]{\mleft\lbrack #1 \mright)}



%
\newcommand{\Rn}{\mathbb{R}^n}
\newcommand{\Cn}{\mathbb{C}^n}

\newcommand{\Rm}{\mathbb{R}^m}
\newcommand{\Cm}{\mathbb{C}^m}


\newcommand{\supp}{\textrm{supp}\,} 

\newcommand{\ifufu}{\,\textrm {iff} \, \it}


\newcommand{\proj}[1]{\it{p}_{#1}}
\newcommand{\projs}[2]{\it{p}_{#1,\ldots,#2}}
\newcommand{\projproj}[2]{\it{p}_{#1,#2}}

\newcommand{\push}{_{\#}}

%可測空間
\newcommand{\stdProbSp}{\paren{\Omega, \mathcal{F}, P}}

%微分作用素
\newcommand{\ddt}{\frac{d}{dt}}
\newcommand{\ddx}{\frac{d}{dx}}
\newcommand{\ddy}{\frac{d}{dy}}

\newcommand{\delt}{\frac{\partial}{\partial t}}
\newcommand{\delx}{\frac{\partial}{\partial x}}

%ハイフン
\newcommand{\hyphen}{\text{-}}

%displaystyle
\newcommand{\dstyle}{\displaystyle}

%⇔, ⇒, \UTF{21D0}%
\newcommand{\LR}{\Leftrightarrow}
\newcommand{\naraba}{\Rightarrow}
\newcommand{\gyaku}{\Leftarrow}

%理由
\newcommand{\naze}[1]{\paren{\because {\mathop{ #1 }}}}

%ベクトル解析
\newcommand{\grad}{\textrm{grad}}
\renewcommand{\div}{\textrm{div}}

%手抜き
\newcommand{\textif}{\textrm{if}\,\,\,}
\newcommand{\Sgn}{\textrm{Sgn}}
\newcommand{\Ric}{\textrm{Ric}}
\newcommand{\Sec}{\textrm{Sec}}
\newcommand{\Scal}{\textrm{Scal}}
\newcommand{\tr}{\textrm{tr}}
\newcommand{\vol}{\textrm{vol}}
\newcommand{\diam}{\textrm{diam}}
\newcommand{\Med}{\textrm{Med}}
\newcommand{\Leb}{\textrm{Leb}}
\newcommand{\Const}{\textrm{Const}}
\newcommand{\Avg}{\textrm{Avg}}
\renewcommand{\d}{\, d}
\newcommand{\length}{\textrm{length}}
\newcommand{\Func}{\textrm{Func}}
\newcommand{\Ker}{\textrm{Ker}}
\newcommand{\Cone}{\textrm{Cone}}
\newcommand{\hess}{\textrm{hess}}
\newcommand{\esssup}{\textrm{ess}\,\textrm{sup}}

\newcommand{\sub}{\textrm{sub}}
\newcommand{\Par}{\textrm{Par}}


\newcommand{\perpperp}{{\perp \perp}}

\newcommand{\sgyouretsu}[1]{\paren{\begin{smallmatrix} #1 \end{smallmatrix} }}

\renewcommand{\ni}{\hspace{2pt} \textrm{I} \hspace{-5pt} \textrm{I} \hspace{2pt}}





%↓本体↓

\title{セミノルム位相}

\author{}
\date{}

\begin{document}

\maketitle
\scriptsize 

\section{}

\begin{remark}
$X$ でベクトル空間を表す.
\end{remark}

\begin{dfn}(セミノルム).
ノルムの条件から分離性$\norm{x} = 0 \LR x = 0$ を除いたものをセミノルムという.
\end{dfn}

\begin{prop}
ノルムはセミノルムである.
\end{prop}

\begin{dfn}(分離的セミノルム族).
$X$ 上のセミノルムの族$\mathcal P = \cbra{p_\lambda}$ は
\begin{align*} \cbra{x \in X \mid \any p \in \mathcal P \,\, p(x) = 0} = \cbra{0} \end{align*}
を満たす時に, 分離的であるという.
\end{dfn}

\begin{dfn}(セミノルム位相生成族). $X$ 上の分離的セミノルム族$\mathcal P $ に対して 
\begin{align*} \cbra{p_x \mid p \in \mathcal P, x \in X} \end{align*}
をセミノルム位相生成族という. ただし, $p \in \mathcal P , x \in X$ に対して
\begin{align*} p_x (\cdot) \coloneqq p(\cdot - x) \end{align*}
と定めてある.
\end{dfn}

\begin{dfn}(セミノルム位相). 分離的セミノルム族$\mathcal P$ から定まるセミノルム位相生成族を$\tilde {\mathcal P}$ で表す. $\tilde{\mathcal P}$ の任意の元を連続にする最弱の位相(つまり始位相)を, 分離的セミノルム族$\mathcal P$ から定まるセミノルム位相という.
\end{dfn}

\begin{prop} $X$ をベクトル空間, $\mathcal P$ を分離的セミノルム族とする. $X$ に$\mathcal P$ により定まるセミノルム位相を備える. このとき, 
\begin{align*} x_n \rightarrow x \LR p_y (x_n) \rightarrow p_y (x) \,\, (\any p_y \in \tilde {\mathcal P} ) \end{align*}
\end{prop}
\begin{pf*}
$(\naraba)$ は, $p_y \in \tilde{\mathcal P}$ を連続にする位相をいれてあるので明らかに成り立つ. $(\gyaku)$ $x$ の開近傍$V_x$を任意にとる. 始位相において
\begin{align*} \cbra{p_y ^{-1}(U) \subset X \mid p \in \tilde{\mathcal P}, U \subset [0, \infty) \textrm{is open}  } \end{align*}
は準開基であるので, 有限個の$p_1, \cdots , p_k \in \tilde{\mathcal P}$ と開集合$U_1, \ldots , U_k \in [0, \infty)$ を用いて
\begin{align*} p_1^{-1}(U_1)\cap \ldots \cap p_k^{-1}(U_k) \subset V_x\end{align*}
となる$x$ の開近傍がとれる. $p_i x \in U_i$ なので, $U_i$ は$p_i x$ の開近傍なので, $p_i$ の連続性から, ある自然数$N_i$ で
\begin{align*} n \geq N_i \naraba p_i x_n \in U_i \end{align*}
を満たすものがとれる. そこで $N \coloneqq \max \cbra{N_1, \ldots , N_k}$ とすると, 
\begin{align*} n \geq N \naraba p_1(x_n) \in U_1, \ldots p_k(x_n) \in U_k \end{align*}
であるので, $x_n \in p_1^{-1}(U_1)\cap \ldots \cap p_k^{-1}(U_k)$ であるので, $x_n \in V_x$ となる. 
つまるところ, $x$ の任意の開近傍$V_x$ に対して, ある自然数$N$ が存在して $n \geq N$ ならば, $x_n \in V_x$ が成り立つので, 主張が示された. 
\qed
\end{pf*}

\begin{prop}$X$ に分離的セミノルム族$\mathcal P$ から定まるセミノルム位相を備える.
\begin{align*} x_n \rightarrow x \LR p(x_n) \rightarrow p(x) \,\, (\any p \in \mathcal P) \end{align*}
\end{prop}
\begin{pf*} 任意に$p_y \in \tilde{\mathcal P}$ をとる.
\begin{align*} \abs{p_y (x_n) - p_y (x)} = \abs{p(x_n -y ) - p(x-y)} = \abs{p (x_n - x)} = \abs{p(x_n) - p(x)}  \end{align*}
であるので, 
\begin{align*} p(x_n) \rightarrow p(x) \,\, (\any p \in \mathcal P) \LR p_y(x_n) \rightarrow p_y(x) \,\, (\any p_y \in \tilde {\mathcal P} )  \end{align*}
が成り立つことから主張が従う.
\qed
\end{pf*}













\end{document}