\documentclass[10pt, fleqn, label-section=none]{bxjsarticle}

%\usepackage[driver=dvipdfm,hmargin=25truemm,vmargin=25truemm]{geometry}

\setpagelayout{driver=dvipdfm,hmargin=25truemm,vmargin=20truemm}


\usepackage{amsmath}
\usepackage{amssymb}
\usepackage{amsfonts}
\usepackage{amsthm}
\usepackage{mathtools}
\usepackage{mleftright}

\usepackage{ascmac}




\usepackage{otf}

\theoremstyle{definition}
\newtheorem{dfn}{定義}[section]
\newtheorem{ex}[dfn]{例}
\newtheorem{lem}[dfn]{補題}
\newtheorem{prop}[dfn]{命題}
\newtheorem{thm}[dfn]{定理}
\newtheorem{setting}[dfn]{設定}
\newtheorem{notation}[dfn]{記号}
\newtheorem{cor}[dfn]{系}
\newtheorem*{pf*}{証明}
\newtheorem{problem}[dfn]{問題}
\newtheorem*{problem*}{問題}
\newtheorem{remark}[dfn]{注意}
\newtheorem*{claim*}{\underline{claim}}



\newtheorem*{solution*}{解答}

%箇条書きの様式
\renewcommand{\labelenumi}{(\arabic{enumi})}


%

\newcommand{\forany}{\rm{for} \ {}^{\forall}}
\newcommand{\foranyeps}{
\rm{for} \ {}^{\forall}\varepsilon >0}
\newcommand{\foranyk}{
\rm{for} \ {}^{\forall}k}


\newcommand{\any}{{}^{\forall}}
\newcommand{\suchthat}{\, \rm{s.t.} \, \it{}}




\newcommand{\veps}{\varepsilon}
\newcommand{\paren}[1]{\mleft( #1\mright )}
\newcommand{\cbra}[1]{\mleft\{#1\mright\}}
\newcommand{\sbra}[1]{\mleft\lbrack#1\mright\rbrack}
\newcommand{\tbra}[1]{\mleft\langle#1\mright\rangle}
\newcommand{\abs}[1]{\left|#1\right|}
\newcommand{\norm}[1]{\left\|#1\right\|}
\newcommand{\lopen}[1]{\mleft(#1\mright\rbrack}
\newcommand{\ropen}[1]{\mleft\lbrack #1 \mright)}



%
\newcommand{\Rn}{\mathbb{R}^n}
\newcommand{\Cn}{\mathbb{C}^n}

\newcommand{\Rm}{\mathbb{R}^m}
\newcommand{\Cm}{\mathbb{C}^m}


\newcommand{\projs}[2]{\it{p}_{#1,\ldots,#2}}
\newcommand{\projproj}[2]{\it{p}_{#1,#2}}

\newcommand{\proj}[1]{p_{#1}}

%可測空間
\newcommand{\stdProbSp}{\paren{\Omega, \mathcal{F}, P}}

%微分作用素
\newcommand{\ddt}{\frac{d}{dt}}
\newcommand{\ddx}{\frac{d}{dx}}
\newcommand{\ddy}{\frac{d}{dy}}

\newcommand{\delt}{\frac{\partial}{\partial t}}
\newcommand{\delx}{\frac{\partial}{\partial x}}

%ハイフン
\newcommand{\hyphen}{\text{-}}

%displaystyle
\newcommand{\dstyle}{\displaystyle}

%⇔, ⇒, \UTF{21D0}%
\newcommand{\LR}{\Leftrightarrow}
\newcommand{\naraba}{\Rightarrow}
\newcommand{\gyaku}{\Leftarrow}

%理由
\newcommand{\naze}[1]{\paren{\because {\mathop{ #1 }}}}

%
\newcommand{\sankaku}{\hfill $\triangle$}

%
\newcommand{\push}{_{\#}}

%手抜き
\newcommand{\textif}{\textrm{if}\,\,\,}
\newcommand{\Ric}{\textrm{Ric}}
\newcommand{\tr}{\textrm{tr}}
\newcommand{\vol}{\textrm{vol}}
\newcommand{\diam}{\textrm{diam}}
\newcommand{\supp}{\textrm{supp}}
\newcommand{\Med}{\textrm{Med}}
\newcommand{\Leb}{\textrm{Leb}}
\newcommand{\Const}{\textrm{Const}}
\newcommand{\Avg}{\textrm{Avg}}
\newcommand{\id}{\textrm{id}}
\newcommand{\Ker}{\textrm{Ker}}
\newcommand{\im}{\textrm{Im}}
\newcommand{\dil}{\textrm{dil}}
\newcommand{\Ch}{\textrm{Ch}}
\newcommand{\Lip}{\textrm{Lip}}
\newcommand{\Ent}{\textrm{Ent}}
\newcommand{\grad}{\textrm{grad}}
\newcommand{\dom}{\textrm{dom}}
\newcommand{\diag}{\textrm{diag}}

\renewcommand{\;}{\, ; \,}
\renewcommand{\d}{\, {d}}

\newcommand{\gyouretsu}[1]{\begin{pmatrix} #1 \end{pmatrix} }

\renewcommand{\div}{\textrm{div}}


%%図式

\usepackage[dvipdfm,all]{xy}


\newenvironment{claim}[1]{\par\noindent\underline{step:}\space#1}{}
\newenvironment{claimproof}[1]{\par\noindent{($\because$)}\space#1}{\hfill $\blacktriangle $}


\newcommand{\pprime}{{\prime \prime}}

%%マグニチュード


\newcommand{\Mag}{\textrm{Mag}}

\usepackage{mathrsfs}


%%6.13
\def\chint#1{\mathchoice
{\XXint\displaystyle\textstyle{#1}}%
{\XXint\textstyle\scriptstyle{#1}}%
{\XXint\scriptstyle\scriptscriptstyle{#1}}%
{\XXint\scriptscriptstyle\scriptscriptstyle{#1}}%
\!\int}
\def\XXint#1#2#3{{\setbox0=\hbox{$#1{#2#3}{\int}$ }
\vcenter{\hbox{$#2#3$ }}\kern-.6\wd0}}
\def\ddashint{\chint=}
\def\dashint{\chint-}

%%7.13

\usepackage{here}

%7.15
\newcommand{\Span}{\textrm{Span}}


\title{$l_1$ 内在的体積}
\date{}


\author{}


\begin{document}


\maketitle

\section{}

\begin{notation}$\mathbb R^n$ に$l_1$ 距離を備えた距離空間を$l^n_1 = (\mathbb R^n, d_{l_1})$ で表す. 

\end{notation}

\begin{dfn}(座標部分空間). $e_1, \ldots, e_n \in \mathbb R^n$ を標準基底とする. 部分空間$P \subset R^n$ は, $i \in \cbra{1, 2, \ldots, n}$ 個の異なる標準基底$e_{k_1} , \ldots , e_{k_i}$ で, $P = \Span (e_{k_1} , \ldots , e_{k_i}) $ を満たすものがとれるとき, ($i$ 次元の)座標部分空間という. $G^\prime _{n, i}$ で$R^n$ における$i$ 次元座標部分空間全体を表す. ただし, $0$ 次元座標部分空間は, $0 \in \mathbb R^n$ として定める. 
\end{dfn}

\begin{prop}$R^n$ において, 異なる$i$ 次元座標部分空間は, $_nC_i$ 個存在する. すなわち, $\# G^\prime_{n,i} = _n C _i$ である.  

\end{prop}
\begin{pf*}
省略. 
\qed
\end{pf*}

\begin{notation}$P \in G^\prime _{n,i }$ に対して, 
\begin{align*} \pi_P : \mathbb R^n \rightarrow \mathbb R^n \end{align*}
で, $P$ への正射影を現す. 
\end{notation}

\begin{dfn}($l_1$ 内在的体積). $A \subset R^n$ に対して, 
\begin{align*} V^\prime_{n, i} \coloneqq \sum_{P \in G^\prime_{n, i}} \vol_i (\pi_P A) \end{align*}
により, $V^\prime_{n, i }     :   2^X \rightarrow \mathbb R$ を定める. 
\end{dfn}

\begin{ex}$A \coloneqq [0,3] \times [0, 5] \subset \mathbb R^2$ とする. 
\begin{align*} V^\prime_{2, 0} (A) = 1, \quad V^\prime_{2, 1} = 3 + 5 = 8, \quad V^\prime_{2, 2} = 15  \end{align*}
\end{ex}

\begin{ex}$A \coloneqq B(0; r) \subset l^3_1$ とする. 
\begin{align*} &V^\prime_{3, 0}(A) = 1, \quad V^\prime_{3, 1}(A) = r \omega_1 + r \omega_1 + r\omega_1 = 3 r \omega_1 = 3r\cdot 2, \\&V^\prime_{3, 2} (A) = r^2 \omega_2 + r^2 \omega_2 + r^2 \omega_2 = 3 r^2 \omega_2 = 3r^2 \cdot 2, \quad V^\prime_{3, 3} = r^3  \omega_3 = r^3 \cdot \frac{2^3}{2!} \end{align*}
\end{ex}

\begin{remark}$l^n_1$ において, $B(0; r) \subset l^n_1$ の体積は
\begin{align*} \vol_n (B_r) =  \frac{2^n}{n!}\end{align*}
です. 
\end{remark}

\begin{prop}($i$ 次斉次性). $A ( \subset \mathbb R^n )\in \dom(V^\prime_{n, i })  $ に対して, $sA \in  \dom(V^\prime_{n, i }) $ であるならば, 
\begin{align*} V^\prime_{n, i }   (sA) = s^i V^\prime_{n, i }(A)\end{align*}
が成り立つ. 
\end{prop}
\begin{pf*}
省略. 
\qed
\end{pf*}

\begin{prop}(内在性). $\iota^n_{n+1}: \mathbb R^n \rightarrow \mathbb R^{n+1}; (x_1, \ldots, x_n) \mapsto (x_1, \ldots, x_n, 0) $ と定めると, 任意の$A \subset (\mathbb R^n) \in \dom(V^\prime_{n, i }) $ に対して, $\iota^n_{n+1} A \in \dom(V^\prime_{n + 1, i }) $ であるならば, 
\begin{align*} V^\prime_{n, i } (A) =  V^\prime_{n+ 1, i }(\iota^n_{n+1} A)  \end{align*}

\end{prop}
\begin{pf*}
省略. 
\qed
\end{pf*}









\end{document}