\documentclass[twocolumn, landscape, a4paper , 8pt, fleqn, titlepage ]{jsarticle}
\usepackage[driver=dvipdfm,hmargin=20truemm,vmargin=25truemm]{geometry}

\usepackage{amsmath}
\usepackage{amssymb}
\usepackage{amsfonts}
\usepackage{amsthm}
\usepackage{mathtools}
\usepackage{mleftright}

%box
\usepackage{ascmac}

%%
\usepackage{xcolor} 
\usepackage[dvipdfmx]{hyperref}
\usepackage{pxjahyper}
\hypersetup{
setpagesize=false,
 bookmarksnumbered=true,
 bookmarksopen=true,
 colorlinks=true,
 linkcolor=teal,
 citecolor=black,
}
%
%

%


%%図式

\usepackage[dvipdfm,all]{xy}


%%



\usepackage{otf}

\theoremstyle{definition}
\newtheorem{dfn}{定義}[section]
\newtheorem{ex}[dfn]{例}
\newtheorem{lem}[dfn]{補題}
\newtheorem{prop}[dfn]{命題}
\newtheorem{thm}[dfn]{定理}
\newtheorem{cor}[dfn]{系}
\newtheorem*{pf*}{証明}
\newtheorem{problem}[dfn]{問題}
\newtheorem*{problem*}{問題}
\newtheorem{remark}[dfn]{注意}

\newtheorem*{solution*}{解答}

%箇条書きの様式
\renewcommand{\labelenumi}{(\arabic{enumi})}


%

\newcommand{\forany}{\rm{for} \ {}^{\forall}}
\newcommand{\foranyeps}{
\rm{for} \ {}^{\forall}\varepsilon >0}
\newcommand{\foranyk}{
\rm{for} \ {}^{\forall}k}


\newcommand{\any}{{}^{\forall}}
\newcommand{\suchthat}{\, \textrm{s.t.} \, }




\newcommand{\veps}{\varepsilon}
\newcommand{\paren}[1]{\mleft( #1\mright )}
\newcommand{\cbra}[1]{\mleft\{#1\mright\}}
\newcommand{\sbra}[1]{\mleft\lbrack#1\mright\rbrack}
\newcommand{\tbra}[1]{\mleft\langle#1\mright\rangle}

\newcommand{\ntbra}[1]{\langle#1\rangle}

\newcommand{\abs}[1]{\left|#1\right|}
\newcommand{\norm}[1]{\left\|#1\right\|}
\newcommand{\lopen}[1]{\mleft(#1\mright\rbrack}
\newcommand{\ropen}[1]{\mleft\lbrack #1 \mright)}



%
\newcommand{\Rn}{\mathbb{R}^n}
\newcommand{\Cn}{\mathbb{C}^n}

\newcommand{\Rm}{\mathbb{R}^m}
\newcommand{\Cm}{\mathbb{C}^m}


\newcommand{\supp}{\textrm{supp}\,} 

\newcommand{\ifufu}{\,\textrm {iff} \, \it}


\newcommand{\proj}[1]{\it{p}_{#1}}
\newcommand{\projs}[2]{\it{p}_{#1,\ldots,#2}}
\newcommand{\projproj}[2]{\it{p}_{#1,#2}}

\newcommand{\push}{_{\#}}

%可測空間
\newcommand{\stdProbSp}{\paren{\Omega, \mathcal{F}, P}}

%微分作用素
\newcommand{\ddt}{\frac{d}{dt}}
\newcommand{\ddx}{\frac{d}{dx}}
\newcommand{\ddy}{\frac{d}{dy}}

\newcommand{\delt}{\frac{\partial}{\partial t}}
\newcommand{\delx}{\frac{\partial}{\partial x}}

%ハイフン
\newcommand{\hyphen}{\text{-}}

%displaystyle
\newcommand{\dstyle}{\displaystyle}

%⇔, ⇒, \UTF{21D0}%
\newcommand{\LR}{\Leftrightarrow}
\newcommand{\naraba}{\Rightarrow}
\newcommand{\gyaku}{\Leftarrow}

%理由
\newcommand{\naze}[1]{\paren{\because {\mathop{ #1 }}}}

%ベクトル解析
\newcommand{\grad}{\textrm{grad}}
\renewcommand{\div}{\textrm{div}}

%手抜き
\newcommand{\textif}{\textrm{if}\,\,\,}
\newcommand{\Sgn}{\textrm{Sgn}}
\newcommand{\Ric}{\textrm{Ric}}
\newcommand{\Sec}{\textrm{Sec}}
\newcommand{\Scal}{\textrm{Scal}}
\newcommand{\tr}{\textrm{tr}}
\newcommand{\vol}{\textrm{vol}}
\newcommand{\diam}{\textrm{diam}}
\newcommand{\Med}{\textrm{Med}}
\newcommand{\Leb}{\textrm{Leb}}
\newcommand{\Const}{\textrm{Const}}
\newcommand{\Avg}{\textrm{Avg}}
\renewcommand{\d}{\, d}
\newcommand{\length}{\textrm{length}}
\newcommand{\Func}{\textrm{Func}}
\newcommand{\Ker}{\textrm{Ker}}
\newcommand{\Cone}{\textrm{Cone}}
\newcommand{\hess}{\textrm{hess}}
\newcommand{\esssup}{\textrm{ess}\,\textrm{sup}}

\newcommand{\sub}{\textrm{sub}}
\newcommand{\Par}{\textrm{Par}}


\newcommand{\perpperp}{{\perp \perp}}

\newcommand{\sgyouretsu}[1]{\paren{\begin{smallmatrix} #1 \end{smallmatrix} }}

\renewcommand{\ni}{\hspace{2pt} \textrm{I} \hspace{-5pt} \textrm{I} \hspace{2pt}}





%↓本体↓

\title{}

\author{}
\date{}

\begin{document}

\maketitle
\scriptsize 


%%目次%%
%\tableofcontents
%%%%%%

\section{}

以下の, ラウチの比較定理のversionを認めることにする.

\begin{prop}(ラウチ比較定理のver).
$M, \tilde M $ を完備リーマン多様体, $\gamma, \tilde \gamma$ をそれぞれ$\gamma_0 = p \in M, \tilde \gamma_0 = \tilde p \in \tilde M $ なる$M, \tilde M$ の正規測地線とする. $J, \tilde J$ をそれぞれ, $J, \tilde J$ を$J_0, \tilde J_0$ でそれぞれ$\gamma, \tilde \gamma$ に接し, $\norm{J_0} = \norm{\tilde J_0} , \norm{\nabla_t J_0} = \norm{\nabla_t \tilde J_0}, \tbra{\dot \gamma_0, \nabla_t J_0 } = \ntbra{\dot {\tilde{\gamma}}_0 , \nabla_t \tilde J_0 } $ をみたす$\gamma, \tilde \gamma$ に沿ったヤコビ場とする. $t_0(p), \tilde t_0 (\tilde p)$ をそれぞれ$p, \tilde p$ における, $\gamma, \tilde \gamma$ に沿った第一共役値とする. このとき, ($t_0(p) \leq  \tilde t_0 (\tilde p)$が成り立ち, )
\begin{align*} \norm{J_t} \leq \norm{\tilde J_t} \quad (0 \leq t < t_0 (p) )\end{align*}
が成り立つ. 
\end{prop}

\begin{remark}
一般的な用語ではない全くの造語であるが, ここで, $p \in M$ に対して$\exp_p |_{B(o_p; r)}$ がはめ込みとなる$r$ の上限をはめ込み半径, 埋め込み(同相なはめ込み)となる$r$ の上限を埋め込み半径と呼ぶことにする. 
\end{remark}


\begin{prop}
$M ,\tilde M$ を完備リーマン多様体, $p, \tilde p$ をそれぞれ$M, \tilde M$ の一点とする. $p$ のはめ込み半径を$r_1$, $\tilde p$ の埋め込み半径を$r_2$ とし, $r \leq \max{r_1, r_2}$ とする. 
\end{prop}
\begin{pf*}

\qed
\end{pf*}




\begin{dfn}
$p, p_1, p_2 \in \mathbb R^2$ を$p = \frac{1}{2}p_1 + \frac{1}{2} p_2$ をみたす$3$ 点とする. 
\begin{align*} l_p \coloneqq \cbra{t p_1 + (1-t)p_2 \in \mathbb R^2 \mid t \in (0,1)} \end{align*}
を$p$ を中心とする開線分という. また, 単位ベクトル$\frac{p_2 - p_1}{\norm{p_2 - p_1}} \in S^1$ をこの開線分の方向という. 
\end{dfn}

\begin{problem*}$f: \mathbb R^2 \rightarrow \mathbb R^2 $ を$C^\infty$ 級の写像とする. 
$p \in \mathbb R^2$ を任意の点とする. このとき, $p$ を中心とする開線分$l_{p}$ で \\
(条件)任意の$q \in l_p$に対して$\frac{f(q)}{\norm{f(q)}} \in \mathbb R^2$ が$l_p$ の方向と一致しない. \\
を満たすものは存在するか.
\end{problem*}
\begin{solution*}

\qed
\end{solution*}
























\end{document}