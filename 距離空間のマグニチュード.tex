\documentclass[10pt, fleqn, label-section=none]{bxjsarticle}

%\usepackage[driver=dvipdfm,hmargin=25truemm,vmargin=25truemm]{geometry}

\setpagelayout{driver=dvipdfm,hmargin=25truemm,vmargin=20truemm}


\usepackage{amsmath}
\usepackage{amssymb}
\usepackage{amsfonts}
\usepackage{amsthm}
\usepackage{mathtools}
\usepackage{mleftright}

\usepackage{ascmac}




\usepackage{otf}

\theoremstyle{definition}
\newtheorem{dfn}{定義}[section]
\newtheorem{ex}[dfn]{例}
\newtheorem{lem}[dfn]{補題}
\newtheorem{prop}[dfn]{命題}
\newtheorem{thm}[dfn]{定理}
\newtheorem{setting}[dfn]{設定}
\newtheorem{notation}[dfn]{記号}
\newtheorem{cor}[dfn]{系}
\newtheorem*{pf*}{証明}
\newtheorem{problem}[dfn]{問題}
\newtheorem*{problem*}{問題}
\newtheorem{remark}[dfn]{注意}
\newtheorem*{claim*}{\underline{claim}}



\newtheorem*{solution*}{解答}

%箇条書きの様式
\renewcommand{\labelenumi}{(\arabic{enumi})}


%

\newcommand{\forany}{\rm{for} \ {}^{\forall}}
\newcommand{\foranyeps}{
\rm{for} \ {}^{\forall}\varepsilon >0}
\newcommand{\foranyk}{
\rm{for} \ {}^{\forall}k}


\newcommand{\any}{{}^{\forall}}
\newcommand{\suchthat}{\, \rm{s.t.} \, \it{}}




\newcommand{\veps}{\varepsilon}
\newcommand{\paren}[1]{\mleft( #1\mright )}
\newcommand{\cbra}[1]{\mleft\{#1\mright\}}
\newcommand{\sbra}[1]{\mleft\lbrack#1\mright\rbrack}
\newcommand{\tbra}[1]{\mleft\langle#1\mright\rangle}
\newcommand{\abs}[1]{\left|#1\right|}
\newcommand{\norm}[1]{\left\|#1\right\|}
\newcommand{\lopen}[1]{\mleft(#1\mright\rbrack}
\newcommand{\ropen}[1]{\mleft\lbrack #1 \mright)}



%
\newcommand{\Rn}{\mathbb{R}^n}
\newcommand{\Cn}{\mathbb{C}^n}

\newcommand{\Rm}{\mathbb{R}^m}
\newcommand{\Cm}{\mathbb{C}^m}


\newcommand{\projs}[2]{\it{p}_{#1,\ldots,#2}}
\newcommand{\projproj}[2]{\it{p}_{#1,#2}}

\newcommand{\proj}[1]{p_{#1}}

%可測空間
\newcommand{\stdProbSp}{\paren{\Omega, \mathcal{F}, P}}

%微分作用素
\newcommand{\ddt}{\frac{d}{dt}}
\newcommand{\ddx}{\frac{d}{dx}}
\newcommand{\ddy}{\frac{d}{dy}}

\newcommand{\delt}{\frac{\partial}{\partial t}}
\newcommand{\delx}{\frac{\partial}{\partial x}}

%ハイフン
\newcommand{\hyphen}{\text{-}}

%displaystyle
\newcommand{\dstyle}{\displaystyle}

%⇔, ⇒, \UTF{21D0}%
\newcommand{\LR}{\Leftrightarrow}
\newcommand{\naraba}{\Rightarrow}
\newcommand{\gyaku}{\Leftarrow}

%理由
\newcommand{\naze}[1]{\paren{\because {\mathop{ #1 }}}}

%
\newcommand{\sankaku}{\hfill $\triangle$}

%
\newcommand{\push}{_{\#}}

%手抜き
\newcommand{\textif}{\textrm{if}\,\,\,}
\newcommand{\Ric}{\textrm{Ric}}
\newcommand{\tr}{\textrm{tr}}
\newcommand{\vol}{\textrm{vol}}
\newcommand{\diam}{\textrm{diam}}
\newcommand{\supp}{\textrm{supp}}
\newcommand{\Med}{\textrm{Med}}
\newcommand{\Leb}{\textrm{Leb}}
\newcommand{\Const}{\textrm{Const}}
\newcommand{\Avg}{\textrm{Avg}}
\newcommand{\id}{\textrm{id}}
\newcommand{\Ker}{\textrm{Ker}}
\newcommand{\im}{\textrm{Im}}
\newcommand{\dil}{\textrm{dil}}
\newcommand{\Ch}{\textrm{Ch}}
\newcommand{\Lip}{\textrm{Lip}}
\newcommand{\Ent}{\textrm{Ent}}
\newcommand{\grad}{\textrm{grad}}
\newcommand{\dom}{\textrm{dom}}
\newcommand{\diag}{\textrm{diag}}

\renewcommand{\;}{\, ; \,}
\renewcommand{\d}{\, {d}}

\newcommand{\gyouretsu}[1]{\begin{pmatrix} #1 \end{pmatrix} }


%%図式

\usepackage[dvipdfm,all]{xy}


\newenvironment{claim}[1]{\par\noindent\underline{step:}\space#1}{}
\newenvironment{claimproof}[1]{\par\noindent{($\because$)}\space#1}{\hfill $\blacktriangle $}


\newcommand{\pprime}{{\prime \prime}}

%%マグニチュード


\newcommand{\Mag}{\textrm{Mag}}


\title{距離空間のマグニチュード}
\date{}


\author{}


\begin{document}


\maketitle

\section{マグニチュード}

\subsection{有限距離空間のマグニチュード}

\begin{dfn}(類似度行列). $(X, d)$ を有限距離空間とする.
\begin{align*} Z: X \times X \rightarrow \mathbb R; (x, y) \mapsto e^{-d(x,y)}\end{align*}
を$(X, d)$ の類似度行列という. 
\end{dfn}


\begin{dfn}(ウェイト, コウェイト). $(X, d)$ を有限距離空間, $Z$ を$X$ の類似度行列とする. $w: X \rightarrow \mathbb R$ で, 任意の$p \in X$ に対して
\begin{align*} \sum_{x \in X }Z(p, x) w(x) = 1 \end{align*}
を満たすものを, $X$ のウェイトという. また, 任意の$p \in X$ に対して
\begin{align*} \sum_{x \in X}w(x) Z(x, p) = 1 \end{align*}
を満たすものを, $X$ のコウェイトという. 
\end{dfn}

\begin{prop}$(X, d)$ を有限距離空間, $w, v$ を$X$ のウェイトとする. このとき, 
\begin{align*} \sum_{x \in X} w(x) = \sum_{x \in X} v(x)\end{align*}
が成り立つ. 
\end{prop}
\begin{pf*}
\begin{align*} \sum_{x \in X} w(x) = \sum_{x \in X}(\sum_{y \in X}v(y)Z(y, x)) w(x) = \sum_{y \in X} v(y) \end{align*}
\qed
\end{pf*}


\begin{dfn}(有限距離空間のマグニチュード). $(X, d)$ を有限距離空間とする. $X$ のウェイト$w$ が存在する時, $X$ はマグニチュードをもつといい, 
\begin{align*} \Mag(X) \coloneqq \sum_{x \in X } w(x)   \end{align*}
と定め, この値を$X$ のマグニチュードという. 
\end{dfn}

\begin{remark}$X$ のマグニチュードはウェイトのとりかたに依らない. 

\end{remark}



\begin{dfn}(メビウス行列). $(X, d)$ を有限距離空間, $Z$ を$X$ の類似度行列とする. 
\begin{align*} M: X \times X \rightarrow \mathbb R\end{align*} 
で, 任意の$x,y \in X$ に対して
\begin{align*} \sum_{z \in X} Z(x, z) M(z, y) = \delta (x,y) \end{align*}
を満たす写像を$X$ のメビウス行列という. 
\end{dfn}

\begin{prop}\label{1703}$(X, d)$ を有限距離空間とする. $X$ のメビウス行列$M$が存在するならば, $X$ のウェイト$w$は存在し, 
\begin{align*} w(x) = \sum_{y \in X} M(x, y) \quad (\any x \in X)\end{align*}
が成り立つ. 
\end{prop}
\begin{pf*}
実際, 
\begin{align*} \sum_{y \in X} (Z(x,y) \sum_{z \in X} M(y, z) )= \sum_{z \in X} \sum_{y \in X} Z(x, y)M(y, z) = \sum_{z \in X} \delta(x, z) = 1   \end{align*}
が成り立つ. 
\qed
\end{pf*}

\begin{prop}$(X, d)$ を有限距離空間とする. $X$ のウェイトが存在することと, $X$ のメビウス行列が存在することは必要十分である. 
\end{prop}
\begin{pf*}メビウス行列が存在するならば, ウェイトが存在することは既に命題\ref{1703}で示してある. $X$ のウェイトが存在するとする. $X$ の類似度行列を$Z$ とする. 対称行列は対角化可能であるので, 
$\lambda : X \rightarrow \mathbb R$ と$ P: X \times X \rightarrow \mathbb R$
で, 任意の$x \in X$ に対して
\begin{align*} \sum_{y \in X} \lambda (y) w(y) P(x, y) = \sum_{y \in X} P(x, y) \end{align*}
が成り立つ. 固有ベクトルは線型独立なので任意の$x \in X$ に対して, 
\begin{align*} \lambda(x) w(x) = 1\end{align*}
が成り立つ. 故に, $\lambda (x) \neq 0$ である. 固有値に$0$ を含まない対称行列は正則行列であるので, $X$ はメビウス行列をもつ. 
\qed
\end{pf*}



\begin{prop}$(X, d)$ を有限距離空間とする. $X$ のメビウス行列が存在するならば, 
\begin{align*} \Mag(X) = \sum_{(x, y) \in X \times X} M(x,y) \end{align*}
が成り立つ.
\end{prop}
\begin{pf*}
\begin{align*} \sum_{(x, y) \in X \times X} M(x,y) &= \sum_{x \in X} \sum_{y \in X} M(x,y) = \sum_{x \in X} \sum_{y \in X} (M(x,y) \sum_{z \in X} Z(y, z) )w(z) \\ &= \sum_{x\in X} \sum_{y\in X}  \sum_{z\in X} M(x, y)Z(y, z) w(z) =  \sum_{z \in X} w(z)   \end{align*}
\qed
\end{pf*}



\begin{prop}$(X, d)$ を有限距離空間とする. 
\begin{align*} \# \cbra{t  \in (0, \infty) \mid tX \textrm{はマグニチュードを持たない}  } < \infty \end{align*}
が成り立つ. 
\end{prop}
\begin{pf*}$N \coloneqq \#X$ とし, 距離空間$tX = (X, td)$ の類似度行列を$Z^{tX}$ で表すことにする. 
$M_N(\mathbb R)$において, 正則行列全体$GL_N (\mathbb R)$は開集合であるので, 単位行列$I$ の開近傍$U_I$ で$U_I \subset GL_N( \mathbb R)$ を満たすものがとれる. $\lim_{t \rightarrow \infty } Z^{tX} = I$ であるので, 
十分大きな$T$ をとると, $t \in (0, T) \naraba \det Z^{tX} = 0$ が成り立つようにできる. 
\begin{align*} f: \mathbb C \rightarrow \mathbb C; t \mapsto Z^{tX}\end{align*}
により正則関数を定める. $f$ が$(0, T)$ に無限個のゼロ点を持つとすると, 相対点列コンパクトであることから, $\mathbb C$ における収束部分列をもつ. 従って, $f$ が正則関数であることから, 一致の定理より$f$ は$\mathbb C$ 全体で$0$ となるが, これは矛盾である. 故に, ゼロ点は$(0, T)$ に高々有限個しか存在しない. 
\qed
\end{pf*}



\subsection{様々な空間のマグニチュード}



\begin{prop}($2$点空間のマグニチュード). 距離空間$X = (\cbra{p,q}, d)$ のマグニチュードは
\begin{align*} \Mag(X) =  \end{align*}
\end{prop}
\begin{pf*}

\qed
\end{pf*}



\begin{prop}($l_1$直積距離空間のマグニチュード). $(X, d_X), (Y, d_Y)$ を有限距離空間とする. 
\begin{align*} d_{X\times Y}^{l_1} ((x_0, y_0), (x_1, y_1)) \coloneqq d_X (x_0, x_1) + d_Y (y_0, y_1)\end{align*}
とする. 距離空間$X \times_1 Y = (X \times Y, d_{X\times Y}^{l_1} ) $ のマグニチュードは
\begin{align*} \Mag(X \times _1 Y ) = \Mag(X) \Mag(Y) \end{align*}
で与えられる.
\end{prop}
\begin{pf*}

\qed
\end{pf*}




\begin{dfn}(斉次距離空間). 
$(X, d)$ を距離空間とする. 群$G$で$X$に等長かつ推移的に作用するものが存在するとき, $(X, d)$ を斉次距離空間という. 

\end{dfn}




\begin{notation}$(X, d)$ を距離空間, $G$ を$X$ に作用する推移的な群とする. $x, y \in X$ に対して 
\begin{align*} g x = y \end{align*}
を満たす$g$ を$g^{x}_y$ で表すことにする.
\end{notation}




\begin{prop}$G$ を$(X, d)$ に推移的かつ等長な作用をする群とする. 任意の$g \in G$ について
\begin{align*} g: X \rightarrow X\end{align*}
は全単射等長写像である.
\end{prop}
\begin{pf*}任意の$x \in X$ に対して
\begin{align*} g g^{-1}x = x\end{align*}
なので$g$ は全射である. 等長的であるので, 単射である. 故に主張が従う.
\qed
\end{pf*}




\begin{prop}$(X, d)$ を有限な斉次距離空間とする. 任意の$g \in G$ と, 任意の$p, q \in X$ に対して 
\begin{align*}  \sum_{p \in X} d(x, p) = \sum_{p \in X} d(y, p) \end{align*}
が成り立つ.
\end{prop}
\begin{pf*} 
\begin{align*} \sum_{p \in X} d(x, p) = \sum_{p \in X} d(g^x_y x, g^x_y p) = \sum_{p \in X} d(y, g^x_y p) = \sum_{q \in X} d(y, q)  \end{align*}
\qed
\end{pf*}




\begin{prop}$(X, d)$ を有限な斉次距離空間とする. 任意の$p \in X$ に対して 
\begin{align*} \Mag(X) = \frac{\# X}{\sum_{x \in X} d(p, x)}  \end{align*}
が成り立つ.
\end{prop}
\begin{pf*}

\qed
\end{pf*}





\begin{dfn}(scattered space). 有限距離空間$(X, d)$ は任意の$x, y \in X$ に対して$\log(\# X - 1) < d(x, y)$ を満たす時に, scattered 空間という. 

\end{dfn}


\begin{prop}$(X, d)$ を有限な距離空間とする. $X$ がscattered 空間であるならば, 
$X$ のメビウス行列が存在する. すなわち, $X$ はマグニチュードをもつ. 
\end{prop}
\begin{pf*}$X$ の類似度行列を$Z$ とする. $Z$ が正定値であることを示す. 実数$a,b \in \mathbb R$ に対して
\begin{align*} -\abs a \abs b \leq a b \leq \abs a \abs b \end{align*}
であることに注意すると, 
\begin{align*} x^t Z x = \sum_i x_i^2 + \sum_{i \neq j} x_i z_{i, j} x_j &> \sum_i  x_i^2 - \frac{1}{\# X - 1} \sum \abs{x_i} \abs{x_j} \\ &= (1 - \frac{1}{\# X - 1} )\sum_i x_i^2 + \frac{1}{\# X - 1 } \sum_i  x_i^2 - \frac{1}{\# X - 1} \sum_
{i \neq j} \abs{x_i} \abs{x_j} \\&=    (1 - \frac{1}{\# X - 1} )\sum_i x_i^2 + \frac{1}{2(\# X - 1)} \sum_{i \neq j} (\abs x_i  - \abs x_j )^2 \geq 0 \end{align*}
なので, 半正定値である. また, $x^t Z x = 0$ のとき, $x \neq 0$ であるとすると, 
\begin{align*} 0 = x^t Z x > (1 - \frac{1}{\# X - 1} )\sum_i x_i^2 + \frac{1}{2(\# X - 1)} \sum_{i \neq j} (\abs x_i  - \abs x_j )^2  \end{align*}
となり矛盾するので, $x = 0$ である. 従って, $x^t Z x = 0 \LR x = 0$ が成り立つので, 正定値である. 従って$Z$ は正定値対称行列なので, (固有値が全て正であることから)正則行列であり, 逆行列をもつので, 主張が従う.
\qed
\end{pf*}

\begin{dfn}(正定値距離空間). $(X, d)$ を有限な距離空間とする. $X$ は, 類似度行列が正定値行列であるとき, 正定値距離空間という.

\end{dfn}

\begin{prop}正定値距離空間は, マグニチュードをもつ.

\end{prop}
\begin{pf*}
類似度行列が正定値対称行列なので, 逆行列をもつ. 
\qed
\end{pf*}

\begin{prop}scattered 空間は, 正定値距離空間である.

\end{prop}
\begin{pf*}

\qed
\end{pf*}



\subsection{無限距離空間のマグニチュード}

\section{マグニチュードホモロジー}


\section{凸}

\begin{dfn}(凸体). $A \subset \mathbb R^n$ は, コンパクトかつ凸であるとき, 凸体という.

\end{dfn}




\section{不変測度}


\begin{dfn}(不変測度). 位相群$G$ 上のボレル測度$\nu$で,  任意の$A \in \mathcal B (G)$ に対して
\begin{align*} \nu(gA) = \nu(A) \quad (\any g \in G)\end{align*}
を満たすものを, 左不変測度という. 
\begin{align*} \nu(A) = \nu(Ag) \quad (\any g \in G)\end{align*}
を満たすものを, 右不変測度という. 
\begin{align*} \nu(A) = \nu(A^{-1}) \end{align*}
を満たすものをinverse 不変測度という. また, 左不変かつ右不変かつinverse 不変な測度を不変測度という.
\end{dfn}

\begin{dfn}(ハール測度). 位相群$G$ 上のコンパクト集合に対して有限な測度を定めるボレル測度$\nu$で, 左不変かつ正則なものを左ハール測度という. また, 右不変かつ正則なものを右ハール測度という. 左不変かつ右不変かつ正則なものをハール測度という.

\end{dfn}

\begin{dfn}(積分左不変). $G$ を位相群, $\nu$ を$G$ 上の測度, $\mathcal F$ を可測関数の族とする. 任意の$f \in \mathcal F$ に対して
\begin{align*} \int_G f (hg) d \nu(g) = \int_G f (g) d\nu(g) \quad (\any h \in G)\end{align*}
が成り立つ. とき, $G$ は$\mathcal F$ に関して積分左不変であるという.
\end{dfn}



\begin{prop} $G$ を位相群, $\nu$ を$G$ 上の左不変測度とすると, $G$ は可測関数に関して積分左不変である.
\end{prop}
\begin{pf*}
\begin{align*} h^{-1} A = \cbra{x \in G \mid hx \in A} \end{align*}
であることから, 
\begin{align*} g \in h^{-1}A  \LR hg \in A \end{align*}
であるので, 
\begin{align*} \int_G 1_A (hg) d\nu(g) &= \nu(\cbra{g \in G \mid hg \in A}) \\ 
&= \nu(\cbra{g \in G \mid g \in h^{-1}A }) \\
&= \nu(h^{-1}A) = \nu (A) = \int_G 1_A (g) \nu(g) 
\end{align*}
より, 定義関数に対しては成り立つ. 従って, 任意の非負可測関数に対して成り立つ.
\qed
\end{pf*}

\begin{prop} $G$ を位相群, $\mathcal F$ を可測関数の族, $\nu$ を$G$ 上の$\mathcal F$ に関して積分左不変な測度とする. このとき, 任意の$f \in \mathcal F, h \in G$ に対して
\begin{align*} \int_G f(g^{-1} h)d\nu(g) = \int_{G} f(g^{-1}) d\nu(g)  \end{align*}
が成り立つ. 
\end{prop}
\begin{pf*}
\begin{align*} \int_G f(g^{-1} h)d\nu(g) =\int_G f((h^{-1} g)^{-1} ) d\nu(g) = \int_{G} f(g^{-1}) d\nu(g).  \end{align*}
\qed
\end{pf*}

\begin{prop}$G$ を位相群, $\nu$ を$G$ 上の測度とする. $C_c (G; \mathbb R_{\geq 0})$ に関して積分左不変であるならば, 左不変測度である.

\end{prop}
\begin{pf*}

\qed
\end{pf*}


\begin{prop}逆不変測度であることと, 非負$C_c(G)$ に関して積分逆不変であることは必要十分である.

\end{prop}
\begin{pf*}

\qed
\end{pf*}



\begin{prop}第二可算なコンパクト群$G$ 上の左ハール測度$\nu$ は不変測度である.
\end{prop}
\begin{pf*}$\nu(G) = 1$ となるように正規化しておく.  任意の非負連続関数$f$ に対して, 
\begin{align*} \int_G f(g^{-1}) d\nu(g) = \int_G \int_G f(g^{-1} h) d\nu(g) d\nu(h)
 = \int_G \int_G f(g^{-1} h) d\nu(h) d\nu(g) = \int_G f(h) d\nu(h)  \end{align*}
 となり, $C(G; \mathbb R_{\geq 0}) $ に関して積分逆不変性が成り立つ. $G$ はコンパクトなので, $C(G; \mathbb R_{\geq 0}) = C_c (G; \mathbb R_{\geq 0})$ であるので, 前述の命題から, $\nu$ は逆不変である. また, 任意の連続関数$f \in C(G)$ に対して
 \begin{align*}  \int_G f(gh) d\nu (g) = \int_G f(g^{-1}h) d\nu(g) = \int_G f(g^{-1}) d\nu(g) =  \int_G f(g) d\nu(g) \end{align*}
 より右不変でもある.
\qed
\end{pf*}


\begin{notation}(あとで消す). $(G, \nu)$ コンパクト第二可算ハウスドルフ連続推移的, ハール確率測度. 
$(E, \alpha, \rho)$ 第二可算ハウスドルフ, 局所有限ボレル測度, 

\end{notation}

\begin{remark}(あとで消す). $(G, \nu)$ コンパクトリー群ハール確率空間. $(E, \alpha, \rho)$ 多様体, 自明でない局所有限ボレル測度, 局所有限ボレル測度. 

\end{remark}


\begin{prop}$G$ をハウスドルフ空間$E$ に連続かつ推移的に作用するコンパクト群とする. $G, E$ は可算開基をもつ. $\nu$ を$G$ 上のハール確率測度とする. $\rho \neq 0, \alpha $ を$E$ 上の局所有限ボレル測度とする. $\rho$ が$G$不変であるならば, 任意の$A, B \in \mathcal B (E) $ に対して
\begin{align*} \int_G \alpha(A \cap gB) d\nu(g) = \alpha (A) \rho(B) / \rho (E)  \end{align*} 
が成り立つ. 
\end{prop}
\begin{pf*}




\qed
\end{pf*}

\begin{prop}$G$ を集合$X$ に作用する群とし, $B \subset X, h \in G$ とする. このとき, 
\begin{align*} \cbra{hg \in G \mid gp \in B } = \cbra{f \in G \mid fp \in hB }\end{align*}
\end{prop}
\begin{pf*}
\begin{align*} hg \in \textrm{左} \naraba h(gp) \in hB , \quad
fp \in \textrm{右} \naraba h^{-1} fp \in B
\end{align*}
\qed
\end{pf*}



\begin{prop}($G$不変ボレル測度の一意的存在). $G$ をコンパクト群, $\nu$ をハール確率測度とする. (満たされがちな, ある程度良い条件を仮定する. すなわち, $G$ の作用は連続かつ推移的, $G, E$ は第二可算ハウスドルフ空間である. ) このとき, $p \in E$ を適当な点として, 
\begin{align*} \rho(B) \coloneqq \nu (\cbra{g \in G \mid gp \in B})      \quad (B \in \mathcal B (E) ) \end{align*}

は$E$ 上の$G$ 不変ボレル確率測度である. 
\end{prop}
\begin{pf*}(一意性は工事中). 
\begin{align*} \rho(B) &= \nu (  \cbra{g \in G \mid gp \in B}   ) = \nu(h  \cbra{g \in G \mid gp \in B} )   \\
&= \nu(  \cbra{h g \in G \mid gp \in B} ) = \nu(  \cbra{f \in G \mid fp \in hB} ) 
\end{align*}

\qed
\end{pf*}

\begin{remark}推移性は点$p$ のとりかたに依らないところに効いてくる? 

\end{remark}







\end{document}