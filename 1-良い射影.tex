\documentclass[10pt, fleqn, label-section=none]{bxjsarticle}

%\usepackage[driver=dvipdfm,hmargin=25truemm,vmargin=25truemm]{geometry}

\setpagelayout{driver=dvipdfm,hmargin=25truemm,vmargin=20truemm}


\usepackage{amsmath}
\usepackage{amssymb}
\usepackage{amsfonts}
\usepackage{amsthm}
\usepackage{mathtools}
\usepackage{mleftright}

\usepackage{ascmac}




\usepackage{otf}

\theoremstyle{definition}
\newtheorem{dfn}{定義}[section]
\newtheorem{ex}[dfn]{例}
\newtheorem{lem}[dfn]{補題}
\newtheorem{prop}[dfn]{命題}
\newtheorem{thm}[dfn]{定理}
\newtheorem{setting}[dfn]{設定}
\newtheorem{notation}[dfn]{記号}
\newtheorem{cor}[dfn]{系}
\newtheorem*{pf*}{証明}
\newtheorem{problem}[dfn]{問題}
\newtheorem*{problem*}{問題}
\newtheorem{remark}[dfn]{注意}
\newtheorem*{claim*}{\underline{claim}}



\newtheorem*{solution*}{解答}

%箇条書きの様式
\renewcommand{\labelenumi}{(\arabic{enumi})}


%

\newcommand{\forany}{\rm{for} \ {}^{\forall}}
\newcommand{\foranyeps}{
\rm{for} \ {}^{\forall}\varepsilon >0}
\newcommand{\foranyk}{
\rm{for} \ {}^{\forall}k}


\newcommand{\any}{{}^{\forall}}
\newcommand{\suchthat}{\, \rm{s.t.} \, \it{}}




\newcommand{\veps}{\varepsilon}
\newcommand{\paren}[1]{\mleft( #1\mright )}
\newcommand{\cbra}[1]{\mleft\{#1\mright\}}
\newcommand{\sbra}[1]{\mleft\lbrack#1\mright\rbrack}
\newcommand{\tbra}[1]{\mleft\langle#1\mright\rangle}
\newcommand{\abs}[1]{\left|#1\right|}
\newcommand{\norm}[1]{\left\|#1\right\|}
\newcommand{\lopen}[1]{\mleft(#1\mright\rbrack}
\newcommand{\ropen}[1]{\mleft\lbrack #1 \mright)}



%
\newcommand{\Rn}{\mathbb{R}^n}
\newcommand{\Cn}{\mathbb{C}^n}

\newcommand{\Rm}{\mathbb{R}^m}
\newcommand{\Cm}{\mathbb{C}^m}


\newcommand{\projs}[2]{\it{p}_{#1,\ldots,#2}}
\newcommand{\projproj}[2]{\it{p}_{#1,#2}}

\newcommand{\proj}[1]{p_{#1}}

%可測空間
\newcommand{\stdProbSp}{\paren{\Omega, \mathcal{F}, P}}

%微分作用素
\newcommand{\ddt}{\frac{d}{dt}}
\newcommand{\ddx}{\frac{d}{dx}}
\newcommand{\ddy}{\frac{d}{dy}}

\newcommand{\delt}{\frac{\partial}{\partial t}}
\newcommand{\delx}{\frac{\partial}{\partial x}}

%ハイフン
\newcommand{\hyphen}{\text{-}}

%displaystyle
\newcommand{\dstyle}{\displaystyle}

%⇔, ⇒, \UTF{21D0}%
\newcommand{\LR}{\Leftrightarrow}
\newcommand{\naraba}{\Rightarrow}
\newcommand{\gyaku}{\Leftarrow}

%理由
\newcommand{\naze}[1]{\paren{\because {\mathop{ #1 }}}}

%
\newcommand{\sankaku}{\hfill $\triangle$}

%
\newcommand{\push}{_{\#}}

%手抜き
\newcommand{\textif}{\textrm{if}\,\,\,}
\newcommand{\Ric}{\textrm{Ric}}
\newcommand{\tr}{\textrm{tr}}
\newcommand{\vol}{\textrm{vol}}
\newcommand{\diam}{\textrm{diam}}
\newcommand{\supp}{\textrm{supp}}
\newcommand{\Med}{\textrm{Med}}
\newcommand{\Leb}{\textrm{Leb}}
\newcommand{\Const}{\textrm{Const}}
\newcommand{\Avg}{\textrm{Avg}}
\newcommand{\id}{\textrm{id}}
\newcommand{\Ker}{\textrm{Ker}}
\newcommand{\im}{\textrm{Im}}
\newcommand{\dil}{\textrm{dil}}
\newcommand{\Ch}{\textrm{Ch}}
\newcommand{\Lip}{\textrm{Lip}}
\newcommand{\Ent}{\textrm{Ent}}
\newcommand{\grad}{\textrm{grad}}
\newcommand{\dom}{\textrm{dom}}
\newcommand{\diag}{\textrm{diag}}

\renewcommand{\;}{\, ; \,}
\renewcommand{\d}{\, {d}}

\newcommand{\gyouretsu}[1]{\begin{pmatrix} #1 \end{pmatrix} }

\renewcommand{\div}{\textrm{div}}


%%図式

\usepackage[dvipdfm,all]{xy}


\newenvironment{claim}[1]{\par\noindent\underline{step:}\space#1}{}
\newenvironment{claimproof}[1]{\par\noindent{($\because$)}\space#1}{\hfill $\blacktriangle $}


\newcommand{\pprime}{{\prime \prime}}

%%マグニチュード


\newcommand{\Mag}{\textrm{Mag}}

\usepackage{mathrsfs}


%%6.13
\def\chint#1{\mathchoice
{\XXint\displaystyle\textstyle{#1}}%
{\XXint\textstyle\scriptstyle{#1}}%
{\XXint\scriptstyle\scriptscriptstyle{#1}}%
{\XXint\scriptscriptstyle\scriptscriptstyle{#1}}%
\!\int}
\def\XXint#1#2#3{{\setbox0=\hbox{$#1{#2#3}{\int}$ }
\vcenter{\hbox{$#2#3$ }}\kern-.6\wd0}}
\def\ddashint{\chint=}
\def\dashint{\chint-}



\title{1-良い射影}
\date{}


\author{}


\begin{document}


\maketitle

\section{}

\begin{dfn}(良い射影). $A, B \subset X$ を部分集合とする. $a \in A$ は 
\begin{align*} ^\exists \pi(a) \in A \cap B ; b \in B \naraba  d(a, b) = d(a, \pi (a)) + d(\pi (a), b) \end{align*}
であるとき, $B$ に良く射影される. 任意の点$a \in A$ が$B$ に良く射影される, $A$ は$B$ に良く射影されるという. $A$ が$B$ に良く射影され, $B$ が$A$ に良く射影されるとき, $A, B$ は互いに良く射影されるという. 
\end{dfn}

\begin{remark} $a \in A$ が$B$ に良く射影されるとき, 類似度行列$Z$の成分に関して, 当たり前だが任意の$b \in B$ に対して
\begin{align*} z_{a b} = z_{a \pi(a)} z_{\pi(a) b}\end{align*}
が成り立つ. 
\end{remark}

\begin{prop}(和集合のマグニチュード). $A, B \subset X$ を$X$ の有限部分集合で, $A, B, A \cap B$ がそれぞれウェイト$w^A, w^B, w^{A \cap B}$を持つとする. $A, B$ が互いに良く射影されるならば,  
\begin{align*} w(x) \coloneqq \begin{cases} w^A(x) & x \in A \\ w^A(x) + w^B(x) - w^{A \cap B }(x) & x \in A\cap B \\ w^B(x) & x \in B \end{cases} \end{align*}
と定めると, これは$A \cup B$ のウェイトである. 従って, 
\begin{align*} \Mag(A \cup B) = \Mag(A) + \Mag(B) - \Mag(A \cap B) \end{align*}
が成り立つ. 
\end{prop}
\begin{pf*}実際, $w$ がウェイトになることは例えば類似度行列の$a \in A$ 行目に関しては
\begin{align*} \sum_{x \in X} Z(a, x) w(x) &= \sum_{a^\prime \in A} Z(a, a^\prime) w^A(a^\prime)   + \sum_{b \in B} Z(a, b) w^{B} (b) - \sum_{c^\prime \in A \cap B} Z(a, c) w^{A \cap B} (c)  \\& = \sum_{a^\prime \in A} Z(a, a^\prime) w^A(a^\prime)  \\& \quad + \sum_{b \in B} Z(a, \pi(a)) Z(\pi(a), b) w^{B} (b) - \sum_{c \in A \cap B} Z(a, \pi(a)) Z(\pi(a), c) w^{A \cap B} (c)    \\ 
&= 1 + Z(a, \pi(a)) (  \sum_{b \in B}  Z(\pi(a), b) w^{B} (b)  - \sum_{c \in A \cap B} Z(\pi(a), c) w^{A \cap B} (c) ) \\
&= 1 + Z(a, \pi(a)) (1- 1) = 1
\end{align*}
よりわかる. $b \in B$ 行目に関しても全く同様に示される. さらに, この$\sum_{x \in A \cup B} w(x)$ を計算すると, 主張が従う. 
\qed
\end{pf*}

\begin{prop}($1$点で交わる集合同士の和). $A, B \subset X$ を $1$ 点($c \in A\cap B$ で表す. )で共通部分をもつ有限部分集合とする. $A, B$ がマグニチュードをもち, $A, B$ が互いに良く射影されるとき, $A \cup B$ はメビウス行列をもち, 
\begin{align*} \mu_{A \cup B}( x, y) \coloneqq \begin{cases} \mu^A(x, y)  & x, y \in A, (x, y) \neq  (c, c) \\ \mu^B(x, y)  & x, y \in B , (x, y) \neq  (c, c) \\ \mu^A(c, c) + \mu^B (c, c) - 1 & (x, y) = (c, c)  \\ 0 & \textrm{otherwise} \end{cases} \end{align*}
で与えられる. ただし, $A, B$ のメビウス行列を$M^A = (\mu^A_{ij}), M^B = (\mu^B_{ij})$ で表している. 

\end{prop}
\begin{pf*}実際, 適当に成分を並べて
\begin{align*} &Z^{A\cup B} \paren{ \gyouretsu{ M^A & 0 \\ 0 & 0   } + \gyouretsu{ 0 & 0 \\ 0 & M^B   } - \diag(0, \ldots ,0,  1 , 0, \ldots , 0) }  \\ & =   \gyouretsu{ E & 0 \\ 0 & 0   } +  \gyouretsu{ 0 & 0 \\ 0 & E   } -  \diag(0, \ldots ,0,  1 , 0, \ldots , 0)  = E \end{align*}
と計算できる. 
\qed
\end{pf*}










\end{document}