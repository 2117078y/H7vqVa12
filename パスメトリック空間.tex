\documentclass[10pt, fleqn, label-section=none]{bxjsarticle}

%\usepackage[driver=dvipdfm,hmargin=25truemm,vmargin=25truemm]{geometry}

\setpagelayout{driver=dvipdfm,hmargin=25truemm,vmargin=20truemm}


\usepackage{amsmath}
\usepackage{amssymb}
\usepackage{amsfonts}
\usepackage{amsthm}
\usepackage{mathtools}
\usepackage{mleftright}

\usepackage{ascmac}




\usepackage{otf}

\theoremstyle{definition}
\newtheorem{dfn}{定義}[section]
\newtheorem{ex}[dfn]{例}
\newtheorem{lem}[dfn]{補題}
\newtheorem{prop}[dfn]{命題}
\newtheorem{thm}[dfn]{定理}
\newtheorem{setting}[dfn]{設定}
\newtheorem{cor}[dfn]{系}
\newtheorem*{pf*}{証明}
\newtheorem{problem}[dfn]{問題}
\newtheorem*{problem*}{問題}
\newtheorem{remark}[dfn]{注意}
\newtheorem*{claim*}{\underline{claim}}



\newtheorem*{solution*}{解答}

%箇条書きの様式
\renewcommand{\labelenumi}{(\arabic{enumi})}


%

\newcommand{\forany}{\rm{for} \ {}^{\forall}}
\newcommand{\foranyeps}{
\rm{for} \ {}^{\forall}\varepsilon >0}
\newcommand{\foranyk}{
\rm{for} \ {}^{\forall}k}


\newcommand{\any}{{}^{\forall}}
\newcommand{\suchthat}{\, \rm{s.t.} \, \it{}}




\newcommand{\veps}{\varepsilon}
\newcommand{\paren}[1]{\mleft( #1\mright )}
\newcommand{\cbra}[1]{\mleft\{#1\mright\}}
\newcommand{\sbra}[1]{\mleft\lbrack#1\mright\rbrack}
\newcommand{\tbra}[1]{\mleft\langle#1\mright\rangle}
\newcommand{\abs}[1]{\left|#1\right|}
\newcommand{\norm}[1]{\left\|#1\right\|}
\newcommand{\lopen}[1]{\mleft(#1\mright\rbrack}
\newcommand{\ropen}[1]{\mleft\lbrack #1 \mright)}



%
\newcommand{\Rn}{\mathbb{R}^n}
\newcommand{\Cn}{\mathbb{C}^n}

\newcommand{\Rm}{\mathbb{R}^m}
\newcommand{\Cm}{\mathbb{C}^m}


\newcommand{\projs}[2]{\it{p}_{#1,\ldots,#2}}
\newcommand{\projproj}[2]{\it{p}_{#1,#2}}

\newcommand{\proj}[1]{p_{#1}}

%可測空間
\newcommand{\stdProbSp}{\paren{\Omega, \mathcal{F}, P}}

%微分作用素
\newcommand{\ddt}{\frac{d}{dt}}
\newcommand{\ddx}{\frac{d}{dx}}
\newcommand{\ddy}{\frac{d}{dy}}

\newcommand{\delt}{\frac{\partial}{\partial t}}
\newcommand{\delx}{\frac{\partial}{\partial x}}

%ハイフン
\newcommand{\hyphen}{\text{-}}

%displaystyle
\newcommand{\dstyle}{\displaystyle}

%⇔, ⇒, \UTF{21D0}%
\newcommand{\LR}{\Leftrightarrow}
\newcommand{\naraba}{\Rightarrow}
\newcommand{\gyaku}{\Leftarrow}

%理由
\newcommand{\naze}[1]{\paren{\because {\mathop{ #1 }}}}

%
\newcommand{\sankaku}{\hfill $\triangle$}

%
\newcommand{\push}{_{\#}}

%手抜き
\newcommand{\textif}{\textrm{if}\,\,\,}
\newcommand{\Ric}{\textrm{Ric}}
\newcommand{\tr}{\textrm{tr}}
\newcommand{\vol}{\textrm{vol}}
\newcommand{\diam}{\textrm{diam}}
\newcommand{\supp}{\textrm{supp}}
\newcommand{\Med}{\textrm{Med}}
\newcommand{\Leb}{\textrm{Leb}}
\newcommand{\Const}{\textrm{Const}}
\newcommand{\Avg}{\textrm{Avg}}
\newcommand{\id}{\textrm{id}}
\newcommand{\Ker}{\textrm{Ker}}
\newcommand{\im}{\textrm{Im}}




\renewcommand{\;}{\, ; \,}
\renewcommand{\d}{\, {d}}

\newcommand{\gyouretsu}[1]{\begin{pmatrix} #1 \end{pmatrix} }

%%図式

\usepackage[dvipdfm,all]{xy}


\newenvironment{claim}[1]{\par\noindent\underline{step:}\space#1}{}
\newenvironment{claimproof}[1]{\par\noindent{($\because$)}\space#1}{\hfill $\blacktriangle $}


\newcommand{\pprime}{{\prime \prime}}





%%


\title{パスメトリック空間}
\date{}


\author{}


\begin{document}


\maketitle



\section{}

\subsection{パスメトリック空間}




\subsection{パスメトリック空間の定義} 

\begin{dfn}(パスメトリック空間). 距離空間$(X, d)$ は, 任意の二点$x_0, x_1 \in X$ に対して
\begin{align*} d(x_0, x_1) = \inf \cbra{ \sup_{\Pi} \sum d(c(t_i), c(t_{i+1} )  ) \mid c \in C([0,1]; X), c_0 = x_0, c_1 = x_1 }   \end{align*}
が成り立つとき, パスメトリック空間という. ただし, $\sup$ は$[0, 1]$ 区間のあらゆる分割を走る. $C([0,1]; X)$ は$[0,1]$ から$X$ への連続写像全体を表す. 
\end{dfn}

\begin{prop}$(X, d)$ を完備距離空間とする. 任意の$x_0, x_1 \in X$ に対し, 任意の$\veps > 0$ に対して, $x^\prime \in X$ で
\begin{align*} \quad  \sup{ d(x_0 ,  x^\prime) , d(x^\prime , x_1)} \leq \frac{1}{2} d(x_0, x_1) + \veps \end{align*}
 を満たすものが存在するならば, $(X, d)$ はパラメトリック空間である. 
\end{prop}
\begin{pf*}
十分小さい$\veps_1$ に対して
\begin{align*}  \sup{ d(x_0 ,  x_{\frac{1}{2}}     ) , d(x_{\frac{1}{2}}  , x_1)}    & \leq \frac{1}{2} d(x_0, x_1) + \veps \frac{1}{2} d(x_0, x_1)  \\
&= \frac{1}{2} d(x_0, x_1)  (1 + \veps_1) \end{align*}
を満たす$x_{\frac{1}{2}} $ がとれる. 次にこれまた十分小さい$\veps_2$ に対して
\begin{align*} & \sup{ d(x_0 ,  x_{\frac{1}{4}} ) , d(x_{\frac{1}{4}}, x_{\frac{1}{2}}  ), d(x_{\frac{1}{2}}, x_{\frac{3}{4}} ) d(x_{\frac{3}{4}}  , x_1)}     \\
& \leq \frac{1}{2}( \frac{1}{2} d(x_0, x_1)  + \veps \frac{1}{2}d(x_0, x_1)  ) + \veps_2 (\frac{1}{2} d(x_0, x_1)  + \veps_1 \frac{1}{2} d(x_0, x_1) ) \\
&= \frac{1}{2}\frac{1}{2}\frac{1}{2} d(x_0, x_1)  (1 + \veps _1) (1 + \veps_2) \end{align*}
を満たす$x_{\frac{1}{4}} , x_{\frac{3}{4}}$ がとれる. これを繰り返して, $\veps_1, \veps_2, \ldots $ と$x_{\frac{1}{2}} , x_{\frac{1}{4}}, x_{\frac{3}{4}}, x_{\frac{1}{8}} , \ldots   $ を定める. ただし, $\veps_1, \veps_2, \ldots $ は十分小さくとって$\prod _{k = 1}^\infty (1 + \veps_k) $ が発散しないようにしておく. 
$[0, 1]$ に含まれる二進有理数上で
\begin{align*} d(x_{\frac{k}{2^n}} , x_{\frac{k + 1}{2^n} } ) \leq \frac{1}{2^n} \prod _{k = 1}^\infty (1 + \veps_k) \end{align*}
が成り立つ. 二進分数でない$x_r$ の値を, $r$ に収束する二進分数の列$q_1, q_2, \ldots $をとり, $\cbra{x_{q_i}}$ を考えるとコーシー列になるので完備性から収束列となり, その値によって$x_r$ を定める. これにより連続な曲線$x: [0,1] \rightarrow X$ が定まるが,  $\prod _{k = 1}^\infty (1 + \veps_k) $ はいくらでも小さくできるので, パラメトリック空間であることが示される. 
\qed
\end{pf*}



\begin{prop}
$(X, d)$ を距離空間とする. (1)と(2)は必要十分である.\\
(1) 任意の$x_0, x_1 \in X$ に対し, 任意の$\veps > 0$ に対して, $x^\prime \in X$ で
\begin{align*} \quad  \sup{ d(x_0 ,  x^\prime) , d(x^\prime , x_1)} \leq \frac{1}{2} d(x_0, x_1) + \veps \end{align*}
 を満たすものが存在する.  \\
 (2)任意の$x_0, x_1 \in X$に対し, 任意の$r_0 + r_1 \leq d(x,y)$ を満たす$r_0 , r_1 > 0$ に対して
 \begin{align*} \quad d(B(x_0; r_0), B(x_1 ; r_1)) \leq d(x_0, x_1) - r_1 - r_2\end{align*}
 が成り立つ. 
\end{prop}
\begin{pf*}
体調が良いときに埋める. 
\qed
\end{pf*}











\end{document}
