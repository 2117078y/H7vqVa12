\documentclass[10pt, fleqn, label-section=none]{bxjsarticle}

%\usepackage[driver=dvipdfm,hmargin=25truemm,vmargin=25truemm]{geometry}

\setpagelayout{driver=dvipdfm,hmargin=25truemm,vmargin=20truemm}


\usepackage{amsmath}
\usepackage{amssymb}
\usepackage{amsfonts}
\usepackage{amsthm}
\usepackage{mathtools}
\usepackage{mleftright}

\usepackage{ascmac}




\usepackage{otf}

\theoremstyle{definition}
\newtheorem{dfn}{定義}[section]
\newtheorem{ex}[dfn]{例}
\newtheorem{lem}[dfn]{補題}
\newtheorem{prop}[dfn]{命題}
\newtheorem{thm}[dfn]{定理}
\newtheorem{setting}[dfn]{設定}
\newtheorem{notation}[dfn]{記号}
\newtheorem{cor}[dfn]{系}
\newtheorem*{pf*}{証明}
\newtheorem{problem}[dfn]{問題}
\newtheorem*{problem*}{問題}
\newtheorem{remark}[dfn]{注意}
\newtheorem*{claim*}{\underline{claim}}



\newtheorem*{solution*}{解答}

%箇条書きの様式
\renewcommand{\labelenumi}{(\arabic{enumi})}


%

\newcommand{\forany}{\rm{for} \ {}^{\forall}}
\newcommand{\foranyeps}{
\rm{for} \ {}^{\forall}\varepsilon >0}
\newcommand{\foranyk}{
\rm{for} \ {}^{\forall}k}


\newcommand{\any}{{}^{\forall}}
\newcommand{\suchthat}{\, \rm{s.t.} \, \it{}}




\newcommand{\veps}{\varepsilon}
\newcommand{\paren}[1]{\mleft( #1\mright )}
\newcommand{\cbra}[1]{\mleft\{#1\mright\}}
\newcommand{\sbra}[1]{\mleft\lbrack#1\mright\rbrack}
\newcommand{\tbra}[1]{\mleft\langle#1\mright\rangle}
\newcommand{\abs}[1]{\left|#1\right|}
\newcommand{\norm}[1]{\left\|#1\right\|}
\newcommand{\lopen}[1]{\mleft(#1\mright\rbrack}
\newcommand{\ropen}[1]{\mleft\lbrack #1 \mright)}



%
\newcommand{\Rn}{\mathbb{R}^n}
\newcommand{\Cn}{\mathbb{C}^n}

\newcommand{\Rm}{\mathbb{R}^m}
\newcommand{\Cm}{\mathbb{C}^m}


\newcommand{\projs}[2]{\it{p}_{#1,\ldots,#2}}
\newcommand{\projproj}[2]{\it{p}_{#1,#2}}

\newcommand{\proj}[1]{p_{#1}}

%可測空間
\newcommand{\stdProbSp}{\paren{\Omega, \mathcal{F}, P}}

%微分作用素
\newcommand{\ddt}{\frac{d}{dt}}
\newcommand{\ddx}{\frac{d}{dx}}
\newcommand{\ddy}{\frac{d}{dy}}

\newcommand{\delt}{\frac{\partial}{\partial t}}
\newcommand{\delx}{\frac{\partial}{\partial x}}

%ハイフン
\newcommand{\hyphen}{\text{-}}

%displaystyle
\newcommand{\dstyle}{\displaystyle}

%⇔, ⇒, \UTF{21D0}%
\newcommand{\LR}{\Leftrightarrow}
\newcommand{\naraba}{\Rightarrow}
\newcommand{\gyaku}{\Leftarrow}

%理由
\newcommand{\naze}[1]{\paren{\because {\mathop{ #1 }}}}

%
\newcommand{\sankaku}{\hfill $\triangle$}

%
\newcommand{\push}{_{\#}}

%手抜き
\newcommand{\textif}{\textrm{if}\,\,\,}
\newcommand{\Ric}{\textrm{Ric}}
\newcommand{\tr}{\textrm{tr}}
\newcommand{\vol}{\textrm{vol}}
\newcommand{\diam}{\textrm{diam}}
\newcommand{\supp}{\textrm{supp}}
\newcommand{\Med}{\textrm{Med}}
\newcommand{\Leb}{\textrm{Leb}}
\newcommand{\Const}{\textrm{Const}}
\newcommand{\Avg}{\textrm{Avg}}
\newcommand{\id}{\textrm{id}}
\newcommand{\Ker}{\textrm{Ker}}
\newcommand{\im}{\textrm{Im}}
\newcommand{\dil}{\textrm{dil}}
\newcommand{\Ch}{\textrm{Ch}}
\newcommand{\Lip}{\textrm{Lip}}
\newcommand{\Ent}{\textrm{Ent}}
\newcommand{\grad}{\textrm{grad}}
\newcommand{\dom}{\textrm{dom}}
\newcommand{\diag}{\textrm{diag}}

\renewcommand{\;}{\, ; \,}
\renewcommand{\d}{\, {d}}

\newcommand{\gyouretsu}[1]{\begin{pmatrix} #1 \end{pmatrix} }

\renewcommand{\div}{\textrm{div}}


%%図式

\usepackage[dvipdfm,all]{xy}


\newenvironment{claim}[1]{\par\noindent\underline{step:}\space#1}{}
\newenvironment{claimproof}[1]{\par\noindent{($\because$)}\space#1}{\hfill $\blacktriangle $}


\newcommand{\pprime}{{\prime \prime}}

%%マグニチュード


\newcommand{\Mag}{\textrm{Mag}}

\usepackage{mathrsfs}


%%6.13
\def\chint#1{\mathchoice
{\XXint\displaystyle\textstyle{#1}}%
{\XXint\textstyle\scriptstyle{#1}}%
{\XXint\scriptstyle\scriptscriptstyle{#1}}%
{\XXint\scriptscriptstyle\scriptscriptstyle{#1}}%
\!\int}
\def\XXint#1#2#3{{\setbox0=\hbox{$#1{#2#3}{\int}$ }
\vcenter{\hbox{$#2#3$ }}\kern-.6\wd0}}
\def\ddashint{\chint=}
\def\dashint{\chint-}


%%7.13

\usepackage{here}

%7.15
\newcommand{\Span}{\textrm{Span}}

\newcommand{\Conv}{\textrm{Conv}}



\title{距離空間上の力学系の定義}
\date{}


\author{}


\begin{document}


\maketitle

\section{}

\subsection{}

\begin{dfn}(距離空間上の力学系). $(X, d)$ を距離空間とする. 写像の族$\cbra{\varphi_t : X \rightarrow X}_{t \in \mathbb R}$ は, \\
(1)任意の$t \in \mathbb R$ に対して$\varphi_t : X \rightarrow X$ は連続である. \\
(2)任意の$x \in X$ に対して, $t \mapsto \varphi_t x$ は連続である. \\
(3)任意の$x \in X$ に対して, $\varphi_0 x = x$ が成り立つ. \\
(4)任意の$t, s \in \mathbb R$ に対して, $\varphi_t \circ \varphi_s = \varphi_{t +s}$ が成り立つ. \\
を満たす時, $X$ 上のフロー, あるいは力学系という. 

\end{dfn}

\begin{prop} $(X, d)$ を距離空間とする. 写像の族$\cbra{\varphi_t : X \rightarrow X}_{t \in \mathbb R}$ をフローとする. 任意の$t \in \mathbb R$ に対して, $\varphi_t$ は同相写像である. 

\end{prop}
\begin{pf*}$\varphi_t $ に対して$\varphi_{-t}$ は両側逆写像であるので, 全単射であり, ともに連続であるから. 

\qed
\end{pf*}


\begin{dfn}(フローの半軌道). $(X, d)$ を距離空間とする. 写像の族$\cbra{\varphi_t : X \rightarrow X}_{t \in \mathbb R}$ をフローとする. $x \in X$ に対して, 
\begin{align*} \gamma^+ x \coloneqq \cbra{\varphi_t x \mid t \geq 0 }   , \quad \gamma^- x \coloneqq  \cbra{\varphi_t x \mid t \leq 0 } \end{align*}
と定め, それぞれ$x$ の正の半軌道, 負の半軌道という. 

\end{dfn}

\begin{dfn}(距離空間上の半流). $(X, d)$ を距離空間とする. 写像の族$\cbra{\varphi_t : X \rightarrow X}_{t \in \mathbb R_{\geq 0 }   }$ は, \\
(1)任意の$t \in \mathbb R_{\geq 0 }$ に対して$\varphi_t : X \rightarrow X$ は連続である. \\
(2)任意の$x \in X$ に対して, $t \mapsto \varphi_t x$ は連続である. \\
(3)任意の$x \in X$ に対して, $\varphi_0 x = x$ が成り立つ. \\
(4)任意の$t, s \in \mathbb R_{\geq 0 }$ に対して, $\varphi_t \circ \varphi_s = \varphi_{t +s}$ が成り立つ. \\
を満たす時, $X$ 上の半流という. 

\end{dfn}

\begin{dfn}(半流の半軌道). $(X, d)$ を距離空間とする. 写像の族$\cbra{\varphi_t : X \rightarrow X}_{t \in \mathbb R_{\geq 0 } }$ を半流とする. $x \in X$ に対して, 
\begin{align*} \gamma^+ x \coloneqq \cbra{\varphi_t x \mid t \geq 0 } \end{align*}
と定め, $x$ の正の半軌道という. また, 曲線$\cbra{c(t)}_{t \leq 0}$ で$c_0 = x$ かつ
\begin{align*} t \in (- \infty, 0], 0 \leq s \leq -t \naraba \varphi_s (c_t) = c_{t + s} \end{align*}
を満たすものを, $x$ の負の半軌道という. 

\end{dfn}


\begin{dfn}(平衡点). $(X, d)$ を距離空間とする. 写像の族$\cbra{\varphi_t : X \rightarrow X}_{t \in \mathbb R_{\geq 0 } }$ を半流とする. $x \in X$ で
\begin{align*} \varphi_t x = x \quad (\any t \in \mathbb R_{\geq 0}) \end{align*}
を満たすものを, $\varphi_t$ の平衡点という. 

\end{dfn}

\begin{dfn}(正不変集合). $(X, d)$ を距離空間とする. 写像の族$\cbra{\varphi_t : X \rightarrow X}_{t \in \mathbb R_{\geq 0 } }$ を半流とする. $S \subset  X$ で
\begin{align*} \varphi_t S \subset  S \quad (\any t \in \mathbb R_{\geq 0}) \end{align*}
を満たすものを, $\varphi_t$ の正不変集合という. 

\end{dfn}

\begin{dfn}(不変集合). $(X, d)$ を距離空間とする. 写像の族$\cbra{\varphi_t : X \rightarrow X}_{t \in \mathbb R_{\geq 0 } }$ を半流とする. $S \subset  X$ で
\begin{align*} \varphi_t S =  S \quad (\any t \in \mathbb R_{\geq 0}) \end{align*}
を満たすものを, $\varphi_t$ の不変集合という. 

\end{dfn}

\begin{prop} 不変集合ならば, 正不変集合である. 

\end{prop}
\begin{pf*}
定義より明らか. 
\qed
\end{pf*}

\begin{prop}$x \in X$ を平衡点とする. $\cbra{x}$ は不変集合である. 

\end{prop}
\begin{pf*}
定義より明らか. 
\qed
\end{pf*}


\begin{dfn}(リャプノフ安定). $(X, d)$ を距離空間とする. 写像の族$\cbra{\varphi_t : X \rightarrow X}_{t \in \mathbb R_{\geq 0 } }$ を半流とする. $x \in X$は, 任意の$\veps > 0$ に対して, $\delta > 0$ で
\begin{align*} \varphi_t B(x; \delta) \subset B(x; \veps) \quad (0 \leq t) \end{align*}
満たすときリャプノフ安定という. 
\end{dfn}

\begin{dfn}(弱漸近的安定). $(X, d)$ を距離空間とする. 写像の族$\cbra{\varphi_t : X \rightarrow X}_{t \in \mathbb R_{\geq 0 } }$ を半流とする. $x \in X$は, 任意の$\veps > 0$ に対して, $\delta > 0$ で
\begin{align*} y \in B(x; \delta) \naraba, \lim_{t \rightarrow \infty} \varphi_t y = x \end{align*}
を満たすものがとれるとき, 弱漸近的安定という. 
\end{dfn}

\begin{dfn}$(X, d)$ を距離空間とする. 写像の族$\cbra{\varphi_t : X \rightarrow X}_{t \in \mathbb R_{\geq 0 } }$ を半流とする. $x \in X$は, リャプノフ安定かつ弱漸近的安定であるとき, 漸近的安定であるという. 

\end{dfn}

\begin{ex}$2$ 次元球面の赤道上に適当に点$x \in S^2$ をとり, $x$ で接する小円全体を軌道とするフローを考えると, 球面の任意の点$y \in S^2$ に対して$\lim_{t \rightarrow \infty} \varphi_t y = x$ が成り立つが, 明らかにリャプノフ安定ではない. 

\end{ex}

\end{document}