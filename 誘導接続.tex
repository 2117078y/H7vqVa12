\documentclass[10pt, fleqn, label-section=none]{bxjsarticle}

%\usepackage[driver=dvipdfm,hmargin=25truemm,vmargin=25truemm]{geometry}

\setpagelayout{driver=dvipdfm,hmargin=25truemm,vmargin=20truemm}


\usepackage{amsmath}
\usepackage{amssymb}
\usepackage{amsfonts}
\usepackage{amsthm}
\usepackage{mathtools}
\usepackage{mleftright}

\usepackage{ascmac}




\usepackage{otf}

\theoremstyle{definition}
\newtheorem{dfn}{定義}[section]
\newtheorem{ex}[dfn]{例}
\newtheorem{lem}[dfn]{補題}
\newtheorem{prop}[dfn]{命題}
\newtheorem{thm}[dfn]{定理}
\newtheorem{setting}[dfn]{設定}
\newtheorem{cor}[dfn]{系}
\newtheorem*{pf*}{証明}
\newtheorem{problem}[dfn]{問題}
\newtheorem*{problem*}{問題}
\newtheorem{remark}[dfn]{注意}
\newtheorem*{claim*}{\underline{claim}}



\newtheorem*{solution*}{解答}

%箇条書きの様式
\renewcommand{\labelenumi}{(\arabic{enumi})}


%

\newcommand{\forany}{\rm{for} \ {}^{\forall}}
\newcommand{\foranyeps}{
\rm{for} \ {}^{\forall}\varepsilon >0}
\newcommand{\foranyk}{
\rm{for} \ {}^{\forall}k}


\newcommand{\any}{{}^{\forall}}
\newcommand{\suchthat}{\, \rm{s.t.} \, \it{}}




\newcommand{\veps}{\varepsilon}
\newcommand{\paren}[1]{\mleft( #1\mright )}
\newcommand{\cbra}[1]{\mleft\{#1\mright\}}
\newcommand{\sbra}[1]{\mleft\lbrack#1\mright\rbrack}
\newcommand{\tbra}[1]{\mleft\langle#1\mright\rangle}
\newcommand{\abs}[1]{\left|#1\right|}
\newcommand{\norm}[1]{\left\|#1\right\|}
\newcommand{\lopen}[1]{\mleft(#1\mright\rbrack}
\newcommand{\ropen}[1]{\mleft\lbrack #1 \mright)}



%
\newcommand{\Rn}{\mathbb{R}^n}
\newcommand{\Cn}{\mathbb{C}^n}

\newcommand{\Rm}{\mathbb{R}^m}
\newcommand{\Cm}{\mathbb{C}^m}


\newcommand{\projs}[2]{\it{p}_{#1,\ldots,#2}}
\newcommand{\projproj}[2]{\it{p}_{#1,#2}}

\newcommand{\proj}[1]{p_{#1}}

%可測空間
\newcommand{\stdProbSp}{\paren{\Omega, \mathcal{F}, P}}

%微分作用素
\newcommand{\ddt}{\frac{d}{dt}}
\newcommand{\ddx}{\frac{d}{dx}}
\newcommand{\ddy}{\frac{d}{dy}}

\newcommand{\delt}{\frac{\partial}{\partial t}}
\newcommand{\delx}{\frac{\partial}{\partial x}}

%ハイフン
\newcommand{\hyphen}{\text{-}}

%displaystyle
\newcommand{\dstyle}{\displaystyle}

%⇔, ⇒, \UTF{21D0}%
\newcommand{\LR}{\Leftrightarrow}
\newcommand{\naraba}{\Rightarrow}
\newcommand{\gyaku}{\Leftarrow}

%理由
\newcommand{\naze}[1]{\paren{\because {\mathop{ #1 }}}}

%
\newcommand{\sankaku}{\hfill $\triangle$}

%
\newcommand{\push}{_{\#}}

%手抜き
\newcommand{\textif}{\textrm{if}\,\,\,}
\newcommand{\Ric}{\textrm{Ric}}
\newcommand{\tr}{\textrm{tr}}
\newcommand{\vol}{\textrm{vol}}
\newcommand{\diam}{\textrm{diam}}
\newcommand{\supp}{\textrm{supp}}
\newcommand{\Med}{\textrm{Med}}
\newcommand{\Leb}{\textrm{Leb}}
\newcommand{\Const}{\textrm{Const}}
\newcommand{\Avg}{\textrm{Avg}}
\newcommand{\id}{\textrm{id}}
\newcommand{\Ker}{\textrm{Ker}}
\newcommand{\im}{\textrm{Im}}




\renewcommand{\;}{\, ; \,}
\renewcommand{\d}{\, {d}}

\newcommand{\gyouretsu}[1]{\begin{pmatrix} #1 \end{pmatrix} }

%%図式

\usepackage[dvipdfm,all]{xy}


\newenvironment{claim}[1]{\par\noindent\underline{step:}\space#1}{}
\newenvironment{claimproof}[1]{\par\noindent{($\because$)}\space#1}{\hfill $\blacktriangle $}


\newcommand{\pprime}{{\prime \prime}}





%%


\title{誘導接続}
\date{}


\author{}


\begin{document}


\maketitle

\section{}

\begin{dfn}
$N, M$ を可微分多様体, $f: N \rightarrow M$ を$C^\infty$ 級の写像とする.  $\xi : N \rightarrow TM$ なる$C^\infty$級写像で
\begin{align*} \pi_{TM} \circ \xi = f \end{align*}
を満たすものを, $f$ に沿ったベクトル場という. その全体を$\Gamma(f^* TM)$ で表す. 
\end{dfn}

\begin{prop}
$N, M$ を多様体, $f: N \rightarrow M$ を$C^\infty$ 級の写像とする. $M$ の接続$\nabla$ に対して, 写像
\begin{align*} \nabla^{f^*} : \mathfrak{X}(N) \times \Gamma(f^* TM) \rightarrow \Gamma(f^* TM)  \end{align*}
で \\
(1)$\mathbb R$ 上の双線形写像である. \\
(2)$\nabla^{f^*}_{gX} \xi = g \nabla^{f^*}_X \xi$ が成り立つ.\\ 
(3)$\nabla^{f^*}_X (g\xi) = X(g)\xi + g \nabla^{f^*}_X \xi$ が成り立つ.\\
(4)任意の$\xi \in \Gamma(f^* TM)$ に対して, $N$ の開集合$U \subset N$ と$M$ の開集合$V \subset M$ と, $U$ 上のベクトル場$Y$ が$f(U) \subset V$ かつ$\xi = f^*Y$ を満たすならば, 
\begin{align*} \quad (\nabla^{f^*}_X \xi)_p = \nabla _{df_p (X)} Y \quad (p \in U)\end{align*}
が成り立つ. \\
を満たすようなものが一意に存在する. 
\end{prop}
\begin{pf*}(sketch). 
$X \in \mathfrak X (M), \xi \in \Gamma(f^* TM), p \in N$ に対して$\nabla^{f^*}_X \xi$ を, 適当に$f(p)$の周囲の局所座標$(V, y_1, \ldots, y_m)$をとって, $f(U) \subset V$ となるように$p \in N$ の周りの局所座標を$(U, x_1, \ldots , x_n)$ ととる. $X = X^i \partial_i, \xi = \xi^\alpha \delta_\alpha$ と表示することにして, $\Gamma_{\alpha \beta }^\gamma$ を$\nabla$ の$(V, y_1, \ldots, y_m)$ に対応するクリストっフェル記号とする. 
\begin{align*} (\nabla^{f^*} _X \xi)_p \coloneqq ( X^i (\partial_i \xi^\gamma)  + X ^i (\partial_i f^\beta ) \xi^\alpha f^* \Gamma_{\beta \alpha} ^\gamma ) f^* (\delta_\gamma)  \end{align*}
により定めるとよい.
\qed
\end{pf*}






\end{document}