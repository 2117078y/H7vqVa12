\documentclass[10pt, fleqn, label-section=none]{bxjsarticle}

%\usepackage[driver=dvipdfm,hmargin=25truemm,vmargin=25truemm]{geometry}

\setpagelayout{driver=dvipdfm,hmargin=25truemm,vmargin=20truemm}


\usepackage{amsmath}
\usepackage{amssymb}
\usepackage{amsfonts}
\usepackage{amsthm}
\usepackage{mathtools}
\usepackage{mleftright}

\usepackage{ascmac}




\usepackage{otf}

\theoremstyle{definition}
\newtheorem{dfn}{定義}[section]
\newtheorem{ex}[dfn]{例}
\newtheorem{lem}[dfn]{補題}
\newtheorem{prop}[dfn]{命題}
\newtheorem{thm}[dfn]{定理}
\newtheorem{cor}[dfn]{系}
\newtheorem*{pf*}{証明}
\newtheorem{problem}[dfn]{問題}
\newtheorem*{problem*}{問題}
\newtheorem{remark}[dfn]{注意}
\newtheorem*{claim*}{\underline{claim}}



\newtheorem*{solution*}{解答}

%箇条書きの様式
\renewcommand{\labelenumi}{(\arabic{enumi})}


%

\newcommand{\forany}{\rm{for} \ {}^{\forall}}
\newcommand{\foranyeps}{
\rm{for} \ {}^{\forall}\varepsilon >0}
\newcommand{\foranyk}{
\rm{for} \ {}^{\forall}k}


\newcommand{\any}{{}^{\forall}}
\newcommand{\suchthat}{\, \rm{s.t.} \, \it{}}




\newcommand{\veps}{\varepsilon}
\newcommand{\paren}[1]{\mleft( #1\mright )}
\newcommand{\cbra}[1]{\mleft\{#1\mright\}}
\newcommand{\sbra}[1]{\mleft\lbrack#1\mright\rbrack}
\newcommand{\tbra}[1]{\mleft\langle#1\mright\rangle}
\newcommand{\abs}[1]{\left|#1\right|}
\newcommand{\norm}[1]{\left\|#1\right\|}
\newcommand{\lopen}[1]{\mleft(#1\mright\rbrack}
\newcommand{\ropen}[1]{\mleft\lbrack #1 \mright)}



%
\newcommand{\Rn}{\mathbb{R}^n}
\newcommand{\Cn}{\mathbb{C}^n}

\newcommand{\Rm}{\mathbb{R}^m}
\newcommand{\Cm}{\mathbb{C}^m}


\newcommand{\projs}[2]{\it{p}_{#1,\ldots,#2}}
\newcommand{\projproj}[2]{\it{p}_{#1,#2}}

\newcommand{\proj}[1]{p_{#1}}

%可測空間
\newcommand{\stdProbSp}{\paren{\Omega, \mathcal{F}, P}}

%微分作用素
\newcommand{\ddt}{\frac{d}{dt}}
\newcommand{\ddx}{\frac{d}{dx}}
\newcommand{\ddy}{\frac{d}{dy}}

\newcommand{\delt}{\frac{\partial}{\partial t}}
\newcommand{\delx}{\frac{\partial}{\partial x}}

%ハイフン
\newcommand{\hyphen}{\text{-}}

%displaystyle
\newcommand{\dstyle}{\displaystyle}

%⇔, ⇒, \UTF{21D0}%
\newcommand{\LR}{\Leftrightarrow}
\newcommand{\naraba}{\Rightarrow}
\newcommand{\gyaku}{\Leftarrow}

%理由
\newcommand{\naze}[1]{\paren{\because {\mathop{ #1 }}}}

%
\newcommand{\sankaku}{\hfill $\triangle$}

%
\newcommand{\push}{_{\#}}

%手抜き
\newcommand{\textif}{\textrm{if}\,\,\,}
\newcommand{\Ric}{\textrm{Ric}}
\newcommand{\tr}{\textrm{tr}}
\newcommand{\vol}{\textrm{vol}}
\newcommand{\diam}{\textrm{diam}}
\newcommand{\supp}{\textrm{supp}}
\newcommand{\Med}{\textrm{Med}}
\newcommand{\Leb}{\textrm{Leb}}
\newcommand{\Const}{\textrm{Const}}
\newcommand{\Avg}{\textrm{Avg}}
\newcommand{\id}{\textrm{id}}
\newcommand{\Ker}{\textrm{Ker}}
\newcommand{\im}{\textrm{Im}}




\renewcommand{\;}{\, ; \,}
\renewcommand{\d}{\, {d}}

\newcommand{\gyouretsu}[1]{\begin{pmatrix} #1 \end{pmatrix} }

%%図式

\usepackage[dvipdfm,all]{xy}


\newenvironment{claim}[1]{\par\noindent\underline{Step:}\space#1}{}
\newenvironment{claimproof}[1]{\par\noindent{($\because$)}\space#1}{\hfill $\blacktriangle $}


\newcommand{\pprime}{{\prime \prime}}


%%


\title{グロタンディーク群}
\date{}


\author{}


\begin{document}


\maketitle



\section{}

\begin{dfn}
群の条件のうち, 逆元の存在を除いたものを半群という. 
\end{dfn}

\begin{prop}
$H$ を可換半群とする. このとき, 可換群$\tilde H$ と半群準同型$j: H \rightarrow \tilde H$ で, \\
(普遍性)任意の可換群$G$ と半群準同型$f: H \rightarrow G$ に対して$f = \tilde f \circ j$ をみたす$\tilde fi : \tilde H \rightarrow $がとれる.
を満たすものが存在する.
\end{prop}
\begin{pf*}
$H\times H$ に同値関係を$(n,m) \sim (n^\prime , m^\prime) \LR n + m^\prime = m + n^\prime $ で定める.  集合$\tilde H \coloneqq H \times H / \sim$ に演算$+$を
\begin{align*} [n,m] + [n^\prime , m^\prime ] = [n + n^\prime, m + m^\prime ]\end{align*}
で定める. ($(n^\pprime , m^\pprime  ) \in [n^\prime , m^\prime ]$ とすると, $n^\prime + m^\pprime = m^\prime + n^\pprime $ であるので, $n + n^\pprime + m + m^\prime = m + m^\pprime + n + n^\prime $ となるので, $[n + n^\pprime , m + m^\pprime ] = [n + n^\prime , m + m^\prime ]$ となるので, 不良定義でない. ) すると, $\tilde H $ は
\begin{claim}
可換である.
\end{claim}
\begin{claimproof}
明らかである.
\end{claimproof}

\begin{claim}
$[n,n]$ が単位元である. 
\end{claim}
\begin{claimproof}
適当に$[s,t] \in \tilde H$ をとる. $[n, n] + [ s, t] = [n + s, n + t] $ であるが, $(n +s, n+t) \sim (s, t) $ であるので, $[n +s, n+t] = [s, t] $
\end{claimproof}

\begin{claim}
$[n,m]$ の逆元は$[m,n]$ である. 
\end{claim}
\begin{claimproof}
$[n,m] + [m,n] = [n + m , n + m]$ となる. 
\end{claimproof}

\begin{claim}適当な$N \in H$ を用いて
$j : H \rightarrow \tilde H ; n \mapsto [n + N, N] $ と定めると, $j$ は半群準同型である. 
\end{claim}
\begin{claimproof}
$j(n + m) = [n + m + N, N] = [n + m + N + N , N + N] = j(n) + j(m)$ が成り立つ.
\end{claimproof}

\begin{claim}
任意の半群準同型$f: H \rightarrow G$ に対して, $\tilde f ([n,m]) \coloneqq f(n) - f(m)$ により$\tilde f : \tilde H \rightarrow G$ を定めると, $f = \tilde f \circ j$ が成り立つ. 
\end{claim}
\begin{claimproof}
($(n,m) \sim (n^\prime, m^\prime )$ であれば, $f(n^\prime) - f(m^\prime ) = f(n^\prime - m^\prime ) = f( n - m) = f(n) - f(m)$ となるので不良定義ではない.) $\tilde f (j(n)) = \tilde f ([n +N, N]) = f(n + N) - f(N) = f(n + N - N) =  f(n)$ が成り立つ. 
\end{claimproof}
以上により示された.
\qed
\end{pf*}

\begin{dfn}
可換半群$H$ に対して以上のようにして定まる可換群$\tilde H$ を$H$ のグロタンディーク群という. 
\end{dfn}

\begin{remark}
$\tilde H$ に加法を定める際に, いかなる$N \in H$ をとっても, 結局は群として同型となる. 
\end{remark}














\end{document}