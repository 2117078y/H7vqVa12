\documentclass[10pt, fleqn, label-section=none]{bxjsarticle}

%\usepackage[driver=dvipdfm,hmargin=25truemm,vmargin=25truemm]{geometry}

\setpagelayout{driver=dvipdfm,hmargin=25truemm,vmargin=20truemm}


\usepackage{amsmath}
\usepackage{amssymb}
\usepackage{amsfonts}
\usepackage{amsthm}
\usepackage{mathtools}
\usepackage{mleftright}





\usepackage{otf}

\theoremstyle{definition}
\newtheorem{dfn}{定義}[section]
\newtheorem{ex}[dfn]{例}
\newtheorem{lem}[dfn]{補題}
\newtheorem{prop}[dfn]{命題}
\newtheorem{thm}[dfn]{定理}
\newtheorem{cor}[dfn]{系}
\newtheorem*{pf*}{証明}
\newtheorem{problem}[dfn]{問題}
\newtheorem*{problem*}{問題}
\newtheorem{remark}[dfn]{注意}
\newtheorem*{claim*}{\underline{claim}}



\newtheorem*{solution*}{解答}

%箇条書きの様式
\renewcommand{\labelenumi}{(\arabic{enumi})}


%

\newcommand{\forany}{\rm{for} \ {}^{\forall}}
\newcommand{\foranyeps}{
\rm{for} \ {}^{\forall}\varepsilon >0}
\newcommand{\foranyk}{
\rm{for} \ {}^{\forall}k}


\newcommand{\any}{{}^{\forall}}
\newcommand{\suchthat}{\, \rm{s.t.} \, \it{}}




\newcommand{\veps}{\varepsilon}
\newcommand{\paren}[1]{\mleft( #1\mright )}
\newcommand{\cbra}[1]{\mleft\{#1\mright\}}
\newcommand{\sbra}[1]{\mleft\lbrack#1\mright\rbrack}
\newcommand{\tbra}[1]{\mleft\langle#1\mright\rangle}
\newcommand{\abs}[1]{\left|#1\right|}
\newcommand{\norm}[1]{\left\|#1\right\|}
\newcommand{\lopen}[1]{\mleft(#1\mright\rbrack}
\newcommand{\ropen}[1]{\mleft\lbrack #1 \mright)}



%
\newcommand{\Rn}{\mathbb{R}^n}
\newcommand{\Cn}{\mathbb{C}^n}

\newcommand{\Rm}{\mathbb{R}^m}
\newcommand{\Cm}{\mathbb{C}^m}


\newcommand{\projs}[2]{\it{p}_{#1,\ldots,#2}}
\newcommand{\projproj}[2]{\it{p}_{#1,#2}}

\newcommand{\proj}[1]{p_{#1}}

%可測空間
\newcommand{\stdProbSp}{\paren{\Omega, \mathcal{F}, P}}

%微分作用素
\newcommand{\ddt}{\frac{d}{dt}}
\newcommand{\ddx}{\frac{d}{dx}}
\newcommand{\ddy}{\frac{d}{dy}}

\newcommand{\delt}{\frac{\partial}{\partial t}}
\newcommand{\delx}{\frac{\partial}{\partial x}}

%ハイフン
\newcommand{\hyphen}{\text{-}}

%displaystyle
\newcommand{\dstyle}{\displaystyle}

%⇔, ⇒, \UTF{21D0}%
\newcommand{\LR}{\Leftrightarrow}
\newcommand{\naraba}{\Rightarrow}
\newcommand{\gyaku}{\Leftarrow}

%理由
\newcommand{\naze}[1]{\paren{\because {\mathop{ #1 }}}}

%
\newcommand{\sankaku}{\hfill $\triangle$}

%
\newcommand{\push}{_{\#}}

%手抜き
\newcommand{\textif}{\textrm{if}\,\,\,}
\newcommand{\Ric}{\textrm{Ric}}
\newcommand{\tr}{\textrm{tr}}
\newcommand{\vol}{\textrm{vol}}
\newcommand{\diam}{\textrm{diam}}
\newcommand{\supp}{\textrm{supp}}
\newcommand{\Med}{\textrm{Med}}
\newcommand{\Leb}{\textrm{Leb}}
\newcommand{\Const}{\textrm{Const}}
\newcommand{\Avg}{\textrm{Avg}}
\renewcommand{\;}{\, ; \,}
\renewcommand{\d}{\, {d}}


\title{局所等長写像と被覆}
\date{}


\author{}


\begin{document}

\maketitle

\maketitle


\section{}

\begin{prop}
$(M,g), (N,h)$ を同じ$n$次元の連結リーマン多様体とする. \\$M$ が完備で, 局所等長写像$f: (M,g) \rightarrow (N,h)$ が存在するとき, 次が成り立つ.\\
($1$)$N$ は完備である. \quad 
($2$)$f$ は全射である. \quad ($3$) $f$ は被覆写像である.
\end{prop}
\begin{pf*}
$(1)$局所等長写像は測地線を保存するので, 任意に$N$ の測地線$\gamma^N$をとる. $f^{-1}(\gamma^N_0)$ を始点, $df^{-1}_{\gamma^N_0} (\dot \gamma_0)$ を始方向とする$M$ の測地線を$\gamma^M$ で表すと, $f\circ \gamma^M = \gamma^N$ が$M$ の完備性より$\mathbb R$ 上で成り立つ. 従って, $N$ の測地線は$\mathbb R$ を定義域に含むので$N$ は完備である. \\
$(2)$適当に$2$点$q_1, q_2 \in N$ をとると, ある正規測地線$\gamma^N$で$\gamma^N_0 = q_1, \gamma^N_l = q_2$ を満たすものがとれる. ($1$)の証明と同様に, 対応する測地線を$\gamma^M$ で表すと, $f(\gamma^M_l) = \gamma^N_l = q_2$ となるので, $f$ は全射である. \\
($3$) 任意に$q \in N$ をとる. $r > 0$ を$q$ における単射半径より小さくとる. $\cbra{p_\alpha} = f^{-1}(q)$ とする. $f$ が測地線の長さを保つことに注意すると, $f^{-1} (B(q; r)) \subset \bigcup B(p_\alpha ; r)$ と$B(p_\alpha) \subset f^{-1} (B(q; r))$ が成り立つので, $\bigcup B(p_\alpha ; r) = f^{-1} (B(q; r))$ が成り立つ. また, $exp_{p_\alpha}$ の$B(p_\alpha ; r)$ への制限は微分同相となることから, $f = \exp_{p_\alpha}^{-1} \circ df_{p_\alpha}\circ \exp_q $ の$B(Op_\alpha ; r)$ への制限は微分同相である. また, $p^\prime \in B(p_\alpha ; r) \cap B(p_\beta ; r)$ がとれるとする(背理法). $p^\prime$ から$p_\alpha$ への測地線$\gamma^{M,\alpha}$と, $p^\prime$ から$p_\beta$ への測地線$\gamma^{M,\beta}$をそれぞれとって, $N$ へうつすと, ともに$f(p^\prime) \in N$ から$q \in N$ への測地線($\gamma^N$ とする)であるので, $N$ において互いに一致する. $df_{p^\prime}$ は同型写像であるので, $\gamma^{M,\alpha}, \gamma^{M,\beta}$ の始方向はともに$df_{p^\prime}^{-1}(\dot \gamma^N_0)$ であるので, $\gamma^{M,\alpha} = \gamma^{M,\beta}$ であるので, $p_\alpha = p_\beta$ となり$\alpha = \beta$ なので矛盾する.
$\alpha \neq \beta \naraba B(p_\alpha ; r) \cup B(p_\beta ; r) $ である. よって$f$ は被覆写像である. 
\qed
\end{pf*}














\end{document}