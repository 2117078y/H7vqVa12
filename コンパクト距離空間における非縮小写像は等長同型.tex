\documentclass[10pt, fleqn, label-section=none]{bxjsarticle}

%\usepackage[driver=dvipdfm,hmargin=25truemm,vmargin=25truemm]{geometry}

\setpagelayout{driver=dvipdfm,hmargin=25truemm,vmargin=20truemm}


\usepackage{amsmath}
\usepackage{amssymb}
\usepackage{amsfonts}
\usepackage{amsthm}
\usepackage{mathtools}
\usepackage{mleftright}

\usepackage{ascmac}




\usepackage{otf}

\theoremstyle{definition}
\newtheorem{dfn}{定義}[section]
\newtheorem{ex}[dfn]{例}
\newtheorem{lem}[dfn]{補題}
\newtheorem{prop}[dfn]{命題}
\newtheorem{thm}[dfn]{定理}
\newtheorem{setting}[dfn]{設定}
\newtheorem{notation}[dfn]{記号}
\newtheorem{cor}[dfn]{系}
\newtheorem*{pf*}{証明}
\newtheorem{problem}[dfn]{問題}
\newtheorem*{problem*}{問題}
\newtheorem{remark}[dfn]{注意}
\newtheorem*{claim*}{\underline{claim}}



\newtheorem*{solution*}{解答}

%箇条書きの様式
\renewcommand{\labelenumi}{(\arabic{enumi})}


%

\newcommand{\forany}{\rm{for} \ {}^{\forall}}
\newcommand{\foranyeps}{
\rm{for} \ {}^{\forall}\varepsilon >0}
\newcommand{\foranyk}{
\rm{for} \ {}^{\forall}k}


\newcommand{\any}{{}^{\forall}}
\newcommand{\suchthat}{\, \rm{s.t.} \, \it{}}




\newcommand{\veps}{\varepsilon}
\newcommand{\paren}[1]{\mleft( #1\mright )}
\newcommand{\cbra}[1]{\mleft\{#1\mright\}}
\newcommand{\sbra}[1]{\mleft\lbrack#1\mright\rbrack}
\newcommand{\tbra}[1]{\mleft\langle#1\mright\rangle}
\newcommand{\abs}[1]{\left|#1\right|}
\newcommand{\norm}[1]{\left\|#1\right\|}
\newcommand{\lopen}[1]{\mleft(#1\mright\rbrack}
\newcommand{\ropen}[1]{\mleft\lbrack #1 \mright)}



%
\newcommand{\Rn}{\mathbb{R}^n}
\newcommand{\Cn}{\mathbb{C}^n}

\newcommand{\Rm}{\mathbb{R}^m}
\newcommand{\Cm}{\mathbb{C}^m}


\newcommand{\projs}[2]{\it{p}_{#1,\ldots,#2}}
\newcommand{\projproj}[2]{\it{p}_{#1,#2}}

\newcommand{\proj}[1]{p_{#1}}

%可測空間
\newcommand{\stdProbSp}{\paren{\Omega, \mathcal{F}, P}}

%微分作用素
\newcommand{\ddt}{\frac{d}{dt}}
\newcommand{\ddx}{\frac{d}{dx}}
\newcommand{\ddy}{\frac{d}{dy}}

\newcommand{\delt}{\frac{\partial}{\partial t}}
\newcommand{\delx}{\frac{\partial}{\partial x}}

%ハイフン
\newcommand{\hyphen}{\text{-}}

%displaystyle
\newcommand{\dstyle}{\displaystyle}

%⇔, ⇒, \UTF{21D0}%
\newcommand{\LR}{\Leftrightarrow}
\newcommand{\naraba}{\Rightarrow}
\newcommand{\gyaku}{\Leftarrow}

%理由
\newcommand{\naze}[1]{\paren{\because {\mathop{ #1 }}}}

%
\newcommand{\sankaku}{\hfill $\triangle$}

%
\newcommand{\push}{_{\#}}

%手抜き
\newcommand{\textif}{\textrm{if}\,\,\,}
\newcommand{\Ric}{\textrm{Ric}}
\newcommand{\tr}{\textrm{tr}}
\newcommand{\vol}{\textrm{vol}}
\newcommand{\diam}{\textrm{diam}}
\newcommand{\supp}{\textrm{supp}}
\newcommand{\Med}{\textrm{Med}}
\newcommand{\Leb}{\textrm{Leb}}
\newcommand{\Const}{\textrm{Const}}
\newcommand{\Avg}{\textrm{Avg}}
\newcommand{\id}{\textrm{id}}
\newcommand{\Ker}{\textrm{Ker}}
\newcommand{\im}{\textrm{Im}}
\newcommand{\dil}{\textrm{dil}}
\newcommand{\Ch}{\textrm{Ch}}
\newcommand{\Lip}{\textrm{Lip}}
\newcommand{\Ent}{\textrm{Ent}}
\newcommand{\grad}{\textrm{grad}}
\newcommand{\dom}{\textrm{dom}}
\newcommand{\diag}{\textrm{diag}}

\renewcommand{\;}{\, ; \,}
\renewcommand{\d}{\, {d}}

\newcommand{\gyouretsu}[1]{\begin{pmatrix} #1 \end{pmatrix} }

\renewcommand{\div}{\textrm{div}}


%%図式

\usepackage[dvipdfm,all]{xy}


\newenvironment{claim}[1]{\par\noindent\underline{step:}\space#1}{}
\newenvironment{claimproof}[1]{\par\noindent{($\because$)}\space#1}{\hfill $\blacktriangle $}


\newcommand{\pprime}{{\prime \prime}}

%%マグニチュード


\newcommand{\Mag}{\textrm{Mag}}

\usepackage{mathrsfs}


%%6.13
\def\Xint#1{\mathchoice
{\XXint\displaystyle\textstyle{#1}}%
{\XXint\textstyle\scriptstyle{#1}}%
{\XXint\scriptstyle\scriptscriptstyle{#1}}%
{\XXint\scriptscriptstyle\scriptscriptstyle{#1}}%
\!\int}
\def\XXint#1#2#3{{\setbox0=\hbox{$#1{#2#3}{\int}$ }
\vcenter{\hbox{$#2#3$ }}\kern-.6\wd0}}
\def\ddashint{\Xint=}
\def\dashint{\Xint-}



\title{コンパクト距離空間における非縮小写像は等長同型}
\date{}


\author{}


\begin{document}


\maketitle

\section{}

\begin{dfn}$(X, d)$ を距離空間とする. $f: X \rightarrow X$ は
\begin{align*} d(x, y) \leq d(f(x) , f(y)) \quad (x, y \in X) \end{align*}
を満たす時に, 非縮小写像という. 
\end{dfn}

\begin{prop}非縮小写像は単射である. 

\end{prop}
\begin{pf*}
明らかである. 
\qed
\end{pf*}

\begin{prop}$(X, d)$ をコンパクト距離空間とする. $f: X \rightarrow X$ を非縮小写像とする. このとき, $f$ は等長写像である. 

\end{prop}
\begin{pf*}$x \in X$ を適当にとる. $f^n (x)$ を考えると, $X$ はコンパクトなので適当な収束部分列$f^{n(k)}(x)$がとれる. 適当にさらに部分列をとって, $n(2) - n(1) < n(3) - n(2) <  \ldots$ となるようにとる.  
\begin{align*} f^{n((k+1)) - n(k)}(x) \end{align*}
を考える
\begin{align*}   d(f^{n((k+1)) - n(k)}(x), x) \leq d(f^{n((k+1))}(x), f^{n(k)}(x) )   \end{align*}
が成り立つので, $f^{n((k+1)) - n(k)}(x) $ は$x$ に収束する. 
適当な$x, y \in X$ に対して同様にして$f^{n((k+1)) - n(k)}(x) , f^{n((k+1)) - n(k)}(y) $ をとると, 
\begin{align*} d(f(x), f(y)) \leq d(f^{n((k+1)) - n(k)}(x) , f^{n((k+1)) - n(k)}(y) ) \end{align*}
が成り立つので, 極限をとると, 
\begin{align*} d(f(x), f(y)) \leq d(x, y) \end{align*}
が成り立つ. 
\qed
\end{pf*}


\begin{prop}$(X, d)$ をコンパクト距離空間とする. $f: X \rightarrow X$ を非縮小写像とする. このとき, $f(X)$ はコンパクトである. 
\end{prop}
\begin{pf*} $f$ が等長写像であるから. 
\qed
\end{pf*}


\begin{prop}$(X, d)$ をコンパクト距離空間とする. $f: X \rightarrow X$ を非縮小写像とする. このとき, $f$ は全射である. 

\end{prop}
\begin{pf*}全射でないと仮定する. $X \setminus f(X)$ が空でないので, $x_0 \in X \setminus f(X)$ がとれる. 
\begin{align*} x_1 \coloneqq f(x_0), \quad x_2 \coloneqq f(x_1), \ldots \end{align*}
と定める. 
\begin{claim}
\begin{align*}x_n \notin f^{n+1}(X) \end{align*}
\end{claim}
\begin{claimproof}
帰納法によりを示す. $x_0 \notin f(X)$ は成り立つ. 
\begin{align*} x_0 \notin f(X), x_1 \notin f^2(X), \ldots, x_{n -1} \notin f^{n}(X) \end{align*}
であるとき, $x_n \in f^{n+1} (X)$ とすると, 適当な点$y$ で$f^{n+1}(y) = x_n$ を満たすものがとれる. 
\begin{align*} f^n (x_0) = x_n = f^{n+1}(y) \end{align*}
であるので, $f$ が単射であることから, 
\begin{align*} f^{n-1} (x_0) = x_{n-1} = f^n (y) \end{align*}
が成り立つ. 従って, $x_{n-1} \in f^n (X)$ となるので矛盾する. 
\end{claimproof}
従って, $x_n \in f^n(X) \setminus f^{n+1}(X)$ である. 
\qed
\end{pf*}



\end{document}