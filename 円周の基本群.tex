\documentclass[10pt, fleqn, label-section=none]{bxjsarticle}

%\usepackage[driver=dvipdfm,hmargin=25truemm,vmargin=25truemm]{geometry}

\setpagelayout{driver=dvipdfm,hmargin=25truemm,vmargin=20truemm}


\usepackage{amsmath}
\usepackage{amssymb}
\usepackage{amsfonts}
\usepackage{amsthm}
\usepackage{mathtools}
\usepackage{mleftright}

\usepackage{ascmac}




\usepackage{otf}

\theoremstyle{definition}
\newtheorem{dfn}{定義}[section]
\newtheorem{ex}[dfn]{例}
\newtheorem{lem}[dfn]{補題}
\newtheorem{prop}[dfn]{命題}
\newtheorem{thm}[dfn]{定理}
\newtheorem{setting}[dfn]{設定}
\newtheorem{cor}[dfn]{系}
\newtheorem*{pf*}{証明}
\newtheorem{problem}[dfn]{問題}
\newtheorem*{problem*}{問題}
\newtheorem{remark}[dfn]{注意}
\newtheorem*{claim*}{\underline{claim}}



\newtheorem*{solution*}{解答}

%箇条書きの様式
\renewcommand{\labelenumi}{(\arabic{enumi})}


%

\newcommand{\forany}{\rm{for} \ {}^{\forall}}
\newcommand{\foranyeps}{
\rm{for} \ {}^{\forall}\varepsilon >0}
\newcommand{\foranyk}{
\rm{for} \ {}^{\forall}k}


\newcommand{\any}{{}^{\forall}}
\newcommand{\suchthat}{\, \rm{s.t.} \, \it{}}




\newcommand{\veps}{\varepsilon}
\newcommand{\paren}[1]{\mleft( #1\mright )}
\newcommand{\cbra}[1]{\mleft\{#1\mright\}}
\newcommand{\sbra}[1]{\mleft\lbrack#1\mright\rbrack}
\newcommand{\tbra}[1]{\mleft\langle#1\mright\rangle}
\newcommand{\abs}[1]{\left|#1\right|}
\newcommand{\norm}[1]{\left\|#1\right\|}
\newcommand{\lopen}[1]{\mleft(#1\mright\rbrack}
\newcommand{\ropen}[1]{\mleft\lbrack #1 \mright)}



%
\newcommand{\Rn}{\mathbb{R}^n}
\newcommand{\Cn}{\mathbb{C}^n}

\newcommand{\Rm}{\mathbb{R}^m}
\newcommand{\Cm}{\mathbb{C}^m}


\newcommand{\projs}[2]{\it{p}_{#1,\ldots,#2}}
\newcommand{\projproj}[2]{\it{p}_{#1,#2}}

\newcommand{\proj}[1]{p_{#1}}

%可測空間
\newcommand{\stdProbSp}{\paren{\Omega, \mathcal{F}, P}}

%微分作用素
\newcommand{\ddt}{\frac{d}{dt}}
\newcommand{\ddx}{\frac{d}{dx}}
\newcommand{\ddy}{\frac{d}{dy}}

\newcommand{\delt}{\frac{\partial}{\partial t}}
\newcommand{\delx}{\frac{\partial}{\partial x}}

%ハイフン
\newcommand{\hyphen}{\text{-}}

%displaystyle
\newcommand{\dstyle}{\displaystyle}

%⇔, ⇒, \UTF{21D0}%
\newcommand{\LR}{\Leftrightarrow}
\newcommand{\naraba}{\Rightarrow}
\newcommand{\gyaku}{\Leftarrow}

%理由
\newcommand{\naze}[1]{\paren{\because {\mathop{ #1 }}}}

%
\newcommand{\sankaku}{\hfill $\triangle$}

%
\newcommand{\push}{_{\#}}

%手抜き
\newcommand{\textif}{\textrm{if}\,\,\,}
\newcommand{\Ric}{\textrm{Ric}}
\newcommand{\tr}{\textrm{tr}}
\newcommand{\vol}{\textrm{vol}}
\newcommand{\diam}{\textrm{diam}}
\newcommand{\supp}{\textrm{supp}}
\newcommand{\Med}{\textrm{Med}}
\newcommand{\Leb}{\textrm{Leb}}
\newcommand{\Const}{\textrm{Const}}
\newcommand{\Avg}{\textrm{Avg}}
\newcommand{\id}{\textrm{id}}
\newcommand{\Ker}{\textrm{Ker}}
\newcommand{\im}{\textrm{Im}}
\newcommand{\dil}{\textrm{dil}}
\newcommand{\Ch}{\textrm{Ch}}
\newcommand{\Lip}{\textrm{Lip}}
\newcommand{\Ent}{\textrm{Ent}}
\newcommand{\grad}{\textrm{grad}}
\newcommand{\dom}{\textrm{dom}}

\renewcommand{\;}{\, ; \,}
\renewcommand{\d}{\, {d}}

\newcommand{\gyouretsu}[1]{\begin{pmatrix} #1 \end{pmatrix} }


%%図式

\usepackage[dvipdfm,all]{xy}


\newenvironment{claim}[1]{\par\noindent\underline{step:}\space#1}{}
\newenvironment{claimproof}[1]{\par\noindent{($\because$)}\space#1}{\hfill $\blacktriangle $}


\newcommand{\pprime}{{\prime \prime}}





%%


\title{$S^1$の基本群}
\date{}


\author{}


\begin{document}


\maketitle

\section{}



\begin{dfn}(道の連結). $f, g:[0,1] \rightarrow X$ に対して
\begin{align*} f \natural g \coloneqq \begin{cases} f(2t) & 0 \leq t \leq \frac{1}{2} \\ g(2t -1) & \frac{1}{2} \leq t \leq 1 \end{cases} \end{align*}
\end{dfn}



\section{}

\begin{setting}
\begin{align*}
& c_n: [0,1] \rightarrow S^1; t \mapsto (\cos 2 \pi n t, \sin 2 \pi n t) \\ 
& \pi: \mathbb R \rightarrow S^1 ; s \mapsto (\cos 2 \pi s, \sin 2 \pi s) \\
& \tilde c_n : [0, 1] \rightarrow \mathbb R; t \mapsto nt
\end{align*}
\end{setting}


\begin{prop}
$\pi_1 (S^1)$ は$c(t) \coloneqq (\cos 2 \pi t, \sin 2 \pi s)$ のホモトピー類が生成する無限巡回群と同型である. 
\end{prop}
\begin{pf*}
\begin{claim}
\begin{align*} [c_1]^n = [c_n]\end{align*}
\end{claim}
\begin{claimproof}
明らかである. 
\end{claimproof}

\begin{claim}
任意の$l \in \pi(S^1, (1, 0) )$ に対して, $n \in \mathbb Z$ で$l \simeq c_n$ を満たすものが存在する. 従って
\begin{align*} [l] = [c_n] \end{align*}
が成り立つ. 
\end{claim}
\begin{claimproof}
$l: [0, 1] \rightarrow S^1$ を$(1, 0)$ を基点とするループとする. $\mathbb R$ へのリフト$\tilde l : [0, 1] \rightarrow \mathbb R$ がとれる. ループ$l$ は時刻$1$ で基点に戻ってくるので, その持ち上げについて$\tilde l_1 \in \mathbb Z$ である. $\tilde c_{\tilde l_1} $ を考え, 
\begin{align*} (1 - t)\tilde l + t c_{\tilde l_1} \end{align*}
を考えると, これは$\tilde l$ から$\tilde c_{\tilde l} $ へのホモトピーである.
\end{claimproof}

\begin{claim}($\tilde l $ は適当な$c_n$ とホモトピックであるが, 別の$c_m$ とホモトピックであるかもしれない. )
\begin{align*} f  \simeq c_n, f \simeq c_m \naraba n = m. \end{align*}
\end{claim}
\begin{claimproof}
$c_n$ から$c_m$ へのホモトピーを$f_t$ とする. $\mathbb R$ へのリフト$\tilde f : [0, 1] \rightarrow \mathbb R$ がとれる. $\tilde f_t (1)$ は$t$ によらず同じである. $\tilde f_0 (1) = n, \tilde f_1 (1) = m$ であるので, $n = m$ が成り立つ. 
\end{claimproof}

\qed
\end{pf*}





\end{document}