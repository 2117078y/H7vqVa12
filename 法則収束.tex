\documentclass[10pt, fleqn, label-section=none]{bxjsarticle}

%\usepackage[driver=dvipdfm,hmargin=25truemm,vmargin=25truemm]{geometry}

\setpagelayout{driver=dvipdfm,hmargin=25truemm,vmargin=20truemm}


\usepackage{amsmath}
\usepackage{amssymb}
\usepackage{amsfonts}
\usepackage{amsthm}
\usepackage{mathtools}
\usepackage{mleftright}

\usepackage{ascmac}




\usepackage{otf}

\theoremstyle{definition}
\newtheorem{dfn}{定義}[section]
\newtheorem{ex}[dfn]{例}
\newtheorem{lem}[dfn]{補題}
\newtheorem{prop}[dfn]{命題}
\newtheorem{thm}[dfn]{定理}
\newtheorem{setting}[dfn]{設定}
\newtheorem{cor}[dfn]{系}
\newtheorem*{pf*}{証明}
\newtheorem{problem}[dfn]{問題}
\newtheorem*{problem*}{問題}
\newtheorem{remark}[dfn]{注意}
\newtheorem*{claim*}{\underline{claim}}



\newtheorem*{solution*}{解答}

%箇条書きの様式
\renewcommand{\labelenumi}{(\arabic{enumi})}


%

\newcommand{\forany}{\rm{for} \ {}^{\forall}}
\newcommand{\foranyeps}{
\rm{for} \ {}^{\forall}\varepsilon >0}
\newcommand{\foranyk}{
\rm{for} \ {}^{\forall}k}


\newcommand{\any}{{}^{\forall}}
\newcommand{\suchthat}{\, \rm{s.t.} \, \it{}}




\newcommand{\veps}{\varepsilon}
\newcommand{\paren}[1]{\mleft( #1\mright )}
\newcommand{\cbra}[1]{\mleft\{#1\mright\}}
\newcommand{\sbra}[1]{\mleft\lbrack#1\mright\rbrack}
\newcommand{\tbra}[1]{\mleft\langle#1\mright\rangle}
\newcommand{\abs}[1]{\left|#1\right|}
\newcommand{\norm}[1]{\left\|#1\right\|}
\newcommand{\lopen}[1]{\mleft(#1\mright\rbrack}
\newcommand{\ropen}[1]{\mleft\lbrack #1 \mright)}



%
\newcommand{\Rn}{\mathbb{R}^n}
\newcommand{\Cn}{\mathbb{C}^n}

\newcommand{\Rm}{\mathbb{R}^m}
\newcommand{\Cm}{\mathbb{C}^m}


\newcommand{\projs}[2]{\it{p}_{#1,\ldots,#2}}
\newcommand{\projproj}[2]{\it{p}_{#1,#2}}

\newcommand{\proj}[1]{p_{#1}}

%可測空間
\newcommand{\stdProbSp}{\paren{\Omega, \mathcal{F}, P}}

%微分作用素
\newcommand{\ddt}{\frac{d}{dt}}
\newcommand{\ddx}{\frac{d}{dx}}
\newcommand{\ddy}{\frac{d}{dy}}

\newcommand{\delt}{\frac{\partial}{\partial t}}
\newcommand{\delx}{\frac{\partial}{\partial x}}

%ハイフン
\newcommand{\hyphen}{\text{-}}

%displaystyle
\newcommand{\dstyle}{\displaystyle}

%⇔, ⇒, \UTF{21D0}%
\newcommand{\LR}{\Leftrightarrow}
\newcommand{\naraba}{\Rightarrow}
\newcommand{\gyaku}{\Leftarrow}

%理由
\newcommand{\naze}[1]{\paren{\because {\mathop{ #1 }}}}

%
\newcommand{\sankaku}{\hfill $\triangle$}

%
\newcommand{\push}{_{\#}}

%手抜き
\newcommand{\textif}{\textrm{if}\,\,\,}
\newcommand{\Ric}{\textrm{Ric}}
\newcommand{\tr}{\textrm{tr}}
\newcommand{\vol}{\textrm{vol}}
\newcommand{\diam}{\textrm{diam}}
\newcommand{\supp}{\textrm{supp}}
\newcommand{\Med}{\textrm{Med}}
\newcommand{\Leb}{\textrm{Leb}}
\newcommand{\Const}{\textrm{Const}}
\newcommand{\Avg}{\textrm{Avg}}
\newcommand{\id}{\textrm{id}}
\newcommand{\Ker}{\textrm{Ker}}
\newcommand{\im}{\textrm{Im}}




\renewcommand{\;}{\, ; \,}
\renewcommand{\d}{\, {d}}

\newcommand{\gyouretsu}[1]{\begin{pmatrix} #1 \end{pmatrix} }

%%図式

\usepackage[dvipdfm,all]{xy}


\newenvironment{claim}[1]{\par\noindent\underline{step:}\space#1}{}
\newenvironment{claimproof}[1]{\par\noindent{($\because$)}\space#1}{\hfill $\blacktriangle $}


\newcommand{\pprime}{{\prime \prime}}





%%


\title{法則収束}
\date{}


\author{}


\begin{document}


\maketitle



\section{}


\begin{setting}
$X_\# \mu = \mu \circ X^{-1}$ という表記を用いる. 
\end{setting}

\begin{dfn}
\, 
\begin{enumerate}
\item $X$ を確率空間$\stdProbSp$ 上の実数値確率変数とするとき,  確率測度$X_{\#} P $ を$X$ の分布(あるいは法則)という.
\item $\mu \in \mathcal{P}\paren{\Rn}$( $ \mathbb R^n$ のボレル確率測度 ) に対して, 
\begin{align*} F\paren{x} \coloneqq \mu \paren {\prod_{i=1}^{n}\lopen{\infty , x_i}}\qquad \paren{x = (x_1 ,\ldots , x_n) \in \Rn}\end{align*}
により定まる関数$F: \mathbb R^n \rightarrow \mathbb R$を$\mu$ の分布関数という.
\end{enumerate}
\end{dfn}

\begin{dfn}
$\mu_k , \mu \in \mathcal{P}\paren{\Rn}$ の分布関数をそれぞれ$F_k, F$ とする. \\
$\mu_k$ が$\mu$ に分布収束(法則収束)する. $:\LR$ $\lim_k F_k\paren{x} = F\paren{x} \quad \paren{\forany x : F\mathop{の連続点}}$ が成り立つ.
\end{dfn}

\begin{prop} 
$\mu_k , \mu \in \mathcal{P}\paren{\Rn}$ の分布関数をそれぞれ$F_k, F$ とする. \\
$\mu_k$ が$\mu$ に弱収束する. $\LR$ $\mu_k$ が $\mu$ に分布収束(法則収束)する.

\end{prop}
\begin{pf*}
\, \\
$\paren{\naraba} \, f_i\paren{x} \coloneqq \sup \cbra{f\paren{y} - i \cdot d\paren{x, y} \mid y \in \Rn}, f_j\paren{x} \coloneqq \inf \cbra{f\paren{y} + j \cdot d\paren{x, y} \mid y \in \Rn}$ とすると, \\$\cbra{f_i}, \cbra{f_j}$ はそれぞれ
$1_{\lopen{- \infty, x}}, 1_{\paren{\infty, x}}$ に各点収束する有界連続関数の減少列と増大列であり, \\
$\int f_j d\mu_k \leq F_k \paren{x} \leq \int f_j d\mu_k$ が成り立ち, kについて極限をとることで, \\
$\int f_j d\mu \leq \liminf_k F_k \paren{x} \leq \limsup_k F_k \paren{x} \leq \int f_j d\mu$ が成り立つ. i,j について極限をとることで, \\
$\mu \paren{-\infty, x} \leq \liminf_k F_k \paren{x} \leq \limsup_k F_k \paren{x} \leq \mu \lopen{-\infty, x} $ が成り立ち, xがFの連続点である時にはこれらの不等号は実際には等号であることから結論が従う. \\
$\paren{\gyaku} \, $
$\abs{\displaystyle  \int_{\Rn} f d\mu_n - \displaystyle  \int_{\Rn} f d\mu} 
\leq \abs{\displaystyle\int_{ \lopen{-\infty, a} \cup \paren{b, \infty}} \hspace{-20pt} f d\mu_n} + \abs{\displaystyle \int_{\lopen{-\infty, a} \cup \paren{b, \infty}} \hspace{-20pt} f d\mu} + \displaystyle \sum_{k=0}^{m-1} \abs{ \int_{ \lopen{a_k , a_{k+1} }} \hspace{-20pt} f d\mu_n -  \int_{ \lopen{a_k , a_{k+1}} } \hspace{-20pt} f d\mu } \\
\leq \abs{\displaystyle\int_{ \lopen{-\infty, a} \cup \paren{b, \infty}} \hspace{-10pt} f d\mu_n} + \abs{\displaystyle \int_{\lopen{-\infty, a} \cup \paren{b, \infty}} \hspace{-10pt} f d\mu} \\ \quad \quad \quad + \displaystyle \sum_{k=0}^{m-1} \abs{ \int_{ \lopen{a_k , a_{k+1} }} \hspace{-30pt} f \paren{x} - f\paren{a_k} + f\paren{a_k } d\mu_n -  \int_{ \lopen{a_k , a_{k+1}} } \hspace{-30pt} f \paren{x} - f\paren{a_k} + f\paren{a_k } d\mu } \\
$ より, $a, b, a_1, \ldots , a_m$ をかなりうまくとればよい.

\qed
\end{pf*}






\end{document}
