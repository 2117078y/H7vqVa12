\documentclass[10pt, fleqn, label-section=none]{bxjsarticle}

%\usepackage[driver=dvipdfm,hmargin=25truemm,vmargin=25truemm]{geometry}

\setpagelayout{driver=dvipdfm,hmargin=25truemm,vmargin=20truemm}


\usepackage{amsmath}
\usepackage{amssymb}
\usepackage{amsfonts}
\usepackage{amsthm}
\usepackage{mathtools}
\usepackage{mleftright}

\usepackage{ascmac}




\usepackage{otf}

\theoremstyle{definition}
\newtheorem{dfn}{定義}[section]
\newtheorem{ex}[dfn]{例}
\newtheorem{lem}[dfn]{補題}
\newtheorem{prop}[dfn]{命題}
\newtheorem{thm}[dfn]{定理}
\newtheorem{setting}[dfn]{設定}
\newtheorem{notation}[dfn]{記号}
\newtheorem{cor}[dfn]{系}
\newtheorem*{pf*}{証明}
\newtheorem{problem}[dfn]{問題}
\newtheorem*{problem*}{問題}
\newtheorem{remark}[dfn]{注意}
\newtheorem*{claim*}{\underline{claim}}



\newtheorem*{solution*}{解答}

%箇条書きの様式
\renewcommand{\labelenumi}{(\arabic{enumi})}


%

\newcommand{\forany}{\rm{for} \ {}^{\forall}}
\newcommand{\foranyeps}{
\rm{for} \ {}^{\forall}\varepsilon >0}
\newcommand{\foranyk}{
\rm{for} \ {}^{\forall}k}


\newcommand{\any}{{}^{\forall}}
\newcommand{\suchthat}{\, \rm{s.t.} \, \it{}}




\newcommand{\veps}{\varepsilon}
\newcommand{\paren}[1]{\mleft( #1\mright )}
\newcommand{\cbra}[1]{\mleft\{#1\mright\}}
\newcommand{\sbra}[1]{\mleft\lbrack#1\mright\rbrack}
\newcommand{\tbra}[1]{\mleft\langle#1\mright\rangle}
\newcommand{\abs}[1]{\left|#1\right|}
\newcommand{\norm}[1]{\left\|#1\right\|}
\newcommand{\lopen}[1]{\mleft(#1\mright\rbrack}
\newcommand{\ropen}[1]{\mleft\lbrack #1 \mright)}



%
\newcommand{\Rn}{\mathbb{R}^n}
\newcommand{\Cn}{\mathbb{C}^n}

\newcommand{\Rm}{\mathbb{R}^m}
\newcommand{\Cm}{\mathbb{C}^m}


\newcommand{\projs}[2]{\it{p}_{#1,\ldots,#2}}
\newcommand{\projproj}[2]{\it{p}_{#1,#2}}

\newcommand{\proj}[1]{p_{#1}}

%可測空間
\newcommand{\stdProbSp}{\paren{\Omega, \mathcal{F}, P}}

%微分作用素
\newcommand{\ddt}{\frac{d}{dt}}
\newcommand{\ddx}{\frac{d}{dx}}
\newcommand{\ddy}{\frac{d}{dy}}

\newcommand{\delt}{\frac{\partial}{\partial t}}
\newcommand{\delx}{\frac{\partial}{\partial x}}

%ハイフン
\newcommand{\hyphen}{\text{-}}

%displaystyle
\newcommand{\dstyle}{\displaystyle}

%⇔, ⇒, \UTF{21D0}%
\newcommand{\LR}{\Leftrightarrow}
\newcommand{\naraba}{\Rightarrow}
\newcommand{\gyaku}{\Leftarrow}

%理由
\newcommand{\naze}[1]{\paren{\because {\mathop{ #1 }}}}

%
\newcommand{\sankaku}{\hfill $\triangle$}

%
\newcommand{\push}{_{\#}}

%手抜き
\newcommand{\textif}{\textrm{if}\,\,\,}
\newcommand{\Ric}{\textrm{Ric}}
\newcommand{\tr}{\textrm{tr}}
\newcommand{\vol}{\textrm{vol}}
\newcommand{\diam}{\textrm{diam}}
\newcommand{\supp}{\textrm{supp}}
\newcommand{\Med}{\textrm{Med}}
\newcommand{\Leb}{\textrm{Leb}}
\newcommand{\Const}{\textrm{Const}}
\newcommand{\Avg}{\textrm{Avg}}
\newcommand{\id}{\textrm{id}}
\newcommand{\Ker}{\textrm{Ker}}
\newcommand{\im}{\textrm{Im}}
\newcommand{\dil}{\textrm{dil}}
\newcommand{\Ch}{\textrm{Ch}}
\newcommand{\Lip}{\textrm{Lip}}
\newcommand{\Ent}{\textrm{Ent}}
\newcommand{\grad}{\textrm{grad}}
\newcommand{\dom}{\textrm{dom}}
\newcommand{\diag}{\textrm{diag}}

\renewcommand{\;}{\, ; \,}
\renewcommand{\d}{\, {d}}

\newcommand{\gyouretsu}[1]{\begin{pmatrix} #1 \end{pmatrix} }

\renewcommand{\div}{\textrm{div}}


%%図式

\usepackage[dvipdfm,all]{xy}


\newenvironment{claim}[1]{\par\noindent\underline{step:}\space#1}{}
\newenvironment{claimproof}[1]{\par\noindent{($\because$)}\space#1}{\hfill $\blacktriangle $}


\newcommand{\pprime}{{\prime \prime}}

%%マグニチュード


\newcommand{\Mag}{\textrm{Mag}}

\usepackage{mathrsfs}


%%6.13
\def\Xint#1{\mathchoice
{\XXint\displaystyle\textstyle{#1}}%
{\XXint\textstyle\scriptstyle{#1}}%
{\XXint\scriptstyle\scriptscriptstyle{#1}}%
{\XXint\scriptscriptstyle\scriptscriptstyle{#1}}%
\!\int}
\def\XXint#1#2#3{{\setbox0=\hbox{$#1{#2#3}{\int}$ }
\vcenter{\hbox{$#2#3$ }}\kern-.6\wd0}}
\def\ddashint{\Xint=}
\def\dashint{\Xint-}



\title{一様非凸性とsnowflake}
\date{}


\author{}


\begin{document}


\maketitle

\section{}


\begin{dfn}(一様非凸). $0 < \delta < \frac{1}{2}$ とする. $(X, d)$ を距離空間とする. 任意の2点$x, y \in X$ に対して, $\lambda(x, y) \in (0, 1)$ で 
\begin{align*} B(x;  (\lambda + \delta) d(x, y)) \cap B(y; (1 - \lambda + \delta) d(x, y)) = \varnothing \end{align*}
を満たすものが存在する時に, $(X, d)$ は$\delta $-一様非凸(UNC)であるという. 
\end{dfn}

\begin{remark}
$1 - \lambda , \lambda \leq \delta$ の時は, $B(x;  (\lambda + \delta) d(x, y)) \cap B(y; (1 - \lambda + \delta) d(x, y)) \neq \varnothing$ であるので, $\lambda \in (\delta, 1- \delta)$ に条件を置き換えてもよい. 
\end{remark}

\begin{prop}$0 < \delta < \frac{1}{2}$ とすると, \\
(1)$4 \delta ^2 - \delta \leq \delta $ が成り立つ. \\
(2)$0 \leq 4 \delta (\frac{1}{2} - \delta) \leq \frac{1}{4}$が成り立つ.
\end{prop}
\begin{pf*}
計算するだけ. 
\qed
\end{pf*}


\begin{prop}$(X, d)$ を距離空間とする. $(X, d)$ が$\delta $UNC であるならば, 
 $(X, d)$ は $p$-snowflakeである. ただし, 
 \begin{align*} p \coloneqq \frac{\log 2}{\log 2 - (1 + 4 \delta ^2)} \end{align*}
 である. 
\end{prop}
\begin{pf*}
\begin{align*} D \coloneqq 4d(\frac{1}{2} - \delta) , \quad  c \coloneqq D ^p \end{align*}
と定める. $N + 1$ 個の点$x_0, \ldots, x_N$ に対して
\begin{align*} c d^p(x_0, x_N) \leq \sum d^p(x_{i}, x_{i + 1}) \end{align*}
が成り立つことを帰納法により示す. $N = 1$ の時, $c \leq 1$ より明らかに成り立つ. $1, \ldots, N$ 個の点に対して成り立つとする. 任意に$x, y \in X$ をとり, $N + 1$ 個の点列$x = x_0, x_1, \ldots , x_N = y$ をとる. $\delta $-UNC であることから, 適当な$\lambda \in (\delta , 1- \delta)$ で, 
\begin{align*} B(x;  (\lambda + \delta) d(x, y)) \cap B(y; (1 - \lambda + \delta) d(x, y)) = \varnothing \end{align*}
を満たすものがとれる. $B(x; (\lambda + \delta) d(x, y) ) \setminus B(x; (\lambda + 4\delta^2 - \delta) d(x, y) )$ に$\cbra{x = x_0, x_1, \ldots, x_N = y}$ の点が含まれるかどうかに応じて, 
\\(1)任意の$x_k \in \cbra{x = x_0, x_1, \ldots, x_N = y}$ に対して
\begin{align*} d(x, x_k) \leq (\lambda + 4 \delta ^2 - \delta) d(x, y) \end{align*} または
\begin{align*}  (\lambda + \delta) d(x, y) \leq d(x, x_k) \end{align*}
が成り立つ. 
あるいは
\\(2)ある$x_k \in \cbra{x = x_0, x_1, \ldots, x_N = y}$ に対して, 
\begin{align*} (\lambda + 4 \delta ^2 - \delta )d(x, y) \leq d(x, x_k), \quad d(x_k, y) \geq (1 - \lambda + \delta ) d(x, y)\end{align*}
が成り立つ. \\
(1)の場合, $x_{k}, x_{k + 1} \in \cbra{x = x_0, x_1, \ldots, x_N = y}$ で
\begin{align*} d(x, x_{k}) \leq (\lambda + 4 \delta ^2 - \delta) d(x, y), \quad (\lambda + \delta ) d(x, y) \leq d(x, x_{k+ 1})\end{align*}
を満たすものがとれる. 
\begin{align*} D d(x, y) \leq  d(x, x_{k + 1}) - d(x, x_k) \leq d(x_k, x_{k + 1}) \end{align*}
が成り立つので, 
\begin{align*} D^p d^p(x, y) \leq  d^p(x_k, x_{k + 1}) \leq \sum d^p(x_o, x_{i + 1})  \end{align*}
が成り立つ. \\
(2)の場合, 帰納法の仮定より, 
\begin{align*} D^p d^p(x, x_k) \leq \sum_[i = 0]^{k-1} d^p(x_i, x_{i + 1})   , \quad D^p d^p(x_k, y) \leq \sum_[i = k]^{N-1} d^p(x_i, x_{i + 1}) \end{align*}
が成り立つので, 
\begin{align*} D^p (d^p(x, x_k) + d^p(x_k , y)) \leq \sum d^p (x_i, x_{i + 1}) \end{align*}
が成り立つ. 
\begin{align*} d(x, x_{k}) \leq (\lambda + 4 \delta ^2 - \delta) d(x, y), \quad (\lambda + \delta ) d(x, y) \leq d(x, x_{k+ 1})\end{align*}
より, 
\begin{align*} (1 + 4 \delta ^2 ) d(x, y) \leq d(x, x_k) + d(x_k ,y)  \end{align*}
であるので, $p$ に対して, 
\begin{align*}d^p(x, y) \leq d^p(x, x_k) + d^p (x_k, y) \end{align*}
が成り立つ. 従って, 
\begin{align*} D^p d^p(x, y) \leq D^p (d^p(x, x_k) + d^p(x_k , y)) \leq \sum d^p (x_i, x_{i + 1}) \end{align*}
が成り立つ. 以上により主張が従う. 
\qed
\end{pf*}









\end{document}