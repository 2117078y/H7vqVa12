\documentclass[10pt, fleqn, label-section=none]{bxjsarticle}

%\usepackage[driver=dvipdfm,hmargin=25truemm,vmargin=25truemm]{geometry}

\setpagelayout{driver=dvipdfm,hmargin=25truemm,vmargin=20truemm}


\usepackage{amsmath}
\usepackage{amssymb}
\usepackage{amsfonts}
\usepackage{amsthm}
\usepackage{mathtools}
\usepackage{mleftright}

\usepackage{ascmac}




\usepackage{otf}

\theoremstyle{definition}
\newtheorem{dfn}{定義}[section]
\newtheorem{ex}[dfn]{例}
\newtheorem{lem}[dfn]{補題}
\newtheorem{prop}[dfn]{命題}
\newtheorem{thm}[dfn]{定理}
\newtheorem{setting}[dfn]{設定}
\newtheorem{cor}[dfn]{系}
\newtheorem*{pf*}{証明}
\newtheorem{problem}[dfn]{問題}
\newtheorem*{problem*}{問題}
\newtheorem{remark}[dfn]{注意}
\newtheorem*{claim*}{\underline{claim}}



\newtheorem*{solution*}{解答}

%箇条書きの様式
\renewcommand{\labelenumi}{(\arabic{enumi})}


%

\newcommand{\forany}{\rm{for} \ {}^{\forall}}
\newcommand{\foranyeps}{
\rm{for} \ {}^{\forall}\varepsilon >0}
\newcommand{\foranyk}{
\rm{for} \ {}^{\forall}k}


\newcommand{\any}{{}^{\forall}}
\newcommand{\suchthat}{\, \rm{s.t.} \, \it{}}




\newcommand{\veps}{\varepsilon}
\newcommand{\paren}[1]{\mleft( #1\mright )}
\newcommand{\cbra}[1]{\mleft\{#1\mright\}}
\newcommand{\sbra}[1]{\mleft\lbrack#1\mright\rbrack}
\newcommand{\tbra}[1]{\mleft\langle#1\mright\rangle}
\newcommand{\abs}[1]{\left|#1\right|}
\newcommand{\norm}[1]{\left\|#1\right\|}
\newcommand{\lopen}[1]{\mleft(#1\mright\rbrack}
\newcommand{\ropen}[1]{\mleft\lbrack #1 \mright)}



%
\newcommand{\Rn}{\mathbb{R}^n}
\newcommand{\Cn}{\mathbb{C}^n}

\newcommand{\Rm}{\mathbb{R}^m}
\newcommand{\Cm}{\mathbb{C}^m}


\newcommand{\projs}[2]{\it{p}_{#1,\ldots,#2}}
\newcommand{\projproj}[2]{\it{p}_{#1,#2}}

\newcommand{\proj}[1]{p_{#1}}

%可測空間
\newcommand{\stdProbSp}{\paren{\Omega, \mathcal{F}, P}}

%微分作用素
\newcommand{\ddt}{\frac{d}{dt}}
\newcommand{\ddx}{\frac{d}{dx}}
\newcommand{\ddy}{\frac{d}{dy}}

\newcommand{\delt}{\frac{\partial}{\partial t}}
\newcommand{\delx}{\frac{\partial}{\partial x}}

%ハイフン
\newcommand{\hyphen}{\text{-}}

%displaystyle
\newcommand{\dstyle}{\displaystyle}

%⇔, ⇒, \UTF{21D0}%
\newcommand{\LR}{\Leftrightarrow}
\newcommand{\naraba}{\Rightarrow}
\newcommand{\gyaku}{\Leftarrow}

%理由
\newcommand{\naze}[1]{\paren{\because {\mathop{ #1 }}}}

%
\newcommand{\sankaku}{\hfill $\triangle$}

%
\newcommand{\push}{_{\#}}

%手抜き
\newcommand{\textif}{\textrm{if}\,\,\,}
\newcommand{\Ric}{\textrm{Ric}}
\newcommand{\tr}{\textrm{tr}}
\newcommand{\vol}{\textrm{vol}}
\newcommand{\diam}{\textrm{diam}}
\newcommand{\supp}{\textrm{supp}}
\newcommand{\Med}{\textrm{Med}}
\newcommand{\Leb}{\textrm{Leb}}
\newcommand{\Const}{\textrm{Const}}
\newcommand{\Avg}{\textrm{Avg}}
\newcommand{\id}{\textrm{id}}
\newcommand{\Ker}{\textrm{Ker}}
\newcommand{\im}{\textrm{Im}}
\newcommand{\dil}{\textrm{dil}}
\newcommand{\Ch}{\textrm{Ch}}
\newcommand{\Lip}{\textrm{Lip}}
\newcommand{\Ent}{\textrm{Ent}}
\newcommand{\grad}{\textrm{grad}}
\newcommand{\dom}{\textrm{dom}}

\renewcommand{\;}{\, ; \,}
\renewcommand{\d}{\, {d}}

\newcommand{\gyouretsu}[1]{\begin{pmatrix} #1 \end{pmatrix} }


%%図式

\usepackage[dvipdfm,all]{xy}


\newenvironment{claim}[1]{\par\noindent\underline{step:}\space#1}{}
\newenvironment{claimproof}[1]{\par\noindent{($\because$)}\space#1}{\hfill $\blacktriangle $}


\newcommand{\pprime}{{\prime \prime}}





%%


\title{確率測度の正則性}
\date{}


\author{}


\begin{document}


\maketitle


\section{}
\begin{prop}(基本的な不等式1).
\begin{align*} \mu(A \setminus B) &= \mu (A \cap B^c) = 1 - \mu((A \cap B^c)^c) \\
&= 1 - \mu(A^c \cup B) \geq 1 - \mu(A^c) - \mu(B) =  \mu(A) - \mu(B). \end{align*}
\end{prop}
\begin{pf*}
全くその通り.
\qed
\end{pf*}

\begin{prop}(基本的な不等式2). $B \subset C \subset A$ とする. 
\begin{align*} \mu(A \setminus B) \leq \mu (A \setminus C) + \mu (C \setminus B)\end{align*}
が成り立つ.
\end{prop}
\begin{pf*}
\begin{align*}  \mu(A \setminus B) = \mu ((A \setminus C) \cup (C \setminus B)) \end{align*}
より従う.
\qed
\end{pf*}

\begin{prop}(基本的な不等式3). 
\begin{align*} \mu(A \setminus (B \cap C) ) \leq \mu(B^c) + \mu(A \setminus C)
 \end{align*}
\end{prop}
\begin{pf*}
\begin{align*}\mu(A \setminus (B \cap C) ) 
&= \mu(A \cap (B^c \cup C^c) ) = \mu(A \cap B^c \cup A \cap C^c) \\&\leq  \mu(A \cap B^c) + \mu( A \cap C^c) 
\leq  \mu(B^c) + \mu( A \cap C^c) 
\end{align*}
より従う.
\qed
\end{pf*}


\section{}

\begin{dfn}(正則な測度).
$(X, \Sigma)$ を可測空間とする. 測度$\mu$ は, 任意の可測集合$A \in \Sigma$ に対して, 任意の$\veps$ に対して
\begin{align*} &(1)F\subset A \subset G \\ &(2)\mu(G \setminus F) < \veps \end{align*}
を満たす閉集合$F$ と開集合$G$ が存在するとき, 正則であるという. 
\end{dfn}

\begin{dfn}(内部正則性).
$(X, \Sigma)$ を可測空間とする. 測度$\mu$ は, 任意の可測集合$A \in \Sigma$ に対して, 任意の$\veps$ に対して
\begin{align*} &(1)F\subset A \\ &(2) \mu(A) - \veps < \mu(F) \leq \mu(A) \end{align*}
を満たす閉集合$F$が存在するとき, 内部正則であるという. 
\end{dfn}

\begin{dfn}(外部正則性). 
$(X, \Sigma)$ を可測空間とする. 測度$\mu$ は, 任意の可測集合$A \in \Sigma$ に対して, 任意の$\veps$ に対して
\begin{align*} &(1)A\subset G \\ &(2) \mu(A) \leq \mu(G) < \mu(A) + \veps \end{align*}
を満たす閉集合$F$が存在するとき, 外部正則であるという. 
\end{dfn}

\begin{prop}
距離空間におけるボレル確率測度は正則である. 
\end{prop}
\begin{pf*}

\begin{align*} \mathcal S \coloneqq \cbra{A \subset X \mid \any \veps > 0, ^\exists F:\textrm{open}, ^\exists G:\textrm{close} ; \mu( F \setminus G) < \veps }\end{align*}
とおく. 
\begin{claim}
$\mathcal{S}$ は全ての閉集合を含む.
\end{claim}
\begin{claimproof}
$A$ を閉集合とする. 十分大きな$N$をとって
\begin{align*} A \subset A \subset B(A; \frac{1}{N})      \end{align*}
かつ $\mu(F \setminus B(A; \frac{1}{N}) ) < \veps$ が成り立つようにできる. 
\end{claimproof}
\begin{claim}
$\mathcal{S}$ は$\sigma$加法族である. 
\end{claim}
\begin{claimproof}
$\varnothing \subset \varnothing \subset \varnothing$ により$\varnothing \in \mathcal S$. $A \in \mathcal S$ とすると, $F \subset A \subset G$ で条件を満たすものがとれる. $G^c \subset A^c \subset F^c$ を考えることで, $A^c \in \mathcal S$. $A_n \in \mathcal S$ とする. $F_n \subset A_n \subset G_n$ で条件を満たすものがとれる. 
十分大きな$N$をとって
\begin{align*} \bigcup_{n \leq N} F_n \subset \bigcup_{n \in \mathbb N}  F_n \subset \bigcup_{n \in \mathbb N}  A_n \subset \bigcup_{n \in \mathbb N}  G_n \end{align*}
を考え, 
\begin{align*} \mu(\bigcup_{n \in \mathbb N}  G_n \setminus \bigcup_{n \leq N}  F_n  ) 
\leq \mu(\bigcup_{n \in \mathbb N}  G_n \setminus \bigcup_{n \in \mathbb N}  F_n ) + \mu (\bigcup_{n \in \mathbb N}  F_n  \setminus \bigcup_{n \leq N} F_n)  \end{align*}
を用いれば良い.
\end{claimproof}

故に, $\mathcal S$ は$\mathcal B(X)$ よりも大きな$\sigma$ 加法族であるので, $\mathcal B (X) \subset \mathcal S$ となり, 
$\mathcal B(X)$ が主張における条件をみたす.
\qed
\end{pf*}




\begin{prop}
$\mu$ が正則なボレル確率測度であるならば, 任意の可測集合$A$ に対して
\begin{align*} \mu(A) = \sup \cbra{\mu (F) \mid F;\textrm{close}, F \subset A. }\end{align*}
が成り立つ. 
\end{prop}
\begin{pf*}
任意の$\veps $ に対して$F \subset A \subset G, \mu (G \setminus F) < \veps$ なる閉集合と開集合がとれる. 
\begin{align*} \mu(A) - \mu(F) \leq  \mu(A \setminus F) < \veps \end{align*} 
より
\begin{align*} \mu(A) - \veps < \mu(F) < \mu(A) \end{align*}
が成り立つ.
\qed
\end{pf*}

\begin{prop}
$\mu$ が正則なボレル確率測度であるならば, 任意の可測集合$A$ に対して
\begin{align*} \mu(A) = \inf \cbra{\mu (G) \mid G;\textrm{close}, A \subset G. }\end{align*}
が成り立つ. 
\end{prop}
\begin{pf*}
閉集合のときの証明をパクればよい.
\qed
\end{pf*}

\begin{prop}
任意の可測集合$A$ に対して
\begin{align*} \mu(A) = \sup \cbra{\mu (F) \mid F;\textrm{close}, F \subset A. }   \end{align*}
が成り立つことと, $\mu$ が内部正則であることは必要十分である.
\end{prop}
\begin{pf*}
上限の定義より明らか.
\qed
\end{pf*}

\begin{prop}
任意の可測集合$A$ に対して
\begin{align*} \mu(A) = \inf \cbra{\mu (G) \mid G;\textrm{open}, A \subset G. }\end{align*}
が成り立つことと, $\mu$ が外部正則であることは必要十分である.
\end{prop}
\begin{pf*}
下限の定義より明らか.
\qed
\end{pf*}



\begin{prop}
\begin{align*} f(x) \coloneqq (1 - \frac{1}{\veps} d(x, F))^+  \end{align*}
と有界一様連続関数を定める. 
\begin{align*} 1_{F} \leq f \leq 1_{     B(F; \veps)      } \end{align*}
が成り立つ.
\end{prop}
\begin{pf*}
$f$ はだいたい, $0$ から$1$ の間の値をとる台形の関数, 
\qed
\end{pf*}

\begin{prop}
$\mu, \nu$ を内部正則なボレル確率測度とする. 
\begin{align*} \mu f = \nu f \quad (\any f \in UC_{b} (X) )\end{align*}
が成り立つならば, $\mu = \nu$が成り立つ. 
\end{prop}
\begin{pf*}任意の閉集合$F$ に対して, 
\begin{align*} f(x) \coloneqq (1 - \frac{1}{\veps} d(x, F))^+  \end{align*}
と定めると, 
\begin{align*} \mu(F) \leq \mu f(f) = \nu (f) \leq \nu(B(F; \veps) )  \end{align*}
が成り立つ. 測度の連続性から, 
\begin{align*} \mu(F) \leq \nu (F) \end{align*}
が成り立つ. 逆の不等式$\nu(F) \leq \mu(F)$ も成り立つので, 
\begin{align*} \mu(F) = \nu(F) \end{align*}
が成り立つ. 任意の可測集合$A$に対して
\begin{align*} \mu(A) = \sup \cbra{\mu (F) \mid F;\textrm{close}, F \subset A. } =  \sup \cbra{\nu (F) \mid F;\textrm{close}, F \subset A. } = \nu(A) \end{align*}
であるので, $\mu(A) = \nu(A)$ が成り立つ.
\qed
\end{pf*}

\begin{prop}
$\mu, \nu$ を内部正則なボレル確率測度とする. 
\begin{align*} \mu f = \nu f \quad (\any f \in C_{b} (X) )\end{align*}
が成り立つならば, $\mu = \nu$が成り立つ. 
\end{prop}
\begin{pf*}
\begin{align*} UC_b(X) \subset C_b(X) \end{align*}
より明らかである.
\qed
\end{pf*}

\begin{prop}
$\mu, \nu$ を正則なボレル確率測度とする. 
\begin{align*} \mu f = \nu f \quad (\any f \in C_{b} (X) )\end{align*}
が成り立つならば, $\mu = \nu$が成り立つ. 
\end{prop}
\begin{pf*}
正則ならば内部正則なので, 明らかである.
\qed
\end{pf*}


\begin{dfn}(緊密性). 
$(X, \Sigma)$ を可測空間とする. 確率測度$\mu$ は, 任意の可測集合$A \in \Sigma$ に対して, 任意の$\veps$ に対して
\begin{align*} &(1)K\subset A \\ &(2) \mu(A) - \veps < \mu(K) \leq \mu(A) \end{align*}
を満たすコンパクト集合$K$が存在するとき, 緊密であるという. 

\end{dfn}

\begin{prop}
$(X, \Sigma)$ を可測空間とする. 確率測度$\mu$ が緊密であることの必要十分条件は, 
任意の$\veps > 0$ に対してコンパクト集合$K \subset X$ で 
\begin{align*} 1 - \veps < P(K) \leq 1 \end{align*}
を満たすものが存在することである.
\end{prop}
\begin{pf*}
$\naraba$ は$A = S$ とすれば明らか. $\gyaku$ を示す. $A$ に対して
$F \subset A $ で$\mu (A \setminus F) < \veps / 2$ となる閉集合$F$ がとれる. $\mu(K^c) < \veps / 2$ となるコンパクト集合$K$ がとれる. (基本的な不等式$3$)に注意すると, 
\begin{align*} \mu(A) - \mu(K \cap F ) \leq \mu(A \setminus (K \cap F)) \leq \mu(K^c) + \mu(A \setminus F)  < \veps/2 + \veps /2 =  \veps \end{align*}
が成り立つ.
\qed
\end{pf*}



\begin{prop}
$X$ が完備可分な距離空間であるならば, ボレル確率測度は緊密である. 
\end{prop}
\begin{pf*}
$X$ は可分なので, 任意の半径$\frac{1}{i}$ に対して, 可算個の開球$\cbra{B_{ij}}_{j \in \mathbb N}$ で$X = \bigcup_j B_{ij}$ と被覆できる. 故に, 十分大きな$J(i)$ をとって
\begin{align*} \mu \paren{   ( \bigcup_{j \leq J(i)} B_{ij} )^c   } < \frac{\veps}{2^i} \end{align*}
とできる. 故に,
\begin{align*} \mu \paren{ (\bigcap_{i \in \mathbb N}   \bigcup_{j \leq J(i)} B_{ij} )^c }  =  \mu \paren{ \bigcup_{i \in \mathbb N}  ( \bigcup_{j \leq J(i)} B_{ij} )^c }  < \veps .
  \end{align*}
 であり, $\bigcap_{i \in \mathbb N}   \bigcup_{j \leq J(i)} B_{ij}  $ は完備な空間のなかの全有界な集合であるので, 相対コンパクトである. 閉包をとっても測度は変わらないので, $K = \textrm{cl}(\bigcap_{i \in \mathbb N}   \bigcup_{j \leq J(i)} B_{ij})$ とすると, $\mu(K^c) < \veps$ が成り立つ.
\qed
\end{pf*}



\end{document}