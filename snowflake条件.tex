\documentclass[10pt, fleqn, label-section=none]{bxjsarticle}

%\usepackage[driver=dvipdfm,hmargin=25truemm,vmargin=25truemm]{geometry}

\setpagelayout{driver=dvipdfm,hmargin=25truemm,vmargin=20truemm}


\usepackage{amsmath}
\usepackage{amssymb}
\usepackage{amsfonts}
\usepackage{amsthm}
\usepackage{mathtools}
\usepackage{mleftright}

\usepackage{ascmac}




\usepackage{otf}

\theoremstyle{definition}
\newtheorem{dfn}{定義}[section]
\newtheorem{ex}[dfn]{例}
\newtheorem{lem}[dfn]{補題}
\newtheorem{prop}[dfn]{命題}
\newtheorem{thm}[dfn]{定理}
\newtheorem{setting}[dfn]{設定}
\newtheorem{notation}[dfn]{記号}
\newtheorem{cor}[dfn]{系}
\newtheorem*{pf*}{証明}
\newtheorem{problem}[dfn]{問題}
\newtheorem*{problem*}{問題}
\newtheorem{remark}[dfn]{注意}
\newtheorem*{claim*}{\underline{claim}}



\newtheorem*{solution*}{解答}

%箇条書きの様式
\renewcommand{\labelenumi}{(\arabic{enumi})}


%

\newcommand{\forany}{\rm{for} \ {}^{\forall}}
\newcommand{\foranyeps}{
\rm{for} \ {}^{\forall}\varepsilon >0}
\newcommand{\foranyk}{
\rm{for} \ {}^{\forall}k}


\newcommand{\any}{{}^{\forall}}
\newcommand{\suchthat}{\, \rm{s.t.} \, \it{}}




\newcommand{\veps}{\varepsilon}
\newcommand{\paren}[1]{\mleft( #1\mright )}
\newcommand{\cbra}[1]{\mleft\{#1\mright\}}
\newcommand{\sbra}[1]{\mleft\lbrack#1\mright\rbrack}
\newcommand{\tbra}[1]{\mleft\langle#1\mright\rangle}
\newcommand{\abs}[1]{\left|#1\right|}
\newcommand{\norm}[1]{\left\|#1\right\|}
\newcommand{\lopen}[1]{\mleft(#1\mright\rbrack}
\newcommand{\ropen}[1]{\mleft\lbrack #1 \mright)}



%
\newcommand{\Rn}{\mathbb{R}^n}
\newcommand{\Cn}{\mathbb{C}^n}

\newcommand{\Rm}{\mathbb{R}^m}
\newcommand{\Cm}{\mathbb{C}^m}


\newcommand{\projs}[2]{\it{p}_{#1,\ldots,#2}}
\newcommand{\projproj}[2]{\it{p}_{#1,#2}}

\newcommand{\proj}[1]{p_{#1}}

%可測空間
\newcommand{\stdProbSp}{\paren{\Omega, \mathcal{F}, P}}

%微分作用素
\newcommand{\ddt}{\frac{d}{dt}}
\newcommand{\ddx}{\frac{d}{dx}}
\newcommand{\ddy}{\frac{d}{dy}}

\newcommand{\delt}{\frac{\partial}{\partial t}}
\newcommand{\delx}{\frac{\partial}{\partial x}}

%ハイフン
\newcommand{\hyphen}{\text{-}}

%displaystyle
\newcommand{\dstyle}{\displaystyle}

%⇔, ⇒, \UTF{21D0}%
\newcommand{\LR}{\Leftrightarrow}
\newcommand{\naraba}{\Rightarrow}
\newcommand{\gyaku}{\Leftarrow}

%理由
\newcommand{\naze}[1]{\paren{\because {\mathop{ #1 }}}}

%
\newcommand{\sankaku}{\hfill $\triangle$}

%
\newcommand{\push}{_{\#}}

%手抜き
\newcommand{\textif}{\textrm{if}\,\,\,}
\newcommand{\Ric}{\textrm{Ric}}
\newcommand{\tr}{\textrm{tr}}
\newcommand{\vol}{\textrm{vol}}
\newcommand{\diam}{\textrm{diam}}
\newcommand{\supp}{\textrm{supp}}
\newcommand{\Med}{\textrm{Med}}
\newcommand{\Leb}{\textrm{Leb}}
\newcommand{\Const}{\textrm{Const}}
\newcommand{\Avg}{\textrm{Avg}}
\newcommand{\id}{\textrm{id}}
\newcommand{\Ker}{\textrm{Ker}}
\newcommand{\im}{\textrm{Im}}
\newcommand{\dil}{\textrm{dil}}
\newcommand{\Ch}{\textrm{Ch}}
\newcommand{\Lip}{\textrm{Lip}}
\newcommand{\Ent}{\textrm{Ent}}
\newcommand{\grad}{\textrm{grad}}
\newcommand{\dom}{\textrm{dom}}
\newcommand{\diag}{\textrm{diag}}

\renewcommand{\;}{\, ; \,}
\renewcommand{\d}{\, {d}}

\newcommand{\gyouretsu}[1]{\begin{pmatrix} #1 \end{pmatrix} }

\renewcommand{\div}{\textrm{div}}


%%図式

\usepackage[dvipdfm,all]{xy}


\newenvironment{claim}[1]{\par\noindent\underline{step:}\space#1}{}
\newenvironment{claimproof}[1]{\par\noindent{($\because$)}\space#1}{\hfill $\blacktriangle $}


\newcommand{\pprime}{{\prime \prime}}

%%マグニチュード


\newcommand{\Mag}{\textrm{Mag}}

\usepackage{mathrsfs}


%%6.13
\def\Xint#1{\mathchoice
{\XXint\displaystyle\textstyle{#1}}%
{\XXint\textstyle\scriptstyle{#1}}%
{\XXint\scriptstyle\scriptscriptstyle{#1}}%
{\XXint\scriptscriptstyle\scriptscriptstyle{#1}}%
\!\int}
\def\XXint#1#2#3{{\setbox0=\hbox{$#1{#2#3}{\int}$ }
\vcenter{\hbox{$#2#3$ }}\kern-.6\wd0}}
\def\ddashint{\Xint=}
\def\dashint{\Xint-}



\title{snowflake条件}
\date{}


\author{}


\begin{document}


\maketitle

\section{}

\begin{dfn}($L^p$距離空間). 距離空間$(X, d)$ は$(X, d^p)$ が距離空間であるときに, $L^p$ 距離空間という.  また, $d$ を$L^p$ 距離関数という. 
\end{dfn}

\begin{dfn}$d, d^\prime$ を$X$ 上の距離関数とする. $L \geq  1$ で
\begin{align*} \frac{1}{L} d(x, y) \leq d^\prime(x, y) \leq L d(x, y) \quad (x, y \in X) \end{align*} 
を満たすものが存在する時に, $d, d^\prime$ は($L$-)双リプシッツ同値であるという. 
\end{dfn}

\begin{dfn}($p$-snowflake). $p > 1$ とする. 距離空間$(X, d)$ は, $d$ と双リプシッツ同値な$L^p$距離関数$d^\prime$ が存在する時に, $p$-snowflake空間という. 適当な$p > 1$ に対して$p$-snowflake空間となるとき, 単にsnowflake空間という. 
\end{dfn}

\begin{dfn}($p$-snowflake条件). $p > 1$ とする. 距離空間$(X, d)$ は$0< c < 1$ で, 任意の有限個の点$x_0, \cdots , x_N \in X$ に対して
\begin{align*} c d^p(x_0, x_N) \leq \sum_{i = 0}^{N-1} d^p (x_i, x_{i+1})  \end{align*}
を満たすものが存在するとき, $p$-snowflake条件を満たすという. 
\end{dfn}


\begin{prop}$p> 1$ とする. 距離空間$(X, d)$ に対して次は同値である. \\
(1)$(X, d)$ は$p$-snowflake空間である. \\
(2)$(X, d)$ は$p$-snowflake条件を満たす. 

\end{prop}
\begin{pf*}$(\naraba)$. $d^\prime$ を$d$ と双リプシッツ同値な$L^p$ 距離とする. 
\begin{align*} \frac{1}{L} d(x, y) \leq d^\prime (x, y) \leq L d(x, y)\end{align*}
が成り立つので, 有限個の点$x_0, \ldots, x_{N + 1}$ に対して, 
\begin{align*}  \frac{1}{L^p} d^p(x_0, x_N)  \leq  {d^\prime}^p (x_0, x_N) \leq \sum {d ^\prime } ^p(x_i, x_{i + 1}) \leq L^p \sum d^p(x_i, x_{i + 1}) \end{align*}
が成り立つので, $c = \frac{1}{L^{2p}} > 0$ ととれば$p$-snowflake条件を満たす. \\
$(\gyaku)$. $p$-snowflake条件を成立させる定数を$0 < c < 1$ とする. 
\begin{align*} d^\prime(x, y) \coloneqq \inf \cbra{ \sum_{i = 0}^{N-1} d^p(x_i, x_{i + 1})}^{\frac{1}{p}}    \end{align*}
と定める(ただし, 下限は, 有限個の点$x_0, \ldots, x_N$ で$x_0 = x, x_N = y$ であるもの全体を走る). すると, $0 < c < 1, p > 1$ であることに注意すると, 
\begin{align*} c^{\frac{1}{p}} d(x, y) =  \inf \paren{c d^p(x, y)}^{\frac{1}{p}} \leq \inf \paren{ \sum_{i = 0}^{N-1} d^p(x_i, x_{i + 1})}^{\frac{1}{p}}   \leq d(x, y) \leq \frac{1}{c^{\frac{1}{p} }} d(x, y) \end{align*}
が成り立つ. 
\qed
\end{pf*}











\end{document}