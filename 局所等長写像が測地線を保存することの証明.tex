\documentclass[10pt, fleqn, label-section=none]{bxjsarticle}

%\usepackage[driver=dvipdfm,hmargin=25truemm,vmargin=25truemm]{geometry}

\setpagelayout{driver=dvipdfm,hmargin=25truemm,vmargin=20truemm}

\usepackage{amsmath}
\usepackage{amssymb}
\usepackage{amsfonts}
\usepackage{amsthm}
\usepackage{mathtools}
\usepackage{mleftright}

%box
\usepackage{ascmac}

%%
\usepackage{xcolor} 
\usepackage[dvipdfmx]{hyperref}
\usepackage{pxjahyper}
\hypersetup{
setpagesize=false,
 bookmarksnumbered=true,
 bookmarksopen=true,
 colorlinks=true,
 linkcolor=teal,
 citecolor=black,
}
%
%

%


%%図式

\usepackage[dvipdfm,all]{xy}


%%



\usepackage{otf}

\theoremstyle{definition}
\newtheorem{dfn}{定義}[section]
\newtheorem{ex}[dfn]{例}
\newtheorem{lem}[dfn]{補題}
\newtheorem{prop}[dfn]{命題}
\newtheorem{thm}[dfn]{定理}
\newtheorem{cor}[dfn]{系}
\newtheorem*{pf*}{証明}
\newtheorem{problem}[dfn]{問題}
\newtheorem*{problem*}{問題}
\newtheorem{remark}[dfn]{注意}

\newtheorem*{solution*}{解答}

%箇条書きの様式
\renewcommand{\labelenumi}{(\arabic{enumi})}


%

\newcommand{\forany}{\rm{for} \ {}^{\forall}}
\newcommand{\foranyeps}{
\rm{for} \ {}^{\forall}\varepsilon >0}
\newcommand{\foranyk}{
\rm{for} \ {}^{\forall}k}


\newcommand{\any}{{}^{\forall}}
\newcommand{\suchthat}{\, \textrm{s.t.} \, }




\newcommand{\veps}{\varepsilon}
\newcommand{\paren}[1]{\mleft( #1\mright )}
\newcommand{\cbra}[1]{\mleft\{#1\mright\}}
\newcommand{\sbra}[1]{\mleft\lbrack#1\mright\rbrack}
\newcommand{\tbra}[1]{\mleft\langle#1\mright\rangle}

\newcommand{\ntbra}[1]{\langle#1\rangle}

\newcommand{\abs}[1]{\left|#1\right|}
\newcommand{\norm}[1]{\left\|#1\right\|}
\newcommand{\lopen}[1]{\mleft(#1\mright\rbrack}
\newcommand{\ropen}[1]{\mleft\lbrack #1 \mright)}



%
\newcommand{\Rn}{\mathbb{R}^n}
\newcommand{\Cn}{\mathbb{C}^n}

\newcommand{\Rm}{\mathbb{R}^m}
\newcommand{\Cm}{\mathbb{C}^m}


\newcommand{\supp}{\textrm{supp}\,} 

\newcommand{\ifufu}{\,\textrm {iff} \, \it}


\newcommand{\proj}[1]{\it{p}_{#1}}
\newcommand{\projs}[2]{\it{p}_{#1,\ldots,#2}}
\newcommand{\projproj}[2]{\it{p}_{#1,#2}}

\newcommand{\push}{_{\#}}

%可測空間
\newcommand{\stdProbSp}{\paren{\Omega, \mathcal{F}, P}}

%微分作用素
\newcommand{\ddt}{\frac{d}{dt}}
\newcommand{\ddx}{\frac{d}{dx}}
\newcommand{\ddy}{\frac{d}{dy}}

\newcommand{\delt}{\frac{\partial}{\partial t}}
\newcommand{\delx}{\frac{\partial}{\partial x}}

%ハイフン
\newcommand{\hyphen}{\text{-}}

%displaystyle
\newcommand{\dstyle}{\displaystyle}

%⇔, ⇒, \UTF{21D0}%
\newcommand{\LR}{\Leftrightarrow}
\newcommand{\naraba}{\Rightarrow}
\newcommand{\gyaku}{\Leftarrow}

%理由
\newcommand{\naze}[1]{\paren{\because {\mathop{ #1 }}}}

%ベクトル解析
\newcommand{\grad}{\textrm{grad}}
\renewcommand{\div}{\textrm{div}}

%手抜き
\newcommand{\textif}{\textrm{if}\,\,\,}
\newcommand{\Sgn}{\textrm{Sgn}}
\newcommand{\Ric}{\textrm{Ric}}
\newcommand{\Sec}{\textrm{Sec}}
\newcommand{\Scal}{\textrm{Scal}}
\newcommand{\tr}{\textrm{tr}}
\newcommand{\vol}{\textrm{vol}}
\newcommand{\diam}{\textrm{diam}}
\newcommand{\Med}{\textrm{Med}}
\newcommand{\Leb}{\textrm{Leb}}
\newcommand{\Const}{\textrm{Const}}
\newcommand{\Avg}{\textrm{Avg}}
\renewcommand{\d}{\, d}
\newcommand{\length}{\textrm{length}}
\newcommand{\Func}{\textrm{Func}}
\newcommand{\Ker}{\textrm{Ker}}
\newcommand{\Cone}{\textrm{Cone}}
\newcommand{\hess}{\textrm{hess}}
\newcommand{\esssup}{\textrm{ess}\,\textrm{sup}}

\newcommand{\sub}{\textrm{sub}}
\newcommand{\Par}{\textrm{Par}}


\newcommand{\perpperp}{{\perp \perp}}

\newcommand{\sgyouretsu}[1]{\paren{\begin{smallmatrix} #1 \end{smallmatrix} }}

\renewcommand{\ni}{\hspace{2pt} \textrm{I} \hspace{-5pt} \textrm{I} \hspace{2pt}}





%↓本体↓

\title{局所等長写像が測地線を保存することの証明}

\author{}
\date{}

\begin{document}

\maketitle
\scriptsize 


%%目次%%
%\tableofcontents
%%%%%%

\section{}

\begin{remark}
接続は全てレビチビタ接続を採用する. 
\end{remark}


\begin{dfn}(局所等長写像).
$(M,g_M)$ から$(N, g_N)$ への局所微分同相写像$f$ で, $f^* g_N = g_M$ をみたすものを, 局所等長写像という.
\end{dfn}

\begin{dfn}(等長写像).
$(M,g_M)$ から$(N, g_N)$ への微分同相写像$f$ で, $f^* g_N = g_M$ をみたすものを, 等長写像という.
\end{dfn}


\begin{prop}
$f: (M,g_M) \rightarrow (N,g_N)$ を局所微分同相写像とする. このとき, 任意の点$p \in M, f(p) \in N$ のまわりの座標近傍$U, V$で
\begin{align*} df_q {\partial_i} _q  = {\delta_i }_{f(q) }  \quad (\any q \in U)\end{align*}
を満たすものが存在する. ただし, $\partial_i, \delta_i$ はそれぞれの局所座標の第$i$ 成分に関する偏微分により定まる接ベクトルである. 
\end{prop}
\begin{pf*}
実際, $p$ の周りで微分同相な開集合のペアを$U, V$ とする. $U $ における局所座標写像を$(x^1, \ldots, x^n)$ とする. このとき, $V$ に対しては座標を$(x^1 \circ f^{-1}, \ldots, x^n \circ f^{-1})$ で定める. すると, $N$ 上の滑らかな実数値関数$F$ に対して
\begin{align*} & \delta_i F = \frac{\partial F \circ f \circ \varphi^{-1} }{ \partial x^i} \varphi \circ f^{-1} = (df \partial_i ) F   \end{align*}
が成り立つ. 
\qed
\end{pf*}

\begin{prop}
$f: (M,g_M) \rightarrow (N,g_N)$ を局所微分同相写像とする. このとき, 任意の点$p \in M, f(p) \in N$ のまわりで  
\begin{align*} g^M_{ij}(q) = g^N_{ij}(f(q)) \quad (\any q \in U) \end{align*}
をみたす局所座標がとれる. 
\end{prop}
\begin{pf*}
直前の命題の通りに局所座標をとると, 
\begin{align*} &g^N_{ij}( f(q) ) = g^N_{f(q)} ({\delta_i}_{f(q)}, { \delta_j}_{f(q)}  ) = g^N_{f(q)} ({df \partial_i}_{q}, {df \partial_j}_{q}  ) \\&\hspace{32pt} = (f^* g^N_q ) ({\partial_i}_q, {\partial_j}_q) =   g^M _q  ({\partial_i}_q, {\partial_j}_q) = g^M _{ij} (q) \end{align*}
\qed
\end{pf*}



\begin{prop}
$M$ の測地線の, 局所等長写像による像は$N$ の測地線である.
\end{prop}
\begin{pf*}
はじめの命題と同様の局所座標をとる. すると, クリストッフェル記号はリーマン計量によって決定されるので, ${\Gamma^ M }_{ij}^k (q) = {\Gamma^N }_{ij}^k (f(q)) $ が成り立つ. $M$ における測地線$\gamma $ は
\begin{align*} \partial_t ^2 (x^k \circ \gamma) + ({\Gamma^M} _{ij}^k \circ \gamma) \partial_t (x^i \circ \gamma) \partial_t (x^j \circ \gamma) = 0\end{align*}
なる$(M,g)$ の測地線方程式を満たす. 従って, 曲線 $f \circ \gamma$ は
\begin{align*}  &\partial_t ^2 (x^k \circ f^{-1} \circ f \circ \gamma) + ({\Gamma^N}_{ij}^k \circ f \circ  \gamma) \partial_t (x^i \circ f^{-1} \circ f \circ\gamma) \partial_t (x^j \circ f^{-1} \circ f \circ \gamma) \\&= \partial_t ^2 (x^k \circ \gamma) + ({\Gamma^M}_{ij}^k  \circ \gamma) \partial_t (x^i \circ \gamma) \partial_t (x^j \circ \gamma) = 0    \end{align*}
となるので, $(N, g_N)$ の測地線方程式をみたす. 
\qed
\end{pf*}






















\end{document}