\documentclass[10pt, fleqn, label-section=none]{bxjsarticle}

%\usepackage[driver=dvipdfm,hmargin=25truemm,vmargin=25truemm]{geometry}

\setpagelayout{driver=dvipdfm,hmargin=25truemm,vmargin=20truemm}


\usepackage{amsmath}
\usepackage{amssymb}
\usepackage{amsfonts}
\usepackage{amsthm}
\usepackage{mathtools}
\usepackage{mleftright}

\usepackage{ascmac}




\usepackage{otf}

\theoremstyle{definition}
\newtheorem{dfn}{定義}[section]
\newtheorem{ex}[dfn]{例}
\newtheorem{lem}[dfn]{補題}
\newtheorem{prop}[dfn]{命題}
\newtheorem{thm}[dfn]{定理}
\newtheorem{cor}[dfn]{系}
\newtheorem*{pf*}{証明}
\newtheorem{problem}[dfn]{問題}
\newtheorem*{problem*}{問題}
\newtheorem{remark}[dfn]{注意}
\newtheorem*{claim*}{\underline{claim}}



\newtheorem*{solution*}{解答}

%箇条書きの様式
\renewcommand{\labelenumi}{(\arabic{enumi})}


%

\newcommand{\forany}{\rm{for} \ {}^{\forall}}
\newcommand{\foranyeps}{
\rm{for} \ {}^{\forall}\varepsilon >0}
\newcommand{\foranyk}{
\rm{for} \ {}^{\forall}k}


\newcommand{\any}{{}^{\forall}}
\newcommand{\suchthat}{\, \rm{s.t.} \, \it{}}




\newcommand{\veps}{\varepsilon}
\newcommand{\paren}[1]{\mleft( #1\mright )}
\newcommand{\cbra}[1]{\mleft\{#1\mright\}}
\newcommand{\sbra}[1]{\mleft\lbrack#1\mright\rbrack}
\newcommand{\tbra}[1]{\mleft\langle#1\mright\rangle}
\newcommand{\abs}[1]{\left|#1\right|}
\newcommand{\norm}[1]{\left\|#1\right\|}
\newcommand{\lopen}[1]{\mleft(#1\mright\rbrack}
\newcommand{\ropen}[1]{\mleft\lbrack #1 \mright)}



%
\newcommand{\Rn}{\mathbb{R}^n}
\newcommand{\Cn}{\mathbb{C}^n}

\newcommand{\Rm}{\mathbb{R}^m}
\newcommand{\Cm}{\mathbb{C}^m}


\newcommand{\projs}[2]{\it{p}_{#1,\ldots,#2}}
\newcommand{\projproj}[2]{\it{p}_{#1,#2}}

\newcommand{\proj}[1]{p_{#1}}

%可測空間
\newcommand{\stdProbSp}{\paren{\Omega, \mathcal{F}, P}}

%微分作用素
\newcommand{\ddt}{\frac{d}{dt}}
\newcommand{\ddx}{\frac{d}{dx}}
\newcommand{\ddy}{\frac{d}{dy}}

\newcommand{\delt}{\frac{\partial}{\partial t}}
\newcommand{\delx}{\frac{\partial}{\partial x}}

%ハイフン
\newcommand{\hyphen}{\text{-}}

%displaystyle
\newcommand{\dstyle}{\displaystyle}

%⇔, ⇒, \UTF{21D0}%
\newcommand{\LR}{\Leftrightarrow}
\newcommand{\naraba}{\Rightarrow}
\newcommand{\gyaku}{\Leftarrow}

%理由
\newcommand{\naze}[1]{\paren{\because {\mathop{ #1 }}}}

%
\newcommand{\sankaku}{\hfill $\triangle$}

%
\newcommand{\push}{_{\#}}

%手抜き
\newcommand{\textif}{\textrm{if}\,\,\,}
\newcommand{\Ric}{\textrm{Ric}}
\newcommand{\tr}{\textrm{tr}}
\newcommand{\vol}{\textrm{vol}}
\newcommand{\diam}{\textrm{diam}}
\newcommand{\supp}{\textrm{supp}}
\newcommand{\Med}{\textrm{Med}}
\newcommand{\Leb}{\textrm{Leb}}
\newcommand{\Const}{\textrm{Const}}
\newcommand{\Avg}{\textrm{Avg}}
\newcommand{\id}{\textrm{id}}
\newcommand{\Ker}{\textrm{Ker}}
\newcommand{\im}{\textrm{Im}}




\renewcommand{\;}{\, ; \,}
\renewcommand{\d}{\, {d}}

\newcommand{\gyouretsu}[1]{\begin{pmatrix} #1 \end{pmatrix} }

%%図式

\usepackage[dvipdfm,all]{xy}


\newenvironment{claim}[1]{\par\noindent\underline{Step:}\space#1}{}
\newenvironment{claimproof}[1]{\par\noindent{($\because$)}\space#1}{\hfill $\blacktriangle $}


\newcommand{\pprime}{{\prime \prime}}


%%


\title{有限補集合位相}
\date{}


\author{}


\begin{document}


\maketitle



\section{有限補集合位相}



\begin{dfn}
$\cbra{\varnothing} \cup \cbra{U \subset X \mid U^c \textrm{が有限集合}}$ なる位相を有限補集合位相(cofinite位相)という.
\end{dfn}

有限補集合位相を$\mathcal O_{cf}$ で表すことにする.

\begin{prop}\label{d19a}
$X$ が有限集合でないならば, $(X, \mathcal O_{cf})$ はハウスドルフ空間でない. 
\end{prop}
\begin{pf*}
ハウスドルフ空間であると仮定する. 好きに異なる二点$p, q \in \mathbb R$ をとる. $p,q$ それぞれの開近傍$U_p, U_q$ で$U_p \cap U_q = \varnothing$ となるものをとる. $U_q \subset U_p^c$ であるので$U_q$ は有限集合である. 従って, $U_q^c$ は有限集合ではないので開集合でなくなる. よって矛盾である. 
\qed
\end{pf*}

\begin{prop}\label{38bm}
$(X, \mathcal O_{cf})$ はT1空間である. 
\end{prop}
\begin{pf*}
二点$p, q \in X$ に対してそれぞれ$X \setminus \cbra{p}, X \setminus \cbra{q}$ ととればよい. 
\qed
\end{pf*}

\begin{prop}
$(\mathbb R^n , \mathcal O_{cf})$ はT1空間だがハウスドルフ空間ではない. 
\end{prop}
\begin{pf*}
$\mathbb R^n$ は有限集合でない.
\qed
\end{pf*}



\begin{prop}\label{pk2d}
$(X , \mathcal O _{cf})$ はコンパクトである. 
\end{prop}
\begin{pf*}
任意に開被覆$\mathcal U = \cbra{U_\lambda}$ をとる. 適当に$U \in \cbra{U_\lambda} $ をとると, $\mathbb R \setminus U $ は有限集合なので, それを$\cbra{p_1, p_2, \ldots , p_N}$ とする. $p_i$ を含む$\mathcal U \setminus \cbra{U}$ の集合を$U_i$, とすると, $U, U_1, \ldots ,U_N$ で$\mathbb R^n$ を被覆できる. 
\qed
\end{pf*}

\begin{dfn}$V \subset X, p \in X$ とする. $V$ は, 
$p$の開近傍$U_p$で$U_p \subset V$ をみたすものが存在するとき, $p$ の近傍であるという. 
\end{dfn}

\begin{prop}\label{77n7}
$X$ が不可算集合であるならば, $(X, \mathcal O_{cf})$ は第一可算公理を満たさない. 
\end{prop}
\begin{pf*}
第一可算公理を満たすとする. 適当に$p \in \mathbb R^n $ をとり, $p$ の可算基本近傍系$\mathcal V_p \coloneqq \cbra{V \mid V \textrm{は} p \textrm{の近傍}}$ をとる. 
$\cbra{p}\cup (\bigcup_{v_i \subset \mathcal V_p} V_i  ^c )$ は可算集合となるので, 不可算集合$\mathbb R ^n $ と一致しない. そこで, 
\begin{align*}q \in \mathbb R ^n  \setminus \paren{ \cbra{p}\cup (\bigcup_{v_i \subset \mathcal V_p} V_i  ^c  )  } \end{align*}
をとる. $q \in V_i$ となるので$V_i \not\subset \mathbb R \setminus \cbra{q}$ である. 一方で, $p \in \mathbb R^n \setminus \cbra{q}$ であり, $\mathbb R^n \setminus \cbra{q}$ の補集合は有限であるので, これは$p$ の開近傍である. 故に近傍であるので, $V_i \in \mathcal V_p$ で$V_i \subset \mathbb R^n \setminus \cbra{q}$ となるものがとれるので矛盾する. 
\qed
\end{pf*}

\newpage


\section{可算補集合位相}

\begin{dfn}
$\cbra{\varnothing} \cup \cbra{U \subset X \mid U^c \textrm{が可算集合}}$ なる位相を可算補集合位相(cocountable)という.
\end{dfn}

可算補集合位相を$\mathcal O_{cc}$で表すことにする. 

\begin{prop}
$X$ が不可算集合であるならば, $(\mathbb X, \mathcal O_{cc})$ はハウスドルフ空間ではない.
\end{prop}
\begin{pf*}
命題\ref{d19a}を真似ればよい. 
\qed
\end{pf*}

\begin{prop}
 $(\mathbb X, \mathcal O_{cc})$ はT1空間である.
\end{prop}
\begin{pf*}
命題\ref{38bm} を真似ればよい. 
\qed
\end{pf*}


\begin{prop}
$(\mathbb X, \mathcal O_{cc})$ はリンデレーフである. 
\end{prop}
\begin{pf*}
命題\ref{pk2d}を真似ればよい.
\qed
\end{pf*}

\begin{prop}
$X$ が不可算集合であるならば, $(X, \mathcal O_{cc})$ は第一可算公理を満たさない. 
\end{prop}
\begin{pf*}
命題\ref{77n7}を真似ればよい.
\qed
\end{pf*}






\end{document}