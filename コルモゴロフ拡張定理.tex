\documentclass[10pt, fleqn, label-section=none]{bxjsarticle}

%\usepackage[driver=dvipdfm,hmargin=25truemm,vmargin=25truemm]{geometry}

\setpagelayout{driver=dvipdfm,hmargin=25truemm,vmargin=20truemm}


\usepackage{amsmath}
\usepackage{amssymb}
\usepackage{amsfonts}
\usepackage{amsthm}
\usepackage{mathtools}
\usepackage{mleftright}

\usepackage{ascmac}




\usepackage{otf}

\theoremstyle{definition}
\newtheorem{dfn}{定義}[section]
\newtheorem{ex}[dfn]{例}
\newtheorem{lem}[dfn]{補題}
\newtheorem{prop}[dfn]{命題}
\newtheorem{thm}[dfn]{定理}
\newtheorem{setting}[dfn]{設定}
\newtheorem{cor}[dfn]{系}
\newtheorem*{pf*}{証明}
\newtheorem{problem}[dfn]{問題}
\newtheorem*{problem*}{問題}
\newtheorem{remark}[dfn]{注意}
\newtheorem*{claim*}{\underline{claim}}



\newtheorem*{solution*}{解答}

%箇条書きの様式
\renewcommand{\labelenumi}{(\arabic{enumi})}


%

\newcommand{\forany}{\rm{for} \ {}^{\forall}}
\newcommand{\foranyeps}{
\rm{for} \ {}^{\forall}\varepsilon >0}
\newcommand{\foranyk}{
\rm{for} \ {}^{\forall}k}


\newcommand{\any}{{}^{\forall}}
\newcommand{\suchthat}{\, \rm{s.t.} \, \it{}}




\newcommand{\veps}{\varepsilon}
\newcommand{\paren}[1]{\mleft( #1\mright )}
\newcommand{\cbra}[1]{\mleft\{#1\mright\}}
\newcommand{\sbra}[1]{\mleft\lbrack#1\mright\rbrack}
\newcommand{\tbra}[1]{\mleft\langle#1\mright\rangle}
\newcommand{\abs}[1]{\left|#1\right|}
\newcommand{\norm}[1]{\left\|#1\right\|}
\newcommand{\lopen}[1]{\mleft(#1\mright\rbrack}
\newcommand{\ropen}[1]{\mleft\lbrack #1 \mright)}



%
\newcommand{\Rn}{\mathbb{R}^n}
\newcommand{\Cn}{\mathbb{C}^n}

\newcommand{\Rm}{\mathbb{R}^m}
\newcommand{\Cm}{\mathbb{C}^m}


\newcommand{\projs}[2]{\it{p}_{#1,\ldots,#2}}
\newcommand{\projproj}[2]{\it{p}_{#1,#2}}

\newcommand{\proj}[1]{p_{#1}}

%可測空間
\newcommand{\stdProbSp}{\paren{\Omega, \mathcal{F}, P}}

%微分作用素
\newcommand{\ddt}{\frac{d}{dt}}
\newcommand{\ddx}{\frac{d}{dx}}
\newcommand{\ddy}{\frac{d}{dy}}

\newcommand{\delt}{\frac{\partial}{\partial t}}
\newcommand{\delx}{\frac{\partial}{\partial x}}

%ハイフン
\newcommand{\hyphen}{\text{-}}

%displaystyle
\newcommand{\dstyle}{\displaystyle}

%⇔, ⇒, \UTF{21D0}%
\newcommand{\LR}{\Leftrightarrow}
\newcommand{\naraba}{\Rightarrow}
\newcommand{\gyaku}{\Leftarrow}

%理由
\newcommand{\naze}[1]{\paren{\because {\mathop{ #1 }}}}

%
\newcommand{\sankaku}{\hfill $\triangle$}

%
\newcommand{\push}{_{\#}}

%手抜き
\newcommand{\textif}{\textrm{if}\,\,\,}
\newcommand{\Ric}{\textrm{Ric}}
\newcommand{\tr}{\textrm{tr}}
\newcommand{\vol}{\textrm{vol}}
\newcommand{\diam}{\textrm{diam}}
\newcommand{\supp}{\textrm{supp}}
\newcommand{\Med}{\textrm{Med}}
\newcommand{\Leb}{\textrm{Leb}}
\newcommand{\Const}{\textrm{Const}}
\newcommand{\Avg}{\textrm{Avg}}
\newcommand{\id}{\textrm{id}}
\newcommand{\Ker}{\textrm{Ker}}
\newcommand{\im}{\textrm{Im}}




\renewcommand{\;}{\, ; \,}
\renewcommand{\d}{\, {d}}

\newcommand{\gyouretsu}[1]{\begin{pmatrix} #1 \end{pmatrix} }

%%図式

\usepackage[dvipdfm,all]{xy}


\newenvironment{claim}[1]{\par\noindent\underline{step:}\space#1}{}
\newenvironment{claimproof}[1]{\par\noindent{($\because$)}\space#1}{\hfill $\blacktriangle $}


\newcommand{\pprime}{{\prime \prime}}





%%


\title{コルモゴロフ拡張定理}
\date{}


\author{}


\begin{document}


\maketitle



\section{}

こっから射影をpで書きたいからという理由で, 確率測度をPとかではなく普通に$\mu$ で表す頻度が増える.

\begin{thm}
(Hopf の拡張定理) \\
有限加法族$ F _0$ 上の有限加法的確率測度$\mu$ が$\sigma\paren{ F _0}$ 上の$\sigma$ 加法的確率測度(即ち, 単に確率測度)に一意に拡張できる. \\
$\LR$ 任意の単調減少列$\cbra{A_i} \subset \mathcal{A}$ について, $\lim_n \mu\paren{A_n} > 0$ ならば, $\bigcap_i A_i \neq \varnothing$ が成り立つ.

\end{thm}
\begin{pf*}
続編に収録.
\qed
\end{pf*}


\begin{thm}
(Kolmogorov の拡張定理)\\
$\mu_n$を$\paren{\mathbb{R}^n, \mathcal{B}\paren{\mathbb{R}^n}}$ 上の確率測度とする. \\
$\mu_n = {p_{1,\ldots,n}}_{\#} \mu_{n+k} \quad \paren{\foranyk \geq 1}
\naraba \, ^\exists \mu :\paren{\mathbb{R}^\infty, \mathcal{B}\paren{\mathbb{R}^\infty}}\mathop{上の確率測度} \suchthat \mu_n = {\it{p}_{1,\ldots,n}}_{\#}\mu \quad \forany n $
\end{thm}
\begin{pf*}
\, \\
$\mathcal{S} \coloneqq \cbra{{\it{p}_{1,\ldots,n}}^{-1}\paren{A} \mid A \in \mathcal{B}\paren{\mathbb{R}^n}, n \in \mathbb{N}}$ 上の有限加法的確率測度$\mu ^{\prime}$ を
${\it{p}_{1,\ldots,n}}_{\#} \mu^{\prime} \coloneqq \mu_n$ で定める. \\
任意の単調減少列$\cbra{S_i} \subset \mathcal{S}$ について, $\lim_n \mu^{\prime} \paren{S_k} > 0$ ならば, $\bigcap_k S_k \neq \varnothing$ を示せば, Hopf の拡張定理より$\sigma\paren{\mathcal{A}} = \mathcal{B} \paren{\Rn}$ 上の確率測度に拡張できる. 従って, $\alpha \coloneqq \lim_n \mu^{\prime} \paren{S_k} > 0$ として以下それを示す. \\
減少列$\cbra{S_i = {\it{p}_{1,\ldots,n}}\paren{A_i}}$ をとり, 各kに対して$A_k \in \mathcal{B}\paren{\mathbb{R}^k}$ となるよう添字を付けなおして, 部分列をとる. \\
$A_k$ をコンパクト集合$K_k$で内側から近似しておく. \\
$\{\mu^{\prime}\paren{{\projs{1}{k}}^{-1}\paren{A_k} \setminus {\projs{1}{k}}^{-1} \paren{K_k} } = \mu^{\prime} \paren{{\projs{1}{k}}^{-1}\paren{A_k \setminus K_k}}  \\ \quad \quad = {\projs{1}{k}}_{\#} \mu ^{\prime} \paren{A_k \setminus K_k} \mu_k \paren{A_k \setminus K_k} \leq \frac{\alpha}{2^{k+1}} \mathop{が成立.} \}$
\begin{align*}
\mu ^{\prime} \paren{\bigcap_{k=1}^{n} \cbra{{\projs{1}{k}}^{-1} \paren{K_k}}} 
&\geq \mu ^{\prime}\paren{ {\projs{1}{n}}^{-1} \paren{A_n} } - \mu^{\prime}\paren{{\projs{1}{n}}^{-1}  \paren{A_n} \cap \paren{\bigcap_{k=1}^{n} {\projs{1}{k}}^{-1}\paren{K_k}}^c} \\
&= \mu ^{\prime}\paren{ {\projs{1}{n}}^{-1} \paren{A_n} } - \mu^{\prime}\paren{\bigcup_{k=1}^{n} {\projs{1}{n}}^{-1} \paren{A_n}  \cap \paren{{\projs{1}{k}}^{-1}\paren{K_k}}^c} \\
&= \mu ^{\prime}\paren{ {\projs{1}{n}}^{-1} \paren{A_n} } - \sum_{k=1}^{n} \mu^{\prime}\paren{{\projs{1}{n}}^{-1}  \paren{A_n} \cap \paren{{\projs{1}{k}}^{-1}\paren{K_k}}^c} \\
&= \alpha - \frac{\alpha}{2} \qquad = \qquad \frac{\alpha}{2} \qquad > \qquad 0
\end{align*}
$\paren{\mathop{第一不等号は,} K_k \subset A_n \mathop{故に, \bigcap_{k=1}^{n}{\projs{1}{k}}^{-1} \paren{K_k} \subset {\projs{1}{n}}^{-1}\paren{A_n} } \mathop{だから.}}$\\
$\bigcap_{k=1}^{n} {\projs{1}{k}}^{-1} \paren{K_k} \neq \varnothing$ なので, 各nに対して, \\
$x^n \coloneqq \paren{{x_{1}}^{n}, {x_{2}}^{n}, \ldots } \in \bigcap_{k=1}^{n} {\projs{1}{k}}^{-1} \paren{K_k} \subset \mathbb{R}^{\infty}$ をとり$\cbra{x^n}\subset \mathbb{R}^{\infty}$ という列をつくると, \\
$n \leq m \naraba x^m \in \bigcap_{k=1}^{n} {\projs{1}{k}}^{-1}$ なので, $\projs{1}{n}\paren{x^m} \in K_1 ,\ldots, K_n $ である. 従って, \\
${x_{1}}^{1}, {x_{1}}^{2}, {x_{1}}^{3}, \ldots \in K_1 $\\
${x_{2}}^{1}, {x_{2}}^{2}, {x_{2}}^{3}, \ldots \in K_2 $\\
${x_{3}}^{1}, {x_{3}}^{2}, {x_{3}}^{3}, \ldots \in K_3 $\\
となるので, うまく収束部分列をとる操作を繰り返すことで, $\paren{x_1, \ldots , x_n } \in \bigcap_{k=1}^{n} K_k$ である. \\
故に, $x = \paren{x_1, x_2\ldots } \in \bigcap_{k=1}^{\infty} K_k$ なる元をとると, $x \in \bigcap S_k$ が成り立つ.

\qed
\end{pf*}





\end{document}