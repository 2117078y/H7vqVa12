\documentclass[10pt, fleqn, label-section=none]{bxjsarticle}

%\usepackage[driver=dvipdfm,hmargin=25truemm,vmargin=25truemm]{geometry}

\setpagelayout{driver=dvipdfm,hmargin=25truemm,vmargin=20truemm}


\usepackage{amsmath}
\usepackage{amssymb}
\usepackage{amsfonts}
\usepackage{amsthm}
\usepackage{mathtools}
\usepackage{mleftright}

\usepackage{ascmac}




\usepackage{otf}

\theoremstyle{definition}
\newtheorem{dfn}{定義}[section]
\newtheorem{ex}[dfn]{例}
\newtheorem{lem}[dfn]{補題}
\newtheorem{prop}[dfn]{命題}
\newtheorem{thm}[dfn]{定理}
\newtheorem{setting}[dfn]{設定}
\newtheorem{notation}[dfn]{記号}
\newtheorem{cor}[dfn]{系}
\newtheorem*{pf*}{証明}
\newtheorem{problem}[dfn]{問題}
\newtheorem*{problem*}{問題}
\newtheorem{remark}[dfn]{注意}
\newtheorem*{claim*}{\underline{claim}}



\newtheorem*{solution*}{解答}

%箇条書きの様式
\renewcommand{\labelenumi}{(\arabic{enumi})}


%

\newcommand{\forany}{\rm{for} \ {}^{\forall}}
\newcommand{\foranyeps}{
\rm{for} \ {}^{\forall}\varepsilon >0}
\newcommand{\foranyk}{
\rm{for} \ {}^{\forall}k}


\newcommand{\any}{{}^{\forall}}
\newcommand{\suchthat}{\, \rm{s.t.} \, \it{}}




\newcommand{\veps}{\varepsilon}
\newcommand{\paren}[1]{\mleft( #1\mright )}
\newcommand{\cbra}[1]{\mleft\{#1\mright\}}
\newcommand{\sbra}[1]{\mleft\lbrack#1\mright\rbrack}
\newcommand{\tbra}[1]{\mleft\langle#1\mright\rangle}
\newcommand{\abs}[1]{\left|#1\right|}
\newcommand{\norm}[1]{\left\|#1\right\|}
\newcommand{\lopen}[1]{\mleft(#1\mright\rbrack}
\newcommand{\ropen}[1]{\mleft\lbrack #1 \mright)}



%
\newcommand{\Rn}{\mathbb{R}^n}
\newcommand{\Cn}{\mathbb{C}^n}

\newcommand{\Rm}{\mathbb{R}^m}
\newcommand{\Cm}{\mathbb{C}^m}


\newcommand{\projs}[2]{\it{p}_{#1,\ldots,#2}}
\newcommand{\projproj}[2]{\it{p}_{#1,#2}}

\newcommand{\proj}[1]{p_{#1}}

%可測空間
\newcommand{\stdProbSp}{\paren{\Omega, \mathcal{F}, P}}

%微分作用素
\newcommand{\ddt}{\frac{d}{dt}}
\newcommand{\ddx}{\frac{d}{dx}}
\newcommand{\ddy}{\frac{d}{dy}}

\newcommand{\delt}{\frac{\partial}{\partial t}}
\newcommand{\delx}{\frac{\partial}{\partial x}}

%ハイフン
\newcommand{\hyphen}{\text{-}}

%displaystyle
\newcommand{\dstyle}{\displaystyle}

%⇔, ⇒, \UTF{21D0}%
\newcommand{\LR}{\Leftrightarrow}
\newcommand{\naraba}{\Rightarrow}
\newcommand{\gyaku}{\Leftarrow}

%理由
\newcommand{\naze}[1]{\paren{\because {\mathop{ #1 }}}}

%
\newcommand{\sankaku}{\hfill $\triangle$}

%
\newcommand{\push}{_{\#}}

%手抜き
\newcommand{\textif}{\textrm{if}\,\,\,}
\newcommand{\Ric}{\textrm{Ric}}
\newcommand{\tr}{\textrm{tr}}
\newcommand{\vol}{\textrm{vol}}
\newcommand{\diam}{\textrm{diam}}
\newcommand{\supp}{\textrm{supp}}
\newcommand{\Med}{\textrm{Med}}
\newcommand{\Leb}{\textrm{Leb}}
\newcommand{\Const}{\textrm{Const}}
\newcommand{\Avg}{\textrm{Avg}}
\newcommand{\id}{\textrm{id}}
\newcommand{\Ker}{\textrm{Ker}}
\newcommand{\im}{\textrm{Im}}
\newcommand{\dil}{\textrm{dil}}
\newcommand{\Ch}{\textrm{Ch}}
\newcommand{\Lip}{\textrm{Lip}}
\newcommand{\Ent}{\textrm{Ent}}
\newcommand{\grad}{\textrm{grad}}
\newcommand{\dom}{\textrm{dom}}
\newcommand{\diag}{\textrm{diag}}

\renewcommand{\;}{\, ; \,}
\renewcommand{\d}{\, {d}}

\newcommand{\gyouretsu}[1]{\begin{pmatrix} #1 \end{pmatrix} }

\renewcommand{\div}{\textrm{div}}


%%図式

\usepackage[dvipdfm,all]{xy}


\newenvironment{claim}[1]{\par\noindent\underline{step:}\space#1}{}
\newenvironment{claimproof}[1]{\par\noindent{($\because$)}\space#1}{\hfill $\blacktriangle $}


\newcommand{\pprime}{{\prime \prime}}

%%マグニチュード


\newcommand{\Mag}{\textrm{Mag}}

\usepackage{mathrsfs}


%%6.13
\def\chint#1{\mathchoice
{\XXint\displaystyle\textstyle{#1}}%
{\XXint\textstyle\scriptstyle{#1}}%
{\XXint\scriptstyle\scriptscriptstyle{#1}}%
{\XXint\scriptscriptstyle\scriptscriptstyle{#1}}%
\!\int}
\def\XXint#1#2#3{{\setbox0=\hbox{$#1{#2#3}{\int}$ }
\vcenter{\hbox{$#2#3$ }}\kern-.6\wd0}}
\def\ddashint{\chint=}
\def\dashint{\chint-}


%%7.13

\usepackage{here}

%7.15
\newcommand{\Span}{\textrm{Span}}

\newcommand{\Conv}{\textrm{Conv}}

%7.27

%9.4
\newcommand{\sing}{\textrm{sing}}



\title{}
\date{}


\author{}


\begin{document}


\maketitle

\section{}


$x \in \mathbb R, r >0$ に対して
\begin{align*}  B(x; r) \coloneqq \cbra{y \in \mathbb R \mid \abs{x -y} < r}  \end{align*}
と定める. 

(1) $\cbra{B(x; 1) }_{x \in F}$ は$F$ の開被覆であり, $F$ はコンパクトであるので, 
適当な$x_1, \ldots , x_N \in F$ で
\begin{align*} F \subset B(x_1; 1) \cup B(x_2; 1) \cup \cdots \cup B(x_N ; 1) \end{align*}
を満たすものが存在する. $x_1, \ldots, x_N$ のうち最大の実数を$x_M$ とし, 最小の実数を$x_m$ とする.
\begin{align*} R \coloneqq \max \cbra{ \abs{x_M}, \abs{x_m}}\end{align*}
と定めると, 
\begin{align*} F \subset B(x_1; 1) \cup B(x_2; 1) \cup \cdots \cup B(x_N ; 1)  \subset B(0; R+1)\end{align*}
が成り立つので, $F$ は有界集合である. 
\\

(2)$x \in F^c$ を任意にとる. $y \in F$ に対して, $B(x; \abs{x-y}/3) \cap B(y; \abs{x-y}/3) = \varnothing$ が成り立つ. $\cbra{B(y; \abs{x-y}/3)}_{y \in F}$ は$F$ の開被覆であり, $F$はコンパクトであるので, 
適当な$y_1, \ldots, y_N \in F$ で$F \subset B(y_1; \abs{x - y_1}/3) \cup B(y_2; \abs{x - y_2}/3)  \cup \cdots \cup B(y_N; \abs{x - y_N}/3) $ を満たすものがとれる. 
\begin{align*} U_x \coloneqq  B(x; \abs{x - y_1}/3) \cap B(x; \abs{x - y_2}/3) \cap \cdots \cap B(x; \abs{x - y_N}/3)  \end{align*}
と定めると, 
\begin{align*} \paren{B(y_1; \abs{x - y_1}/3) \cup B(y_2; \abs{x - y_2}/3)  \cup \cdots \cup B(y_N; \abs{x - y_N}/3) } \cap U_x = \varnothing \end{align*}
であるので, $F \cap U_x = \varnothing $ が成り立つ. $R \coloneqq \min \cbra{\abs{x - y_1}, \abs{x - y_2}, \ldots, \abs{x - y_N} }$ とすると, 
\begin{align*} B(x; R/3) \subset U_x\end{align*}
が成り立つので, $U_x$ は$x$ を含む開集合である. 従って, $F^c$ は開集合であるので, $F$ は閉集合である. \\

(3)$F^c$ は開集合なので, 任意の点$q \in \mathbb F^c$ に対して$\delta_q > 0$ で$B(q; \delta_q) \subset F^c$ を満たすものが存在する. $\bigcup_{q \in F^c \cap \mathbb Q} B(q; \delta_q) = F^c$ であるので, $F^c$ は高々可算個の開区間の和で表される. 高々可算個である集合族$\cbra{B(q; \delta_q)}_{q \in F^c \cap \mathbb Q}$ をあらためて$\mathcal B = \cbra{B_1, B_2, \ldots , B_n, \ldots }$ と表すことにする. 
$B_k, B_l \in \mathcal B$ に対して
適当な有限個の$B_{k_0}, \ldots , B_{k_N} \in \mathcal B$ で, 
\begin{align*} B_{k_0} = B_k, B_{k_1} \cap B_{k_2} \neq \varnothing, \ldots, B_{k_{N-1}} \cap B_{k_N} \neq \varnothing, B_{k_N} = B_l\end{align*}
を満たすものが存在する時に
\begin{align*} B_k \sim B_l \end{align*}
であると定めると, これは同値関係である. $[B] \in \mathcal B / \sim$ に対して
\begin{align*} U_{[B] }\coloneqq \bigcup_{B \in [B]} B \end{align*}
と定めると, $[B] \neq [B^\prime ] $ であれば, $U_{[B]} \cap U_{[B^\prime]} = \varnothing$ である. 
\begin{align*} F^c = \bigsqcup_{[B] \in \mathcal B / \sim} U_{[B]}   \end{align*}
が成り立つので, $F^c$ は高々可算個の互いに交わらない開集合の和で表される. 
\end{document}