\documentclass[10pt, fleqn, label-section=none]{bxjsarticle}

%\usepackage[driver=dvipdfm,hmargin=25truemm,vmargin=25truemm]{geometry}

\setpagelayout{driver=dvipdfm,hmargin=25truemm,vmargin=20truemm}


\usepackage{amsmath}
\usepackage{amssymb}
\usepackage{amsfonts}
\usepackage{amsthm}
\usepackage{mathtools}
\usepackage{mleftright}

\usepackage{ascmac}




\usepackage{otf}

\theoremstyle{definition}
\newtheorem{dfn}{定義}[section]
\newtheorem{ex}[dfn]{例}
\newtheorem{lem}[dfn]{補題}
\newtheorem{prop}[dfn]{命題}
\newtheorem{thm}[dfn]{定理}
\newtheorem{setting}[dfn]{設定}
\newtheorem{cor}[dfn]{系}
\newtheorem*{pf*}{証明}
\newtheorem{problem}[dfn]{問題}
\newtheorem*{problem*}{問題}
\newtheorem{remark}[dfn]{注意}
\newtheorem*{claim*}{\underline{claim}}



\newtheorem*{solution*}{解答}

%箇条書きの様式
\renewcommand{\labelenumi}{(\arabic{enumi})}


%

\newcommand{\forany}{\rm{for} \ {}^{\forall}}
\newcommand{\foranyeps}{
\rm{for} \ {}^{\forall}\varepsilon >0}
\newcommand{\foranyk}{
\rm{for} \ {}^{\forall}k}


\newcommand{\any}{{}^{\forall}}
\newcommand{\suchthat}{\, \rm{s.t.} \, \it{}}




\newcommand{\veps}{\varepsilon}
\newcommand{\paren}[1]{\mleft( #1\mright )}
\newcommand{\cbra}[1]{\mleft\{#1\mright\}}
\newcommand{\sbra}[1]{\mleft\lbrack#1\mright\rbrack}
\newcommand{\tbra}[1]{\mleft\langle#1\mright\rangle}
\newcommand{\abs}[1]{\left|#1\right|}
\newcommand{\norm}[1]{\left\|#1\right\|}
\newcommand{\lopen}[1]{\mleft(#1\mright\rbrack}
\newcommand{\ropen}[1]{\mleft\lbrack #1 \mright)}



%
\newcommand{\Rn}{\mathbb{R}^n}
\newcommand{\Cn}{\mathbb{C}^n}

\newcommand{\Rm}{\mathbb{R}^m}
\newcommand{\Cm}{\mathbb{C}^m}


\newcommand{\projs}[2]{\it{p}_{#1,\ldots,#2}}
\newcommand{\projproj}[2]{\it{p}_{#1,#2}}

\newcommand{\proj}[1]{p_{#1}}

%可測空間
\newcommand{\stdProbSp}{\paren{\Omega, \mathcal{F}, P}}

%微分作用素
\newcommand{\ddt}{\frac{d}{dt}}
\newcommand{\ddx}{\frac{d}{dx}}
\newcommand{\ddy}{\frac{d}{dy}}

\newcommand{\delt}{\frac{\partial}{\partial t}}
\newcommand{\delx}{\frac{\partial}{\partial x}}

%ハイフン
\newcommand{\hyphen}{\text{-}}

%displaystyle
\newcommand{\dstyle}{\displaystyle}

%⇔, ⇒, \UTF{21D0}%
\newcommand{\LR}{\Leftrightarrow}
\newcommand{\naraba}{\Rightarrow}
\newcommand{\gyaku}{\Leftarrow}

%理由
\newcommand{\naze}[1]{\paren{\because {\mathop{ #1 }}}}

%
\newcommand{\sankaku}{\hfill $\triangle$}

%
\newcommand{\push}{_{\#}}

%手抜き
\newcommand{\textif}{\textrm{if}\,\,\,}
\newcommand{\Ric}{\textrm{Ric}}
\newcommand{\tr}{\textrm{tr}}
\newcommand{\vol}{\textrm{vol}}
\newcommand{\diam}{\textrm{diam}}
\newcommand{\supp}{\textrm{supp}}
\newcommand{\Med}{\textrm{Med}}
\newcommand{\Leb}{\textrm{Leb}}
\newcommand{\Const}{\textrm{Const}}
\newcommand{\Avg}{\textrm{Avg}}
\newcommand{\id}{\textrm{id}}
\newcommand{\Ker}{\textrm{Ker}}
\newcommand{\im}{\textrm{Im}}
\newcommand{\dil}{\textrm{dil}}
\newcommand{\Ch}{\textrm{Ch}}
\newcommand{\Lip}{\textrm{Lip}}
\newcommand{\Ent}{\textrm{Ent}}
\newcommand{\grad}{\textrm{grad}}
\newcommand{\dom}{\textrm{dom}}

\renewcommand{\;}{\, ; \,}
\renewcommand{\d}{\, {d}}

\newcommand{\gyouretsu}[1]{\begin{pmatrix} #1 \end{pmatrix} }


%%図式

\usepackage[dvipdfm,all]{xy}


\newenvironment{claim}[1]{\par\noindent\underline{step:}\space#1}{}
\newenvironment{claimproof}[1]{\par\noindent{($\because$)}\space#1}{\hfill $\blacktriangle $}


\newcommand{\pprime}{{\prime \prime}}





%%


\title{$S^2$の基本群}
\date{}


\author{}


\begin{document}


\maketitle

\section{}


\begin{prop}
$X$ が基点$p \in X$ を含む弧状連結な開集合$\cbra{U_\lambda}$ の和で表され, 任意の共通部分$U_\lambda \cap U_ \lambda \mu$ は弧状連結であるとする. このとき, $X$ の任意のループは, 適当な$U_\lambda$ に含まれるループの連結とホモトピックである. (たとえば, $U_\lambda$ に含まれるループと, $U_\mu$ に含まれるループの連結など.) 
\end{prop}
\begin{pf*}
$p\in X $ を基点とするループ$f: [0, 1] \rightarrow X$ をとる. $f([0,1])$ はコンパクトなので,  有限部分被覆$\cbra{U_i}$ がとれる. 任意の $t \in [0, 1]$ に対して$f(t)$ は適当な$U_i$ に含まれているので, $f$ の連続性から, $t$ の開近$V_t$で, $f(V_t)$ が$U_i$ に含まれるものがとれる. $(a, b) = V_t$ の境界の点$a, b$ の像$f(a), f(b)$が境界$\partial U_i$ に属する場合は, $(a, b)$ を十分小さく縮めて $(a + \veps, b - \veps)$ とし, これを改めて$V_t$ とすることで, $f(\overline V_t) \subset U_i$ となるようにしておく. このような$V_t$ の族は$[0, 1]$ の開被覆であるので, コンパクト性から有限部分被覆をとる. 
\begin{align*} V_{i_1}, V_{i_2} , \ldots , V_N \end{align*}
と適当にうまく並べて, $[s_{i_2}, t_{i_2}] = V_{i_2}$ と表しておいて, 
\begin{align*} [0, t_{i_1}], [t_{i_1}, t_{i_2}]  , \ldots , [t_{N}, 1] \end{align*}
と分割することで, $f([t_{i_n}, t_{i_m}]) $ が適当な$U_\lambda$ に含まれるようにしておく. 
\begin{align*} t_{i_k}  \in V_{i_k} \cap V_{i_{k+1}} \end{align*} 
であり, $f(t_{i_k}) \in U_\alpha \cap U_\beta$ であり, $U_\alpha \cap U_\beta$ は弧状連結なので, $p$ と$f(t_{i_k}) $を結ぶ道$c_k$ がとれる. 
\begin{align*} f_1 \coloneqq [0, t_{i_1}], f_2 \coloneqq f|_{[t_{i_1}, t_{i_2}]}, \ldots , f_{N+1} \coloneqq f|_{[t_{N}, 1]} \end{align*}
と定め, 
\begin{align*} f_1 \natural \bar c_1 \natural c_1 \natural f_2 \natural \bar c_2 \natural c_2 \natural f_3 \cdots \natural c_{N}  \natural f_{N+1}  \end{align*}
を考えると, これがもとめる
\qed
\end{pf*}


\begin{prop}
\begin{align*} \pi_1 (S^2) = 0 \end{align*}
\end{prop}
\begin{pf*}
$S^2$ は適当に赤道上に基点$p$ をとって, 北半球を少し広げたもの($N$ で表す)と, 南半球を少し広げたもの($S$ で表す)で被覆する. それらは前述の命題の条件を満たしているので$S^2$ の任意のループは$N$ のループと$S$ のループの連結とホモトピックである. $N, S$ はともに$\mathbb R^2$ と同相であるので, これらのループは可縮である. 従って, $S^2$ の任意のループは自明なループとホモトピックである. 
\qed
\end{pf*}






\end{document}