\documentclass[10pt, fleqn, label-section=none]{bxjsarticle}

%\usepackage[driver=dvipdfm,hmargin=25truemm,vmargin=25truemm]{geometry}

\setpagelayout{driver=dvipdfm,hmargin=25truemm,vmargin=20truemm}


\usepackage{amsmath}
\usepackage{amssymb}
\usepackage{amsfonts}
\usepackage{amsthm}
\usepackage{mathtools}
\usepackage{mleftright}

\usepackage{ascmac}




\usepackage{otf}

\theoremstyle{definition}
\newtheorem{dfn}{定義}[section]
\newtheorem{ex}[dfn]{例}
\newtheorem{lem}[dfn]{補題}
\newtheorem{prop}[dfn]{命題}
\newtheorem{thm}[dfn]{定理}
\newtheorem{setting}[dfn]{設定}
\newtheorem{notation}[dfn]{記号}
\newtheorem{cor}[dfn]{系}
\newtheorem*{pf*}{証明}
\newtheorem{problem}[dfn]{問題}
\newtheorem*{problem*}{問題}
\newtheorem{remark}[dfn]{注意}
\newtheorem*{claim*}{\underline{claim}}



\newtheorem*{solution*}{解答}

%箇条書きの様式
\renewcommand{\labelenumi}{(\arabic{enumi})}


%

\newcommand{\forany}{\rm{for} \ {}^{\forall}}
\newcommand{\foranyeps}{
\rm{for} \ {}^{\forall}\varepsilon >0}
\newcommand{\foranyk}{
\rm{for} \ {}^{\forall}k}


\newcommand{\any}{{}^{\forall}}
\newcommand{\suchthat}{\, \rm{s.t.} \, \it{}}




\newcommand{\veps}{\varepsilon}
\newcommand{\paren}[1]{\mleft( #1\mright )}
\newcommand{\cbra}[1]{\mleft\{#1\mright\}}
\newcommand{\sbra}[1]{\mleft\lbrack#1\mright\rbrack}
\newcommand{\tbra}[1]{\mleft\langle#1\mright\rangle}
\newcommand{\abs}[1]{\left|#1\right|}
\newcommand{\norm}[1]{\left\|#1\right\|}
\newcommand{\lopen}[1]{\mleft(#1\mright\rbrack}
\newcommand{\ropen}[1]{\mleft\lbrack #1 \mright)}



%
\newcommand{\Rn}{\mathbb{R}^n}
\newcommand{\Cn}{\mathbb{C}^n}

\newcommand{\Rm}{\mathbb{R}^m}
\newcommand{\Cm}{\mathbb{C}^m}


\newcommand{\projs}[2]{\it{p}_{#1,\ldots,#2}}
\newcommand{\projproj}[2]{\it{p}_{#1,#2}}

\newcommand{\proj}[1]{p_{#1}}

%可測空間
\newcommand{\stdProbSp}{\paren{\Omega, \mathcal{F}, P}}

%微分作用素
\newcommand{\ddt}{\frac{d}{dt}}
\newcommand{\ddx}{\frac{d}{dx}}
\newcommand{\ddy}{\frac{d}{dy}}

\newcommand{\delt}{\frac{\partial}{\partial t}}
\newcommand{\delx}{\frac{\partial}{\partial x}}

%ハイフン
\newcommand{\hyphen}{\text{-}}

%displaystyle
\newcommand{\dstyle}{\displaystyle}

%⇔, ⇒, \UTF{21D0}%
\newcommand{\LR}{\Leftrightarrow}
\newcommand{\naraba}{\Rightarrow}
\newcommand{\gyaku}{\Leftarrow}

%理由
\newcommand{\naze}[1]{\paren{\because {\mathop{ #1 }}}}

%
\newcommand{\sankaku}{\hfill $\triangle$}

%
\newcommand{\push}{_{\#}}

%手抜き
\newcommand{\textif}{\textrm{if}\,\,\,}
\newcommand{\Ric}{\textrm{Ric}}
\newcommand{\tr}{\textrm{tr}}
\newcommand{\vol}{\textrm{vol}}
\newcommand{\diam}{\textrm{diam}}
\newcommand{\supp}{\textrm{supp}}
\newcommand{\Med}{\textrm{Med}}
\newcommand{\Leb}{\textrm{Leb}}
\newcommand{\Const}{\textrm{Const}}
\newcommand{\Avg}{\textrm{Avg}}
\newcommand{\id}{\textrm{id}}
\newcommand{\Ker}{\textrm{Ker}}
\newcommand{\im}{\textrm{Im}}
\newcommand{\dil}{\textrm{dil}}
\newcommand{\Ch}{\textrm{Ch}}
\newcommand{\Lip}{\textrm{Lip}}
\newcommand{\Ent}{\textrm{Ent}}
\newcommand{\grad}{\textrm{grad}}
\newcommand{\dom}{\textrm{dom}}
\newcommand{\diag}{\textrm{diag}}

\renewcommand{\;}{\, ; \,}
\renewcommand{\d}{\, {d}}

\newcommand{\gyouretsu}[1]{\begin{pmatrix} #1 \end{pmatrix} }

\renewcommand{\div}{\textrm{div}}


%%図式

\usepackage[dvipdfm,all]{xy}


\newenvironment{claim}[1]{\par\noindent\underline{step:}\space#1}{}
\newenvironment{claimproof}[1]{\par\noindent{($\because$)}\space#1}{\hfill $\blacktriangle $}


\newcommand{\pprime}{{\prime \prime}}

%%マグニチュード


\newcommand{\Mag}{\textrm{Mag}}

\usepackage{mathrsfs}


%%6.13
\def\chint#1{\mathchoice
{\XXint\displaystyle\textstyle{#1}}%
{\XXint\textstyle\scriptstyle{#1}}%
{\XXint\scriptstyle\scriptscriptstyle{#1}}%
{\XXint\scriptscriptstyle\scriptscriptstyle{#1}}%
\!\int}
\def\XXint#1#2#3{{\setbox0=\hbox{$#1{#2#3}{\int}$ }
\vcenter{\hbox{$#2#3$ }}\kern-.6\wd0}}
\def\ddashint{\chint=}
\def\dashint{\chint-}


%%7.13

\usepackage{here}

%7.15
\newcommand{\Span}{\textrm{Span}}

\newcommand{\Conv}{\textrm{Conv}}

%7.27

%9.4
\newcommand{\sing}{\textrm{sing}}

%
\newcommand{\C}[2]{{}_{#1}C_{#2} }


\title{ユークリッド空間の地面に対するトレース作用素}
\date{}


\author{}


\begin{document}


\maketitle

\section{}

\begin{notation}$x = (x^\prime , x_n) \subset \mathbb R^n$ という表記をとることにする. 

\end{notation}

\begin{prop}$s < -1/2$ とする. このとき, 

\begin{align*} \int_{\mathbb R} \tbra{x}^{2s} dx_n = \tbra{x^\prime }^{2s + 1} \int_{\mathbb R} \tbra{t}^{2s} dt \end{align*}

が成り立つ.

\end{prop}
\begin{pf*}

\begin{align*} x_n = \tbra{x^\prime } t\end{align*}
により変数変換すると, 

\begin{align*} \tbra{x}^2 ~ \tbra{x^\prime } ^2 + x_n ^2 = \tbra{x^\prime} ^2 + \tbra{x^\prime}^2 t^2 = \paren{1 + t^2} \tbra{x^\prime} ^2 = \tbra{t}^2 \tbra{x^\prime} ^2 \end{align*}

\begin{align*}  \int_{\mathbb R} \tbra{x}^{2s} dx_n = \int \tbra{t}^{2s}\tbra{x^\prime}^{2s} \tbra{x^\prime} dt  \end{align*}

が成り立つ. 

\qed
\end{pf*}

\begin{notation}
\begin{align*} \Gamma_0 : C^\infty(\mathbb R^n) \rightarrow C^\infty(\mathbb R^{n-1}) \end{align*}

を, 

\begin{align*} \Gamma_0 f (x^\prime ) \coloneqq f(x^\prime, 0)\end{align*}

により定める. 
\end{notation}

\begin{prop}$f \in C_c^\infty (\mathbb R^n)$ に対して, 

\begin{align*} F(\Gamma_0 f) (\xi ^\prime) = \int_{\mathbb R} Ff(\xi^\prime, \xi_n) d\xi_n \end{align*}

が成り立つ. 

\end{prop}
\begin{pf*}

\begin{align*} \Gamma_0 f (x^\prime) &= \frac{1}{(2 \pi)^{n/2}} \int_{\mathbb R^n} Ff(\xi)e^{i(x^\prime, 0) \xi} d\xi 
\\& = \frac{1}{(2 \pi)^{n/2}} \int_{\mathbb R^{n-1}} \paren{\int_{\mathbb R}Ff(\xi^\prime, \xi_n) d\xi_n }e^{ix^\prime \xi ^\prime} d\xi^\prime  \end{align*}

であるので, 

\begin{align*} F(\Gamma_0 f) (\xi ^\prime) = \int_{\mathbb R} Ff(\xi^\prime, \xi_n) d\xi_n \end{align*}

が成り立つ. 


\qed
\end{pf*}


\begin{prop}$s > 1/2$ とする. $f \in C_c^\infty (\mathbb R^n) $ に対して, 

\begin{align*} \norm{\Gamma_0 f}_{H_{s - 1/2}(\mathbb R^{n-1})} \lesssim_s  \norm{f}_{H_s(\mathbb R^n)}  \end{align*}

が成り立つ. 

\end{prop}
\begin{pf*}

\begin{align*} F(\Gamma_0 f) (\xi ^\prime) &= \int_{\mathbb R} Ff(\xi^\prime, \xi_n) d\xi_n  \\& = \int \tbra{\xi}^{-s} \tbra{\xi}^s Ff(\xi) d \xi_n  \end{align*}
であるので, 

\begin{align*} \abs{F(\Gamma_0 f) (\xi ^\prime)}^2 &\leq \int_{\mathbb R} \tbra{\xi}^{-2s} d \xi_n  \int_{\mathbb R} \tbra{\xi}^{2s} (Ff(\xi^\prime, \xi_n) )^2d\xi_n 
\\&= \tbra{\xi}^{-2s + 1} \int_{\mathbb R } \tbra{t}^{-2s} dt \int_{\mathbb R}  \tbra{\xi}^{2s} (Ff(\xi^\prime, \xi_n) )^2 d\xi_n    \end{align*}

であるので, 

\begin{align*} \abs{ \tbra{\xi}^{2s - 1} \abs{F(\Gamma_0 f) (\xi ^\prime)}^2} \leq  \int_{\mathbb R } \tbra{t}^{-2s} dt \int_{\mathbb R}  \tbra{\xi}^{2s} Ff(\xi^\prime, \xi_n) d\xi_n   \end{align*}

両辺を積分することで, 

\begin{align*} \int_{\mathbb R^n }\tbra{\xi}^{2s - 1}   \abs{F(\Gamma_0 f) (\xi ^\prime)}^2  d\xi^\prime &\lesssim_s \int_{\mathbb R^{n-1}}\paren{  \int_{\mathbb R}  \tbra{\xi}^{2s} (Ff(\xi^\prime, \xi_n))^2 d\xi_n  } d\xi^\prime 
\\&= \int_{\mathbb R^n} \tbra{\xi}^{2s} (Ff(\xi))^2 d\xi   \end{align*}

より主張が従う. 

\qed
\end{pf*}







\end{document}