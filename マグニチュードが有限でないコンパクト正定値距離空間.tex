\documentclass[10pt, fleqn, label-section=none]{bxjsarticle}

%\usepackage[driver=dvipdfm,hmargin=25truemm,vmargin=25truemm]{geometry}

\setpagelayout{driver=dvipdfm,hmargin=25truemm,vmargin=20truemm}


\usepackage{amsmath}
\usepackage{amssymb}
\usepackage{amsfonts}
\usepackage{amsthm}
\usepackage{mathtools}
\usepackage{mleftright}

\usepackage{ascmac}




\usepackage{otf}

\theoremstyle{definition}
\newtheorem{dfn}{定義}[section]
\newtheorem{ex}[dfn]{例}
\newtheorem{lem}[dfn]{補題}
\newtheorem{prop}[dfn]{命題}
\newtheorem{thm}[dfn]{定理}
\newtheorem{setting}[dfn]{設定}
\newtheorem{notation}[dfn]{記号}
\newtheorem{cor}[dfn]{系}
\newtheorem*{pf*}{証明}
\newtheorem{problem}[dfn]{問題}
\newtheorem*{problem*}{問題}
\newtheorem{remark}[dfn]{注意}
\newtheorem*{claim*}{\underline{claim}}



\newtheorem*{solution*}{解答}

%箇条書きの様式
\renewcommand{\labelenumi}{(\arabic{enumi})}


%

\newcommand{\forany}{\rm{for} \ {}^{\forall}}
\newcommand{\foranyeps}{
\rm{for} \ {}^{\forall}\varepsilon >0}
\newcommand{\foranyk}{
\rm{for} \ {}^{\forall}k}


\newcommand{\any}{{}^{\forall}}
\newcommand{\suchthat}{\, \rm{s.t.} \, \it{}}




\newcommand{\veps}{\varepsilon}
\newcommand{\paren}[1]{\mleft( #1\mright )}
\newcommand{\cbra}[1]{\mleft\{#1\mright\}}
\newcommand{\sbra}[1]{\mleft\lbrack#1\mright\rbrack}
\newcommand{\tbra}[1]{\mleft\langle#1\mright\rangle}
\newcommand{\abs}[1]{\left|#1\right|}
\newcommand{\norm}[1]{\left\|#1\right\|}
\newcommand{\lopen}[1]{\mleft(#1\mright\rbrack}
\newcommand{\ropen}[1]{\mleft\lbrack #1 \mright)}



%
\newcommand{\Rn}{\mathbb{R}^n}
\newcommand{\Cn}{\mathbb{C}^n}

\newcommand{\Rm}{\mathbb{R}^m}
\newcommand{\Cm}{\mathbb{C}^m}


\newcommand{\projs}[2]{\it{p}_{#1,\ldots,#2}}
\newcommand{\projproj}[2]{\it{p}_{#1,#2}}

\newcommand{\proj}[1]{p_{#1}}

%可測空間
\newcommand{\stdProbSp}{\paren{\Omega, \mathcal{F}, P}}

%微分作用素
\newcommand{\ddt}{\frac{d}{dt}}
\newcommand{\ddx}{\frac{d}{dx}}
\newcommand{\ddy}{\frac{d}{dy}}

\newcommand{\delt}{\frac{\partial}{\partial t}}
\newcommand{\delx}{\frac{\partial}{\partial x}}

%ハイフン
\newcommand{\hyphen}{\text{-}}

%displaystyle
\newcommand{\dstyle}{\displaystyle}

%⇔, ⇒, \UTF{21D0}%
\newcommand{\LR}{\Leftrightarrow}
\newcommand{\naraba}{\Rightarrow}
\newcommand{\gyaku}{\Leftarrow}

%理由
\newcommand{\naze}[1]{\paren{\because {\mathop{ #1 }}}}

%
\newcommand{\sankaku}{\hfill $\triangle$}

%
\newcommand{\push}{_{\#}}

%手抜き
\newcommand{\textif}{\textrm{if}\,\,\,}
\newcommand{\Ric}{\textrm{Ric}}
\newcommand{\tr}{\textrm{tr}}
\newcommand{\vol}{\textrm{vol}}
\newcommand{\diam}{\textrm{diam}}
\newcommand{\supp}{\textrm{supp}}
\newcommand{\Med}{\textrm{Med}}
\newcommand{\Leb}{\textrm{Leb}}
\newcommand{\Const}{\textrm{Const}}
\newcommand{\Avg}{\textrm{Avg}}
\newcommand{\id}{\textrm{id}}
\newcommand{\Ker}{\textrm{Ker}}
\newcommand{\im}{\textrm{Im}}
\newcommand{\dil}{\textrm{dil}}
\newcommand{\Ch}{\textrm{Ch}}
\newcommand{\Lip}{\textrm{Lip}}
\newcommand{\Ent}{\textrm{Ent}}
\newcommand{\grad}{\textrm{grad}}
\newcommand{\dom}{\textrm{dom}}
\newcommand{\diag}{\textrm{diag}}

\renewcommand{\;}{\, ; \,}
\renewcommand{\d}{\, {d}}

\newcommand{\gyouretsu}[1]{\begin{pmatrix} #1 \end{pmatrix} }

\renewcommand{\div}{\textrm{div}}


%%図式

\usepackage[dvipdfm,all]{xy}


\newenvironment{claim}[1]{\par\noindent\underline{step:}\space#1}{}
\newenvironment{claimproof}[1]{\par\noindent{($\because$)}\space#1}{\hfill $\blacktriangle $}


\newcommand{\pprime}{{\prime \prime}}

%%マグニチュード


\newcommand{\Mag}{\textrm{Mag}}

\usepackage{mathrsfs}


%%6.13
\def\chint#1{\mathchoice
{\XXint\displaystyle\textstyle{#1}}%
{\XXint\textstyle\scriptstyle{#1}}%
{\XXint\scriptstyle\scriptscriptstyle{#1}}%
{\XXint\scriptscriptstyle\scriptscriptstyle{#1}}%
\!\int}
\def\XXint#1#2#3{{\setbox0=\hbox{$#1{#2#3}{\int}$ }
\vcenter{\hbox{$#2#3$ }}\kern-.6\wd0}}
\def\ddashint{\chint=}
\def\dashint{\chint-}


%%7.13

\usepackage{here}

%7.15
\newcommand{\Span}{\textrm{Span}}

\newcommand{\Conv}{\textrm{Conv}}

%7.27




\title{マグニチュードが有限でないコンパクト正定値距離空間}
\date{}


\author{}


\begin{document}


\maketitle

\section{}

\begin{prop}$l_1$ 距離を備えた数列空間において, 
\begin{align*} 
\\&J_\infty \coloneqq \cbra{\frac{\pi^2}{6}} \times \cbra{0} \times \cbra{0} \times \cdots 
\\&J_1 \coloneqq [0, 1] \times \cbra{0} \times \cbra{0} \times \cdots  
\\& J_2 \coloneqq [1, 1 + \frac{1}{2^2}] \times [0, \frac{1}{2^2}] \times \cbra{0} \times \cbra{0} \times \cdots 
\\&J_3 \coloneqq [1 + \frac{1}{2^2}, 1 + \frac{1}{2^2} + \frac{1}{3^2} ] \times \cbra{0} \times [0, \frac{1}{3^2}]  \times [0, \frac{1}{3^2}] \times \cbra{0} \times \cdots 
\\& \cdots
\end{align*}
と定めると, 
\begin{align*} J_\infty \cup \paren{\bigcup_{i \in \mathbb N} J_i }  \end{align*}
はマグニチュードが有限でないコンパクト正定値距離空間である. 
\end{prop}
\begin{pf*}任意に点列$p_n$をとると, $(\frac{\pi^2}{6}, 0, \ldots )$ が集積点でないときは, 適当に十分大きな$N \in \mathbb N$ に対して $\cbra{p_n}\subset  \bigcup_{i \in \mathbb N} J_i$ であり, $\bigcup_{i \in \mathbb N} J_i$ はコンパクトであるので収束部分列をもつ. $(\frac{\pi^2}{6}, 0, \ldots, )$ が集積点であるとき, $\frac{\pi^2}{6}$ に収束する収束部分列をもつ. また, 
\begin{align*} z \coloneqq (\sum_{i= 1}^n \frac{1}{k^2}, 0, 0, \ldots  )  \in \paren{\bigcup_{i = 1}^{n } J_i } \cap J_{n+1} \end{align*}
と定めると, 任意に
\begin{align*}x =  (x_1, x_2, \ldots , x_m, 0, 0, \ldots, ) \in \bigcup_{i = 1}^{n } J_i  \end{align*}
をとると(ただし, $m  \coloneqq 1 + 2 + \cdots + n-1$), 任意の点$y = (y_1, 0, \ldots , 0 , y_{m +1}, y_{m+2}, \ldots, y_{m+n}, 0, \ldots )  \in J_{n + 1}$ に対して, 

\begin{align*} d(x, y) &= \abs{y_1 - x_1} + \abs{x_2} +  \cdots +  \abs{x_m} + \abs{y_{m+1}} + \cdots \abs{y_{m+n}} \\&= \abs{\sum_{i= 1}^n \frac{1}{k^2} - x_1} + \abs{y_{m+1} - \sum_{i= 1}^n \frac{1}{k^2} }  + \abs{0 - x_2} +  \cdots +  \abs{0- x_m} +  \abs{y_{m+1} - 0 } + \cdots \abs{y_{m+n} - 0} 
\\&=  d(x, z) + d(z, y)   \end{align*}
が成り立つので, $\bigcup_{i = 1}^{n } J_i$ は$J_{n+1}$ に対して良く射影される. 同様にして$J_{n+1}$ が$\bigcup_{i = 1}^{n } J_i$ に良く射影されるので, 互いに良く射影される. 従って, マグニチュードを計算すると, 適当な$N \in \mathbb N $ に対して
\begin{align*} &\Mag \paren{J_\infty \cup \paren{\bigcup_{i \in \mathbb N} J_i } } 
\\& \geq \Mag (\paren{\bigcup_{i = 1}^N J_i } )
\\&= (1 + \frac{1}{2}) + (1 + \frac{1}{2}\cdot \frac{1}{2^2} )^2 - 1 + (1 + \frac{1}{2}\cdot \frac{1}{3^2} )^3 - 1 + (1 + \frac{1}{2}\cdot \frac{1}{4^2} )^4 -1 + \cdots +(1 + \frac{1}{2}\cdot \frac{1}{N^2} )^N -1 
\\&\geq 1 + \frac{1}{2} (1 +  2 \cdot \frac{1}{2^2} + 3 \cdot \frac{1}{3^2} + \cdots N \cdot \frac{1}{N^2} )     \end{align*}
が成り立つので, マグニチュードは有限ではない. 
\qed
\end{pf*}





\end{document}

