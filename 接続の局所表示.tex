\documentclass[10pt, fleqn, label-section=none]{bxjsarticle}

%\usepackage[driver=dvipdfm,hmargin=25truemm,vmargin=25truemm]{geometry}

\setpagelayout{driver=dvipdfm,hmargin=25truemm,vmargin=20truemm}


\usepackage{amsmath}
\usepackage{amssymb}
\usepackage{amsfonts}
\usepackage{amsthm}
\usepackage{mathtools}
\usepackage{mleftright}

\usepackage{ascmac}




\usepackage{otf}

\theoremstyle{definition}
\newtheorem{dfn}{定義}[section]
\newtheorem{ex}[dfn]{例}
\newtheorem{lem}[dfn]{補題}
\newtheorem{prop}[dfn]{命題}
\newtheorem{thm}[dfn]{定理}
\newtheorem{cor}[dfn]{系}
\newtheorem*{pf*}{証明}
\newtheorem{problem}[dfn]{問題}
\newtheorem*{problem*}{問題}
\newtheorem{remark}[dfn]{注意}
\newtheorem*{claim*}{\underline{claim}}



\newtheorem*{solution*}{解答}

%箇条書きの様式
\renewcommand{\labelenumi}{(\arabic{enumi})}


%

\newcommand{\forany}{\rm{for} \ {}^{\forall}}
\newcommand{\foranyeps}{
\rm{for} \ {}^{\forall}\varepsilon >0}
\newcommand{\foranyk}{
\rm{for} \ {}^{\forall}k}


\newcommand{\any}{{}^{\forall}}
\newcommand{\suchthat}{\, \rm{s.t.} \, \it{}}




\newcommand{\veps}{\varepsilon}
\newcommand{\paren}[1]{\mleft( #1\mright )}
\newcommand{\cbra}[1]{\mleft\{#1\mright\}}
\newcommand{\sbra}[1]{\mleft\lbrack#1\mright\rbrack}
\newcommand{\tbra}[1]{\mleft\langle#1\mright\rangle}
\newcommand{\abs}[1]{\left|#1\right|}
\newcommand{\norm}[1]{\left\|#1\right\|}
\newcommand{\lopen}[1]{\mleft(#1\mright\rbrack}
\newcommand{\ropen}[1]{\mleft\lbrack #1 \mright)}



%
\newcommand{\Rn}{\mathbb{R}^n}
\newcommand{\Cn}{\mathbb{C}^n}

\newcommand{\Rm}{\mathbb{R}^m}
\newcommand{\Cm}{\mathbb{C}^m}


\newcommand{\projs}[2]{\it{p}_{#1,\ldots,#2}}
\newcommand{\projproj}[2]{\it{p}_{#1,#2}}

\newcommand{\proj}[1]{p_{#1}}

%可測空間
\newcommand{\stdProbSp}{\paren{\Omega, \mathcal{F}, P}}

%微分作用素
\newcommand{\ddt}{\frac{d}{dt}}
\newcommand{\ddx}{\frac{d}{dx}}
\newcommand{\ddy}{\frac{d}{dy}}

\newcommand{\delt}{\frac{\partial}{\partial t}}
\newcommand{\delx}{\frac{\partial}{\partial x}}

%ハイフン
\newcommand{\hyphen}{\text{-}}

%displaystyle
\newcommand{\dstyle}{\displaystyle}

%⇔, ⇒, \UTF{21D0}%
\newcommand{\LR}{\Leftrightarrow}
\newcommand{\naraba}{\Rightarrow}
\newcommand{\gyaku}{\Leftarrow}

%理由
\newcommand{\naze}[1]{\paren{\because {\mathop{ #1 }}}}

%
\newcommand{\sankaku}{\hfill $\triangle$}

%
\newcommand{\push}{_{\#}}

%手抜き
\newcommand{\textif}{\textrm{if}\,\,\,}
\newcommand{\Ric}{\textrm{Ric}}
\newcommand{\tr}{\textrm{tr}}
\newcommand{\vol}{\textrm{vol}}
\newcommand{\diam}{\textrm{diam}}
\newcommand{\supp}{\textrm{supp}}
\newcommand{\Med}{\textrm{Med}}
\newcommand{\Leb}{\textrm{Leb}}
\newcommand{\Const}{\textrm{Const}}
\newcommand{\Avg}{\textrm{Avg}}



\renewcommand{\;}{\, ; \,}
\renewcommand{\d}{\, {d}}



\title{接続の局所表示と第二構造方程式}
\date{}


\author{}


\begin{document}


\maketitle



\section{}

\begin{remark}
$M$ で滑らかな多様体を表す. $\pi: E \rightarrow M$ で階数$r$ の滑らかなベクトル束を表す. $U \subset M$ を開集合, $e_1, \ldots, e_r$ を$E$ の$U$ における局所フレームとする. 
\end{remark}

\begin{dfn}(接続形式).
$U$ 上の$1$形式の族$\cbra{\omega_i^j}$で
\begin{align*} \nabla_X e_i = \omega_i^j (X) e_j \quad (\any X \in \mathfrak X (U))\end{align*}
を満たすものを接続形式という.
\end{dfn}

\begin{dfn}(曲率形式). 
$U$ 上の$2$形式の族$\cbra{\Omega_i^j}$ で
\begin{align*} R(X,Y)e_i = \Omega_i^j(X,Y) e_j \quad (\any X, Y \in \mathfrak X (U))\end{align*}
を満たすものを曲率形式という. 
\end{dfn}

\begin{prop}(第二構造方程式).
\begin{align*} \Omega_i^j (X,Y) = d\omega_i^j (X,Y) - \omega_i^k \wedge \omega_k^j (X,Y) \quad (\any X, Y \in \mathfrak X (U)) \end{align*}
\end{prop}
\begin{pf*}
\begin{align*} \Omega_i^j (X,Y) e_j &= R(X, Y) e_i = \nabla_X \nabla_Y e_i - \nabla_Y \nabla_X e_i - \nabla_{[X,Y]} e_i \\&= \nabla_X (\omega_i^j(Y) e_j) - \nabla_Y (\omega_i^j (X) e_j) 
\\&= (  X(\omega_i^j(Y)) e_j + \omega_i^j (Y) \omega_j^k (X) e_k   ) - (Y(\omega _i^j (X)) e_j  + \omega_i^j (X) \omega_j^k (Y) e_k  )  - \omega _i^j ([X,Y]) e_j
\\&= (  X(\omega_i^j(Y)) e_j + \omega_i^k (Y) \omega_k^j (X) e_j   ) - (Y(\omega _i^j (X)) e_j  + \omega_i^k (X) \omega_k^j (Y) e_j  )  - \omega _i^j ([X,Y]) e_j  
\\&= (  X(\omega_i^j(Y)) - Y(\omega _i^j (X))  -   \omega _i^j ([X,Y])  ) e_j + ( \omega_i^k (Y) \omega_k^j (X) - \omega_i^k (X) \omega_k^j (Y) ) e_j  
\\&= d\omega_i^j (X,Y) e_j - \omega_i^k \wedge \omega_k^j (X,Y) e_j 
\\&= (d\omega_i^j (X,Y)  - \omega_i^k \wedge \omega_k^j (X,Y) )e_j \end{align*}
\qed
\end{pf*}









\end{document}