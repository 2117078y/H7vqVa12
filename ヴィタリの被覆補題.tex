\documentclass[10pt, fleqn, label-section=none]{bxjsarticle}

%\usepackage[driver=dvipdfm,hmargin=25truemm,vmargin=25truemm]{geometry}

\setpagelayout{driver=dvipdfm,hmargin=25truemm,vmargin=20truemm}


\usepackage{amsmath}
\usepackage{amssymb}
\usepackage{amsfonts}
\usepackage{amsthm}
\usepackage{mathtools}
\usepackage{mleftright}

\usepackage{ascmac}




\usepackage{otf}

\theoremstyle{definition}
\newtheorem{dfn}{定義}[section]
\newtheorem{ex}[dfn]{例}
\newtheorem{lem}[dfn]{補題}
\newtheorem{prop}[dfn]{命題}
\newtheorem{thm}[dfn]{定理}
\newtheorem{setting}[dfn]{設定}
\newtheorem{cor}[dfn]{系}
\newtheorem*{pf*}{証明}
\newtheorem{problem}[dfn]{問題}
\newtheorem*{problem*}{問題}
\newtheorem{remark}[dfn]{注意}
\newtheorem*{claim*}{\underline{claim}}



\newtheorem*{solution*}{解答}

%箇条書きの様式
\renewcommand{\labelenumi}{(\arabic{enumi})}


%

\newcommand{\forany}{\rm{for} \ {}^{\forall}}
\newcommand{\foranyeps}{
\rm{for} \ {}^{\forall}\varepsilon >0}
\newcommand{\foranyk}{
\rm{for} \ {}^{\forall}k}


\newcommand{\any}{{}^{\forall}}
\newcommand{\suchthat}{\, \rm{s.t.} \, \it{}}




\newcommand{\veps}{\varepsilon}
\newcommand{\paren}[1]{\mleft( #1\mright )}
\newcommand{\cbra}[1]{\mleft\{#1\mright\}}
\newcommand{\sbra}[1]{\mleft\lbrack#1\mright\rbrack}
\newcommand{\tbra}[1]{\mleft\langle#1\mright\rangle}
\newcommand{\abs}[1]{\left|#1\right|}
\newcommand{\norm}[1]{\left\|#1\right\|}
\newcommand{\lopen}[1]{\mleft(#1\mright\rbrack}
\newcommand{\ropen}[1]{\mleft\lbrack #1 \mright)}



%
\newcommand{\Rn}{\mathbb{R}^n}
\newcommand{\Cn}{\mathbb{C}^n}

\newcommand{\Rm}{\mathbb{R}^m}
\newcommand{\Cm}{\mathbb{C}^m}


\newcommand{\projs}[2]{\it{p}_{#1,\ldots,#2}}
\newcommand{\projproj}[2]{\it{p}_{#1,#2}}

\newcommand{\proj}[1]{p_{#1}}

%可測空間
\newcommand{\stdProbSp}{\paren{\Omega, \mathcal{F}, P}}

%微分作用素
\newcommand{\ddt}{\frac{d}{dt}}
\newcommand{\ddx}{\frac{d}{dx}}
\newcommand{\ddy}{\frac{d}{dy}}

\newcommand{\delt}{\frac{\partial}{\partial t}}
\newcommand{\delx}{\frac{\partial}{\partial x}}

%ハイフン
\newcommand{\hyphen}{\text{-}}

%displaystyle
\newcommand{\dstyle}{\displaystyle}

%⇔, ⇒, \UTF{21D0}%
\newcommand{\LR}{\Leftrightarrow}
\newcommand{\naraba}{\Rightarrow}
\newcommand{\gyaku}{\Leftarrow}

%理由
\newcommand{\naze}[1]{\paren{\because {\mathop{ #1 }}}}

%
\newcommand{\sankaku}{\hfill $\triangle$}

%
\newcommand{\push}{_{\#}}

%手抜き
\newcommand{\textif}{\textrm{if}\,\,\,}
\newcommand{\Ric}{\textrm{Ric}}
\newcommand{\tr}{\textrm{tr}}
\newcommand{\vol}{\textrm{vol}}
\newcommand{\diam}{\textrm{diam}}
\newcommand{\supp}{\textrm{supp}}
\newcommand{\Med}{\textrm{Med}}
\newcommand{\Leb}{\textrm{Leb}}
\newcommand{\Const}{\textrm{Const}}
\newcommand{\Avg}{\textrm{Avg}}
\newcommand{\id}{\textrm{id}}
\newcommand{\Ker}{\textrm{Ker}}
\newcommand{\im}{\textrm{Im}}
\newcommand{\dil}{\textrm{dil}}
\newcommand{\Ch}{\textrm{Ch}}
\newcommand{\Lip}{\textrm{Lip}}
\newcommand{\Ent}{\textrm{Ent}}
\newcommand{\grad}{\textrm{grad}}
\newcommand{\dom}{\textrm{dom}}

\renewcommand{\;}{\, ; \,}
\renewcommand{\d}{\, {d}}

\newcommand{\gyouretsu}[1]{\begin{pmatrix} #1 \end{pmatrix} }


%%図式

\usepackage[dvipdfm,all]{xy}


\newenvironment{claim}[1]{\par\noindent\underline{step:}\space#1}{}
\newenvironment{claimproof}[1]{\par\noindent{($\because$)}\space#1}{\hfill $\blacktriangle $}


\newcommand{\pprime}{{\prime \prime}}





%%


\title{ヴィタリの被覆補題}
\date{}


\author{}


\begin{document}


\maketitle

\section{}



\begin{prop}(有限被覆補題). 距離空間$X $ で考える. $B_1, \ldots , B_N$ を適当な半径(同じとは限らない)の有限個の球とする. このとき, 部分族
\begin{align*} B_{k_1}, \ldots , B_{k_m}\end{align*}
で, 互いにdisjointで, 
\begin{align*} \cup B_i \subset 3 B_{k_j}\end{align*}
を満たし, 
任意の$B_i$ に対して$B_i \subset 3B_{k_{j(i)} }$ を満たす
ものが存在する. 
\end{prop}
\begin{pf*}
$N = 1$ のとき, 明らかに成り立つ. 帰納法 で示すわけだけど, 具体的なシチュエーションをみてみる. 正式な証明はこれを眺めてたら作れるとおもう. $N = 10$ のとき成り立つとする. $N = 11$ のときを考える. 半径最大の球が$B_9$ だったとする. $B_9$ と交わるのが$B_1,B_4, B_5, B_6, B_9$ で, $B_2, B_3, B_7, B_8,  B_{10}$は$B_9$ と交わらないとする. 帰納法の仮定から$B_2, B_3, B_7, B_8, B_{10}$ の中から条件をみたす部分族がとれる. それが$B_2, B_3, B_8$ だったとする. 
\begin{align*} &B_1 \cup B_4 \cup B_5 \cup B_6 \cup B_9 \subset 3B_9 \\ &B_2 \cup B_3 \cup B_7 \cup B_8 \cup B_{10} \subset 3 B_2 \cup 3B_3 \cup 3B_8  \end{align*}
みたいな状況になっている. つまるところ, 
\begin{align*} B_9, B_2, B_3, B_8\end{align*}
が求める部分族となる.  
\qed
\end{pf*}


\begin{prop}(無限被覆補題). 第二可算, あるいは可分な距離空間$X$ で考える. $\mathcal B$ を, 
\begin{align*} \sup \cbra{ \diam B \mid B \in \mathcal B} < \infty \end{align*}
である球の族とする. このとき, 部分族$\mathcal B ^\prime $で, $\mathcal B ^\prime $ に属する球は互いにdisjointであり, 
\begin{align*} \cup_{B \in \mathcal B} \subset \cup_{B^\prime \in \mathcal B^\prime } 5 B ^\prime \end{align*}
を満たし, 任意の$B \in \mathcal B$ に対して, $B \subset 5 B^\prime $ をみたす$B^\prime \in \mathcal B ^\prime$ がとれるようなものが存在する. 
\end{prop}
\begin{pf*}

気合い. 

\qed
\end{pf*}









\end{document}