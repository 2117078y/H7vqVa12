\documentclass[10pt, fleqn, label-section=none]{bxjsarticle}

%\usepackage[driver=dvipdfm,hmargin=25truemm,vmargin=25truemm]{geometry}

\setpagelayout{driver=dvipdfm,hmargin=25truemm,vmargin=20truemm}


\usepackage{amsmath}
\usepackage{amssymb}
\usepackage{amsfonts}
\usepackage{amsthm}
\usepackage{mathtools}
\usepackage{mleftright}

\usepackage{ascmac}




\usepackage{otf}

\theoremstyle{definition}
\newtheorem{dfn}{定義}[section]
\newtheorem{ex}[dfn]{例}
\newtheorem{lem}[dfn]{補題}
\newtheorem{prop}[dfn]{命題}
\newtheorem{thm}[dfn]{定理}
\newtheorem{setting}[dfn]{設定}
\newtheorem{cor}[dfn]{系}
\newtheorem*{pf*}{証明}
\newtheorem{problem}[dfn]{問題}
\newtheorem*{problem*}{問題}
\newtheorem{remark}[dfn]{注意}
\newtheorem*{claim*}{\underline{claim}}



\newtheorem*{solution*}{解答}

%箇条書きの様式
\renewcommand{\labelenumi}{(\arabic{enumi})}


%

\newcommand{\forany}{\rm{for} \ {}^{\forall}}
\newcommand{\foranyeps}{
\rm{for} \ {}^{\forall}\varepsilon >0}
\newcommand{\foranyk}{
\rm{for} \ {}^{\forall}k}


\newcommand{\any}{{}^{\forall}}
\newcommand{\suchthat}{\, \rm{s.t.} \, \it{}}




\newcommand{\veps}{\varepsilon}
\newcommand{\paren}[1]{\mleft( #1\mright )}
\newcommand{\cbra}[1]{\mleft\{#1\mright\}}
\newcommand{\sbra}[1]{\mleft\lbrack#1\mright\rbrack}
\newcommand{\tbra}[1]{\mleft\langle#1\mright\rangle}
\newcommand{\abs}[1]{\left|#1\right|}
\newcommand{\norm}[1]{\left\|#1\right\|}
\newcommand{\lopen}[1]{\mleft(#1\mright\rbrack}
\newcommand{\ropen}[1]{\mleft\lbrack #1 \mright)}



%
\newcommand{\Rn}{\mathbb{R}^n}
\newcommand{\Cn}{\mathbb{C}^n}

\newcommand{\Rm}{\mathbb{R}^m}
\newcommand{\Cm}{\mathbb{C}^m}


\newcommand{\projs}[2]{\it{p}_{#1,\ldots,#2}}
\newcommand{\projproj}[2]{\it{p}_{#1,#2}}

\newcommand{\proj}[1]{p_{#1}}

%可測空間
\newcommand{\stdProbSp}{\paren{\Omega, \mathcal{F}, P}}

%微分作用素
\newcommand{\ddt}{\frac{d}{dt}}
\newcommand{\ddx}{\frac{d}{dx}}
\newcommand{\ddy}{\frac{d}{dy}}

\newcommand{\delt}{\frac{\partial}{\partial t}}
\newcommand{\delx}{\frac{\partial}{\partial x}}

%ハイフン
\newcommand{\hyphen}{\text{-}}

%displaystyle
\newcommand{\dstyle}{\displaystyle}

%⇔, ⇒, \UTF{21D0}%
\newcommand{\LR}{\Leftrightarrow}
\newcommand{\naraba}{\Rightarrow}
\newcommand{\gyaku}{\Leftarrow}

%理由
\newcommand{\naze}[1]{\paren{\because {\mathop{ #1 }}}}

%
\newcommand{\sankaku}{\hfill $\triangle$}

%
\newcommand{\push}{_{\#}}

%手抜き
\newcommand{\textif}{\textrm{if}\,\,\,}
\newcommand{\Ric}{\textrm{Ric}}
\newcommand{\tr}{\textrm{tr}}
\newcommand{\vol}{\textrm{vol}}
\newcommand{\diam}{\textrm{diam}}
\newcommand{\supp}{\textrm{supp}}
\newcommand{\Med}{\textrm{Med}}
\newcommand{\Leb}{\textrm{Leb}}
\newcommand{\Const}{\textrm{Const}}
\newcommand{\Avg}{\textrm{Avg}}
\newcommand{\id}{\textrm{id}}
\newcommand{\Ker}{\textrm{Ker}}
\newcommand{\im}{\textrm{Im}}
\newcommand{\dil}{\textrm{dil}}
\newcommand{\Ch}{\textrm{Ch}}
\newcommand{\Lip}{\textrm{Lip}}
\newcommand{\Ent}{\textrm{Ent}}
\newcommand{\grad}{\textrm{grad}}
\newcommand{\dom}{\textrm{dom}}

\renewcommand{\;}{\, ; \,}
\renewcommand{\d}{\, {d}}

\newcommand{\gyouretsu}[1]{\begin{pmatrix} #1 \end{pmatrix} }


%%図式

\usepackage[dvipdfm,all]{xy}


\newenvironment{claim}[1]{\par\noindent\underline{step:}\space#1}{}
\newenvironment{claimproof}[1]{\par\noindent{($\because$)}\space#1}{\hfill $\blacktriangle $}


\newcommand{\pprime}{{\prime \prime}}





%%


\title{被覆空間への連続曲線のリフト}
\date{}


\author{}


\begin{document}


\maketitle

\section{}

\begin{prop}(自明な被覆空間への連続曲線のリフト). 第$2$成分への射影によりつくられる被覆空間$\pi: \mathbb N \times B \rightarrow B$ は, 
任意の連続曲線$c: [0, 1] \rightarrow B$ と, 任意の$\tilde p \in \pi^{-1} c_0$ に対して, 
連続曲線$\tilde c: [0,1] \rightarrow \mathbb N \times B$ で,  $\tilde c_ 0= \tilde p$ かつ, $\pi \circ \tilde c = c$ を満たす連続曲線が存在する.

\end{prop}
\begin{pf*}$\tilde p \in \cbra{n} \times B$ とする. $\pi | _{\cbra{n} \times B} $ は同相写像なので, 逆写像$\pi | _{\cbra{n} \times B} ^{-1}$ をもつ. 
\begin{align*} \tilde c \coloneqq \pi | _{\cbra{n} \times B} ^{-1} \circ c \end{align*}
と連続曲線を定めると, $\tilde c_ 0 = \tilde p$ であり, 
\begin{align*}  \pi \circ \tilde c =  c\end{align*}
となるので, これが求める連続曲線である. 
\qed
\end{pf*}

\begin{remark}
大域的に自明な構造をもっていれば, リフトは簡単に構成できた. 
一般の被覆空間の場合でも, 局所的に自明な構造をもっているので, 部分的にリフトを構成して, 有限回張り合わせていけばよい. 
\end{remark}



\begin{prop}$[a,b]$ の任意の開被覆$\cbra{U_\lambda}$ に対して, 
\begin{align*} a = s_0 < s_1 < \cdots < s_N < b, \quad a < t_0 < t_1, \cdots < t_N = b \end{align*}
で, $s_i < t_{i-1} < s_{i+1} < t_i$ を満たし, 
\begin{align*} [s_0, t_0) \subset ^\exists U_{\lambda(0)}, (s_1, t_1) \subset ^\exists U_{\lambda(1)}, \ldots , (s_N, t_N] \subset ^\exists U_{\lambda(N)}  \end{align*}
となる$s_n, t_n$ が存在する.
\end{prop}
\begin{pf*}
コンパクト空間$[a, b]$ の開被覆$\cbra{U_\lambda}$ のルベーグ数を$\delta$ とする. $N$ を十分大きくとって
\begin{align*} \frac{3(b-a)}{2m + 1} < \delta  \end{align*}
としておく. 
\begin{align*} s_i \coloneqq a + \frac{2i -1}{2m + 1} (b-a), \quad t_j \coloneqq a + \frac{2j + 2}{1m + 1}(b-a) \end{align*}
と定めると, 各区間の幅が$\frac{3(b-a)}{2m + 1}$ 以下なので, ルベーグの補題より主張が従う.
\qed
\end{pf*}



\begin{prop}(リフトの存在). $\pi : E \rightarrow B$ を被覆空間とする. 任意の連続曲線$c: [a,b] \rightarrow B$ に対して, 任意の$\tilde p \in \pi ^{-1} (c_a) $ に対して
$\tilde c: [a,b] \rightarrow E$ で, $\tilde c_ a = \tilde p$ かつ, $\pi \circ \tilde c = c$ を満たす連続曲線が存在する.

\end{prop}
\begin{pf*}
工事中.
\qed
\end{pf*}











\end{document}