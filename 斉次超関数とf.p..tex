\documentclass[10pt, fleqn, label-section=none]{bxjsarticle}

%\usepackage[driver=dvipdfm,hmargin=25truemm,vmargin=25truemm]{geometry}

\setpagelayout{driver=dvipdfm,hmargin=25truemm,vmargin=20truemm}


\usepackage{amsmath}
\usepackage{amssymb}
\usepackage{amsfonts}
\usepackage{amsthm}
\usepackage{mathtools}
\usepackage{mleftright}

\usepackage{ascmac}




\usepackage{otf}

\theoremstyle{definition}
\newtheorem{dfn}{定義}[section]
\newtheorem{ex}[dfn]{例}
\newtheorem{lem}[dfn]{補題}
\newtheorem{prop}[dfn]{命題}
\newtheorem{thm}[dfn]{定理}
\newtheorem{setting}[dfn]{設定}
\newtheorem{notation}[dfn]{記号}
\newtheorem{cor}[dfn]{系}
\newtheorem*{pf*}{証明}
\newtheorem{problem}[dfn]{問題}
\newtheorem*{problem*}{問題}
\newtheorem{remark}[dfn]{注意}
\newtheorem*{claim*}{\underline{claim}}



\newtheorem*{solution*}{解答}

%箇条書きの様式
\renewcommand{\labelenumi}{(\arabic{enumi})}


%

\newcommand{\forany}{\rm{for} \ {}^{\forall}}
\newcommand{\foranyeps}{
\rm{for} \ {}^{\forall}\varepsilon >0}
\newcommand{\foranyk}{
\rm{for} \ {}^{\forall}k}


\newcommand{\any}{{}^{\forall}}
\newcommand{\suchthat}{\, \rm{s.t.} \, \it{}}




\newcommand{\veps}{\varepsilon}
\newcommand{\paren}[1]{\mleft( #1\mright )}
\newcommand{\cbra}[1]{\mleft\{#1\mright\}}
\newcommand{\sbra}[1]{\mleft\lbrack#1\mright\rbrack}
\newcommand{\tbra}[1]{\mleft\langle#1\mright\rangle}
\newcommand{\abs}[1]{\left|#1\right|}
\newcommand{\norm}[1]{\left\|#1\right\|}
\newcommand{\lopen}[1]{\mleft(#1\mright\rbrack}
\newcommand{\ropen}[1]{\mleft\lbrack #1 \mright)}



%
\newcommand{\Rn}{\mathbb{R}^n}
\newcommand{\Cn}{\mathbb{C}^n}

\newcommand{\Rm}{\mathbb{R}^m}
\newcommand{\Cm}{\mathbb{C}^m}


\newcommand{\projs}[2]{\it{p}_{#1,\ldots,#2}}
\newcommand{\projproj}[2]{\it{p}_{#1,#2}}

\newcommand{\proj}[1]{p_{#1}}

%可測空間
\newcommand{\stdProbSp}{\paren{\Omega, \mathcal{F}, P}}

%微分作用素
\newcommand{\ddt}{\frac{d}{dt}}
\newcommand{\ddx}{\frac{d}{dx}}
\newcommand{\ddy}{\frac{d}{dy}}

\newcommand{\delt}{\frac{\partial}{\partial t}}
\newcommand{\delx}{\frac{\partial}{\partial x}}

%ハイフン
\newcommand{\hyphen}{\text{-}}

%displaystyle
\newcommand{\dstyle}{\displaystyle}

%⇔, ⇒, \UTF{21D0}%
\newcommand{\LR}{\Leftrightarrow}
\newcommand{\naraba}{\Rightarrow}
\newcommand{\gyaku}{\Leftarrow}

%理由
\newcommand{\naze}[1]{\paren{\because {\mathop{ #1 }}}}

%
\newcommand{\sankaku}{\hfill $\triangle$}

%
\newcommand{\push}{_{\#}}

%手抜き
\newcommand{\textif}{\textrm{if}\,\,\,}
\newcommand{\Ric}{\textrm{Ric}}
\newcommand{\tr}{\textrm{tr}}
\newcommand{\vol}{\textrm{vol}}
\newcommand{\diam}{\textrm{diam}}
\newcommand{\supp}{\textrm{supp}}
\newcommand{\Med}{\textrm{Med}}
\newcommand{\Leb}{\textrm{Leb}}
\newcommand{\Const}{\textrm{Const}}
\newcommand{\Avg}{\textrm{Avg}}
\newcommand{\id}{\textrm{id}}
\newcommand{\Ker}{\textrm{Ker}}
\newcommand{\im}{\textrm{Im}}
\newcommand{\dil}{\textrm{dil}}
\newcommand{\Ch}{\textrm{Ch}}
\newcommand{\Lip}{\textrm{Lip}}
\newcommand{\Ent}{\textrm{Ent}}
\newcommand{\grad}{\textrm{grad}}
\newcommand{\dom}{\textrm{dom}}
\newcommand{\diag}{\textrm{diag}}

\renewcommand{\;}{\, ; \,}
\renewcommand{\d}{\, {d}}

\newcommand{\gyouretsu}[1]{\begin{pmatrix} #1 \end{pmatrix} }

\renewcommand{\div}{\textrm{div}}


%%図式

\usepackage[dvipdfm,all]{xy}


\newenvironment{claim}[1]{\par\noindent\underline{step:}\space#1}{}
\newenvironment{claimproof}[1]{\par\noindent{($\because$)}\space#1}{\hfill $\blacktriangle $}


\newcommand{\pprime}{{\prime \prime}}

%%マグニチュード


\newcommand{\Mag}{\textrm{Mag}}

\usepackage{mathrsfs}


%%6.13
\def\chint#1{\mathchoice
{\XXint\displaystyle\textstyle{#1}}%
{\XXint\textstyle\scriptstyle{#1}}%
{\XXint\scriptstyle\scriptscriptstyle{#1}}%
{\XXint\scriptscriptstyle\scriptscriptstyle{#1}}%
\!\int}
\def\XXint#1#2#3{{\setbox0=\hbox{$#1{#2#3}{\int}$ }
\vcenter{\hbox{$#2#3$ }}\kern-.6\wd0}}
\def\ddashint{\chint=}
\def\dashint{\chint-}


%%7.13

\usepackage{here}

%7.15
\newcommand{\Span}{\textrm{Span}}

\newcommand{\Conv}{\textrm{Conv}}

%7.27

%9.4
\newcommand{\sing}{\textrm{sing}}

%
\newcommand{\C}[2]{{}_{#1}C_{#2} }

%10.1
\newcommand{\fp}{\textrm{f.p.}}


\title{斉次超関数とf.p.}
\date{}


\author{}


\begin{document}


\maketitle

\section{}
\subsection{}

\begin{setting}特に始域をかいてないときは, $\mathbb R^n$ を指すことにする. 

\end{setting}

\begin{dfn}$u \in \mathcal D^\prime $ に対して, $M_t u \in \mathcal D^\prime $ を

\begin{align*} (M_tu, \varphi) = \frac{1}{\abs t^n} (u, M_{1/t} \varphi) \quad (t \neq 0, \,\,\, \varphi \in \mathcal D)  \end{align*}

により定義する. 

\end{dfn}

\begin{remark}これは, 変数変換を通して$(u(t \cdot ), \varphi) = (u, \frac{1}{\abs t^n}  \varphi ((1/t)\cdot)  )$ が成り立つことを逆手にとって定義している. 

\end{remark}

\begin{dfn}(斉次超関数). $u \in \mathcal D^\prime$ で, 任意の$t > 0$ に対して超関数の意味で

\begin{align*} M_t u = t^a u\end{align*}

を満たすものを$a$ 次斉次超関数という. 

\end{dfn}

\begin{remark}

\end{remark}

\begin{dfn}(関数の有限部分). $g:(0, \infty)  \rightarrow \mathbb C$ とする. 異なる非負実部複素数$a_1, \ldots, a_N$ (ただし, いずれかひとつは$0$ でない) と, (異なるとは限らない) 複素数$ b_1, \ldots , b_{N+2}$ で

\begin{align*} g(\veps) = \sum_{j = 1}^N \frac{b_j}{\veps^{a_j}} + b_{N+1} \log \veps + b_{N+2} + o(1) \quad(\veps \downarrow 0)  \end{align*}

を満たすものが存在するときに, $b_{N+2}$ を$g$ の有限部分といい, 

\begin{align*} \textrm{f.p.}_{\veps \downarrow 0 }   \end{align*}

により表す. この定義の仕方は, $a_j, b_j$ の取り方によらずユニークである.


\end{dfn}

\begin{prop}

\begin{align*} \lim_{\veps \downarrow 0} \paren{\sum_{j=1}^N \frac{b_j}{\veps^{a_j}} + b_{N+1} \log \veps + b_{N+2}} = 0\end{align*}
ならば, 全ての$j$ について$b_j = 0 $  が成り立つ. 
\end{prop}
\begin{pf*}
strongly elliptic systems and boundary integral equations, 補題$5.2$ .
\qed
\end{pf*}

\begin{remark}$g$ が$r > 0$ の周りでローラン展開可能で, 負べきの項が有限個であれば, これは単に$0$ 次の項と一致する. 

\end{remark}

\begin{remark}負ベキの項と, 対数項がない場合は, これは単に

\begin{align*} \lim_{\veps \downarrow 0 }\end{align*}

と一致する. 

\end{remark}


\subsection{}

\begin{dfn}(有限部分拡張).  $u \in L^1_{\textrm{loc}}(\mathbb R^n \setminus \cbra{0})$ に対して, 

\begin{align*} (\fp u, \varphi) = \fp_{\veps \downarrow 0} \int_{\norm x > \veps } ju(x)\varphi(x) dx \quad (\any \varphi \in \mathcal D (\mathbb R^n) )\end{align*}

により, $\fp u$ を定める. 

\end{dfn}






\end{document}
