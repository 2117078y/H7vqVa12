\documentclass[10pt, fleqn, label-section=none]{bxjsarticle}

%\usepackage[driver=dvipdfm,hmargin=25truemm,vmargin=25truemm]{geometry}

\setpagelayout{driver=dvipdfm,hmargin=25truemm,vmargin=20truemm}


\usepackage{amsmath}
\usepackage{amssymb}
\usepackage{amsfonts}
\usepackage{amsthm}
\usepackage{mathtools}
\usepackage{mleftright}

\usepackage{ascmac}




\usepackage{otf}

\theoremstyle{definition}
\newtheorem{dfn}{定義}[section]
\newtheorem{ex}[dfn]{例}
\newtheorem{lem}[dfn]{補題}
\newtheorem{prop}[dfn]{命題}
\newtheorem{thm}[dfn]{定理}
\newtheorem{setting}[dfn]{設定}
\newtheorem{notation}[dfn]{記号}
\newtheorem{cor}[dfn]{系}
\newtheorem*{pf*}{証明}
\newtheorem{problem}[dfn]{問題}
\newtheorem*{problem*}{問題}
\newtheorem{remark}[dfn]{注意}
\newtheorem*{claim*}{\underline{claim}}



\newtheorem*{solution*}{解答}

%箇条書きの様式
\renewcommand{\labelenumi}{(\arabic{enumi})}


%

\newcommand{\forany}{\rm{for} \ {}^{\forall}}
\newcommand{\foranyeps}{
\rm{for} \ {}^{\forall}\varepsilon >0}
\newcommand{\foranyk}{
\rm{for} \ {}^{\forall}k}


\newcommand{\any}{{}^{\forall}}
\newcommand{\suchthat}{\, \rm{s.t.} \, \it{}}




\newcommand{\veps}{\varepsilon}
\newcommand{\paren}[1]{\mleft( #1\mright )}
\newcommand{\cbra}[1]{\mleft\{#1\mright\}}
\newcommand{\sbra}[1]{\mleft\lbrack#1\mright\rbrack}
\newcommand{\tbra}[1]{\mleft\langle#1\mright\rangle}
\newcommand{\abs}[1]{\left|#1\right|}
\newcommand{\norm}[1]{\left\|#1\right\|}
\newcommand{\lopen}[1]{\mleft(#1\mright\rbrack}
\newcommand{\ropen}[1]{\mleft\lbrack #1 \mright)}



%
\newcommand{\Rn}{\mathbb{R}^n}
\newcommand{\Cn}{\mathbb{C}^n}

\newcommand{\Rm}{\mathbb{R}^m}
\newcommand{\Cm}{\mathbb{C}^m}


\newcommand{\projs}[2]{\it{p}_{#1,\ldots,#2}}
\newcommand{\projproj}[2]{\it{p}_{#1,#2}}

\newcommand{\proj}[1]{p_{#1}}

%可測空間
\newcommand{\stdProbSp}{\paren{\Omega, \mathcal{F}, P}}

%微分作用素
\newcommand{\ddt}{\frac{d}{dt}}
\newcommand{\ddx}{\frac{d}{dx}}
\newcommand{\ddy}{\frac{d}{dy}}

\newcommand{\delt}{\frac{\partial}{\partial t}}
\newcommand{\delx}{\frac{\partial}{\partial x}}

%ハイフン
\newcommand{\hyphen}{\text{-}}

%displaystyle
\newcommand{\dstyle}{\displaystyle}

%⇔, ⇒, \UTF{21D0}%
\newcommand{\LR}{\Leftrightarrow}
\newcommand{\naraba}{\Rightarrow}
\newcommand{\gyaku}{\Leftarrow}

%理由
\newcommand{\naze}[1]{\paren{\because {\mathop{ #1 }}}}

%
\newcommand{\sankaku}{\hfill $\triangle$}

%
\newcommand{\push}{_{\#}}

%手抜き
\newcommand{\textif}{\textrm{if}\,\,\,}
\newcommand{\Ric}{\textrm{Ric}}
\newcommand{\tr}{\textrm{tr}}
\newcommand{\vol}{\textrm{vol}}
\newcommand{\diam}{\textrm{diam}}
\newcommand{\supp}{\textrm{supp}}
\newcommand{\Med}{\textrm{Med}}
\newcommand{\Leb}{\textrm{Leb}}
\newcommand{\Const}{\textrm{Const}}
\newcommand{\Avg}{\textrm{Avg}}
\newcommand{\id}{\textrm{id}}
\newcommand{\Ker}{\textrm{Ker}}
\newcommand{\im}{\textrm{Im}}
\newcommand{\dil}{\textrm{dil}}
\newcommand{\Ch}{\textrm{Ch}}
\newcommand{\Lip}{\textrm{Lip}}
\newcommand{\Ent}{\textrm{Ent}}
\newcommand{\grad}{\textrm{grad}}
\newcommand{\dom}{\textrm{dom}}
\newcommand{\diag}{\textrm{diag}}

\renewcommand{\;}{\, ; \,}
\renewcommand{\d}{\, {d}}

\newcommand{\gyouretsu}[1]{\begin{pmatrix} #1 \end{pmatrix} }

\renewcommand{\div}{\textrm{div}}


%%図式

\usepackage[dvipdfm,all]{xy}


\newenvironment{claim}[1]{\par\noindent\underline{step:}\space#1}{}
\newenvironment{claimproof}[1]{\par\noindent{($\because$)}\space#1}{\hfill $\blacktriangle $}


\newcommand{\pprime}{{\prime \prime}}

%%マグニチュード


\newcommand{\Mag}{\textrm{Mag}}

\usepackage{mathrsfs}


%%6.13
\def\chint#1{\mathchoice
{\XXint\displaystyle\textstyle{#1}}%
{\XXint\textstyle\scriptstyle{#1}}%
{\XXint\scriptstyle\scriptscriptstyle{#1}}%
{\XXint\scriptscriptstyle\scriptscriptstyle{#1}}%
\!\int}
\def\XXint#1#2#3{{\setbox0=\hbox{$#1{#2#3}{\int}$ }
\vcenter{\hbox{$#2#3$ }}\kern-.6\wd0}}
\def\ddashint{\chint=}
\def\dashint{\chint-}


%%7.13

\usepackage{here}

%7.15
\newcommand{\Span}{\textrm{Span}}

\newcommand{\Conv}{\textrm{Conv}}

%7.27

%9.4
\newcommand{\sing}{\textrm{sing}}

%
\newcommand{\C}[2]{{}_{#1}C_{#2} }


\title{コンパクト台をもつ関数の原点における積分表示}
\date{}


\author{}


\begin{document}


\maketitle

\section{}

\begin{remark}$f: \mathbb [0, \infty) \rightarrow \mathbb R$ を$C^\infty $ 級のコンパクト台をもつ関数とすると, 

\begin{align*}\sbra{t^{k-1} f^{(k-1)} (t)  }_0^\infty  \end{align*}

こういう値は, 十分大きい$R$ をとれば, $R^{k-1} f^(k-1) (R) = 0$ となることに注意する. 

\end{remark}


\begin{prop}$f: \mathbb [0, \infty) \rightarrow \mathbb R$ を$C^k$ 級のコンパクト台をもつ関数とする. この時, 

\begin{align*} f(0) = \frac{(-1)^k }{(k-1)!} \int_0^\infty t^{k-1}f^{(k)} (t) d t \end{align*}

が成り立つ. 

\end{prop}
\begin{pf*}$k = 1$ のとき, $- \int_0^\infty f^\prime (t) dt = f(0)$ なので成り立つ. 部分積分を行うことで, 

\begin{align*} \frac{(-1)^k }{(k-1)!} \int_0^\infty t^{k-1}f^{(k)} (t) d t &= \frac{(-1)^k}{(k-1)!} \paren{ \sbra{t^{k-1} f^{(k-1)} (t)  }_0^\infty  - \int_0^\infty (k-1) t^{k-2}f^{(k-1)} dt }
\\&= \frac{(-1)^k}{(k-1)!} \paren{ 0 - 0  - \int_0^\infty (k-1) t^{k-2}f^{(k-1)} dt }  \end{align*}

が成り立つので, 帰納法を用いれば主張が従う. 

\qed
\end{pf*}




\begin{prop} $C^\infty$ 級写像$\varphi(x): S^{n-1} \rightarrow \mathbb R$ を

\begin{align*} \int_{S^{n-1}} \varphi (x) dx = \frac{(-1)^k}{(k-1)!}   \end{align*} 
を満たすものとする. このとき, 

$u \in C_c^k (\mathbb R^n) $ に対して, 

\begin{align*} u(x) = \int_{\mathbb R^n} \varphi(\frac{y}{\norm y})  \norm y^{-n}   \paren{ \sum_{\abs \alpha = k } \partial^\alpha u(x+y) y^\alpha }   dy        \end{align*}

が成り立つ. 

\end{prop}
\begin{pf*}$v \in S^{n-1}$ に対して, 

\begin{align*} u(x) &= \frac{(-1)^k}{(k-1)!} \int_0^\infty t^{k-1} (\partial_t ) ^k u (x + tv) dt 
\\&=    \frac{(-1)^k}{(k-1)!} \int_0^\infty t^{k-1} \paren{ \sum_{\abs \alpha = k } \partial^\alpha u(x+tv) v^\alpha }         dt    \end{align*}

であるので, 両辺に$\varphi(v)$ をかけて$\int_{S^{n-1}}$ で積分すると, 

\begin{align*} u(x) &= \int_{S^{n-1}} \varphi(v) \int_0^\infty  t^{-n}t^{k} \paren{ \sum_{\abs \alpha = k } \partial^\alpha u(x+tv) v^\alpha }   t^{n-1} dt dv
\\&=  \int_{\mathbb R^n} \varphi(\frac{y}{\norm y})  \norm y^{-n} \norm y ^{k} \paren{ \sum_{\abs \alpha = k } \partial^\alpha u(x+y) {\frac{y}{\norm y}}^\alpha }   dy
\\&= \int_{\mathbb R^n} \varphi(\frac{y}{\norm y})  \norm y^{-n}   \paren{ \sum_{\abs \alpha = k } \partial^\alpha u(x+y) y^\alpha }   dy    \end{align*}

となる. $2$ つめの等号は極座標変換の逆を行った($t^{n-1} dt dv \mapsto dy $),

\qed
\end{pf*}






\begin{remark}$C^\infty$ 級写像$\varphi(x): S^{n-1} \rightarrow \mathbb R$ で

\begin{align*} \int_{S^{n-1}} \varphi (x) dx = \frac{(-1)^k}{(k-1)!}   \end{align*}

を満たすものは存在するのかということについては, $\int_{S^{n-1}} \varphi (x) dx  < \infty$ を適当に低数倍すりゃつくれるので, こういう関数はたくさん存在する. 

\end{remark}



\begin{prop}$C^\infty$ 級写像$\varphi(x): S^{n-1} \rightarrow \mathbb R$ を

\begin{align*} \int_{S^{n-1}} \varphi (x) dx = \frac{(-1)^k}{(k-1)!}   \end{align*} 
を満たすものとする. $\chi \in C_c^\infty[0, \infty)$ とする. 

このとき, 

$u \in C^k (\mathbb R^n) $ に対して, 

\begin{align*} u(x) = \int_{\mathbb R^n} \varphi(\frac{y}{\norm y}) \norm{y}^{-n} \sum_{l = 0}^k  \norm{y}^{k-l} \paren{ \C{k}{l} \chi^{(k-l)}(\norm y) \sum_{\abs \alpha = l} \partial^\alpha u(x+ y) {y}^\alpha }dy        \end{align*}

が成り立つ. 

\end{prop}
\begin{pf*} これまでと同様にして, 

\begin{align*}
u(x) &= u(x)\chi(0) 
\\&= \frac{(-1)^k}{(k-1)!} \int_0^\infty t^{k-1} (\partial_t ) ^k (\chi(t)u (x + tv)) dt 
\\&=  \frac{(-1)^k}{(k-1)!} \int_0^\infty t^{k-1} \sum_{l=0}^k \paren{\C{k}{l} \chi^{(k-l)} (t) \sum_{\abs \alpha = l} \partial^\alpha u(x + tv)v ^\alpha } dt
\\&=   \frac{(-1)^k}{(k-1)!} \int_0^\infty t^{-n} t^{k}t^{n-1} \sum_{l=0}^k \paren{\C{k}{l} \chi^{(k-l)}(t) \sum_{\abs \alpha = l} \partial^\alpha u(x + tv)v ^\alpha } dt
 \end{align*}

となるので, これまでと同様に両辺に$\varphi(v)$ をかけて$\int_{S^{n-1}}$ で積分をして,  極座標変換の逆を行うと, 

\begin{align*}
 \int_{\mathbb R^n} ^{k} \varphi(\frac{y}{\norm y}) \norm y^{-n} \norm y^{k}  \sum_{l=0}^k \paren{\C{k}{l} \chi^{k-l} (\norm y) \sum_{\abs \alpha = l} \partial^\alpha u(x + y) \paren{\frac{y}{\norm y}} ^\alpha } dy
 \end{align*}
 となるので, 主張が従う. 

\qed
\end{pf*}













\end{document}