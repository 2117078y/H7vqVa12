\documentclass[10pt, fleqn, label-section=none]{bxjsarticle}

%\usepackage[driver=dvipdfm,hmargin=25truemm,vmargin=25truemm]{geometry}

\setpagelayout{driver=dvipdfm,hmargin=25truemm,vmargin=20truemm}


\usepackage{amsmath}
\usepackage{amssymb}
\usepackage{amsfonts}
\usepackage{amsthm}
\usepackage{mathtools}
\usepackage{mleftright}

\usepackage{ascmac}




\usepackage{otf}

\theoremstyle{definition}
\newtheorem{dfn}{定義}[section]
\newtheorem{ex}[dfn]{例}
\newtheorem{lem}[dfn]{補題}
\newtheorem{prop}[dfn]{命題}
\newtheorem{thm}[dfn]{定理}
\newtheorem{setting}[dfn]{設定}
\newtheorem{notation}[dfn]{記号}
\newtheorem{cor}[dfn]{系}
\newtheorem*{pf*}{証明}
\newtheorem{problem}[dfn]{問題}
\newtheorem*{problem*}{問題}
\newtheorem{remark}[dfn]{注意}
\newtheorem*{claim*}{\underline{claim}}



\newtheorem*{solution*}{解答}

%箇条書きの様式
\renewcommand{\labelenumi}{(\arabic{enumi})}


%

\newcommand{\forany}{\rm{for} \ {}^{\forall}}
\newcommand{\foranyeps}{
\rm{for} \ {}^{\forall}\varepsilon >0}
\newcommand{\foranyk}{
\rm{for} \ {}^{\forall}k}


\newcommand{\any}{{}^{\forall}}
\newcommand{\suchthat}{\, \rm{s.t.} \, \it{}}




\newcommand{\veps}{\varepsilon}
\newcommand{\paren}[1]{\mleft( #1\mright )}
\newcommand{\cbra}[1]{\mleft\{#1\mright\}}
\newcommand{\sbra}[1]{\mleft\lbrack#1\mright\rbrack}
\newcommand{\tbra}[1]{\mleft\langle#1\mright\rangle}
\newcommand{\abs}[1]{\left|#1\right|}
\newcommand{\norm}[1]{\left\|#1\right\|}
\newcommand{\lopen}[1]{\mleft(#1\mright\rbrack}
\newcommand{\ropen}[1]{\mleft\lbrack #1 \mright)}



%
\newcommand{\Rn}{\mathbb{R}^n}
\newcommand{\Cn}{\mathbb{C}^n}

\newcommand{\Rm}{\mathbb{R}^m}
\newcommand{\Cm}{\mathbb{C}^m}


\newcommand{\projs}[2]{\it{p}_{#1,\ldots,#2}}
\newcommand{\projproj}[2]{\it{p}_{#1,#2}}

\newcommand{\proj}[1]{p_{#1}}

%可測空間
\newcommand{\stdProbSp}{\paren{\Omega, \mathcal{F}, P}}

%微分作用素
\newcommand{\ddt}{\frac{d}{dt}}
\newcommand{\ddx}{\frac{d}{dx}}
\newcommand{\ddy}{\frac{d}{dy}}

\newcommand{\delt}{\frac{\partial}{\partial t}}
\newcommand{\delx}{\frac{\partial}{\partial x}}

%ハイフン
\newcommand{\hyphen}{\text{-}}

%displaystyle
\newcommand{\dstyle}{\displaystyle}

%⇔, ⇒, \UTF{21D0}%
\newcommand{\LR}{\Leftrightarrow}
\newcommand{\naraba}{\Rightarrow}
\newcommand{\gyaku}{\Leftarrow}

%理由
\newcommand{\naze}[1]{\paren{\because {\mathop{ #1 }}}}

%
\newcommand{\sankaku}{\hfill $\triangle$}

%
\newcommand{\push}{_{\#}}

%手抜き
\newcommand{\textif}{\textrm{if}\,\,\,}
\newcommand{\Ric}{\textrm{Ric}}
\newcommand{\tr}{\textrm{tr}}
\newcommand{\vol}{\textrm{vol}}
\newcommand{\diam}{\textrm{diam}}
\newcommand{\supp}{\textrm{supp}}
\newcommand{\Med}{\textrm{Med}}
\newcommand{\Leb}{\textrm{Leb}}
\newcommand{\Const}{\textrm{Const}}
\newcommand{\Avg}{\textrm{Avg}}
\newcommand{\id}{\textrm{id}}
\newcommand{\Ker}{\textrm{Ker}}
\newcommand{\im}{\textrm{Im}}
\newcommand{\dil}{\textrm{dil}}
\newcommand{\Ch}{\textrm{Ch}}
\newcommand{\Lip}{\textrm{Lip}}
\newcommand{\Ent}{\textrm{Ent}}
\newcommand{\grad}{\textrm{grad}}
\newcommand{\dom}{\textrm{dom}}
\newcommand{\diag}{\textrm{diag}}

\renewcommand{\;}{\, ; \,}
\renewcommand{\d}{\, {d}}

\newcommand{\gyouretsu}[1]{\begin{pmatrix} #1 \end{pmatrix} }


%%図式

\usepackage[dvipdfm,all]{xy}


\newenvironment{claim}[1]{\par\noindent\underline{step:}\space#1}{}
\newenvironment{claimproof}[1]{\par\noindent{($\because$)}\space#1}{\hfill $\blacktriangle $}


\newcommand{\pprime}{{\prime \prime}}

%%マグニチュード


\newcommand{\Mag}{\textrm{Mag}}


\title{回転群と運動群のハール測度}
\date{}


\author{}


\begin{document}


\maketitle

\section{}

\begin{remark}多分, 一般に, 局所コンパクト位相群にはハール測度が存在する.

\end{remark}

\begin{notation}
\begin{align*} LI_d \coloneqq \cbra{(x_1, \ldots, x_d) \in S^{d-1} \times \cdots \times S^{d-1} \mid x_1, \ldots, x_d \textrm{は線型独立である.}} 
\end{align*}
とする. 
\begin{align*} \varphi^{pgs}: LI_d \rightarrow SO_d  \end{align*}
を以下のように定める. $(x_1, \ldots, x_d)$ に対して,添字順に直交化して$(z_1, \ldots, z_d) \in (S^{d-1} )^d$ を得て, 次に$z_d$ のみに$+1$ か$-1$ をかけて$(\tilde z_1, \ldots, \tilde z_d) \in (S^{d-1} )^d$ を得る. 
\begin{align*} \exists R_{\theta } \in SO_d  ; R_{\theta } (e_1, \ldots, e_d) = (\tilde z_1, \ldots, \tilde z_d)  \end{align*}
であるので, $(x_1, \ldots , x_d) LI_d$ にこの$R_\theta \in SO_d$ を対応させる写像として$\varphi^{pgs}$ を定める. 続いて, その拡張
\begin{align*} \tilde \varphi^{pgs}:  (S^{d-1})^d \rightarrow SO_d  \end{align*}
を, $LI_d$ に対してはそのまま$\varphi^{pgs}$ で, $(S^{d-1})^d \setminus LI_d$ に対しては自明な回転($id_{\mathbb R^d}$)を与える, とすることで定める. 
\end{notation}

\begin{dfn}(球面ルベーグ測度). ボレル可測空間$(S^{d-1}, \mathcal B (S^{d-1})    )  $ に, $B \in \mathcal B S^{d-1}$ に対して
\begin{align*} \sigma(A) \coloneqq  d \Leb_d (\cbra{sx \in \mathbb R^d \mid x \in B , 0 \leq s \leq 1} )\end{align*}
によりボレル測度$\sigma$ を定める. これを球面ルベーグ測度という. 
\end{dfn}


\begin{prop}(回転群のハール測度). 回転群$SO_d$ 上にはハール確率測度が一意に存在する. 

\end{prop}
\begin{pf*}


\begin{align*} \tilde \varphi^{psd} : (S^{d-1})^d \rightarrow SO_d \end{align*}
は, $LI_d$ 上で, 任意の$R_\theta \in SO_d$ に対して
\begin{align*} \tilde \varphi^{psd}(R_\theta x_1, \ldots , R_\theta x_d) = R_\theta \tilde \varphi^{psd} (x_1, \ldots x_d) \end{align*} が成り立つ. 



$(S^{d-1})^d$ 上のボレル測度を, 積測度
\begin{align*} \sigma^d \coloneqq \sigma \otimes \cdots \otimes \sigma \end{align*}
で定める. 

\begin{claim}$(S^{d-1})^d \setminus LI_d$ は$\sigma^d$に関して測度ゼロの集合である. 
\end{claim}
\begin{claimproof}

非正則行列$M$ に対して, $tEM$ は任意の$t \in [0,1] $ に対して非正則行列であるので, $Set \coloneqq \cup_{s \in [0,1]} s ((S^{d-1})^d \setminus LI_d  )$ は非正則行列に含まれるので, その$d^2$ 次元ルベーグ測度はゼロである. 
$\sigma^d = d^2 Leb_{d^2}$ であるので, 主張が従う. 
\end{claimproof}

\begin{claim}$\tilde \varphi^{psd}$ はボレル可測である. 

\end{claim}
\begin{claimproof}
任意に開集合$B \subset SO_d$ をとる. $ B \cap SO_d\setminus \cbra{Id_{\mathbb R^d} } , B \cap  \cbra{Id_{\mathbb R^d} } $ はそれぞれ開集合と$1$点であり, $\tilde \varphi^{psd} : (S^{d-1})^d $ の$LI_d, (S^{d-1})^d$ への制限はそれぞれ連続写像(直交化は多項式で表されるので)と定値写像である. $LI_d$ への制限は連続写像なので, その逆像は$LI_d$ の開集合だが, $LI_d$ 自体が開集合なので, 開集合である$LI_d$ との共通部分は開集合であり従ってボレル集合である. $1$点の方の逆像は$LI_d$ の補集合であり, それは閉集合なのでボレル集合である. 最後にボレル集合とボレル集合の合併はボレル集合であるので, 結局逆像がボレル集合なのでボレル可測である.
\end{claimproof}

$\tilde \varphi^{psd}: (S^{d-1})^d \rightarrow SO_d$ による押し出しによって有限測度を
\begin{align*} \check \nu = \tilde \varphi^{psd}_\#  \sigma^d \end{align*}
と定める. 

\begin{claim}$\check \nu$ 

\end{claim}
\begin{claimproof}任意に$R_{\theta_0} \in SO_d$ と非負可測関数$f \geq 0$ をとる. 
\begin{align*} \int_{SO_d} f(R_{\theta_0} R_\theta ) d\check \nu (R_\theta )   &= \int_{(S^{d-1})^d} f(R_{\theta_0}  \tilde \varphi^{psd} (x_1, \ldots , x_d) ) d\sigma^d  \\
&=   \int_{(S^{d-1})^d} f(  \tilde \varphi^{psd} (R_{\theta_0}x_1, \ldots , R_{\theta_0} x_d) ) d\sigma^d  \\
&= \int_{S^{d-1}}  \cdots \int_{S^{d-1}} f(  \tilde \varphi^{psd} (R_{\theta_0}x_1, \ldots , R_{\theta_0} x_d) ) d\sigma(x_1) \ldots, d\sigma(x_d) \\
&= \int_{S^{d-1}}  \cdots \int_{S^{d-1}} f(  \tilde \varphi^{psd} (x_1, \ldots , x_d) ) d\sigma(x_1) \ldots, d\sigma(x_d) \\
&= \int_{SO_d} f(R_\theta) d \tilde \nu (R_\theta ) 
\end{align*}
が成り立つ. 最後の等号は, ルベーグ球面測度が$R_\theta$ 不変であることを用いた.  
\end{claimproof}

従って, これは可測関数に関して積分左不変, 
最後に, $\tilde \nu$ は有限なので, 適当に確率測度となるように正規化すればよい. 
\qed
\end{pf*}


\begin{dfn}(運動群のハール群). 
\begin{align*} \gamma : \mathbb R^d \times SO_d \rightarrow G_d; (x, R_\theta) \mapsto \tau_x \circ R_\theta \end{align*}
と定める(ただし, $\tau_x$ は$x$ 分の平行移動).
$\lambda$ を$d$ 次元単位キューブ$C^d$ にルベーグ測度を制限したもの, $\nu$ を$SO_d$ 上のハール確率測度とする. 
\begin{align*} \mu = \gamma_# (\lambda \otimes \nu)\end{align*}
により, $G_d$ の測度を定める. 
\end{dfn}














\end{document}