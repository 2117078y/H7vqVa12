\documentclass[10pt, fleqn, label-section=none]{bxjsarticle}

%\usepackage[driver=dvipdfm,hmargin=25truemm,vmargin=25truemm]{geometry}

\setpagelayout{driver=dvipdfm,hmargin=25truemm,vmargin=20truemm}


\usepackage{amsmath}
\usepackage{amssymb}
\usepackage{amsfonts}
\usepackage{amsthm}
\usepackage{mathtools}
\usepackage{mleftright}

\usepackage{ascmac}




\usepackage{otf}

\theoremstyle{definition}
\newtheorem{dfn}{定義}[section]
\newtheorem{ex}[dfn]{例}
\newtheorem{lem}[dfn]{補題}
\newtheorem{prop}[dfn]{命題}
\newtheorem{thm}[dfn]{定理}
\newtheorem{setting}[dfn]{設定}
\newtheorem{notation}[dfn]{記号}
\newtheorem{cor}[dfn]{系}
\newtheorem*{pf*}{証明}
\newtheorem{problem}[dfn]{問題}
\newtheorem*{problem*}{問題}
\newtheorem{remark}[dfn]{注意}
\newtheorem*{claim*}{\underline{claim}}



\newtheorem*{solution*}{解答}

%箇条書きの様式
\renewcommand{\labelenumi}{(\arabic{enumi})}


%

\newcommand{\forany}{\rm{for} \ {}^{\forall}}
\newcommand{\foranyeps}{
\rm{for} \ {}^{\forall}\varepsilon >0}
\newcommand{\foranyk}{
\rm{for} \ {}^{\forall}k}


\newcommand{\any}{{}^{\forall}}
\newcommand{\suchthat}{\, \rm{s.t.} \, \it{}}




\newcommand{\veps}{\varepsilon}
\newcommand{\paren}[1]{\mleft( #1\mright )}
\newcommand{\cbra}[1]{\mleft\{#1\mright\}}
\newcommand{\sbra}[1]{\mleft\lbrack#1\mright\rbrack}
\newcommand{\tbra}[1]{\mleft\langle#1\mright\rangle}
\newcommand{\abs}[1]{\left|#1\right|}
\newcommand{\norm}[1]{\left\|#1\right\|}
\newcommand{\lopen}[1]{\mleft(#1\mright\rbrack}
\newcommand{\ropen}[1]{\mleft\lbrack #1 \mright)}



%
\newcommand{\Rn}{\mathbb{R}^n}
\newcommand{\Cn}{\mathbb{C}^n}

\newcommand{\Rm}{\mathbb{R}^m}
\newcommand{\Cm}{\mathbb{C}^m}


\newcommand{\projs}[2]{\it{p}_{#1,\ldots,#2}}
\newcommand{\projproj}[2]{\it{p}_{#1,#2}}

\newcommand{\proj}[1]{p_{#1}}

%可測空間
\newcommand{\stdProbSp}{\paren{\Omega, \mathcal{F}, P}}

%微分作用素
\newcommand{\ddt}{\frac{d}{dt}}
\newcommand{\ddx}{\frac{d}{dx}}
\newcommand{\ddy}{\frac{d}{dy}}

\newcommand{\delt}{\frac{\partial}{\partial t}}
\newcommand{\delx}{\frac{\partial}{\partial x}}

%ハイフン
\newcommand{\hyphen}{\text{-}}

%displaystyle
\newcommand{\dstyle}{\displaystyle}

%⇔, ⇒, \UTF{21D0}%
\newcommand{\LR}{\Leftrightarrow}
\newcommand{\naraba}{\Rightarrow}
\newcommand{\gyaku}{\Leftarrow}

%理由
\newcommand{\naze}[1]{\paren{\because {\mathop{ #1 }}}}

%
\newcommand{\sankaku}{\hfill $\triangle$}

%
\newcommand{\push}{_{\#}}

%手抜き
\newcommand{\textif}{\textrm{if}\,\,\,}
\newcommand{\Ric}{\textrm{Ric}}
\newcommand{\tr}{\textrm{tr}}
\newcommand{\vol}{\textrm{vol}}
\newcommand{\diam}{\textrm{diam}}
\newcommand{\supp}{\textrm{supp}}
\newcommand{\Med}{\textrm{Med}}
\newcommand{\Leb}{\textrm{Leb}}
\newcommand{\Const}{\textrm{Const}}
\newcommand{\Avg}{\textrm{Avg}}
\newcommand{\id}{\textrm{id}}
\newcommand{\Ker}{\textrm{Ker}}
\newcommand{\im}{\textrm{Im}}
\newcommand{\dil}{\textrm{dil}}
\newcommand{\Ch}{\textrm{Ch}}
\newcommand{\Lip}{\textrm{Lip}}
\newcommand{\Ent}{\textrm{Ent}}
\newcommand{\grad}{\textrm{grad}}
\newcommand{\dom}{\textrm{dom}}
\newcommand{\diag}{\textrm{diag}}

\renewcommand{\;}{\, ; \,}
\renewcommand{\d}{\, {d}}

\newcommand{\gyouretsu}[1]{\begin{pmatrix} #1 \end{pmatrix} }

\renewcommand{\div}{\textrm{div}}


%%図式

\usepackage[dvipdfm,all]{xy}


\newenvironment{claim}[1]{\par\noindent\underline{step:}\space#1}{}
\newenvironment{claimproof}[1]{\par\noindent{($\because$)}\space#1}{\hfill $\blacktriangle $}


\newcommand{\pprime}{{\prime \prime}}

%%マグニチュード


\newcommand{\Mag}{\textrm{Mag}}

\usepackage{mathrsfs}


\title{距離空間のヒルベルト空間への等長埋め込み}
\date{}


\author{}


\begin{document}


\maketitle

\section{}

\begin{dfn}(Conditionally of negative type). $X$ を位相空間とする. 連続関数$\phi: X \times X \rightarrow \mathbb R$ は\\
(1)$x \in X \naraba \phi (x, x) = 0.$ \\
(2)$x, y \in X \naraba \phi (x, y) = \phi(y, x).$ \\
(3)$n \in N, x_1, \ldots, x_n \in X, c_1, \ldots , c_n \in \mathbb R, \sum c_i = 0 \naraba, \sum\sum c_i c_j \phi (x_i, x_j) \leq 0.$ \\
を満たす時に, CND(conditionally of negative type)核という. 

\end{dfn}

\begin{prop}(GNS構成法). $X$ を位相空間, 連続関数$\phi: X \times X \rightarrow \mathbb R$ をCND核, $x_0 \in X$とする. このとき, ヒルベルト空間と連続関数の組$(H, f)$で, 
\begin{align*} \phi(x, y) = \norm{f(x) - f(y)}^2 \quad (x, y \in X) \end{align*}
を満たすものが存在する. 
\end{prop}
\begin{pf*}
\begin{align*} V \coloneqq \cbra{ \lambda: X \rightarrow \mathbb R \mid \# [\lambda \neq 0] < \infty, \sum_{x \in X} \lambda(x) = 0 } \end{align*}
双線形写像を
\begin{align*} \tbra{\lambda, \xi} \coloneqq - \frac{1}{2} \sum_{x \in X}\sum_{y \in X} \lambda(x)\xi(y) \phi(x, y)   \end{align*}
で定める. 
\begin{align*} N \coloneqq \cbra{\lambda : X \rightarrow \mathbb R  \mid  \tbra{\lambda, \lambda} = 0 } \end{align*}
と定める. 
\begin{align*} \lambda_1, \lambda_2 \in N \naraba (\lambda_1 + \lambda_2, \lambda_1 + \lambda_2) \leq 2 \norm{\lambda_1} \norm{\lambda_2} = 0  \end{align*}
となることから, $N$ は部分空間であることに注意する. 
 $V / N $ に$ \tbra{[\lambda], [\xi] } \coloneqq \tbra{\lambda, \xi}$ で内積を定める. この内積に関して, 完備化してできるヒルベルト空間を$(H, \tbra{, })$ とする. 
\begin{align*} f: X \rightarrow H; x \mapsto [\delta_x - \delta_{x_0}]\end{align*}
と定めると, 
\begin{align*} \norm{f(x) - f(y)}^2 &= \norm{[ (\delta_x - \delta_{x_0}) -   (\delta_y - \delta_{x_0})    ]}^2 \\&= \norm{ [\delta_x - \delta_y]  }^2 \\&= \norm{\delta_x}^2 - 2\tbra{\delta_x, \delta_y} + \norm{\delta_y} ^2  \\&= -\frac{1}{2} \paren{\phi(x,x) - 2 \phi(x,y) + \phi(y,y)} \\&= \phi(x, y) \end{align*}
が成り立つ. 
\qed
\end{pf*}











\end{document}