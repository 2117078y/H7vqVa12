\documentclass[10pt, fleqn, label-section=none]{bxjsarticle}

%\usepackage[driver=dvipdfm,hmargin=25truemm,vmargin=25truemm]{geometry}

\setpagelayout{driver=dvipdfm,hmargin=25truemm,vmargin=20truemm}


\usepackage{amsmath}
\usepackage{amssymb}
\usepackage{amsfonts}
\usepackage{amsthm}
\usepackage{mathtools}
\usepackage{mleftright}

\usepackage{ascmac}




\usepackage{otf}

\theoremstyle{definition}
\newtheorem{dfn}{定義}[section]
\newtheorem{ex}[dfn]{例}
\newtheorem{lem}[dfn]{補題}
\newtheorem{prop}[dfn]{命題}
\newtheorem{thm}[dfn]{定理}
\newtheorem{setting}[dfn]{設定}
\newtheorem{cor}[dfn]{系}
\newtheorem*{pf*}{証明}
\newtheorem{problem}[dfn]{問題}
\newtheorem*{problem*}{問題}
\newtheorem{remark}[dfn]{注意}
\newtheorem*{claim*}{\underline{claim}}



\newtheorem*{solution*}{解答}

%箇条書きの様式
\renewcommand{\labelenumi}{(\arabic{enumi})}


%

\newcommand{\forany}{\rm{for} \ {}^{\forall}}
\newcommand{\foranyeps}{
\rm{for} \ {}^{\forall}\varepsilon >0}
\newcommand{\foranyk}{
\rm{for} \ {}^{\forall}k}


\newcommand{\any}{{}^{\forall}}
\newcommand{\suchthat}{\, \rm{s.t.} \, \it{}}




\newcommand{\veps}{\varepsilon}
\newcommand{\paren}[1]{\mleft( #1\mright )}
\newcommand{\cbra}[1]{\mleft\{#1\mright\}}
\newcommand{\sbra}[1]{\mleft\lbrack#1\mright\rbrack}
\newcommand{\tbra}[1]{\mleft\langle#1\mright\rangle}
\newcommand{\abs}[1]{\left|#1\right|}
\newcommand{\norm}[1]{\left\|#1\right\|}
\newcommand{\lopen}[1]{\mleft(#1\mright\rbrack}
\newcommand{\ropen}[1]{\mleft\lbrack #1 \mright)}



%
\newcommand{\Rn}{\mathbb{R}^n}
\newcommand{\Cn}{\mathbb{C}^n}

\newcommand{\Rm}{\mathbb{R}^m}
\newcommand{\Cm}{\mathbb{C}^m}


\newcommand{\projs}[2]{\it{p}_{#1,\ldots,#2}}
\newcommand{\projproj}[2]{\it{p}_{#1,#2}}

\newcommand{\proj}[1]{p_{#1}}

%可測空間
\newcommand{\stdProbSp}{\paren{\Omega, \mathcal{F}, P}}

%微分作用素
\newcommand{\ddt}{\frac{d}{dt}}
\newcommand{\ddx}{\frac{d}{dx}}
\newcommand{\ddy}{\frac{d}{dy}}

\newcommand{\delt}{\frac{\partial}{\partial t}}
\newcommand{\delx}{\frac{\partial}{\partial x}}

%ハイフン
\newcommand{\hyphen}{\text{-}}

%displaystyle
\newcommand{\dstyle}{\displaystyle}

%⇔, ⇒, \UTF{21D0}%
\newcommand{\LR}{\Leftrightarrow}
\newcommand{\naraba}{\Rightarrow}
\newcommand{\gyaku}{\Leftarrow}

%理由
\newcommand{\naze}[1]{\paren{\because {\mathop{ #1 }}}}

%
\newcommand{\sankaku}{\hfill $\triangle$}

%
\newcommand{\push}{_{\#}}

%手抜き
\newcommand{\textif}{\textrm{if}\,\,\,}
\newcommand{\Ric}{\textrm{Ric}}
\newcommand{\tr}{\textrm{tr}}
\newcommand{\vol}{\textrm{vol}}
\newcommand{\diam}{\textrm{diam}}
\newcommand{\supp}{\textrm{supp}}
\newcommand{\Med}{\textrm{Med}}
\newcommand{\Leb}{\textrm{Leb}}
\newcommand{\Const}{\textrm{Const}}
\newcommand{\Avg}{\textrm{Avg}}
\newcommand{\id}{\textrm{id}}
\newcommand{\Ker}{\textrm{Ker}}
\newcommand{\im}{\textrm{Im}}
\newcommand{\dil}{\textrm{dil}}
\newcommand{\Ch}{\textrm{Ch}}
\newcommand{\Lip}{\textrm{Lip}}
\newcommand{\Ent}{\textrm{Ent}}
\newcommand{\grad}{\textrm{grad}}
\newcommand{\dom}{\textrm{dom}}

\renewcommand{\;}{\, ; \,}
\renewcommand{\d}{\, {d}}

\newcommand{\gyouretsu}[1]{\begin{pmatrix} #1 \end{pmatrix} }

%%図式

\usepackage[dvipdfm,all]{xy}


\newenvironment{claim}[1]{\par\noindent\underline{step:}\space#1}{}
\newenvironment{claimproof}[1]{\par\noindent{($\because$)}\space#1}{\hfill $\blacktriangle $}


\newcommand{\pprime}{{\prime \prime}}





%%


\title{3次元球面から2次元球面へのホップ写像}
\date{}


\author{}


\begin{document}


\maketitle




\section{}

\subsection{3次元球面の中の2次元トーラス}


\begin{align*} S^3 = \cbra{(z^1, z^2) \in \mathbb C ^2 \mid \abs{z^1}^2  + \abs{z^2} ^2 = 1 }\end{align*}

を, あえて

\begin{align*} S^3 = \cbra{(\cos (\frac{\xi}{2}) e^{i \theta_1  } , \sin (\frac{\xi}{2}) e^{i \theta_2  } ) \in \mathbb C ^2 \mid 0 \leq \frac{\xi}{2} \leq \frac{\pi}{2}, \theta_1, \theta_2 \in \mathbb S^1 }  \end{align*}

と表す.

\begin{align*} T_{\frac{\xi}{2}} \coloneqq \cbra{(\cos (\frac{\xi}{2}) e^{i \theta_1  } , \sin (\frac{\xi}{2}) e^{i \theta_2  } ) \in \mathbb C ^2 \mid \theta_1, \theta_2 \in \mathbb S^1 } \end{align*}

と定めると, これは$T^2 = S^1 \times S^1$ と微分同相である. (ただし, $\xi = 0, \pi$ の時は, 退化して$S^1$ と微分同相である.)

\subsection{$S^3$のU(1)による軌道空間}

あたりまえだが, $U(1) = \cbra{a \in \mathbb C \mid \abs{a} = 1}$ である.

右作用$: S^3 \times U(1) \rightarrow S^3$ を

\begin{align*} ( (z^1, z^2) , a) \mapsto (z^1 a, z^2 a)\end{align*}

により定める.

\begin{prop}
$(z^1, z^2), (w^1, w^2)$ が同じ軌道上にある. $\LR$ $\frac{z^1}{z^2} = \frac{w^1}{w^2} \in \hat{\mathbb C}.$ \\
(但し, $\hat{\mathbb C} = \mathbb C \cup \cbra{\infty}$ であり, $z^2 = 0$ のときは$z^1/ z^2 = \infty$ であると定める. )
\end{prop}
\begin{pf*}
明らかである. 
\qed
\end{pf*}

つまり, 軌道全体と$\hat{\mathbb C}$ が全単射である.

\begin{align*} \mathcal P : S^3 \rightarrow S^2 = S^3 / U(1); (z^1, z^2) \mapsto \hat{\rho_1} \paren{\frac{z^1}{z^2}} \end{align*}
と定める. (但し, $\hat{\rho_1} : \hat{\mathbb C} \rightarrow S^2$ は立体射影.)

これをホップ写像という. 





















\end{document}