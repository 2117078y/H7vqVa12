\documentclass[10pt, fleqn, label-section=none]{bxjsarticle}

%\usepackage[driver=dvipdfm,hmargin=25truemm,vmargin=25truemm]{geometry}

\setpagelayout{driver=dvipdfm,hmargin=25truemm,vmargin=20truemm}


\usepackage{amsmath}
\usepackage{amssymb}
\usepackage{amsfonts}
\usepackage{amsthm}
\usepackage{mathtools}
\usepackage{mleftright}

\usepackage{ascmac}




\usepackage{otf}

\theoremstyle{definition}
\newtheorem{dfn}{定義}[section]
\newtheorem{ex}[dfn]{例}
\newtheorem{lem}[dfn]{補題}
\newtheorem{prop}[dfn]{命題}
\newtheorem{thm}[dfn]{定理}
\newtheorem{setting}[dfn]{設定}
\newtheorem{cor}[dfn]{系}
\newtheorem*{pf*}{証明}
\newtheorem{problem}[dfn]{問題}
\newtheorem*{problem*}{問題}
\newtheorem{remark}[dfn]{注意}
\newtheorem*{claim*}{\underline{claim}}



\newtheorem*{solution*}{解答}

%箇条書きの様式
\renewcommand{\labelenumi}{(\arabic{enumi})}


%

\newcommand{\forany}{\rm{for} \ {}^{\forall}}
\newcommand{\foranyeps}{
\rm{for} \ {}^{\forall}\varepsilon >0}
\newcommand{\foranyk}{
\rm{for} \ {}^{\forall}k}


\newcommand{\any}{{}^{\forall}}
\newcommand{\suchthat}{\, \rm{s.t.} \, \it{}}




\newcommand{\veps}{\varepsilon}
\newcommand{\paren}[1]{\mleft( #1\mright )}
\newcommand{\cbra}[1]{\mleft\{#1\mright\}}
\newcommand{\sbra}[1]{\mleft\lbrack#1\mright\rbrack}
\newcommand{\tbra}[1]{\mleft\langle#1\mright\rangle}
\newcommand{\abs}[1]{\left|#1\right|}
\newcommand{\norm}[1]{\left\|#1\right\|}
\newcommand{\lopen}[1]{\mleft(#1\mright\rbrack}
\newcommand{\ropen}[1]{\mleft\lbrack #1 \mright)}



%
\newcommand{\Rn}{\mathbb{R}^n}
\newcommand{\Cn}{\mathbb{C}^n}

\newcommand{\Rm}{\mathbb{R}^m}
\newcommand{\Cm}{\mathbb{C}^m}


\newcommand{\projs}[2]{\it{p}_{#1,\ldots,#2}}
\newcommand{\projproj}[2]{\it{p}_{#1,#2}}

\newcommand{\proj}[1]{p_{#1}}

%可測空間
\newcommand{\stdProbSp}{\paren{\Omega, \mathcal{F}, P}}

%微分作用素
\newcommand{\ddt}{\frac{d}{dt}}
\newcommand{\ddx}{\frac{d}{dx}}
\newcommand{\ddy}{\frac{d}{dy}}

\newcommand{\delt}{\frac{\partial}{\partial t}}
\newcommand{\delx}{\frac{\partial}{\partial x}}

%ハイフン
\newcommand{\hyphen}{\text{-}}

%displaystyle
\newcommand{\dstyle}{\displaystyle}

%⇔, ⇒, \UTF{21D0}%
\newcommand{\LR}{\Leftrightarrow}
\newcommand{\naraba}{\Rightarrow}
\newcommand{\gyaku}{\Leftarrow}

%理由
\newcommand{\naze}[1]{\paren{\because {\mathop{ #1 }}}}

%
\newcommand{\sankaku}{\hfill $\triangle$}

%
\newcommand{\push}{_{\#}}

%手抜き
\newcommand{\textif}{\textrm{if}\,\,\,}
\newcommand{\Ric}{\textrm{Ric}}
\newcommand{\tr}{\textrm{tr}}
\newcommand{\vol}{\textrm{vol}}
\newcommand{\diam}{\textrm{diam}}
\newcommand{\supp}{\textrm{supp}}
\newcommand{\Med}{\textrm{Med}}
\newcommand{\Leb}{\textrm{Leb}}
\newcommand{\Const}{\textrm{Const}}
\newcommand{\Avg}{\textrm{Avg}}
\newcommand{\id}{\textrm{id}}
\newcommand{\Ker}{\textrm{Ker}}
\newcommand{\im}{\textrm{Im}}




\renewcommand{\;}{\, ; \,}
\renewcommand{\d}{\, {d}}

\newcommand{\gyouretsu}[1]{\begin{pmatrix} #1 \end{pmatrix} }

%%図式

\usepackage[dvipdfm,all]{xy}


\newenvironment{claim}[1]{\par\noindent\underline{step:}\space#1}{}
\newenvironment{claimproof}[1]{\par\noindent{($\because$)}\space#1}{\hfill $\blacktriangle $}


\newcommand{\pprime}{{\prime \prime}}





%%


\title{ネットによる連続写像の特徴づけ}
\date{}


\author{}


\begin{document}

\maketitle

\section{}

\subsection{前置き}

\begin{setting}
$X$ で適当な位相空間を表す. 
\end{setting}

\begin{dfn}(ネット). 有向集合$\Lambda$ と$x: \Lambda \rightarrow X$ の組$(x, \Lambda)$ を$X$ のネットという. これを単に$\cbra{x_\lambda}_{\lambda \in \Lambda}$ で表す. 

\end{dfn}

\begin{dfn}(共終). $A$ を順序集合, $B$ を$A$ の部分集合とする. 任意の$a \in A$ に対して, $b \in B$ で
\begin{align*} a \leq b \end{align*}
を満たすものが存在するとき, $B$ は$A$ と共終であるという. 
\end{dfn}

\begin{dfn}(強共終). $A$ を順序集合, $B$ を$A$ の部分集合とする. 任意の$a \in A$ に対して, $b_0 \in B$ で
\begin{align*} b \geq b_0 \naraba a \leq b \end{align*}
を満たすものが存在するとき, $B$ は$A$ と強共終であるという. 

\end{dfn}

\begin{dfn}(部分ネット). $(x, \Lambda)$ を$X$ のネットとする. $\Lambda^\prime $ を有向集合, $\varphi: \Lambda^\prime \rightarrow \Lambda$ とする.  $(x, \varphi(\Lambda^\prime))$ は, $\varphi(\Lambda^\prime)$ が$\Lambda$ と強共終であるとき, $(x, \Lambda)$ の部分ネットという. これを単に$\cbra{x_{\varphi(\lambda^\prime)}}_{\lambda^\prime \in \Lambda^\prime }$ で表す. 

\end{dfn}

\begin{dfn}(補有限回属する). $\cbra{x_\lambda}_{\lambda \in \Lambda}$ を$X$ のネット, $S \subset X$ を$X$ の部分集合とする. $\cbra{x_\lambda}$  は $\lambda_0 \in \Lambda$ で
\begin{align*} \lambda \geq \lambda_0 \naraba x_{\lambda} \in S \end{align*}
を満たすとき, $S$ に補有限回属するという. 
\end{dfn}

\begin{dfn}(頻繁に属する). $\cbra{x_\lambda}_{\lambda \in \Lambda}$ を$X$ のネット, $S \subset X$ を$X$ の部分集合とする. $\cbra{x_\lambda}$  は 任意の$\lambda \in \Lambda$ に対して, $\lambda^\prime \geq \lambda $ で
\begin{align*} x_{\lambda^\prime} \in S \end{align*}
を満たすとき, $S$ に頻繁に属するという. 
\end{dfn}

\begin{dfn}(普遍ネット). $X$ のネット$\cbra{x_\lambda}$ は, 任意の部分集合$S \subset X$ に対して, $S$ に補有限回属するか, あるいは$S^c$ に補有限回属するとき, 普遍ネットであるという.  

\end{dfn}

\begin{dfn}(収束点). $\cbra{x_\lambda}$ を$X$ のネットとし, $a \in X$ とする.  $a$ の任意の近傍$V_a$ に対して, $\cbra{x_\lambda}$ が$V_a$ に補有限回属するとき, $a$ を$\cbra{x_\lambda}$ の収束点という. $a = \lim x_\lambda$ と表す. 

\end{dfn}

\begin{prop}(閉包のネットによる特徴づけ). 
$S \subset X$ を部分集合とする. $x \in \bar S$ であることと, $x$ に収束する$S$ のネットが存在することは必要十分である. 
\end{prop}
\begin{pf*}
$\naraba$ を示す. 
$x$ は$S$ の閉包に属しているので, $x$ の任意の近傍$V$ に対して, $x_{V} \in S \cap V$ なる点がとれる. $x$ の近傍全体$\mathcal N _x$に, $V \leq  U :\LR V \supset U$ により順序を定めて有向集合とする. すると, $\cbra{x_V}_{V \in \mathcal N _x}$ は$x$ に収束する$S$ のネットである. $\gyaku$ を示す. $x \in X \setminus  \bar S$ であると, $X \setminus \bar S$ は閉集合なので, 小さい$x$ の開近傍$U_x$で$\bar S$ と共通部分を持たないものをとると, $U_x$ に補有限回属する$S$ のネットはとれないので$x$ に収束することに矛盾する.
\qed
\end{pf*}

\begin{dfn}(堆積点).  $\cbra{x_\lambda}$ を$X$ のネットとし, $a \in X$ とする.  $a$ の任意の近傍$V_a$ に対して, $\cbra{x_\lambda}$ が$V_a$ に頻繁に属するとき, $a$ を$\cbra{x_\lambda}$ の堆積点という. 

\end{dfn}



\subsection{本編}

\begin{prop}(連続写像のネットによる特徴づけ). $X, Y$ を位相空間, $f : X \rightarrow Y$ , $x \in X$とする. $f$ が連続であることと, $X$ の任意の収束ネット$\cbra{x_\lambda}$に対して$\cbra{fx_\lambda}$ が$f (\lim x_\lambda )$ に収束する$Y$ の収束ネットとなることは, 必要十分である. 
\end{prop}
\begin{pf*}
$\naraba$ を示す. $X$ の, $x \in X$ に収束するネット$\cbra{x_\lambda}$ をとる. 任意に$f(x)$ の近傍$U$をとり, その逆像を$V_x$ とする. $V_x$ は$x$ の近傍であるので, $\cbra{x_\lambda}$ は補有限回$V_x$ に属する. 従って, $\cbra{fx_\lambda}$ は$U$ に補有限回属するので, 示された. $\gyaku$ 連続でない点$x$ があるとする. $f(x)$ の近傍$U$ で, $x$ の任意の近傍の$f$ による像が$U$ に含まれないものがとれる. そこで, $x$ の任意の近傍$V$ に対して$y \in f(V) \setminus U$ が取れるので, $x_V \in V$ で $fx_V = y$ を満たすものがとれる. $x$ の近傍全体$\mathcal N _x$に, $V \leq  U :\LR V \supset U$ により順序を定めて有向集合とする. すると, $\cbra{x_V}_{V \in \mathcal N _x}$ は$x$ に収束する$X$ のネットである. 一方で, 任意の$V$ に対して$fx_V$ は$fx$ の近傍$U$ に含まれないので, $fx$ には収束しないので矛盾する. 
\qed
\end{pf*}











\end{document}