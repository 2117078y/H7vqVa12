\documentclass[10pt, fleqn, label-section=none]{bxjsarticle}

%\usepackage[driver=dvipdfm,hmargin=25truemm,vmargin=25truemm]{geometry}

\setpagelayout{driver=dvipdfm,hmargin=25truemm,vmargin=20truemm}


\usepackage{amsmath}
\usepackage{amssymb}
\usepackage{amsfonts}
\usepackage{amsthm}
\usepackage{mathtools}
\usepackage{mleftright}

\usepackage{ascmac}




\usepackage{otf}

\theoremstyle{definition}
\newtheorem{dfn}{定義}[section]
\newtheorem{ex}[dfn]{例}
\newtheorem{lem}[dfn]{補題}
\newtheorem{prop}[dfn]{命題}
\newtheorem{thm}[dfn]{定理}
\newtheorem{setting}[dfn]{設定}
\newtheorem{cor}[dfn]{系}
\newtheorem*{pf*}{証明}
\newtheorem{problem}[dfn]{問題}
\newtheorem*{problem*}{問題}
\newtheorem{remark}[dfn]{注意}
\newtheorem*{claim*}{\underline{claim}}



\newtheorem*{solution*}{解答}

%箇条書きの様式
\renewcommand{\labelenumi}{(\arabic{enumi})}


%

\newcommand{\forany}{\rm{for} \ {}^{\forall}}
\newcommand{\foranyeps}{
\rm{for} \ {}^{\forall}\varepsilon >0}
\newcommand{\foranyk}{
\rm{for} \ {}^{\forall}k}


\newcommand{\any}{{}^{\forall}}
\newcommand{\suchthat}{\, \rm{s.t.} \, \it{}}




\newcommand{\veps}{\varepsilon}
\newcommand{\paren}[1]{\mleft( #1\mright )}
\newcommand{\cbra}[1]{\mleft\{#1\mright\}}
\newcommand{\sbra}[1]{\mleft\lbrack#1\mright\rbrack}
\newcommand{\tbra}[1]{\mleft\langle#1\mright\rangle}
\newcommand{\abs}[1]{\left|#1\right|}
\newcommand{\norm}[1]{\left\|#1\right\|}
\newcommand{\lopen}[1]{\mleft(#1\mright\rbrack}
\newcommand{\ropen}[1]{\mleft\lbrack #1 \mright)}



%
\newcommand{\Rn}{\mathbb{R}^n}
\newcommand{\Cn}{\mathbb{C}^n}

\newcommand{\Rm}{\mathbb{R}^m}
\newcommand{\Cm}{\mathbb{C}^m}


\newcommand{\projs}[2]{\it{p}_{#1,\ldots,#2}}
\newcommand{\projproj}[2]{\it{p}_{#1,#2}}

\newcommand{\proj}[1]{p_{#1}}

%可測空間
\newcommand{\stdProbSp}{\paren{\Omega, \mathcal{F}, P}}

%微分作用素
\newcommand{\ddt}{\frac{d}{dt}}
\newcommand{\ddx}{\frac{d}{dx}}
\newcommand{\ddy}{\frac{d}{dy}}

\newcommand{\delt}{\frac{\partial}{\partial t}}
\newcommand{\delx}{\frac{\partial}{\partial x}}

%ハイフン
\newcommand{\hyphen}{\text{-}}

%displaystyle
\newcommand{\dstyle}{\displaystyle}

%⇔, ⇒, \UTF{21D0}%
\newcommand{\LR}{\Leftrightarrow}
\newcommand{\naraba}{\Rightarrow}
\newcommand{\gyaku}{\Leftarrow}

%理由
\newcommand{\naze}[1]{\paren{\because {\mathop{ #1 }}}}

%
\newcommand{\sankaku}{\hfill $\triangle$}

%
\newcommand{\push}{_{\#}}

%手抜き
\newcommand{\textif}{\textrm{if}\,\,\,}
\newcommand{\Ric}{\textrm{Ric}}
\newcommand{\tr}{\textrm{tr}}
\newcommand{\vol}{\textrm{vol}}
\newcommand{\diam}{\textrm{diam}}
\newcommand{\supp}{\textrm{supp}}
\newcommand{\Med}{\textrm{Med}}
\newcommand{\Leb}{\textrm{Leb}}
\newcommand{\Const}{\textrm{Const}}
\newcommand{\Avg}{\textrm{Avg}}
\newcommand{\id}{\textrm{id}}
\newcommand{\Ker}{\textrm{Ker}}
\newcommand{\im}{\textrm{Im}}




\renewcommand{\;}{\, ; \,}
\renewcommand{\d}{\, {d}}

\newcommand{\gyouretsu}[1]{\begin{pmatrix} #1 \end{pmatrix} }

%%図式

\usepackage[dvipdfm,all]{xy}


\newenvironment{claim}[1]{\par\noindent\underline{step:}\space#1}{}
\newenvironment{claimproof}[1]{\par\noindent{($\because$)}\space#1}{\hfill $\blacktriangle $}


\newcommand{\pprime}{{\prime \prime}}





%%


\title{確率変数の収束}
\date{}


\author{}


\begin{document}


\maketitle



\section{}

\subsection{}



特に断りのない限り, $\cbra{ X_n}$ は$\paren{\Omega,  F , P}$ 上の確率変数の列を表す. 
\begin{dfn}
\label{}
\quad \\
$\cbra{X_n}$ が$X$ に概収束する. $:\LR P\paren{\lim_n X_n = X} = 1$ が成り立つ.\\
$\cbra{X_n}$ が$X$ に確率収束する. $:\LR \foranyeps, \lim_n \mathit{P}\paren{\abs{X_n - X} > \veps} = 0$ が成り立つ.
\end{dfn}

しばらくの間, $A_n \paren{\veps} \coloneqq \sbra{ \abs{X_n - X} > \veps}$ とする.
\begin{prop}
$\sbra{\lim_n X_n = X} = \bigcap_k \bigcup_N \bigcap_{n\geq N} \paren{{A_n\paren{\frac{1}{k}}}}^c$ である.
\end{prop}
\begin{pf*}
極限の定義より直ちに従う.
\qed
\end{pf*}


\begin{prop}
$\cbra{X_n}$ が$X$ に概収束する. $\naraba$ $X$に確率収束する.
\begin{pf*} \quad \\
$\cbra{X_n}$ が$X$ に概収束するならば, 
\begin{align*}
0 &= P\paren{\sbra{\lim_n X_n = X}^c} 
= P\paren{\bigcup_k \bigcap_N \bigcup_{n\geq N} \paren{ {A_n\paren{\frac{1}{k} } } } } \\
&\geq P\paren{\bigcap_N \bigcup_{n \geq N} \paren{A_n \paren{\frac{1}{k}} }} \quad {\paren{\because \mathop{単調性}}}\\
&\geq \limsup_n \mathit{P}\paren{\paren{A_n \paren{\frac{1}{k}} }} \quad {\paren{\because \rm{Fatou}}}
\end{align*}
が成り立つ. 故に, 
$\foranyk, \, \lim_n \mathit{P}\paren{A_n \paren{\frac{1}{k}} } = 0$ が成り立つ.
\qed
\end{pf*}
\end{prop}

\begin{prop}
$\foranyk, \, \mathit{P}\paren{\bigcap_N \bigcup_{n\geq N} A_n\paren{ \frac{1}{k} } } = 0 \naraba \cbra{X_n}$ が$X$ に概収束する.
\end{prop}
\begin{pf*}
$P\paren{\bigcup_k \bigcap_N \bigcup_{n\geq N} \paren{ {A_n\paren{\frac{1}{k} } } } } \leq \sum_k P\paren{\bigcap_N \bigcup_{n\geq N} A_n\paren{ \frac{1}{k} } } = \sum_k 0 = 0$
\qed
\end{pf*}

\begin{prop}
$\foranyk , \lim_N P \paren{\sup_{n\geq N} \abs{X_n - X} > \frac{1}{k}} = 0 \LR \cbra{X_n}$ が$X$に概収束する.
\end{prop}
\begin{pf*} 下方連続性より, 
$P\paren{\bigcap_N \bigcup_{n\geq N} A_n\paren{ \frac{1}{k} } }=\lim_N P \paren{\bigcup_{n\geq N} A_n\paren{ \frac{1}{k} }}$ .
\qed
\end{pf*}

\begin{prop}(Chevyshevの不等式) \\
$f:\paren{0, \infty}\rightarrow \paren{0, \infty},\mathop{単調増加. } \, X:f\paren{\abs{X}} \in L^1\paren{P} \mathop{を満たす確率変数, }$ $a > 0$ に対して, 
\begin{align*}
P\paren{\abs{X} > a} \leq \frac{1}{f\paren{a}} E\paren{f\paren{\abs{X}}}
\end{align*}
が成り立つ.

\end{prop}
\begin{pf*} 
$f\paren{a}E\paren{1_{\sbra{\abs{X}>a}}} = E\paren{f\paren{a}1_{\sbra{\abs{X}>a}}}\leq E\paren{f\paren{\abs{X}} 1_{\sbra{\abs{X}>a}}} \leq E\paren{f\paren{\abs{X}}}$
\qed
\end{pf*}


\begin{prop}
$\cbra{X_n}$ が$X$ に$L^p$ 収束する. $\naraba$ $X$ に確率収束する.  
\end{prop}
\begin{pf*}
$P\paren{\abs{X_n - X} > \veps} \leq \frac{1}{\veps^p} E\paren{\abs{X_n - X}^{p}} $
\qed
\end{pf*}

\begin{prop}
\label{1817}
$\cbra{X_n}$ に対して, 以下が成り立つ. 
\begin{align*}
X \mathop{に確率収束 } \LR \lim_{n} E\paren{\frac{\abs{X_n - X} }{1 + \abs{X_n -X } } } = 0
\end{align*}
\end{prop}
\begin{pf*}
\quad \\
$\paren{\naraba} \quad A_n\coloneqq \sbra{\abs{X_n - X} > \veps}$ とすると, nを十分大きくとると次が成り立つ.
\begin{align*}
E\paren{\frac{\abs{X_n - X} }{1 + \abs{X_n -X } } } 
&= E\paren{\frac{\abs{X_n - X} }{1 + \abs{X_n -X } } 1_{A_n}} + E\paren{\frac{\abs{X_n - X} }{1 + \abs{X_n -X } } 1_{{A_n }^c}} \\
&\leq E\paren{1 \cdot 1_{A_n}} + E\paren{\veps \cdot 1_{{A_n }^c}} \\
&\leq E\paren{1 \cdot 1_{A_n}} + E\paren{\veps \cdot 1_{\Omega}} \\
&\leq \veps + \veps \qquad \paren{\because \mathop{確率収束するので} P\paren{A_n} はいくらでも小さくできる.}
\end{align*}
$\paren{\gyaku} $
\begin{align*}
P\paren{\abs{X_n -X} > \veps} \leq \frac{\veps}{1 + \veps} E\paren{\frac{\abs{X_n - X} }{1 + \abs{X_n -X } } } 
\qquad \naze{チェビシェフの不等式}
\end{align*}
\qed
\end{pf*}

\begin{thm}
$\cbra{X_n}$ が$X$ に確率収束する. $\naraba$ $X$ に概収束する部分列がとれる.
\end{thm}
\begin{pf*}
命題\ref{1817} より, 十分大きなnをとると, $E\paren{\frac{\abs{X_n - X} }{1 + \abs{X_n -X } } } \leq \frac{1}{k^2}$ とできるので, \\
適当に部分列を取ることで, $\sum E\paren{\frac{\abs{X_n - X} }{1 + \abs{X_n -X } } } \leq \sum \frac{1}{k^2} < \infty$ が成り立つ. \\
$\sum E\paren{\frac{\abs{X_n - X} }{1 + \abs{X_n -X } } } = E\paren{\sum \frac{\abs{X_n - X} }{1 + \abs{X_n -X } } } $ 故に, $\sum \frac{\abs{X_n - X} }{1 + \abs{X_n -X } } $ は可積分なのでa.s.で有限の値を取る. \\
即ち, $\lim_n \paren{\frac{\abs{X_n - X} }{1 + \abs{X_n -X } } } = 0$がa.s.で成り立つので, $\lim_n \abs{X_n - X} = 0$ がa.s.で成り立つ.


\qed
\end{pf*}










\end{document}