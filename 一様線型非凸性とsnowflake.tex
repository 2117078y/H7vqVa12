\documentclass[10pt, fleqn, label-section=none]{bxjsarticle}

%\usepackage[driver=dvipdfm,hmargin=25truemm,vmargin=25truemm]{geometry}

\setpagelayout{driver=dvipdfm,hmargin=25truemm,vmargin=20truemm}


\usepackage{amsmath}
\usepackage{amssymb}
\usepackage{amsfonts}
\usepackage{amsthm}
\usepackage{mathtools}
\usepackage{mleftright}

\usepackage{ascmac}




\usepackage{otf}

\theoremstyle{definition}
\newtheorem{dfn}{定義}[section]
\newtheorem{ex}[dfn]{例}
\newtheorem{lem}[dfn]{補題}
\newtheorem{prop}[dfn]{命題}
\newtheorem{thm}[dfn]{定理}
\newtheorem{setting}[dfn]{設定}
\newtheorem{notation}[dfn]{記号}
\newtheorem{cor}[dfn]{系}
\newtheorem*{pf*}{証明}
\newtheorem{problem}[dfn]{問題}
\newtheorem*{problem*}{問題}
\newtheorem{remark}[dfn]{注意}
\newtheorem*{claim*}{\underline{claim}}



\newtheorem*{solution*}{解答}

%箇条書きの様式
\renewcommand{\labelenumi}{(\arabic{enumi})}


%

\newcommand{\forany}{\rm{for} \ {}^{\forall}}
\newcommand{\foranyeps}{
\rm{for} \ {}^{\forall}\varepsilon >0}
\newcommand{\foranyk}{
\rm{for} \ {}^{\forall}k}


\newcommand{\any}{{}^{\forall}}
\newcommand{\suchthat}{\, \rm{s.t.} \, \it{}}




\newcommand{\veps}{\varepsilon}
\newcommand{\paren}[1]{\mleft( #1\mright )}
\newcommand{\cbra}[1]{\mleft\{#1\mright\}}
\newcommand{\sbra}[1]{\mleft\lbrack#1\mright\rbrack}
\newcommand{\tbra}[1]{\mleft\langle#1\mright\rangle}
\newcommand{\abs}[1]{\left|#1\right|}
\newcommand{\norm}[1]{\left\|#1\right\|}
\newcommand{\lopen}[1]{\mleft(#1\mright\rbrack}
\newcommand{\ropen}[1]{\mleft\lbrack #1 \mright)}



%
\newcommand{\Rn}{\mathbb{R}^n}
\newcommand{\Cn}{\mathbb{C}^n}

\newcommand{\Rm}{\mathbb{R}^m}
\newcommand{\Cm}{\mathbb{C}^m}


\newcommand{\projs}[2]{\it{p}_{#1,\ldots,#2}}
\newcommand{\projproj}[2]{\it{p}_{#1,#2}}

\newcommand{\proj}[1]{p_{#1}}

%可測空間
\newcommand{\stdProbSp}{\paren{\Omega, \mathcal{F}, P}}

%微分作用素
\newcommand{\ddt}{\frac{d}{dt}}
\newcommand{\ddx}{\frac{d}{dx}}
\newcommand{\ddy}{\frac{d}{dy}}

\newcommand{\delt}{\frac{\partial}{\partial t}}
\newcommand{\delx}{\frac{\partial}{\partial x}}

%ハイフン
\newcommand{\hyphen}{\text{-}}

%displaystyle
\newcommand{\dstyle}{\displaystyle}

%⇔, ⇒, \UTF{21D0}%
\newcommand{\LR}{\Leftrightarrow}
\newcommand{\naraba}{\Rightarrow}
\newcommand{\gyaku}{\Leftarrow}

%理由
\newcommand{\naze}[1]{\paren{\because {\mathop{ #1 }}}}

%
\newcommand{\sankaku}{\hfill $\triangle$}

%
\newcommand{\push}{_{\#}}

%手抜き
\newcommand{\textif}{\textrm{if}\,\,\,}
\newcommand{\Ric}{\textrm{Ric}}
\newcommand{\tr}{\textrm{tr}}
\newcommand{\vol}{\textrm{vol}}
\newcommand{\diam}{\textrm{diam}}
\newcommand{\supp}{\textrm{supp}}
\newcommand{\Med}{\textrm{Med}}
\newcommand{\Leb}{\textrm{Leb}}
\newcommand{\Const}{\textrm{Const}}
\newcommand{\Avg}{\textrm{Avg}}
\newcommand{\id}{\textrm{id}}
\newcommand{\Ker}{\textrm{Ker}}
\newcommand{\im}{\textrm{Im}}
\newcommand{\dil}{\textrm{dil}}
\newcommand{\Ch}{\textrm{Ch}}
\newcommand{\Lip}{\textrm{Lip}}
\newcommand{\Ent}{\textrm{Ent}}
\newcommand{\grad}{\textrm{grad}}
\newcommand{\dom}{\textrm{dom}}
\newcommand{\diag}{\textrm{diag}}

\renewcommand{\;}{\, ; \,}
\renewcommand{\d}{\, {d}}

\newcommand{\gyouretsu}[1]{\begin{pmatrix} #1 \end{pmatrix} }

\renewcommand{\div}{\textrm{div}}


%%図式

\usepackage[dvipdfm,all]{xy}


\newenvironment{claim}[1]{\par\noindent\underline{step:}\space#1}{}
\newenvironment{claimproof}[1]{\par\noindent{($\because$)}\space#1}{\hfill $\blacktriangle $}


\newcommand{\pprime}{{\prime \prime}}

%%マグニチュード


\newcommand{\Mag}{\textrm{Mag}}

\usepackage{mathrsfs}


%%6.13
\def\chint#1{\mathchoice
{\XXint\displaystyle\textstyle{#1}}%
{\XXint\textstyle\scriptstyle{#1}}%
{\XXint\scriptstyle\scriptscriptstyle{#1}}%
{\XXint\scriptscriptstyle\scriptscriptstyle{#1}}%
\!\int}
\def\XXint#1#2#3{{\setbox0=\hbox{$#1{#2#3}{\int}$ }
\vcenter{\hbox{$#2#3$ }}\kern-.6\wd0}}
\def\ddashint{\chint=}
\def\dashint{\chint-}

%%7.13

\usepackage{here}

%7.15
\newcommand{\Span}{\textrm{Span}}

\newcommand{\Conv}{\textrm{Conv}}


\title{一様線型非凸性とsnowflake}
\date{}


\author{}


\begin{document}


\maketitle

\section{}

\begin{dfn}(一様線型非凸). $\eta > 0$ とし, $(V, \norm \cdot)$ をノルム空間とする. $A \subset V$ は 任意の$x, y \in A$ に対して, $c^x_y \in \Conv(\cbra{x, y})$ で
\begin{align*} B(c^x_y, \eta \norm{y- x} ) \cap A = \varnothing \end{align*}
を満たすものが存在する時, $\eta$ - 一様線型非凸(ULNC)であるという. 
\end{dfn}

\begin{prop}$(X, d)$ を$p$-snowflakeとし, $(V, \norm \cdot)$ をノルム空間とする. このとき, 任意の$L$-双リプシッツ写像$f: X \rightarrow V$ に対して, $f(X)$ は$V$ においてULNCである.  

\end{prop}
\begin{pf*}$(X, d)$ は$p$-snowflakeであるので$ 0 < c < 1$ を, 任意の有限個の点$x_0, \ldots , x_n \in X$ に対して
\begin{align*} c(x_0 x_n)^ p \leq \sum(x_i x_i+1)^p\end{align*}
を満たすようにとれる. $N$ を十分大きくとり, 
\begin{align*} N \paren{\frac{2L^2}{N}}^p < c \end{align*}
が成り立つようにしておく. 任意に$2$点$x_0, x_N \in X$をとる. 
\begin{align*} &z_0 \coloneqq fx_0, \quad z_N \coloneqq fx_N, \\ &z_1 \coloneqq z_0 + \frac{1}{N} (z_N - z_0), \\&  z_2 \coloneqq z_0 + \frac{2}{N} (z_N - z_0)  , \\& \vdots   \\&    z_{N-1} \coloneqq z_0 + \frac{N-1}{N} (z_N - z_0)  .                               \end{align*}
と定める. 任意の$i = 1, \ldots , N-1$ に対して, $B(z_i ; \frac{1}{2N} \norm{z_N - z_0}) \cap f(X) \neq \varnothing$ と仮定する(背理法). $i = 1, \ldots , N-1$ に対して$x_i \in B(z_i ; \frac{1}{2N} \norm{z_N - z_0}) \cap f(X)$ がとれる. 
\begin{align*} &\frac{1}{L} x_{i + 1} x_i \leq \norm{fx_{i+ 1} - fx_{i} } \leq \norm{fx_{i+1} - z_{i+1}} + \norm{z_{i + 1} - z_i} + \norm{z_i - fx_i} \\&< \frac{1}{2N} \norm{z_N - z_0} + \frac{1}{N} \norm{z_N - z_0} + \frac{1}{2N} \norm{z_N - z_0} \\&< \frac{2}{N} \norm{z_N - z_0} = \frac{2}{N} \norm{fx_N - fx_0} \leq \frac{2}{N} L x_Nx_0 \end{align*}
が成り立つので, 
\begin{align*} (x_{i +1} x_i) ^p < \paren{ \frac{2L^2}{N}}^p (x_N x_0)^p \end{align*}
が成り立つ. すると, 
\begin{align*} c(x_N x_0)^ p \leq \sum (x_i x_{i+1}) ^p < N \frac{2 L^2}{N}^p (x_n x_0) ^p < c(x_N x_0) ^p \end{align*}
となり矛盾する. つまり, 適当な$i \in \cbra{1, \ldots , N-1}$ で $B(z_i ; \frac{1}{2N} \norm{z_N - z_0}) \cap f(X) = \varnothing$ となるものが存在するので, $f(X)$ は$V$ において$\frac{1}{2N}$-ULNCである. 

\qed
\end{pf*}











\end{document}