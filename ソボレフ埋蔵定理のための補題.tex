\documentclass[10pt, fleqn, label-section=none]{bxjsarticle}

%\usepackage[driver=dvipdfm,hmargin=25truemm,vmargin=25truemm]{geometry}

\setpagelayout{driver=dvipdfm,hmargin=25truemm,vmargin=20truemm}


\usepackage{amsmath}
\usepackage{amssymb}
\usepackage{amsfonts}
\usepackage{amsthm}
\usepackage{mathtools}
\usepackage{mleftright}





\usepackage{otf}

\theoremstyle{definition}
\newtheorem{dfn}{定義}[section]
\newtheorem{ex}[dfn]{例}
\newtheorem{lem}[dfn]{補題}
\newtheorem{prop}[dfn]{命題}
\newtheorem{thm}[dfn]{定理}
\newtheorem{cor}[dfn]{系}
\newtheorem*{pf*}{証明}
\newtheorem{problem}[dfn]{問題}
\newtheorem*{problem*}{問題}
\newtheorem{remark}[dfn]{注意}
\newtheorem*{claim*}{\underline{claim}}



\newtheorem*{solution*}{解答}

%箇条書きの様式
\renewcommand{\labelenumi}{(\arabic{enumi})}


%

\newcommand{\forany}{\rm{for} \ {}^{\forall}}
\newcommand{\foranyeps}{
\rm{for} \ {}^{\forall}\varepsilon >0}
\newcommand{\foranyk}{
\rm{for} \ {}^{\forall}k}


\newcommand{\any}{{}^{\forall}}
\newcommand{\suchthat}{\, \rm{s.t.} \, \it{}}




\newcommand{\veps}{\varepsilon}
\newcommand{\paren}[1]{\mleft( #1\mright )}
\newcommand{\cbra}[1]{\mleft\{#1\mright\}}
\newcommand{\sbra}[1]{\mleft\lbrack#1\mright\rbrack}
\newcommand{\tbra}[1]{\mleft\langle#1\mright\rangle}
\newcommand{\abs}[1]{\left|#1\right|}
\newcommand{\norm}[1]{\left\|#1\right\|}
\newcommand{\lopen}[1]{\mleft(#1\mright\rbrack}
\newcommand{\ropen}[1]{\mleft\lbrack #1 \mright)}



%
\newcommand{\Rn}{\mathbb{R}^n}
\newcommand{\Cn}{\mathbb{C}^n}

\newcommand{\Rm}{\mathbb{R}^m}
\newcommand{\Cm}{\mathbb{C}^m}


\newcommand{\projs}[2]{\it{p}_{#1,\ldots,#2}}
\newcommand{\projproj}[2]{\it{p}_{#1,#2}}

\newcommand{\proj}[1]{p_{#1}}

%可測空間
\newcommand{\stdProbSp}{\paren{\Omega, \mathcal{F}, P}}

%微分作用素
\newcommand{\ddt}{\frac{d}{dt}}
\newcommand{\ddx}{\frac{d}{dx}}
\newcommand{\ddy}{\frac{d}{dy}}

\newcommand{\delt}{\frac{\partial}{\partial t}}
\newcommand{\delx}{\frac{\partial}{\partial x}}

%ハイフン
\newcommand{\hyphen}{\text{-}}

%displaystyle
\newcommand{\dstyle}{\displaystyle}

%⇔, ⇒, \UTF{21D0}%
\newcommand{\LR}{\Leftrightarrow}
\newcommand{\naraba}{\Rightarrow}
\newcommand{\gyaku}{\Leftarrow}

%理由
\newcommand{\naze}[1]{\paren{\because {\mathop{ #1 }}}}

%
\newcommand{\sankaku}{\hfill $\triangle$}

%
\newcommand{\push}{_{\#}}

%手抜き
\newcommand{\textif}{\textrm{if}\,\,\,}
\newcommand{\Ric}{\textrm{Ric}}
\newcommand{\tr}{\textrm{tr}}
\newcommand{\vol}{\textrm{vol}}
\newcommand{\diam}{\textrm{diam}}
\newcommand{\supp}{\textrm{supp}}
\newcommand{\Med}{\textrm{Med}}
\newcommand{\Leb}{\textrm{Leb}}
\newcommand{\Const}{\textrm{Const}}
\newcommand{\Avg}{\textrm{Avg}}
\renewcommand{\;}{\, ; \,}
\renewcommand{\d}{\, {d}}


\title{正則性に関連する不等式1}
\date{}


\author{}


\begin{document}


\maketitle


\section{}

$k$ を相応に大きな非負整数とする. $f $ を$ H^k(\mathbb R^n)$ の元とする. $f \in L^2$ であるので, $L^2$におけるフーリエ変換が可能で, それを$\hat f$ で表すと, $\hat f \in L^2 (\mathbb R^n)$ である. 適当な定数$\Const$ を用いると
\begin{align*} \int_{\mathbb R^n} \abs{\hat f (\xi)}^2 (1 + \abs{\xi}^2 )^k d\xi  \leq \Const \norm{f}_{H^k}^2 \end{align*}
が成り立つ. $k$ が相応に大きいときには, $\int_{\mathbb R^n} (1 + \abs{\xi }^2 )^{-k}  d\xi < \infty$ であることに注意しつつ, 適当に式変形すると
\begin{align*} \paren{\int_{\mathbb R^n} \abs{\hat f (\xi) } d\xi}^2 &\leq \paren{\int_{\mathbb R^n} \abs{\hat f (\xi)} ^2 (1 + \abs{\xi}^2 ) ^k d \xi}\paren{\int_{\mathbb R^n} (1 + \abs{\xi }^2 )^{-k}  d\xi } \\
&\leq \Const \norm{f}^2_{H^k} \paren{\int_{\mathbb R^n} (1 + \abs{\xi }^2 )^{-k}  d\xi } < \infty \end{align*}
であるので, $\int_{\mathbb R^n} \abs{\hat f (\xi) } d\xi < \infty$ であるので, $\hat f \in L^1 (\mathbb R^n)$ である. 従って, $L^1$ におけるフーリエ変換の反転公式から, 殆ど至る所
\begin{align*} f(x) = \frac{1}{(2 \pi)^{\frac{n}{2}} } \int_{\mathbb R^n} \hat f (\xi) e^{ix \xi} d\xi  \end{align*}
が成り立つ.
\begin{align*} \abs {\int_{\mathbb R^n} \hat f (\xi) e^{iy \xi} d\xi - \int_{\mathbb R^n} \hat f (\xi) e^{ix \xi} d\xi } = \abs{\int_{\mathbb R^n} \hat f (\xi) (e^{iy \xi}  - e^{ix \xi} )} d\xi \leq \int_{\mathbb R^n} 2 \abs{\hat f (\xi)}  d\xi   < \infty  \end{align*}
を眺めると, 優収束定理を用いることで$f(x) = \int_{\mathbb R^n} \hat f (\xi) e^{ix \xi} d\xi $ が連続であることがわかる. 
また, 
\begin{align*} \abs{f(x)} = \abs{\int_{\mathbb R^n} \hat f (\xi) e^{ix \xi} d\xi} \leq \Const \norm{f}^2_{H^k} \paren{\int_{\mathbb R^n} (1 + \abs{\xi }^2 )^{-k}  d\xi }  \end{align*}
より, 
\begin{align*} \abs{f(x)} \leq \Const \norm{f}^2_{H^k}  \end{align*}
であるので, 
\begin{align*} \sup_{\mathbb R^n} \abs{f} \leq \Const \norm{f}^2_{H^k } \end{align*}
が成り立つ. さらに, $\abs \alpha \leq m $ であれば, $\partial^\alpha f \in H^{k-m}$ であることに注意しながら変形すると, 
\begin{align*} \sup_{\mathbb R^n} \abs{\partial ^\alpha f} \leq \Const \norm{\partial ^\alpha f}_{H^{k-m} } \leq  \Const \norm{f}_{H^k }  \end{align*}
であるので, 
\begin{align*} \norm{f}_{C^m} \leq \Const \norm{f}^2_{H^k}\end{align*}
が成り立つ. ($k$は具体的には少なくとも$k > \frac{n}{2} + m$ であればよい.)






\end{document}