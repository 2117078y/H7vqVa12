\documentclass[10pt, fleqn, label-section=none]{bxjsarticle}

%\usepackage[driver=dvipdfm,hmargin=25truemm,vmargin=25truemm]{geometry}

\setpagelayout{driver=dvipdfm,hmargin=25truemm,vmargin=20truemm}


\usepackage{amsmath}
\usepackage{amssymb}
\usepackage{amsfonts}
\usepackage{amsthm}
\usepackage{mathtools}
\usepackage{mleftright}





\usepackage{otf}

\theoremstyle{definition}
\newtheorem{dfn}{定義}[section]
\newtheorem{ex}[dfn]{例}
\newtheorem{lem}[dfn]{補題}
\newtheorem{prop}[dfn]{命題}
\newtheorem{thm}[dfn]{定理}
\newtheorem{cor}[dfn]{系}
\newtheorem*{pf*}{証明}
\newtheorem{problem}[dfn]{問題}
\newtheorem*{problem*}{問題}
\newtheorem{remark}[dfn]{注意}
\newtheorem*{claim*}{\underline{claim}}



\newtheorem*{solution*}{解答}

%箇条書きの様式
\renewcommand{\labelenumi}{(\arabic{enumi})}


%

\newcommand{\forany}{\rm{for} \ {}^{\forall}}
\newcommand{\foranyeps}{
\rm{for} \ {}^{\forall}\varepsilon >0}
\newcommand{\foranyk}{
\rm{for} \ {}^{\forall}k}


\newcommand{\any}{{}^{\forall}}
\newcommand{\suchthat}{\, \rm{s.t.} \, \it{}}




\newcommand{\veps}{\varepsilon}
\newcommand{\paren}[1]{\mleft( #1\mright )}
\newcommand{\cbra}[1]{\mleft\{#1\mright\}}
\newcommand{\sbra}[1]{\mleft\lbrack#1\mright\rbrack}
\newcommand{\tbra}[1]{\mleft\langle#1\mright\rangle}
\newcommand{\abs}[1]{\left|#1\right|}
\newcommand{\norm}[1]{\left\|#1\right\|}
\newcommand{\lopen}[1]{\mleft(#1\mright\rbrack}
\newcommand{\ropen}[1]{\mleft\lbrack #1 \mright)}



%
\newcommand{\Rn}{\mathbb{R}^n}
\newcommand{\Cn}{\mathbb{C}^n}

\newcommand{\Rm}{\mathbb{R}^m}
\newcommand{\Cm}{\mathbb{C}^m}


\newcommand{\projs}[2]{\it{p}_{#1,\ldots,#2}}
\newcommand{\projproj}[2]{\it{p}_{#1,#2}}

\newcommand{\proj}[1]{p_{#1}}

%可測空間
\newcommand{\stdProbSp}{\paren{\Omega, \mathcal{F}, P}}

%微分作用素
\newcommand{\ddt}{\frac{d}{dt}}
\newcommand{\ddx}{\frac{d}{dx}}
\newcommand{\ddy}{\frac{d}{dy}}

\newcommand{\delt}{\frac{\partial}{\partial t}}
\newcommand{\delx}{\frac{\partial}{\partial x}}

%ハイフン
\newcommand{\hyphen}{\text{-}}

%displaystyle
\newcommand{\dstyle}{\displaystyle}

%⇔, ⇒, \UTF{21D0}%
\newcommand{\LR}{\Leftrightarrow}
\newcommand{\naraba}{\Rightarrow}
\newcommand{\gyaku}{\Leftarrow}

%理由
\newcommand{\naze}[1]{\paren{\because {\mathop{ #1 }}}}

%
\newcommand{\sankaku}{\hfill $\triangle$}

%
\newcommand{\push}{_{\#}}

%手抜き
\newcommand{\textif}{\textrm{if}\,\,\,}
\newcommand{\Ric}{\textrm{Ric}}
\newcommand{\tr}{\textrm{tr}}
\newcommand{\vol}{\textrm{vol}}
\newcommand{\diam}{\textrm{diam}}
\newcommand{\supp}{\textrm{supp}}
\newcommand{\Med}{\textrm{Med}}
\newcommand{\Leb}{\textrm{Leb}}
\newcommand{\Const}{\textrm{Const}}
\newcommand{\Avg}{\textrm{Avg}}
\renewcommand{\;}{\, ; \,}
\renewcommand{\d}{\, {d}}


\title{}
\date{}


\author{}


\begin{document}


\maketitle


\section{}

\begin{remark} $m$ を非負整数, $\Omega \subset \mathbb R^n$ を開集合とする. 
$C^m (\Omega)$ には
\begin{align*} \norm{f} \coloneqq \max_{\abs \alpha \leq k} \sup_{\Omega} \abs{\partial^\alpha f} \end{align*}
で位相を定める. 
\end{remark}

\begin{remark}
$A,B$ を位相空間$X$ の部分集合とする. $A \subset B$ かつ, $A$ が$B$ の相対コンパクトな部分集合であるときに, $A \Subset B$ で表す. 
\end{remark}


\begin{remark} $m$ を非負整数, $\Omega \subset \mathbb R^n$ を開集合とする. 
相対コンパクト集合$\Omega^\prime \Subset \Omega$ に対して
\begin{align*} \norm{f}_{H^m(\Omega^\prime)} \coloneqq \paren{\sum_{\abs{\alpha} \leq m} \int_{\Omega} \abs{\partial ^\alpha f(x)} ^2 dx}^{\frac{1}{2}}\end{align*}
と定め, $H^m_{loc} (\Omega)$ には, セミノルムの族$\cbra{\norm{\cdot}_{H^m(\Omega^\prime)} \mid \Omega^\prime \Subset \Omega}$ により定まる位相を備える.
\end{remark}


\begin{prop} $m$ を非負整数, $\Omega \subset \mathbb R^n$ を開集合とする. 包含写像
\begin{align*} C^m (\Omega) \rightarrow H^m_{loc} (\Omega) \end{align*}
は連続である. 
\end{prop}
\begin{pf*}
\begin{align*} &\norm{f}_{H^m(\Omega^\prime)}^2 = \sum \norm{\partial^\alpha f}_{L^2(\Omega^\prime)}^2 \leq \Const ( \sum \sup \abs{\partial ^\alpha f}^2 ) \leq \Const ( \max_{\abs{\alpha } \leq m} \sup_{\Omega^\prime} \abs{\partial^\alpha f}^2 ) = \Const \norm{f}^2_{C^m (\Omega^\prime)}\end{align*}
\qed
\end{pf*}

\begin{remark}
$\cbra{H_t}$ をユークリッド空間上の熱半群とする.
\end{remark}


\begin{prop} $u, f \in \mathcal D (\mathbb R^n) $, $k$ を非負整数とする.  
\begin{align*} f = (-\Delta + \textrm{id} )^k u \end{align*}
とすると, $u$ の$x \in \mathbb R^n$ における値は
\begin{align*} u(x) = \int_0^\infty \frac{t^{k-1} e^{-t} }{(k-1)!} H_t f(x) dt\end{align*}
で与えられる.
\end{prop}
\begin{pf*}

\qed
\end{pf*}














\end{document}