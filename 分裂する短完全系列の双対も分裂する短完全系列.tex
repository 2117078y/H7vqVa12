\documentclass[10pt, fleqn, label-section=none]{bxjsarticle}

%\usepackage[driver=dvipdfm,hmargin=25truemm,vmargin=25truemm]{geometry}

\setpagelayout{driver=dvipdfm,hmargin=25truemm,vmargin=20truemm}


\usepackage{amsmath}
\usepackage{amssymb}
\usepackage{amsfonts}
\usepackage{amsthm}
\usepackage{mathtools}
\usepackage{mleftright}

\usepackage{ascmac}




\usepackage{otf}

\theoremstyle{definition}
\newtheorem{dfn}{定義}[section]
\newtheorem{ex}[dfn]{例}
\newtheorem{lem}[dfn]{補題}
\newtheorem{prop}[dfn]{命題}
\newtheorem{thm}[dfn]{定理}
\newtheorem{cor}[dfn]{系}
\newtheorem*{pf*}{証明}
\newtheorem{problem}[dfn]{問題}
\newtheorem*{problem*}{問題}
\newtheorem{remark}[dfn]{注意}
\newtheorem*{claim*}{\underline{claim}}



\newtheorem*{solution*}{解答}

%箇条書きの様式
\renewcommand{\labelenumi}{(\arabic{enumi})}


%

\newcommand{\forany}{\rm{for} \ {}^{\forall}}
\newcommand{\foranyeps}{
\rm{for} \ {}^{\forall}\varepsilon >0}
\newcommand{\foranyk}{
\rm{for} \ {}^{\forall}k}


\newcommand{\any}{{}^{\forall}}
\newcommand{\suchthat}{\, \rm{s.t.} \, \it{}}




\newcommand{\veps}{\varepsilon}
\newcommand{\paren}[1]{\mleft( #1\mright )}
\newcommand{\cbra}[1]{\mleft\{#1\mright\}}
\newcommand{\sbra}[1]{\mleft\lbrack#1\mright\rbrack}
\newcommand{\tbra}[1]{\mleft\langle#1\mright\rangle}
\newcommand{\abs}[1]{\left|#1\right|}
\newcommand{\norm}[1]{\left\|#1\right\|}
\newcommand{\lopen}[1]{\mleft(#1\mright\rbrack}
\newcommand{\ropen}[1]{\mleft\lbrack #1 \mright)}



%
\newcommand{\Rn}{\mathbb{R}^n}
\newcommand{\Cn}{\mathbb{C}^n}

\newcommand{\Rm}{\mathbb{R}^m}
\newcommand{\Cm}{\mathbb{C}^m}


\newcommand{\projs}[2]{\it{p}_{#1,\ldots,#2}}
\newcommand{\projproj}[2]{\it{p}_{#1,#2}}

\newcommand{\proj}[1]{p_{#1}}

%可測空間
\newcommand{\stdProbSp}{\paren{\Omega, \mathcal{F}, P}}

%微分作用素
\newcommand{\ddt}{\frac{d}{dt}}
\newcommand{\ddx}{\frac{d}{dx}}
\newcommand{\ddy}{\frac{d}{dy}}

\newcommand{\delt}{\frac{\partial}{\partial t}}
\newcommand{\delx}{\frac{\partial}{\partial x}}

%ハイフン
\newcommand{\hyphen}{\text{-}}

%displaystyle
\newcommand{\dstyle}{\displaystyle}

%⇔, ⇒, \UTF{21D0}%
\newcommand{\LR}{\Leftrightarrow}
\newcommand{\naraba}{\Rightarrow}
\newcommand{\gyaku}{\Leftarrow}

%理由
\newcommand{\naze}[1]{\paren{\because {\mathop{ #1 }}}}

%
\newcommand{\sankaku}{\hfill $\triangle$}

%
\newcommand{\push}{_{\#}}

%手抜き
\newcommand{\textif}{\textrm{if}\,\,\,}
\newcommand{\Ric}{\textrm{Ric}}
\newcommand{\tr}{\textrm{tr}}
\newcommand{\vol}{\textrm{vol}}
\newcommand{\diam}{\textrm{diam}}
\newcommand{\supp}{\textrm{supp}}
\newcommand{\Med}{\textrm{Med}}
\newcommand{\Leb}{\textrm{Leb}}
\newcommand{\Const}{\textrm{Const}}
\newcommand{\Avg}{\textrm{Avg}}
\newcommand{\id}{\textrm{id}}
\newcommand{\Ker}{\textrm{Ker}}
\newcommand{\im}{\textrm{Im}}




\renewcommand{\;}{\, ; \,}
\renewcommand{\d}{\, {d}}

\newcommand{\gyouretsu}[1]{\begin{pmatrix} #1 \end{pmatrix} }

%%図式

\usepackage[dvipdfm,all]{xy}


\newenvironment{claim}[1]{\par\noindent\underline{Step:}\space#1}{}
\newenvironment{claimproof}[1]{\par\noindent{($\because$)}\space#1}{\hfill $\blacktriangle $}


\title{分裂する短完全系列の双対も分裂する短完全系列}
\date{}


\author{}


\begin{document}


\maketitle



\section{}



$A,B,C$ で適当な環$R$ 上の加群を, $A^\sharp, B^\sharp, C^\sharp$ でそれぞれの双対加群を表す.


\begin{prop}
\begin{align*}  \xymatrix@C=13pt{ A \ar[r]^f &B \ar[r]^g &C \ar[r] & 0}   \end{align*}
を加群の完全系列とする. このとき, 
\begin{align*} \xymatrix@C=13pt{0 \ar[r] & C^\sharp \ar[r]^{g^\sharp}  & B^\sharp \ar[r]^{f^\sharp}   &A^\sharp  }  \end{align*}
も加群の完全系列である. 
\end{prop}
\begin{pf*}

\begin{claim}
$g^\sharp$ は単射である. 
\end{claim}
\begin{claimproof}
$g^\sharp c^\prime = 0$ である$c^\prime \in C^\sharp$ をとる. 任意に$c \in C$をとり, $gb = c$ となる$b \in B$ をとると, $c^\prime (c) = c^\prime (gb) = (g^\sharp c^\prime) (b) = 0$ なので, $c^\prime = 0$である
\end{claimproof}

\begin{claim}
$\im g^\sharp \subset \Ker f^\sharp$
\end{claim}
\begin{claimproof}
$f^\sharp \circ g^\sharp = (g \circ f) ^\sharp = 0$
\end{claimproof}

\begin{claim}
$\Ker f^\sharp \subset \im g^\sharp$
\end{claim}
\begin{claimproof}
$b^\prime \in \Ker f^\sharp$ をとる, $c^\prime \in C^\sharp$ を$c \in C$ に対して, $gb = c$ をみたす$b \in B$ を好きにとって, $c^\prime (c) \coloneqq b^\prime (b)$ とすることで定める. (もし, $gb_1 = gb_2 = c$ となる$b_1 , b_2\in B$ で$b^\prime ( b_1 ) \neq b^\prime ( b_2 )$ なるものがあると不良定義となる. が, $0 = g(b_1 - b_2)$ より$b_1 - b_2 \in \Ker g$ であるので, $b_1 - b_2 \in \im f$ なので, $a \in A$ で$f(a) = b_1 - b_2$ となるものをとると, $b^\prime(b_1 - b_2) = b^\prime (f(a)) = (f^\sharp b^\prime) (a) = 0$ となるので, $b^\prime (b_1 ) = b^\prime(b_2)$ となりきちんと定義されている. ) すると, $c^\prime (c) = c^\prime (gb) = (g^\sharp c^\prime) (b)$ 故に, $g^\sharp c^\prime = b  ^\prime$ となる.  
\end{claimproof}

\qed
\end{pf*}

\begin{prop}
\begin{align*}  \xymatrix@C=13pt{ 0 \ar[r] & A \ar[r]^f &B \ar[r]^g &C \ar[r] & 0}   \end{align*}
が分裂する短完全系列であるならば, 
\begin{align*} \xymatrix@C=13pt{0 \ar[r] & C^\sharp \ar[r]^{g^\sharp}  & B^\sharp \ar[r]^{f^\sharp}   &A^\sharp \ar[r] &0  }  \end{align*}
も分裂する短完全系列である. 
\end{prop}
\begin{pf*}
$h \circ f = \textrm{id}_A$ を満たす$h$ が存在するので, $f ^\sharp \circ h^\sharp = \textrm{id}_A$ より, $f^\sharp $ が全射である. 前述の命題と合わせると, 主張が従う.
\qed
\end{pf*}












\end{document}