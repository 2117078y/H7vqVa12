\documentclass[10pt, fleqn, label-section=none]{bxjsarticle}

%\usepackage[driver=dvipdfm,hmargin=25truemm,vmargin=25truemm]{geometry}

\setpagelayout{driver=dvipdfm,hmargin=25truemm,vmargin=20truemm}


\usepackage{amsmath}
\usepackage{amssymb}
\usepackage{amsfonts}
\usepackage{amsthm}
\usepackage{mathtools}
\usepackage{mleftright}

\usepackage{ascmac}




\usepackage{otf}

\theoremstyle{definition}
\newtheorem{dfn}{定義}[section]
\newtheorem{ex}[dfn]{例}
\newtheorem{lem}[dfn]{補題}
\newtheorem{prop}[dfn]{命題}
\newtheorem{thm}[dfn]{定理}
\newtheorem{setting}[dfn]{設定}
\newtheorem{notation}[dfn]{記号}
\newtheorem{cor}[dfn]{系}
\newtheorem*{pf*}{証明}
\newtheorem{problem}[dfn]{問題}
\newtheorem*{problem*}{問題}
\newtheorem{remark}[dfn]{注意}
\newtheorem*{claim*}{\underline{claim}}



\newtheorem*{solution*}{解答}

%箇条書きの様式
\renewcommand{\labelenumi}{(\arabic{enumi})}


%

\newcommand{\forany}{\rm{for} \ {}^{\forall}}
\newcommand{\foranyeps}{
\rm{for} \ {}^{\forall}\varepsilon >0}
\newcommand{\foranyk}{
\rm{for} \ {}^{\forall}k}


\newcommand{\any}{{}^{\forall}}
\newcommand{\suchthat}{\, \rm{s.t.} \, \it{}}




\newcommand{\veps}{\varepsilon}
\newcommand{\paren}[1]{\mleft( #1\mright )}
\newcommand{\cbra}[1]{\mleft\{#1\mright\}}
\newcommand{\sbra}[1]{\mleft\lbrack#1\mright\rbrack}
\newcommand{\tbra}[1]{\mleft\langle#1\mright\rangle}
\newcommand{\abs}[1]{\left|#1\right|}
\newcommand{\norm}[1]{\left\|#1\right\|}
\newcommand{\lopen}[1]{\mleft(#1\mright\rbrack}
\newcommand{\ropen}[1]{\mleft\lbrack #1 \mright)}



%
\newcommand{\Rn}{\mathbb{R}^n}
\newcommand{\Cn}{\mathbb{C}^n}

\newcommand{\Rm}{\mathbb{R}^m}
\newcommand{\Cm}{\mathbb{C}^m}


\newcommand{\projs}[2]{\it{p}_{#1,\ldots,#2}}
\newcommand{\projproj}[2]{\it{p}_{#1,#2}}

\newcommand{\proj}[1]{p_{#1}}

%可測空間
\newcommand{\stdProbSp}{\paren{\Omega, \mathcal{F}, P}}

%微分作用素
\newcommand{\ddt}{\frac{d}{dt}}
\newcommand{\ddx}{\frac{d}{dx}}
\newcommand{\ddy}{\frac{d}{dy}}

\newcommand{\delt}{\frac{\partial}{\partial t}}
\newcommand{\delx}{\frac{\partial}{\partial x}}

%ハイフン
\newcommand{\hyphen}{\text{-}}

%displaystyle
\newcommand{\dstyle}{\displaystyle}

%⇔, ⇒, \UTF{21D0}%
\newcommand{\LR}{\Leftrightarrow}
\newcommand{\naraba}{\Rightarrow}
\newcommand{\gyaku}{\Leftarrow}

%理由
\newcommand{\naze}[1]{\paren{\because {\mathop{ #1 }}}}

%
\newcommand{\sankaku}{\hfill $\triangle$}

%
\newcommand{\push}{_{\#}}

%手抜き
\newcommand{\textif}{\textrm{if}\,\,\,}
\newcommand{\Ric}{\textrm{Ric}}
\newcommand{\tr}{\textrm{tr}}
\newcommand{\vol}{\textrm{vol}}
\newcommand{\diam}{\textrm{diam}}
\newcommand{\supp}{\textrm{supp}}
\newcommand{\Med}{\textrm{Med}}
\newcommand{\Leb}{\textrm{Leb}}
\newcommand{\Const}{\textrm{Const}}
\newcommand{\Avg}{\textrm{Avg}}
\newcommand{\id}{\textrm{id}}
\newcommand{\Ker}{\textrm{Ker}}
\newcommand{\im}{\textrm{Im}}
\newcommand{\dil}{\textrm{dil}}
\newcommand{\Ch}{\textrm{Ch}}
\newcommand{\Lip}{\textrm{Lip}}
\newcommand{\Ent}{\textrm{Ent}}
\newcommand{\grad}{\textrm{grad}}
\newcommand{\dom}{\textrm{dom}}
\newcommand{\diag}{\textrm{diag}}

\renewcommand{\;}{\, ; \,}
\renewcommand{\d}{\, {d}}

\newcommand{\gyouretsu}[1]{\begin{pmatrix} #1 \end{pmatrix} }

\renewcommand{\div}{\textrm{div}}


%%図式

\usepackage[dvipdfm,all]{xy}


\newenvironment{claim}[1]{\par\noindent\underline{step:}\space#1}{}
\newenvironment{claimproof}[1]{\par\noindent{($\because$)}\space#1}{\hfill $\blacktriangle $}


\newcommand{\pprime}{{\prime \prime}}

%%マグニチュード


\newcommand{\Mag}{\textrm{Mag}}

\usepackage{mathrsfs}


%%6.13
\def\chint#1{\mathchoice
{\XXint\displaystyle\textstyle{#1}}%
{\XXint\textstyle\scriptstyle{#1}}%
{\XXint\scriptstyle\scriptscriptstyle{#1}}%
{\XXint\scriptscriptstyle\scriptscriptstyle{#1}}%
\!\int}
\def\XXint#1#2#3{{\setbox0=\hbox{$#1{#2#3}{\int}$ }
\vcenter{\hbox{$#2#3$ }}\kern-.6\wd0}}
\def\ddashint{\chint=}
\def\dashint{\chint-}

%%7.13

\usepackage{here}

%7.15
\newcommand{\Span}{\textrm{Span}}

\newcommand{\Conv}{\textrm{Conv}}


\title{ピクセルと$l_1$ 凸集合}
\date{}


\author{}


\begin{document}


\maketitle

\section{}

\begin{remark}ちょっとこのノートの完成度低いです. 

\end{remark}


\begin{notation}$(X, d)$ を距離空間とする. 
\begin{align*} &((x, y)) \coloneqq \cbra{z \in X \mid xy = xz + zy, \quad x \neq z \neq y} \\& [[x, y]] \coloneqq \cbra{z \in X \mid xy = xz + zy}\end{align*}
という記号を用いることにする. 
\end{notation}


\begin{notation}$C_n \coloneqq [-1/2, 1/2]^n$ で$n$ 次元単位立方体を表す. $\mathbb H \coloneqq \cbra{z + 1/2 \mid z \in \mathbb Z}$ で半整数全体を表す. 

\end{notation}

\begin{dfn}
\begin{align*} P_x(\lambda) \coloneqq \cbra{\lambda(h + C_n) \mid h \in \mathbb H^n} \end{align*}
を$\lambda$ ピクセル集合族という. $\lambda$ ピクセル集合族の要素を, $\lambda$ピクセル集合という. 
\end{dfn}

\begin{dfn}($\lambda$-ピクセル). $A \subset \mathbb R^n$ に対して, 
\begin{align*} A_{\lambda } \coloneqq \bigcup_{\lambda \in \mathbb H ^n , \lambda(h + C_n) \cap A \neq \varnothing} \lambda(h + C_n)\end{align*}
と定める. 
\end{dfn}

\begin{prop}$I \subset \mathbb R^n$ を$n$ 次元有界閉区間(つまり有界閉区間の$n$ 個の直積)とする. $F \subset \mathbb R^n $ を$x, y \in F$ が
\begin{align*} x + I \cap y + I = \varnothing  \end{align*}
を満たすならば, $z \in F$ で$d(x, z) + d(z, y) = d(x, y), \quad x \neq z, \quad z \neq y$  を満たすものが存在するような閉集合とする. このとき, $F + I$ は$l_1$ 凸集合である. 
\end{prop}
\begin{pf*}
工事中. 
\qed
\end{pf*}

\begin{prop}$A \subset \mathbb R^n$ が$l_1$ 凸集合であるならば, 任意の$\lambda > 0$ に対して, $A_\lambda $ は$l_1$ 凸集合である.  

\end{prop}
\begin{pf*}
\begin{align*} L(\lambda) \coloneqq \cbra{h \in \mathbb H^n \mid \lambda (h + C_n) \cap A \neq \varnothing} \end{align*}
とする. このとき, $X = \lambda L(\lambda) + \lambda C_n$ が成り立つ. $\lambda h, \lambda h^\prime \in  \lambda L(\lambda)$ で$\lambda(h + C_n) \cap \lambda(h^\prime + C_n) = \varnothing$ を満たすものをとる. $\lambda h_1 < \lambda h^\prime_1$ となるようにしておく(ただし, $h_1, h^\prime_1$ はそれぞれ$h, h^\prime$ の第一成分である). 適当に$x \in \lambda(h + C_n) \cap A, x^\prime \in \lambda(h + C_n) \cap A$ をとる. $A$ が$l_1$ 凸であるので, $y \in ((x, x^\prime))$ で
\begin{align*} x_1 \leq \lambda (h_1 + 1/2) < y_1 <  \lambda (h^\prime_1 - 1/2) \leq x^\prime_1  \end{align*}
を満たすものが取れる(ただし, $x_1, x^\prime_1, y_1$ は$x, x^\prime, y$ の第一成分). 適当な$k \in \mathbb H^n$ で, $y \in \lambda(k +  C_n) $ かつ, $k \in [[h, h^\prime]]$ を満たすものがとれる. $k_1$ を$k$ の第一成分とすると, $\lambda (h_1 + 1/2) < y_1 <  \lambda (h^\prime_1 - 1/2)$ より, $h_1 < k_1 < h^\prime_1$ であるので, $h \neq k \neq h^\prime $ であるので, $k \in ((h, h^\prime))$ が成り立つ. $\lambda L(\lambda)$ が閉集合であることから, 前述の命題より, $A_\lambda $ は$l_1$ 凸集合である.
\qed
\end{pf*}

\begin{dfn}$A \subset \mathbb R^n$ は$\lambda$ ピクセル集合の有限和で表されるとき, $\lambda$ -pixellated 集合という. 適当な$\lambda$ に対して$\lambda$-pixellated 集合である集合を, 単にpixellated 集合という. 

\end{dfn}

\begin{prop}コンパクト$l_1$ 凸集合全体において, pixellated $l_1$ 凸集合 全体はハウスドルフ距離に関して稠密である. 

\end{prop}
\begin{pf*}集合$A$ に対して$A_\lambda$ は$\lambda$ を小さくしていけば$A$ にいくらでも近くなるようにとれる. あとはコンパクト性から有限個のピクセル集合で被覆できるかどうかは適当に検討すればいい. 
\qed
\end{pf*}






\end{document}