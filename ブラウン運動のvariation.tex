\documentclass[10pt, fleqn, label-section=none]{bxjsarticle}

%\usepackage[driver=dvipdfm,hmargin=25truemm,vmargin=25truemm]{geometry}

\setpagelayout{driver=dvipdfm,hmargin=25truemm,vmargin=20truemm}


\usepackage{amsmath}
\usepackage{amssymb}
\usepackage{amsfonts}
\usepackage{amsthm}
\usepackage{mathtools}
\usepackage{mleftright}

\usepackage{ascmac}




\usepackage{otf}

\theoremstyle{definition}
\newtheorem{dfn}{定義}[section]
\newtheorem{ex}[dfn]{例}
\newtheorem{lem}[dfn]{補題}
\newtheorem{prop}[dfn]{命題}
\newtheorem{thm}[dfn]{定理}
\newtheorem{setting}[dfn]{設定}
\newtheorem{cor}[dfn]{系}
\newtheorem*{pf*}{証明}
\newtheorem{problem}[dfn]{問題}
\newtheorem*{problem*}{問題}
\newtheorem{remark}[dfn]{注意}
\newtheorem*{claim*}{\underline{claim}}



\newtheorem*{solution*}{解答}

%箇条書きの様式
\renewcommand{\labelenumi}{(\arabic{enumi})}


%

\newcommand{\forany}{\rm{for} \ {}^{\forall}}
\newcommand{\foranyeps}{
\rm{for} \ {}^{\forall}\varepsilon >0}
\newcommand{\foranyk}{
\rm{for} \ {}^{\forall}k}


\newcommand{\any}{{}^{\forall}}
\newcommand{\suchthat}{\, \rm{s.t.} \, \it{}}




\newcommand{\veps}{\varepsilon}
\newcommand{\paren}[1]{\mleft( #1\mright )}
\newcommand{\cbra}[1]{\mleft\{#1\mright\}}
\newcommand{\sbra}[1]{\mleft\lbrack#1\mright\rbrack}
\newcommand{\tbra}[1]{\mleft\langle#1\mright\rangle}
\newcommand{\abs}[1]{\left|#1\right|}
\newcommand{\norm}[1]{\left\|#1\right\|}
\newcommand{\lopen}[1]{\mleft(#1\mright\rbrack}
\newcommand{\ropen}[1]{\mleft\lbrack #1 \mright)}



%
\newcommand{\Rn}{\mathbb{R}^n}
\newcommand{\Cn}{\mathbb{C}^n}

\newcommand{\Rm}{\mathbb{R}^m}
\newcommand{\Cm}{\mathbb{C}^m}


\newcommand{\projs}[2]{\it{p}_{#1,\ldots,#2}}
\newcommand{\projproj}[2]{\it{p}_{#1,#2}}

\newcommand{\proj}[1]{p_{#1}}

%可測空間
\newcommand{\stdProbSp}{\paren{\Omega, \mathcal{F}, P}}

%微分作用素
\newcommand{\ddt}{\frac{d}{dt}}
\newcommand{\ddx}{\frac{d}{dx}}
\newcommand{\ddy}{\frac{d}{dy}}

\newcommand{\delt}{\frac{\partial}{\partial t}}
\newcommand{\delx}{\frac{\partial}{\partial x}}

%ハイフン
\newcommand{\hyphen}{\text{-}}

%displaystyle
\newcommand{\dstyle}{\displaystyle}

%⇔, ⇒, \UTF{21D0}%
\newcommand{\LR}{\Leftrightarrow}
\newcommand{\naraba}{\Rightarrow}
\newcommand{\gyaku}{\Leftarrow}

%理由
\newcommand{\naze}[1]{\paren{\because {\mathop{ #1 }}}}

%
\newcommand{\sankaku}{\hfill $\triangle$}

%
\newcommand{\push}{_{\#}}

%手抜き
\newcommand{\textif}{\textrm{if}\,\,\,}
\newcommand{\Ric}{\textrm{Ric}}
\newcommand{\tr}{\textrm{tr}}
\newcommand{\vol}{\textrm{vol}}
\newcommand{\diam}{\textrm{diam}}
\newcommand{\supp}{\textrm{supp}}
\newcommand{\Med}{\textrm{Med}}
\newcommand{\Leb}{\textrm{Leb}}
\newcommand{\Const}{\textrm{Const}}
\newcommand{\Avg}{\textrm{Avg}}
\newcommand{\id}{\textrm{id}}
\newcommand{\Ker}{\textrm{Ker}}
\newcommand{\im}{\textrm{Im}}




\renewcommand{\;}{\, ; \,}
\renewcommand{\d}{\, {d}}

\newcommand{\gyouretsu}[1]{\begin{pmatrix} #1 \end{pmatrix} }

%%図式

\usepackage[dvipdfm,all]{xy}


\newenvironment{claim}[1]{\par\noindent\underline{step:}\space#1}{}
\newenvironment{claimproof}[1]{\par\noindent{($\because$)}\space#1}{\hfill $\blacktriangle $}


\newcommand{\pprime}{{\prime \prime}}





%%


\title{ブラウウン運動のvariation}
\date{}


\author{}


\begin{document}


\maketitle

\section{}


\begin{setting}
区間$[0,t]$ の$n$分割と, 最大区間幅を 
\begin{align*} &\Delta_n [0,t] \coloneqq \cbra{(t_0^n, t_1^n, \ldots, t_n^n) \in \mathbb R^n \mid 0 = t_0^n \leq t_1^n \leq \cdots \leq t_n^n = t }\\
&\abs{\Delta_n[0,t] } = \max \cbra{\abs{t_{k+1}^n - t_k^n \mid k = 0, \ldots, n-1}}
\end{align*}
により定める. 


\end{setting}


\begin{prop}(ブラウン運動のvariation). $\cbra{B_t}$ を標準ブラウン運動とする. $t \geq 0 $ とする. 任意の, 分割の列 で
\begin{align*} \lim_n \abs{\Delta_n [0,t]} = 0 \end{align*}
を満たすものに対して, $L^2$ の意味で
\begin{align*} \sum_{k=1}^n \paren{B_{t^n_k} - B_{t^n_{k-1} } }^2 \rightarrow  t \end{align*}
が成り立つ.
\end{prop}
\begin{pf*}
\begin{align*} V_n \coloneqq \sum_{k=1} ^n (     B_{t^n_k} - B_{t^n_{k-1} }    ) ^2 \end{align*}
と定める. 
\begin{align*} &E((V_n - t)^2) \\
&= E(V_n^2) - 2tE(V_n) + t^2 \\
&= \sum_{j, k} E( ( B_{t^n_j} - B_{t^n_{j-1} }  ) ^2 ( B_{t^n_k} - B_{t^n_{k-1} }  )^2 ) - 2t^2 + t^2 \\
&= \sum_{k} E(( B_{t^n_k} - B_{t^n_{k-1} }  ) ^4) + 2 \sum_[j < k] E(( B_{t^n_j} - B_{t^n_{j-1} }  ) ^2 ( B_{t^n_k} - B_{t^n_{k-1} }  ) ^2 ) - t^2 \\
&= \sum_k (t^n_k - t^n_{k-1}) ^2 E(B_1 ^4) + 2 \sum_{j < k} (t^n_j - t^n_{j-1}) (t^n_k - t^n_{k-1} ) - t^2 \\
&= 3 \sum_k  (t^n_j - t^n_{j-1}) ^2 + 2 \sum_{j < k} (t^n _j - t^n_{j-1}) (t^n_k - t^n_{k-1}) - t^2 \\
&= 3 \sum_k  (t^n_j - t^n_{j-1}) ^2 + 2 \sum_{j < k} (t^n _j - t^n_{j-1}) (t^n_k - t^n_{k-1})  - (\sum_j (t^n _j - t^n_{j-1})) (\sum_k (t^n _k - t^n_{k-1})) \\
&= 3 \sum_k (t^n_j - t^n_{j-1}) ^2  - \sum_k  (t^n_j - t^n_{j-1}) ^2  \\
&= 2  \sum_k (t^n_j - t^n_{j-1}) ^2 \leq 2t \abs{\Delta_n [0,t]}
 \end{align*}
 となることから, 極限をとることで$L^2$ 収束することが従う. 
\qed
\end{pf*}

\begin{prop}
$\cbra{B_t}$ を標準ブラウン運動とする. $t > 0 $ とする. 任意の, 分割の列 で
\begin{align*} \lim_n \abs{\Delta_n [0,t]} = 0 \end{align*}
を満たすものに対して, 
\begin{align*} \sum_k \abs{B_{t^n_k } - B_{t^n_{k-1}}  }  \rightarrow  \infty   \end{align*}
が$a.s.$で成り立つ. 
\end{prop}
\begin{pf*}
背理法により, ある分割の列で, 発散しないものが存在すると仮定する. 
\begin{align*} M = \sup_n \sum_k \abs{B_{t^n_k } - B_{t^n_{k-1}}  }  < \infty    \end{align*}
であると仮定する. 前述の命題より, $L^2$ の意味で
\begin{align*} \sum_{k=1}^n \paren{B_{t^n_k} - B_{t^n_{k-1} } }^2 \rightarrow  t \end{align*}
が成り立つので, 確率収束の意味でこれが成り立つ. 従って, 適当な部分列をとって$a.s.$で$ \sum_k (B_{t^n_k} - B_{t^n_{k-1} } )^2 $ が$t$ に収束するようにできる.  しかし, 
\begin{align*} \sum_k (B_{t^n_k} - B_{t^n_{k-1} } )^2  \leq M \sup_{1 \leq k \leq n} \abs{B_{t^n_k} - B_{t^n_{k-1} } }         \end{align*}
が成り立ち, $\cbra{B_t}$ の連続性より右辺は分割を細かくすると$0$に$a.s.$で収束する. よって矛盾する. 
\qed
\end{pf*}













\end{document}