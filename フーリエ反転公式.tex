\documentclass[10pt, fleqn, label-section=none]{bxjsarticle}

%\usepackage[driver=dvipdfm,hmargin=25truemm,vmargin=25truemm]{geometry}

\setpagelayout{driver=dvipdfm,hmargin=25truemm,vmargin=20truemm}


\usepackage{amsmath}
\usepackage{amssymb}
\usepackage{amsfonts}
\usepackage{amsthm}
\usepackage{mathtools}
\usepackage{mleftright}

\usepackage{ascmac}




\usepackage{otf}

\theoremstyle{definition}
\newtheorem{dfn}{定義}[section]
\newtheorem{ex}[dfn]{例}
\newtheorem{lem}[dfn]{補題}
\newtheorem{prop}[dfn]{命題}
\newtheorem{thm}[dfn]{定理}
\newtheorem{setting}[dfn]{設定}
\newtheorem{notation}[dfn]{記号}
\newtheorem{cor}[dfn]{系}
\newtheorem*{pf*}{証明}
\newtheorem{problem}[dfn]{問題}
\newtheorem*{problem*}{問題}
\newtheorem{remark}[dfn]{注意}
\newtheorem*{claim*}{\underline{claim}}



\newtheorem*{solution*}{解答}

%箇条書きの様式
\renewcommand{\labelenumi}{(\arabic{enumi})}


%

\newcommand{\forany}{\rm{for} \ {}^{\forall}}
\newcommand{\foranyeps}{
\rm{for} \ {}^{\forall}\varepsilon >0}
\newcommand{\foranyk}{
\rm{for} \ {}^{\forall}k}


\newcommand{\any}{{}^{\forall}}
\newcommand{\suchthat}{\, \rm{s.t.} \, \it{}}




\newcommand{\veps}{\varepsilon}
\newcommand{\paren}[1]{\mleft( #1\mright )}
\newcommand{\cbra}[1]{\mleft\{#1\mright\}}
\newcommand{\sbra}[1]{\mleft\lbrack#1\mright\rbrack}
\newcommand{\tbra}[1]{\mleft\langle#1\mright\rangle}
\newcommand{\abs}[1]{\left|#1\right|}
\newcommand{\norm}[1]{\left\|#1\right\|}
\newcommand{\lopen}[1]{\mleft(#1\mright\rbrack}
\newcommand{\ropen}[1]{\mleft\lbrack #1 \mright)}



%
\newcommand{\Rn}{\mathbb{R}^n}
\newcommand{\Cn}{\mathbb{C}^n}

\newcommand{\Rm}{\mathbb{R}^m}
\newcommand{\Cm}{\mathbb{C}^m}


\newcommand{\projs}[2]{\it{p}_{#1,\ldots,#2}}
\newcommand{\projproj}[2]{\it{p}_{#1,#2}}

\newcommand{\proj}[1]{p_{#1}}

%可測空間
\newcommand{\stdProbSp}{\paren{\Omega, \mathcal{F}, P}}

%微分作用素
\newcommand{\ddt}{\frac{d}{dt}}
\newcommand{\ddx}{\frac{d}{dx}}
\newcommand{\ddy}{\frac{d}{dy}}

\newcommand{\delt}{\frac{\partial}{\partial t}}
\newcommand{\delx}{\frac{\partial}{\partial x}}

%ハイフン
\newcommand{\hyphen}{\text{-}}

%displaystyle
\newcommand{\dstyle}{\displaystyle}

%⇔, ⇒, \UTF{21D0}%
\newcommand{\LR}{\Leftrightarrow}
\newcommand{\naraba}{\Rightarrow}
\newcommand{\gyaku}{\Leftarrow}

%理由
\newcommand{\naze}[1]{\paren{\because {\mathop{ #1 }}}}

%
\newcommand{\sankaku}{\hfill $\triangle$}

%
\newcommand{\push}{_{\#}}

%手抜き
\newcommand{\textif}{\textrm{if}\,\,\,}
\newcommand{\Ric}{\textrm{Ric}}
\newcommand{\tr}{\textrm{tr}}
\newcommand{\vol}{\textrm{vol}}
\newcommand{\diam}{\textrm{diam}}
\newcommand{\supp}{\textrm{supp}}
\newcommand{\Med}{\textrm{Med}}
\newcommand{\Leb}{\textrm{Leb}}
\newcommand{\Const}{\textrm{Const}}
\newcommand{\Avg}{\textrm{Avg}}
\newcommand{\id}{\textrm{id}}
\newcommand{\Ker}{\textrm{Ker}}
\newcommand{\im}{\textrm{Im}}
\newcommand{\dil}{\textrm{dil}}
\newcommand{\Ch}{\textrm{Ch}}
\newcommand{\Lip}{\textrm{Lip}}
\newcommand{\Ent}{\textrm{Ent}}
\newcommand{\grad}{\textrm{grad}}
\newcommand{\dom}{\textrm{dom}}
\newcommand{\diag}{\textrm{diag}}

\renewcommand{\;}{\, ; \,}
\renewcommand{\d}{\, {d}}

\newcommand{\gyouretsu}[1]{\begin{pmatrix} #1 \end{pmatrix} }

\renewcommand{\div}{\textrm{div}}


%%図式

\usepackage[dvipdfm,all]{xy}


\newenvironment{claim}[1]{\par\noindent\underline{step:}\space#1}{}
\newenvironment{claimproof}[1]{\par\noindent{($\because$)}\space#1}{\hfill $\blacktriangle $}


\newcommand{\pprime}{{\prime \prime}}

%%マグニチュード


\newcommand{\Mag}{\textrm{Mag}}

\usepackage{mathrsfs}


%%6.13
\def\Xint#1{\mathchoice
{\XXint\displaystyle\textstyle{#1}}%
{\XXint\textstyle\scriptstyle{#1}}%
{\XXint\scriptstyle\scriptscriptstyle{#1}}%
{\XXint\scriptscriptstyle\scriptscriptstyle{#1}}%
\!\int}
\def\XXint#1#2#3{{\setbox0=\hbox{$#1{#2#3}{\int}$ }
\vcenter{\hbox{$#2#3$ }}\kern-.6\wd0}}
\def\ddashint{\Xint=}
\def\dashint{\Xint-}



\title{フーリエ反転公式}
\date{}


\author{}


\begin{document}


\maketitle

\section{}

\begin{prop}$f \in L^1(\mathbb R), \mathcal F f \in L^1(\mathbb R)$ であるならば, 
\begin{align*} f(x) = \mathcal F^{-1} \mathcal F f(x)  \end{align*}
が成り立つ. 
\end{prop}
\begin{pf*}

\begin{align*} f_t(x)              \coloneqq                  \frac{1}{2 \pi} \int_{\mathbb R} e^{ix \xi} e^{-t \xi ^2} \paren{\int_{\mathbb R} e^{- i \xi y} f(y) dy } d \xi                 \end{align*}
と定める. $e^{ix \xi} e^{-t \xi ^2} e^{- i \xi y} f(y)$ が$y$ に関して可積分で, $e^{- t \xi ^2} \norm{f}_{L^1}$ が$\xi$ に関して可積分であるので, フビニの定理より, 
\begin{align*} f_t (x) &=     \frac{1}{2 \pi} \int_{\mathbb R} \paren{ \int_{\mathbb R} e^{i(x - y) \xi} e^{-t \xi ^2}  d\xi }   f(y)                   dy \\&= \frac{1}{2 \pi} \int_{\mathbb R} \paren{\frac{ \pi }{t}}^{\frac{1}{2}} e^{- \frac{\norm{x - y} ^2 }{4t}}  f(y)      dy     \\& =    \int_{\mathbb R} \frac{1}{(4 \pi t)^{\frac{1}{2}}} e^{- \frac{\norm{x - y} ^2 }{4t}}  f(y)      dy      \end{align*}
が成り立つ. $\frac{1}{(4 \pi t)^{\frac{1}{2}}} e^{- \frac{\norm{x - y} ^2 }{4t}} $ は$\mathbb R$ における熱核であるので, 
\begin{align*} \lim_{t \rightarrow 0} \norm{f_t - f}_{L^1} = 0 \end{align*} 
が成り立つ. 従って, 概収束部分列がとれることと, $\abs{e^{ix\xi}e^{-t\xi^2} \mathcal F f (\xi)} \leq \abs{\mathcal F f(\xi)}  \in L^1$ より優収束定理が成り立つことから, 
\begin{align*} f(x) = \lim_{t \rightarrow 0} f_t (x) =  \frac{1}{2 \pi} \int_{\mathbb R} e^{ix \xi} \paren{\int_{\mathbb R} e^{- i \xi y} f(y) dy } d \xi   = \mathcal F^{-1} \mathcal F f (x) \end{align*}
が成り立つ. 
\qed
\end{pf*}



\begin{prop}$f \in \mathcal S (\mathbb R)$ に対して, 
\begin{align*} f(x) = \mathcal F^{-1} \mathcal F f(x) \end{align*}
が成り立つ. 
\end{prop}
\begin{pf*}
$ S (\mathbb R) \subset L^1 (\mathbb R)$ であることと, 急減少関数のフーリエ変換が急減少関数であることから, 主張が従う. 
\qed
\end{pf*}







\end{document}
