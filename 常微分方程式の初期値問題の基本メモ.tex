\documentclass[10pt, fleqn, label-section=none]{bxjsarticle}

%\usepackage[driver=dvipdfm,hmargin=25truemm,vmargin=25truemm]{geometry}

\setpagelayout{driver=dvipdfm,hmargin=25truemm,vmargin=20truemm}


\usepackage{amsmath}
\usepackage{amssymb}
\usepackage{amsfonts}
\usepackage{amsthm}
\usepackage{mathtools}
\usepackage{mleftright}





\usepackage{otf}

\theoremstyle{definition}
\newtheorem{dfn}{定義}[section]
\newtheorem{ex}[dfn]{例}
\newtheorem{lem}[dfn]{補題}
\newtheorem{prop}[dfn]{命題}
\newtheorem{thm}[dfn]{定理}
\newtheorem{cor}[dfn]{系}
\newtheorem*{pf*}{証明}
\newtheorem{problem}[dfn]{問題}
\newtheorem*{problem*}{問題}
\newtheorem{remark}[dfn]{注意}
\newtheorem*{claim*}{\underline{claim}}



\newtheorem*{solution*}{解答}

%箇条書きの様式
\renewcommand{\labelenumi}{(\arabic{enumi})}


%

\newcommand{\forany}{\rm{for} \ {}^{\forall}}
\newcommand{\foranyeps}{
\rm{for} \ {}^{\forall}\varepsilon >0}
\newcommand{\foranyk}{
\rm{for} \ {}^{\forall}k}


\newcommand{\any}{{}^{\forall}}
\newcommand{\suchthat}{\, \rm{s.t.} \, \it{}}




\newcommand{\veps}{\varepsilon}
\newcommand{\paren}[1]{\mleft( #1\mright )}
\newcommand{\cbra}[1]{\mleft\{#1\mright\}}
\newcommand{\sbra}[1]{\mleft\lbrack#1\mright\rbrack}
\newcommand{\tbra}[1]{\mleft\langle#1\mright\rangle}
\newcommand{\abs}[1]{\left|#1\right|}
\newcommand{\norm}[1]{\left\|#1\right\|}
\newcommand{\lopen}[1]{\mleft(#1\mright\rbrack}
\newcommand{\ropen}[1]{\mleft\lbrack #1 \mright)}



%
\newcommand{\Rn}{\mathbb{R}^n}
\newcommand{\Cn}{\mathbb{C}^n}

\newcommand{\Rm}{\mathbb{R}^m}
\newcommand{\Cm}{\mathbb{C}^m}


\newcommand{\projs}[2]{\it{p}_{#1,\ldots,#2}}
\newcommand{\projproj}[2]{\it{p}_{#1,#2}}

\newcommand{\proj}[1]{p_{#1}}

%可測空間
\newcommand{\stdProbSp}{\paren{\Omega, \mathcal{F}, P}}

%微分作用素
\newcommand{\ddt}{\frac{d}{dt}}
\newcommand{\ddx}{\frac{d}{dx}}
\newcommand{\ddy}{\frac{d}{dy}}

\newcommand{\delt}{\frac{\partial}{\partial t}}
\newcommand{\delx}{\frac{\partial}{\partial x}}

%ハイフン
\newcommand{\hyphen}{\text{-}}

%displaystyle
\newcommand{\dstyle}{\displaystyle}

%⇔, ⇒, \UTF{21D0}%
\newcommand{\LR}{\Leftrightarrow}
\newcommand{\naraba}{\Rightarrow}
\newcommand{\gyaku}{\Leftarrow}

%理由
\newcommand{\naze}[1]{\paren{\because {\mathop{ #1 }}}}

%
\newcommand{\sankaku}{\hfill $\triangle$}

%
\newcommand{\push}{_{\#}}

%手抜き
\newcommand{\textif}{\textrm{if}\,\,\,}
\newcommand{\Ric}{\textrm{Ric}}
\newcommand{\tr}{\textrm{tr}}
\newcommand{\vol}{\textrm{vol}}
\newcommand{\diam}{\textrm{diam}}
\newcommand{\supp}{\textrm{supp}}
\newcommand{\Med}{\textrm{Med}}
\newcommand{\Leb}{\textrm{Leb}}
\newcommand{\Const}{\textrm{Const}}
\newcommand{\Avg}{\textrm{Avg}}
\renewcommand{\;}{\, ; \,}
\renewcommand{\d}{\, {d}}


\title{ODE初期値問題の局所的な解の存在と一意性}
\date{}


\author{}


\begin{document}


\maketitle


\section{}

\begin{remark}
変数の記号が$t$ であるものを時間変数, $x$ であるものを空間変数と呼ぶことにする. また $\abs{\cdot}$ で$L^1$ ノルムを表す. 
\end{remark}

\begin{prop} $f: \mathbb R \times \mathbb R^n \rightarrow \mathbb R^n$ を
\begin{align*} D \coloneqq \cbra{(t,x) \mid \abs t \leq a, \abs{x - x_0} \leq b} \end{align*}
を定義域に含む写像とする. \\
$f$が$D$ で連続かつ, 空間変数に関してリプシッツ連続 であるならば, 常微分方程式の初期値問題
\begin{align*} \begin{cases} \frac{dx}{dt} = f(t,x) \\ x(0) = x_0 \end{cases}\end{align*}
は$\cbra{t \in \mathbb \mid  \abs{t} \leq \min \cbra{a, \frac{b}{\sup_D f }} }$ を定義域に含む解を持つ. 
\end{prop}
\begin{pf*} $M \coloneqq \max \cbra{ \sup_D f_1, \ldots , \sup_D f_n}$ とする.
\begin{align*} x_1 \coloneqq x_0 + \int_0^t f(s, x_0) ds , \quad x_{n+1} (t) \coloneqq x_0 + \int_0^t f(s, x_n (s) ) ds\end{align*}
と定める. ただし, $x_n$ の定義域は, 最低限$\cbra{ t \in \mathbb R \mid \abs{t} \leq \min \cbra{a, \frac{b}{M }} } $ を含めば, 
\begin{align*} \abs{x_n (t) - x_0}  \leq \abs{\int_0^t  \abs{f(s, x_n(s)) } ds } \leq M \abs t  \leq b \end{align*}
と
$x_n $ の値域が$f$ の空間変数の定義域に含まれ, うまく定義される.   帰納法とリプシッツ連続性より
\begin{align*} \abs{x_n(t) - x_n (t)} &\leq \abs{\int_0^t (f(s, x_n (s)) - f(s, x_n (s) ) )  ds} \\& \leq  \abs{ \textrm{Lip} \int_0^t \abs{x_n (s) - x_{n-1}(s) } ds } \\&\leq \Const \frac{\textrm{Lip} ^n t^n} {n!}   \end{align*}
を得るので, $\cbra{x_n}$ はある関数$x$ に一様収束し, 
\begin{align*} \abs{f(t, x_n (t)) - f(t,x(t))} \leq \textrm{Lip} \abs{x_n(t) - x(t)} \end{align*}
であることから, $f(\cdot, x_n(\cdot))$ も一様収束するので, 
\begin{align*}  x_{n+1} (t) = x_0 + \int_0^t f(s, x_n (s) ) ds \end{align*}
の両辺の極限をとることで, $x$ が解であることがわかる. 一意性は他に$y$という解があったら, 定義域上の任意の$t$ で
\begin{align*} \abs{x(t) -  y(t) } \leq \Const \frac{\textrm{Lip} ^n t^n} {n!}     \end{align*}
が任意の$n$について成り立つので, 極限をとることで両者は一致する.
\qed
\end{pf*}















\end{document}