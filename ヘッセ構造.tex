\documentclass[10pt, fleqn, label-section=none]{bxjsarticle}

%\usepackage[driver=dvipdfm,hmargin=25truemm,vmargin=25truemm]{geometry}

\setpagelayout{driver=dvipdfm,hmargin=25truemm,vmargin=20truemm}


\usepackage{amsmath}
\usepackage{amssymb}
\usepackage{amsfonts}
\usepackage{amsthm}
\usepackage{mathtools}
\usepackage{mleftright}

\usepackage{ascmac}




\usepackage{otf}

\theoremstyle{definition}
\newtheorem{dfn}{定義}[section]
\newtheorem{ex}[dfn]{例}
\newtheorem{lem}[dfn]{補題}
\newtheorem{prop}[dfn]{命題}
\newtheorem{thm}[dfn]{定理}
\newtheorem{setting}[dfn]{設定}
\newtheorem{cor}[dfn]{系}
\newtheorem*{pf*}{証明}
\newtheorem{problem}[dfn]{問題}
\newtheorem*{problem*}{問題}
\newtheorem{remark}[dfn]{注意}
\newtheorem*{claim*}{\underline{claim}}



\newtheorem*{solution*}{解答}

%箇条書きの様式
\renewcommand{\labelenumi}{(\arabic{enumi})}


%

\newcommand{\forany}{\rm{for} \ {}^{\forall}}
\newcommand{\foranyeps}{
\rm{for} \ {}^{\forall}\varepsilon >0}
\newcommand{\foranyk}{
\rm{for} \ {}^{\forall}k}


\newcommand{\any}{{}^{\forall}}
\newcommand{\suchthat}{\, \rm{s.t.} \, \it{}}




\newcommand{\veps}{\varepsilon}
\newcommand{\paren}[1]{\mleft( #1\mright )}
\newcommand{\cbra}[1]{\mleft\{#1\mright\}}
\newcommand{\sbra}[1]{\mleft\lbrack#1\mright\rbrack}
\newcommand{\tbra}[1]{\mleft\langle#1\mright\rangle}
\newcommand{\abs}[1]{\left|#1\right|}
\newcommand{\norm}[1]{\left\|#1\right\|}
\newcommand{\lopen}[1]{\mleft(#1\mright\rbrack}
\newcommand{\ropen}[1]{\mleft\lbrack #1 \mright)}



%
\newcommand{\Rn}{\mathbb{R}^n}
\newcommand{\Cn}{\mathbb{C}^n}

\newcommand{\Rm}{\mathbb{R}^m}
\newcommand{\Cm}{\mathbb{C}^m}


\newcommand{\projs}[2]{\it{p}_{#1,\ldots,#2}}
\newcommand{\projproj}[2]{\it{p}_{#1,#2}}

\newcommand{\proj}[1]{p_{#1}}

%可測空間
\newcommand{\stdProbSp}{\paren{\Omega, \mathcal{F}, P}}

%微分作用素
\newcommand{\ddt}{\frac{d}{dt}}
\newcommand{\ddx}{\frac{d}{dx}}
\newcommand{\ddy}{\frac{d}{dy}}

\newcommand{\delt}{\frac{\partial}{\partial t}}
\newcommand{\delx}{\frac{\partial}{\partial x}}

%ハイフン
\newcommand{\hyphen}{\text{-}}

%displaystyle
\newcommand{\dstyle}{\displaystyle}

%⇔, ⇒, \UTF{21D0}%
\newcommand{\LR}{\Leftrightarrow}
\newcommand{\naraba}{\Rightarrow}
\newcommand{\gyaku}{\Leftarrow}

%理由
\newcommand{\naze}[1]{\paren{\because {\mathop{ #1 }}}}

%
\newcommand{\sankaku}{\hfill $\triangle$}

%
\newcommand{\push}{_{\#}}

%手抜き
\newcommand{\textif}{\textrm{if}\,\,\,}
\newcommand{\Ric}{\textrm{Ric}}
\newcommand{\tr}{\textrm{tr}}
\newcommand{\vol}{\textrm{vol}}
\newcommand{\diam}{\textrm{diam}}
\newcommand{\supp}{\textrm{supp}}
\newcommand{\Med}{\textrm{Med}}
\newcommand{\Leb}{\textrm{Leb}}
\newcommand{\Const}{\textrm{Const}}
\newcommand{\Avg}{\textrm{Avg}}
\newcommand{\id}{\textrm{id}}
\newcommand{\Ker}{\textrm{Ker}}
\newcommand{\im}{\textrm{Im}}




\renewcommand{\;}{\, ; \,}
\renewcommand{\d}{\, {d}}

\newcommand{\gyouretsu}[1]{\begin{pmatrix} #1 \end{pmatrix} }

%%図式

\usepackage[dvipdfm,all]{xy}


\newenvironment{claim}[1]{\par\noindent\underline{step:}\space#1}{}
\newenvironment{claimproof}[1]{\par\noindent{($\because$)}\space#1}{\hfill $\blacktriangle $}


\newcommand{\pprime}{{\prime \prime}}





%%


\title{ヘッセ構造}
\date{}


\author{}


\begin{document}


\maketitle



\section{}
\subsection{ヘッセ構造}
\begin{dfn}(平坦接続). 接続$\nabla$ は, 
\begin{align*} T(X,Y) \coloneqq \nabla_X Y - \nabla_Y X - [X, Y], \quad R(X, Y)Z \coloneqq \nabla_X \nabla_Y Z - \nabla_Y \nabla_X Z - \nabla_{[X, Y]} Z\end{align*}
がともに$0$ のとき, 平坦接続という. 多様体と平坦接続の組$(M, \nabla)$ を平坦多様体という. 
\end{dfn}

\begin{remark}
極めて当たり前だが, 平坦接続はレビチビタ接続と一致するとは限らない. レビチビタ接続は, 捩れがなく, 計量と整合的なものである. 
\end{remark}


\begin{dfn}(アファイン座標系). 平坦接続$\nabla$ に対して, 
\begin{align*} \nabla_{\partial_i} \partial_j = 0 \end{align*}
を満たす局所座標系$\cbra{x_1, \ldots , x_n}$ を$\nabla$ のアファイン座標系という. 
\end{dfn}

\begin{dfn}多様体$M$ 上の平坦接続$\nabla$ と擬リーマン計量$g$ の組$(\nabla, g)$ は$D$ の任意のアファイン座標$\cbra{x_1, \ldots , x_n}$ に対して ある関数$f$ で
\begin{align*} g_{ij} = \partial_i \partial_j F \end{align*}
を満たすものが存在する(すなわち, $g = Ddf$ である)とき, ヘッセ構造であるという. またこのとき, $g$ をヘッセ計量, $F$ を$g$ の$\nabla$ に関するポテンシャル, $(M, \nabla, g)$ をヘッセ多様体という. 
\end{dfn}

\begin{remark}
以降, $g$ は単にリーマン計量とするが, 諸々の結果は擬リーマン計量でも成立する. 
\end{remark}

\begin{dfn}(コシュール型計量). ヘッセ計量$g$ は, 閉$1$次微分形式$\omega$ で
\begin{align*} g = D\omega \end{align*}
を満たすものが存在する時に, コシュール型であるという. 
\end{dfn}

\begin{prop}$(M, \nabla^f)$ を平坦多様体, $\nabla^L$ をレビチビタ接続とする. 
\begin{align*} \gamma_X Y \coloneqq \nabla^L _X Y - \nabla^f _X Y \end{align*}
によりテンソル$\gamma$ を定める. 
$\gamma _{\partial _j}  \partial_k = \gamma^i_{jk} \partial_i$ により$\gamma^i_{jk}$ を定める. このとき, 
\begin{align*} \gamma^i_{jk} = \Gamma^i_{jk}\end{align*}
が成り立つ. 
\end{prop}
\begin{pf*}
\begin{align*} \gamma_{\partial_j} \partial_k = \nabla^L _{\partial_j} \partial_k - \nabla^f_{\partial_j} \partial_k   \end{align*}
なのだが, $\nabla^f_{\partial_j} \partial_k = 0$ である. 
\qed
\end{pf*}

\begin{prop}
\begin{align*} \gamma_{ijk} (\coloneqq \gamma^{l}_{ij} g_{lk} ) = \frac{1}{2} \paren{\partial_k g_{ij} + \partial_j g_{ik} - \partial_i g_{jk} } \end{align*} 
が成り立つ. 
\end{prop}
\begin{pf*}クリストッフェル記号に対して
\begin{align*} \Gamma_{ijk} (\coloneqq \Gamma^{l}_{ij} g_{lk} ) = \frac{1}{2} \paren{\partial_k g_{ij} + \partial_j g_{ik} - \partial_i g_{jk} } \end{align*} 
が成り立つことから従う. 
\qed
\end{pf*}

\begin{prop}
$g$がヘッセ計量であることと, 
\begin{align*} \partial_j g_{ij} = \partial_i g_{kj} \end{align*}
が成り立つことは必要十分である. 
\end{prop}
\begin{pf*}
$\naraba$. ポテンシャル$F$ を用いて$g_{ij} = \partial_{ij} F$ と表されることから従う. $\gyaku$. $h_j \coloneqq g_{ij} dx^i$ とおくと, $dh_j = dg_{ij} \wedge dx^i = \sum_{k <i} (\partial_k g_{ij} - \partial_i g_{kj})dx^k \wedge dx^i = 0$ となり閉形式であるので, ポアンカレの補題から局所的に適当な関数$\varphi_j$ を用いて$h_j = d\varphi_j$ と表される. $h \coloneqq \varphi_j dx^j$ も同様に計算すると$dh = 0$ となるので, 再びポアンカレの補題より, 適当な関数$\varphi$ を用いて局所的に$h = d \varphi$ と表される. すると, $\partial_i \partial_j \varphi = \partial_i \varphi_j = g_{ij}$ が成り立つ. 
\qed
\end{pf*}

\begin{prop}$g$ がヘッセ計量であるならば, 
\begin{align*}\gamma^i_{jk} = \frac{1}{2} g^{ir} \partial_k g_{rj}, \quad  \gamma_{ijk} =  \frac{1}{2} \partial_k g_{ij}  \end{align*}
\end{prop}
\begin{pf*}
\begin{align*} \partial_j g_{ij} = \partial_i g_{kj} \end{align*}
が成り立つから. 
\qed
\end{pf*}



\subsection{双対ヘッセ構造(工事中)}

\begin{setting}$\mathbb R^*_n$ を$\mathbb R^n$ の双対ベクトル空間とする. $\mathbb R^n$ の標準アファイン座標系$\cbra{x_1, \ldots, x_n} $($0$ を原点として$e_1, \ldots, e_n$ を大域フレームとするアファイン座標) に関する$\mathbb R^*_n$ の双対アファイン座標系($0$ を原点とし, $e_1. \ldots, e_n$ の双対基底を大域フレームとする座標系)を$\cbra{x^*_1, \ldots, x^*_n}$ とする. $\mathbb R^*_n$ の標準平坦接続を$\nabla^{*f}$ で表すことにする. 

\end{setting}

\begin{dfn}(勾配写像). 領域$\Omega \subset \mathbb R^n$ 上にヘッセ構造$(\nabla^f . g)$が与えられている時, $\Omega$ から$(\mathbb R^*_n, \nabla^{*f})$ への写像$\iota$ を
\begin{align*} x^*i \circ \iota = - \partial_i \varphi \end{align*}
によって(つまり, 局所表示の$i$ 成分が$- \partial_i \varphi$ であるように) さだめる. これを, $(\Omega, \nabla^f, g)$ から$(\mathbb R^*_n , \nabla^{*f})$ への勾配写像という. 
\end{dfn}





\end{document}