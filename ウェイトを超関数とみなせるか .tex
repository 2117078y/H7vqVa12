\documentclass[10pt, fleqn, label-section=none]{bxjsarticle}

%\usepackage[driver=dvipdfm,hmargin=25truemm,vmargin=25truemm]{geometry}

\setpagelayout{driver=dvipdfm,hmargin=25truemm,vmargin=20truemm}


\usepackage{amsmath}
\usepackage{amssymb}
\usepackage{amsfonts}
\usepackage{amsthm}
\usepackage{mathtools}
\usepackage{mleftright}

\usepackage{ascmac}




\usepackage{otf}

\theoremstyle{definition}
\newtheorem{dfn}{定義}[section]
\newtheorem{ex}[dfn]{例}
\newtheorem{lem}[dfn]{補題}
\newtheorem{prop}[dfn]{命題}
\newtheorem{thm}[dfn]{定理}
\newtheorem{setting}[dfn]{設定}
\newtheorem{notation}[dfn]{記号}
\newtheorem{cor}[dfn]{系}
\newtheorem*{pf*}{証明}
\newtheorem{problem}[dfn]{問題}
\newtheorem*{problem*}{問題}
\newtheorem{remark}[dfn]{注意}
\newtheorem*{claim*}{\underline{claim}}



\newtheorem*{solution*}{解答}

%箇条書きの様式
\renewcommand{\labelenumi}{(\arabic{enumi})}


%

\newcommand{\forany}{\rm{for} \ {}^{\forall}}
\newcommand{\foranyeps}{
\rm{for} \ {}^{\forall}\varepsilon >0}
\newcommand{\foranyk}{
\rm{for} \ {}^{\forall}k}


\newcommand{\any}{{}^{\forall}}
\newcommand{\suchthat}{\, \rm{s.t.} \, \it{}}




\newcommand{\veps}{\varepsilon}
\newcommand{\paren}[1]{\mleft( #1\mright )}
\newcommand{\cbra}[1]{\mleft\{#1\mright\}}
\newcommand{\sbra}[1]{\mleft\lbrack#1\mright\rbrack}
\newcommand{\tbra}[1]{\mleft\langle#1\mright\rangle}
\newcommand{\abs}[1]{\left|#1\right|}
\newcommand{\norm}[1]{\left\|#1\right\|}
\newcommand{\lopen}[1]{\mleft(#1\mright\rbrack}
\newcommand{\ropen}[1]{\mleft\lbrack #1 \mright)}



%
\newcommand{\Rn}{\mathbb{R}^n}
\newcommand{\Cn}{\mathbb{C}^n}

\newcommand{\Rm}{\mathbb{R}^m}
\newcommand{\Cm}{\mathbb{C}^m}


\newcommand{\projs}[2]{\it{p}_{#1,\ldots,#2}}
\newcommand{\projproj}[2]{\it{p}_{#1,#2}}

\newcommand{\proj}[1]{p_{#1}}

%可測空間
\newcommand{\stdProbSp}{\paren{\Omega, \mathcal{F}, P}}

%微分作用素
\newcommand{\ddt}{\frac{d}{dt}}
\newcommand{\ddx}{\frac{d}{dx}}
\newcommand{\ddy}{\frac{d}{dy}}

\newcommand{\delt}{\frac{\partial}{\partial t}}
\newcommand{\delx}{\frac{\partial}{\partial x}}

%ハイフン
\newcommand{\hyphen}{\text{-}}

%displaystyle
\newcommand{\dstyle}{\displaystyle}

%⇔, ⇒, \UTF{21D0}%
\newcommand{\LR}{\Leftrightarrow}
\newcommand{\naraba}{\Rightarrow}
\newcommand{\gyaku}{\Leftarrow}

%理由
\newcommand{\naze}[1]{\paren{\because {\mathop{ #1 }}}}

%
\newcommand{\sankaku}{\hfill $\triangle$}

%
\newcommand{\push}{_{\#}}

%手抜き
\newcommand{\textif}{\textrm{if}\,\,\,}
\newcommand{\Ric}{\textrm{Ric}}
\newcommand{\tr}{\textrm{tr}}
\newcommand{\vol}{\textrm{vol}}
\newcommand{\diam}{\textrm{diam}}
\newcommand{\supp}{\textrm{supp}}
\newcommand{\Med}{\textrm{Med}}
\newcommand{\Leb}{\textrm{Leb}}
\newcommand{\Const}{\textrm{Const}}
\newcommand{\Avg}{\textrm{Avg}}
\newcommand{\id}{\textrm{id}}
\newcommand{\Ker}{\textrm{Ker}}
\newcommand{\im}{\textrm{Im}}
\newcommand{\dil}{\textrm{dil}}
\newcommand{\Ch}{\textrm{Ch}}
\newcommand{\Lip}{\textrm{Lip}}
\newcommand{\Ent}{\textrm{Ent}}
\newcommand{\grad}{\textrm{grad}}
\newcommand{\dom}{\textrm{dom}}
\newcommand{\diag}{\textrm{diag}}

\renewcommand{\;}{\, ; \,}
\renewcommand{\d}{\, {d}}

\newcommand{\gyouretsu}[1]{\begin{pmatrix} #1 \end{pmatrix} }

\renewcommand{\div}{\textrm{div}}


%%図式

\usepackage[dvipdfm,all]{xy}


\newenvironment{claim}[1]{\par\noindent\underline{step:}\space#1}{}
\newenvironment{claimproof}[1]{\par\noindent{($\because$)}\space#1}{\hfill $\blacktriangle $}


\newcommand{\pprime}{{\prime \prime}}

%%マグニチュード


\newcommand{\Mag}{\textrm{Mag}}

\usepackage{mathrsfs}


%%6.13
\def\chint#1{\mathchoice
{\XXint\displaystyle\textstyle{#1}}%
{\XXint\textstyle\scriptstyle{#1}}%
{\XXint\scriptstyle\scriptscriptstyle{#1}}%
{\XXint\scriptscriptstyle\scriptscriptstyle{#1}}%
\!\int}
\def\XXint#1#2#3{{\setbox0=\hbox{$#1{#2#3}{\int}$ }
\vcenter{\hbox{$#2#3$ }}\kern-.6\wd0}}
\def\ddashint{\chint=}
\def\dashint{\chint-}


%%7.13

\usepackage{here}

%7.15
\newcommand{\Span}{\textrm{Span}}

\newcommand{\Conv}{\textrm{Conv}}

%7.27

%9.4
\newcommand{\sing}{\textrm{sing}}



\title{ウェイトを超関数とみなせるか}
\date{}


\author{}


\begin{document}


\maketitle

\section{}

\begin{setting}$X = (X, d, \mu)$ を測度距離空間とする. $W_X \coloneqq \cbra{w: X \rightarrow \mathbb R \mid \# [w \neq 0] < \infty}$ とする. 

\end{setting}



\begin{prop}$W_X$ は$X$ 上の有限な台をもつ符号付き測度全体の集合との間に全単射が存在する. 

\end{prop}
\begin{pf*}
明らかである. 
\qed
\end{pf*}

\begin{setting}$k: X \times X \rightarrow \mathbb R_{\geq 0}$ を正定値対称連続関数とする. 

\end{setting}

\begin{setting}$(\cdot, \cdot)_k : W_X \times W_X \rightarrow \mathbb R_{\geq 0} ; (w, v) \mapsto \sum_{x, y} w(x)k(x, y)v(y)$ と定めると, これは$W_X$ の内積を定める.  また, この内積により$W_X$ を完備化した空間を$(\mathcal W_X, (\cdot, \cdot)_k)$ により表す. 
\end{setting}

\begin{dfn}$w \in W_X$ , $f: X \rightarrow \mathbb R$ に対して, 
\begin{align*} w(f) \coloneqq \int_X f(x) \paren{\sum_y k(x, y) w(y)} d\mu  \end{align*}
と定める. 
\end{dfn}

\begin{prop}$w \in W_X$ , $f: X \rightarrow \mathbb R$ とする. 
\begin{align*} w(f) = \int_X f(x) (\delta_x, w)_k d\mu  \end{align*}
が成り立つ. 

\end{prop}
\begin{pf*}
\begin{align*} \sum_y k(x, y) w(y) = (\delta_x, w)_k \end{align*}
より明らか. 
\qed
\end{pf*}




\begin{prop} $w_n \in W_X, w \in \mathcal W_X$ とし, $w_n \rightarrow w \in \mathcal W_X$ とし, $f$ を$X$ 上の有界連続関数とする. $\int_{X} k(x.x)^{1/2} d\mu < \infty$ であるならば,  $w_n(f)$ は$\mathbb R$ におけるコーシー列である. 

\end{prop}
\begin{pf*}

\begin{align*} \abs{w_n f - w_m f}  & = \abs{ \int_X f(x) \paren{\sum_y k(x, y)(w_n(y) - w_m(y))} d\mu } \\& = \int_X \abs{f(x) (\delta_x, (w_n - w_m))_k} d\mu \\&\leq  \abs{\sup f} \int_X \abs{(\delta_x, w_n - w_m)_k } d\mu 
\\& \leq  \abs{\sup f} \norm{w_n - w_m}_k \int_X (\delta_x, \delta_x)_k ^{1/2} d\mu  \\&=  \abs{\sup f} \norm{w_n - w_m}_k \int_X k(x, x) ^{1/2} d\mu   < \infty  \end{align*}

\qed
\end{pf*}

\begin{setting}$k$ を$\int_X k(x,x)^{1/2} d \mu < \infty$ を満たす対称連続関数としておく. 

\end{setting}

\begin{dfn}$w \in \mathcal W_X$ に対して, $w: C_c^\infty \rightarrow \mathbb R$ を
\begin{align*} w(f) \coloneqq \lim w_n (f)  \end{align*}
により定める. ただし, $w_n $ は$w_n \rightarrow w$ をみたす適当な点列$w_n \in W_X $ である. 
\end{dfn}





\end{document}