\documentclass[10pt, fleqn, label-section=none]{bxjsarticle}

%\usepackage[driver=dvipdfm,hmargin=25truemm,vmargin=25truemm]{geometry}

\setpagelayout{driver=dvipdfm,hmargin=25truemm,vmargin=20truemm}


\usepackage{amsmath}
\usepackage{amssymb}
\usepackage{amsfonts}
\usepackage{amsthm}
\usepackage{mathtools}
\usepackage{mleftright}

\usepackage{ascmac}




\usepackage{otf}

\theoremstyle{definition}
\newtheorem{dfn}{定義}[section]
\newtheorem{ex}[dfn]{例}
\newtheorem{lem}[dfn]{補題}
\newtheorem{prop}[dfn]{命題}
\newtheorem{thm}[dfn]{定理}
\newtheorem{setting}[dfn]{設定}
\newtheorem{notation}[dfn]{記号}
\newtheorem{cor}[dfn]{系}
\newtheorem*{pf*}{証明}
\newtheorem{problem}[dfn]{問題}
\newtheorem*{problem*}{問題}
\newtheorem{remark}[dfn]{注意}
\newtheorem*{claim*}{\underline{claim}}



\newtheorem*{solution*}{解答}

%箇条書きの様式
\renewcommand{\labelenumi}{(\arabic{enumi})}


%

\newcommand{\forany}{\rm{for} \ {}^{\forall}}
\newcommand{\foranyeps}{
\rm{for} \ {}^{\forall}\varepsilon >0}
\newcommand{\foranyk}{
\rm{for} \ {}^{\forall}k}


\newcommand{\any}{{}^{\forall}}
\newcommand{\suchthat}{\, \rm{s.t.} \, \it{}}




\newcommand{\veps}{\varepsilon}
\newcommand{\paren}[1]{\mleft( #1\mright )}
\newcommand{\cbra}[1]{\mleft\{#1\mright\}}
\newcommand{\sbra}[1]{\mleft\lbrack#1\mright\rbrack}
\newcommand{\tbra}[1]{\mleft\langle#1\mright\rangle}
\newcommand{\abs}[1]{\left|#1\right|}
\newcommand{\norm}[1]{\left\|#1\right\|}
\newcommand{\lopen}[1]{\mleft(#1\mright\rbrack}
\newcommand{\ropen}[1]{\mleft\lbrack #1 \mright)}



%
\newcommand{\Rn}{\mathbb{R}^n}
\newcommand{\Cn}{\mathbb{C}^n}

\newcommand{\Rm}{\mathbb{R}^m}
\newcommand{\Cm}{\mathbb{C}^m}


\newcommand{\projs}[2]{\it{p}_{#1,\ldots,#2}}
\newcommand{\projproj}[2]{\it{p}_{#1,#2}}

\newcommand{\proj}[1]{p_{#1}}

%可測空間
\newcommand{\stdProbSp}{\paren{\Omega, \mathcal{F}, P}}

%微分作用素
\newcommand{\ddt}{\frac{d}{dt}}
\newcommand{\ddx}{\frac{d}{dx}}
\newcommand{\ddy}{\frac{d}{dy}}

\newcommand{\delt}{\frac{\partial}{\partial t}}
\newcommand{\delx}{\frac{\partial}{\partial x}}

%ハイフン
\newcommand{\hyphen}{\text{-}}

%displaystyle
\newcommand{\dstyle}{\displaystyle}

%⇔, ⇒, \UTF{21D0}%
\newcommand{\LR}{\Leftrightarrow}
\newcommand{\naraba}{\Rightarrow}
\newcommand{\gyaku}{\Leftarrow}

%理由
\newcommand{\naze}[1]{\paren{\because {\mathop{ #1 }}}}

%
\newcommand{\sankaku}{\hfill $\triangle$}

%
\newcommand{\push}{_{\#}}

%手抜き
\newcommand{\textif}{\textrm{if}\,\,\,}
\newcommand{\Ric}{\textrm{Ric}}
\newcommand{\tr}{\textrm{tr}}
\newcommand{\vol}{\textrm{vol}}
\newcommand{\diam}{\textrm{diam}}
\newcommand{\supp}{\textrm{supp}}
\newcommand{\Med}{\textrm{Med}}
\newcommand{\Leb}{\textrm{Leb}}
\newcommand{\Const}{\textrm{Const}}
\newcommand{\Avg}{\textrm{Avg}}
\newcommand{\id}{\textrm{id}}
\newcommand{\Ker}{\textrm{Ker}}
\newcommand{\im}{\textrm{Im}}
\newcommand{\dil}{\textrm{dil}}
\newcommand{\Ch}{\textrm{Ch}}
\newcommand{\Lip}{\textrm{Lip}}
\newcommand{\Ent}{\textrm{Ent}}
\newcommand{\grad}{\textrm{grad}}
\newcommand{\dom}{\textrm{dom}}
\newcommand{\diag}{\textrm{diag}}

\renewcommand{\;}{\, ; \,}
\renewcommand{\d}{\, {d}}

\newcommand{\gyouretsu}[1]{\begin{pmatrix} #1 \end{pmatrix} }


%%図式

\usepackage[dvipdfm,all]{xy}


\newenvironment{claim}[1]{\par\noindent\underline{step:}\space#1}{}
\newenvironment{claimproof}[1]{\par\noindent{($\because$)}\space#1}{\hfill $\blacktriangle $}


\newcommand{\pprime}{{\prime \prime}}

%%マグニチュード


\newcommand{\Mag}{\textrm{Mag}}


\title{次元の異なるベクトル空間のあいだの線型写像のヤコビアン}
\date{}


\author{}


\begin{document}


\maketitle

\section{}

\begin{remark}
内積空間として有限次元のものを考えることにする. 
\end{remark}

\begin{prop}$V$ を内積空間, $\cbra{v_1, \ldots, v_n}, \cbra{w_1, \ldots, w_n} $ をそれぞれ$V$ の基底とする. 正則行列$A$ を
\begin{align*} v = A w\end{align*}
を満たす行列とすると, 
\begin{align*} \det(v_i, v_j) = (\det A)^2 \det (w_i, w_j) \end{align*}
が成り立つ. 
\end{prop}
\begin{pf*}
\begin{align*} \det(v_i, v_j) = \det(v v^t) = \det (Aww^t A^t) = (\det A)^2 \det (w w^t) = (\det A)^2 \det (w_i, w_j) .\end{align*}
\qed
\end{pf*}

\begin{prop}$V$ を内積空間, 任意の$p \in \mathbb N$ 個の元$x_1, \ldots , x_p$ に対して
\begin{align*} \det(x_i, x_j) \leq \prod \norm{x_i} \end{align*}
が成り立つ. 
\end{prop}
\begin{pf*}$v_i, w_i$ を次が成り立つように定めていく. 
\begin{align*} &x_1 = v_1 +  w_1 \quad (v_1 = 0, w_1 = x_1) \\ &x_2 = v_2 + w_2 \quad (v_2 \in \textrm{span}(x_1), w_2 \in \textrm{span}(x_1) ^\perp )\\ &
x_3 = v_3 + w_3 \quad (v_3 \in \textrm{span}(x_1, x_2), w_3 \in \textrm{span}(x_1, x_2) ^\perp0 ) \\& \vdots  
 \end{align*}
 すると, 落ち着いて計算すると$i\neq j \naraba (w_i, w_j) = 0 $ となるので, 
 \begin{align*} \det(x_i, x_j) = \det(w_i, w_j) \leq \prod \norm{w_i}^2 \leq \prod \norm{x_i}^2   \end{align*}
 が成り立つ. 
\qed
\end{pf*}



\begin{dfn}$V, W$ をそれぞれ次元が$m \geq n$ の内積空間とする. $F: V \rightarrow W$ を線型写像とする.
\begin{align*} JF \coloneqq \sup \cbra{  \sqrt{\det ((Fu_i, Fu_j))_{i,j = 1, \ldots ,n} } \mid u_1, \ldots , u_n \textrm{は}V \textrm{の正規直交系}   }\end{align*}
と定め, これを$F$ のヤコビアンという. 
\end{dfn}


\begin{remark}$V$ を内積空間とするとき, 任意の1次独立なベクトルの組$v_1, \ldots, v_n$ に対して$\textrm{span} (v_1, \ldots, v_n)$ の正規直交基底を$u_1, \ldots , u_n$ とすると, $v = A u$ なる正則行列$A$ がとれる. 故に
\begin{align*} \det (v_i, v_j) = (\det A)^2 \det (u_i, u_j) =(\det A)^2  \geq 0 \end{align*}
であるので, 線形写像のヤコビアンの定義でルートをとるところはwell-definedである. 
\end{remark}



\begin{prop}$F$ が全射でないとき, $JF = 0$ である.
\end{prop}
\begin{pf*}
$F$ が全射でない時, $\dim \textrm{Im} F < n$ である. 適当に$V$ の正規直交系$u_1, \ldots , u_n$ をとると, ある添字$i\neq j$ で, $a \in \mathbb R$ で$Fu_i = a Fu_j$ となるものがとれるようなものが存在する. 
\begin{align*} (Fu_k, Fu_i) = (Fu_k, aFu_j) \quad (k = 1, \ldots , n) \end{align*}
が成り立つ. 
\qed
\end{pf*}




\begin{prop}($F$ のヤコビアンのexplicitな表示). $F$ が全射であるとき, $(\ker F)^\perp $の任意の基底$v_1, \ldots, v_n \in (\ker F)^\perp $ に対して, 
\begin{align*} JF = \frac{ \sqrt{\det (Fv_i, Fv_j)  }}{\sqrt{ \det (v_i, v_j) } }  \end{align*}
が成り立つ. 
\end{prop}
\begin{pf*}
$(\ker F)^\perp $ の任意の正規直交基底$u_1, \ldots, u_n \in (\ker F)^\perp$と任意の基底$v_1, \ldots, v_n \in (\ker F)^\perp$, $V$ の任意の正規直交系$w_1, \ldots , w_n \in V$ を適当にとっておく. 

\begin{align*} w_i = w^P_i + w^K_i \quad(w^P_i \in (\ker F)^\perp,  w^K_i \in \ker F) \end{align*}
とおくと, $w^ P = A u$ なる正則行列$A$ がとれる. 
 \begin{align*}  \frac{ \sqrt{\det (Fw^P_i, Fw^P_j)  } }{\sqrt{ \det (w^P_i, w^P_j) } } = \frac{ \sqrt{ \det A \det (Fu_i, Fu_j)  } }{\sqrt{ \det A \det (u_i, u_j) } }  =  \sqrt{\det (Fu_i, Fu_j)  }  \end{align*}
が成り立っていることに注意すると, 
\begin{align*} \det (Fw_i, Fw_j) &= \det(Fw^P_i, Fw^P_j) \leq \det (Fu_i, Fu_j) \det (w^P_i, w^P_j)  \\&
\leq \det (Fu_i, Fu_j) \prod \norm{w^P_i}^2 \\
&\leq \det (Fu_i, Fu_j) \prod \norm{w_i}^2 \\
&= \det (Fu_i, Fu_j) \cdot 1 \\
&= \frac{{\det (Fv_i, Fv_j)  } }{{ \det (v_i, v_j) } } 
\end{align*}
従って, 
\begin{align*} JF \leq \frac{ \sqrt{\det (Fv_i, Fv_j)  }}{\sqrt{ \det (v_i, v_j) } }  \leq JF.  \end{align*}
\qed
\end{pf*}










\end{document}