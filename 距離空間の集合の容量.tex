\documentclass[10pt, fleqn, label-section=none]{bxjsarticle}

%\usepackage[driver=dvipdfm,hmargin=25truemm,vmargin=25truemm]{geometry}

\setpagelayout{driver=dvipdfm,hmargin=25truemm,vmargin=20truemm}


\usepackage{amsmath}
\usepackage{amssymb}
\usepackage{amsfonts}
\usepackage{amsthm}
\usepackage{mathtools}
\usepackage{mleftright}

\usepackage{ascmac}




\usepackage{otf}

\theoremstyle{definition}
\newtheorem{dfn}{定義}[section]
\newtheorem{ex}[dfn]{例}
\newtheorem{lem}[dfn]{補題}
\newtheorem{prop}[dfn]{命題}
\newtheorem{thm}[dfn]{定理}
\newtheorem{setting}[dfn]{設定}
\newtheorem{notation}[dfn]{記号}
\newtheorem{cor}[dfn]{系}
\newtheorem*{pf*}{証明}
\newtheorem{problem}[dfn]{問題}
\newtheorem*{problem*}{問題}
\newtheorem{remark}[dfn]{注意}
\newtheorem*{claim*}{\underline{claim}}



\newtheorem*{solution*}{解答}

%箇条書きの様式
\renewcommand{\labelenumi}{(\arabic{enumi})}


%

\newcommand{\forany}{\rm{for} \ {}^{\forall}}
\newcommand{\foranyeps}{
\rm{for} \ {}^{\forall}\varepsilon >0}
\newcommand{\foranyk}{
\rm{for} \ {}^{\forall}k}


\newcommand{\any}{{}^{\forall}}
\newcommand{\suchthat}{\, \rm{s.t.} \, \it{}}




\newcommand{\veps}{\varepsilon}
\newcommand{\paren}[1]{\mleft( #1\mright )}
\newcommand{\cbra}[1]{\mleft\{#1\mright\}}
\newcommand{\sbra}[1]{\mleft\lbrack#1\mright\rbrack}
\newcommand{\tbra}[1]{\mleft\langle#1\mright\rangle}
\newcommand{\abs}[1]{\left|#1\right|}
\newcommand{\norm}[1]{\left\|#1\right\|}
\newcommand{\lopen}[1]{\mleft(#1\mright\rbrack}
\newcommand{\ropen}[1]{\mleft\lbrack #1 \mright)}



%
\newcommand{\Rn}{\mathbb{R}^n}
\newcommand{\Cn}{\mathbb{C}^n}

\newcommand{\Rm}{\mathbb{R}^m}
\newcommand{\Cm}{\mathbb{C}^m}


\newcommand{\projs}[2]{\it{p}_{#1,\ldots,#2}}
\newcommand{\projproj}[2]{\it{p}_{#1,#2}}

\newcommand{\proj}[1]{p_{#1}}

%可測空間
\newcommand{\stdProbSp}{\paren{\Omega, \mathcal{F}, P}}

%微分作用素
\newcommand{\ddt}{\frac{d}{dt}}
\newcommand{\ddx}{\frac{d}{dx}}
\newcommand{\ddy}{\frac{d}{dy}}

\newcommand{\delt}{\frac{\partial}{\partial t}}
\newcommand{\delx}{\frac{\partial}{\partial x}}

%ハイフン
\newcommand{\hyphen}{\text{-}}

%displaystyle
\newcommand{\dstyle}{\displaystyle}

%⇔, ⇒, \UTF{21D0}%
\newcommand{\LR}{\Leftrightarrow}
\newcommand{\naraba}{\Rightarrow}
\newcommand{\gyaku}{\Leftarrow}

%理由
\newcommand{\naze}[1]{\paren{\because {\mathop{ #1 }}}}

%
\newcommand{\sankaku}{\hfill $\triangle$}

%
\newcommand{\push}{_{\#}}

%手抜き
\newcommand{\textif}{\textrm{if}\,\,\,}
\newcommand{\Ric}{\textrm{Ric}}
\newcommand{\tr}{\textrm{tr}}
\newcommand{\vol}{\textrm{vol}}
\newcommand{\diam}{\textrm{diam}}
\newcommand{\supp}{\textrm{supp}}
\newcommand{\Med}{\textrm{Med}}
\newcommand{\Leb}{\textrm{Leb}}
\newcommand{\Const}{\textrm{Const}}
\newcommand{\Avg}{\textrm{Avg}}
\newcommand{\id}{\textrm{id}}
\newcommand{\Ker}{\textrm{Ker}}
\newcommand{\im}{\textrm{Im}}
\newcommand{\dil}{\textrm{dil}}
\newcommand{\Ch}{\textrm{Ch}}
\newcommand{\Lip}{\textrm{Lip}}
\newcommand{\Ent}{\textrm{Ent}}
\newcommand{\grad}{\textrm{grad}}
\newcommand{\dom}{\textrm{dom}}
\newcommand{\diag}{\textrm{diag}}

\renewcommand{\;}{\, ; \,}
\renewcommand{\d}{\, {d}}

\newcommand{\gyouretsu}[1]{\begin{pmatrix} #1 \end{pmatrix} }

\renewcommand{\div}{\textrm{div}}


%%図式

\usepackage[dvipdfm,all]{xy}


\newenvironment{claim}[1]{\par\noindent\underline{step:}\space#1}{}
\newenvironment{claimproof}[1]{\par\noindent{($\because$)}\space#1}{\hfill $\blacktriangle $}


\newcommand{\pprime}{{\prime \prime}}

%%マグニチュード


\newcommand{\Mag}{\textrm{Mag}}

\usepackage{mathrsfs}


%%6.13
\def\chint#1{\mathchoice
{\XXint\displaystyle\textstyle{#1}}%
{\XXint\textstyle\scriptstyle{#1}}%
{\XXint\scriptstyle\scriptscriptstyle{#1}}%
{\XXint\scriptscriptstyle\scriptscriptstyle{#1}}%
\!\int}
\def\XXint#1#2#3{{\setbox0=\hbox{$#1{#2#3}{\int}$ }
\vcenter{\hbox{$#2#3$ }}\kern-.6\wd0}}
\def\ddashint{\chint=}
\def\dashint{\chint-}


%%7.13

\usepackage{here}

%7.15
\newcommand{\Span}{\textrm{Span}}

\newcommand{\Conv}{\textrm{Conv}}



\title{距離空間の集合の容量}
\date{}


\author{}


\begin{document}


\maketitle

\section{}

\begin{setting}$1 \leq p < \infty$ とし, $\mu $ は$X$ 上の適当な完備ボレル測度で, 任意の開球に対して正の測度を割り当てるものとする. ボレル可測でない集合$A \subset X$に対しては, 外測度により$\mu (A) $ を定めることにする. また, このような測度に対しては, 任意の集合$A \subset X$ に対して, ボレル集合$B \subset X$ で, $\mu(A) = \mu(B)$ を満たすものが存在する. 

\end{setting}



\begin{dfn}$(X, d)$ を距離空間とする. $A \subset X$ に対して, 
\begin{align*} C_p (A) \coloneqq    \inf_{u \in N^{1, p}(X),  \chi_A \leq u|_A } \norm{u}^p _{N^{1, p}(X)}    \end{align*}
と定め, これを$A$ の容量という. 
\end{dfn}

\begin{prop}$(X, d)$ を距離空間とする. $u: X \rightarrow \mathbb R$ に対して, 
\begin{align*} UG(u) \subset UG( \max \cbra{\min \cbra{u, 1}, 0}          )   \end{align*}
が成り立つ. 
\end{prop}
\begin{pf*}$v \coloneqq \max \cbra{\min \cbra{u, 1}, 0}    $ と表すことにする. $g \in UG(u)$ とすると, 任意の弧長パラメータづけられた絶対連続曲線$\gamma$ に対して, 
\begin{align*} \abs{v\circ \gamma(L\gamma) - v \circ \gamma(0)} \leq \abs{u\circ \gamma(L\gamma) - u \circ \gamma(0)} \end{align*}
より主張がしたがう. 
\qed
\end{pf*}

\begin{prop}$(X, d)$ を距離空間とする. $u \in N^{1, p}(X)$ に対して, 
\begin{align*}  \norm{ \max \cbra{\min \cbra{u, 1}, 0}      }_{N^{1, p}(X)} \leq \norm{u}_{N^{1, p}(X)}  \end{align*}
が成り立つ. 
\end{prop}
\begin{pf*}$UG(u) \subset UG( \max \cbra{\min \cbra{u, 1}, 0}        )  $と, $\abs{\max \cbra{\min \cbra{u, 1}, 0}     } < \abs{u}$ より主張が従う. 

\qed
\end{pf*}

\begin{prop}$(X, d)$ を距離空間とし, $A \subset X$ とする. $v: X \rightarrow \mathbb R$ が$1 \leq u|_A $ を満たす$u: X \rightarrow \mathbb R$ を用いて, 
\begin{align*} v =  \max \cbra{\min \cbra{u, 1}, 0}  \end{align*}
と表されるとき, $\chi_A \leq v \leq 1$ を満たす. 
\end{prop}
\begin{pf*}明らかにそう. 

\qed
\end{pf*}


\begin{prop}$(X, d)$ を距離空間とする. $A \subset X$ に対して, 
\begin{align*} C_p (A) =  \inf_{u \in N^{1, p}(X),  \chi_A \leq u \leq 1} \norm{u}^p _{N^{1, p}(X)}  \end{align*}
が成り立つ. 
\end{prop}
\begin{pf*}
\begin{align*} &C_p (A) \leq \inf \cbra{ \norm v ^p \mid \chi_A \leq v \leq 1 }  \\& \leq \inf \cbra{ \norm v ^p  \mid {v = \max \cbra{\min \cbra{u, 1}, 0} , 1 \leq u|_A }  } \\& \leq \inf \cbra{ \norm u ^p \mid  1 \leq u|_A  } = C_p (A)     \end{align*}

\qed
\end{pf*}

\begin{prop}$(X, d)$ を距離空間とする. 
\begin{align*} C_p(\varnothing ) = 0\end{align*}
\end{prop}
\begin{pf*}
$0$ という関数を考えれば良い. 
\qed
\end{pf*}

\begin{prop}$(X, d)$ を距離空間とする. $A \subset X$ とする. 
\begin{align*} \mu (A) \leq C_p (A)  \end{align*}
が成り立つ. 
\end{prop}
\begin{pf*}
$u \in N^{1, p}(X)$ で $ \chi_A \leq u|_A $ を満たすならば,  
\begin{align*} \norm{\chi_A }^p _{N^{1, p}(X)}  \leq   \norm{u}^p _{N^{1, p}(X)}  \end{align*}   
が成り立つ. で, $\norm{\chi_A }^p _{N^{1, p}(X)}  = \mu(A)$ なので, 主張が従う. 
\qed
\end{pf*}









\end{document}