\documentclass[10pt, fleqn, label-section=none]{bxjsarticle}

%\usepackage[driver=dvipdfm,hmargin=25truemm,vmargin=25truemm]{geometry}

\setpagelayout{driver=dvipdfm,hmargin=25truemm,vmargin=20truemm}


\usepackage{amsmath}
\usepackage{amssymb}
\usepackage{amsfonts}
\usepackage{amsthm}
\usepackage{mathtools}
\usepackage{mleftright}

\usepackage{ascmac}




\usepackage{otf}

\theoremstyle{definition}
\newtheorem{dfn}{定義}[section]
\newtheorem{ex}[dfn]{例}
\newtheorem{lem}[dfn]{補題}
\newtheorem{prop}[dfn]{命題}
\newtheorem{thm}[dfn]{定理}
\newtheorem{setting}[dfn]{設定}
\newtheorem{notation}[dfn]{記号}
\newtheorem{cor}[dfn]{系}
\newtheorem*{pf*}{証明}
\newtheorem{problem}[dfn]{問題}
\newtheorem*{problem*}{問題}
\newtheorem{remark}[dfn]{注意}
\newtheorem*{claim*}{\underline{claim}}



\newtheorem*{solution*}{解答}

%箇条書きの様式
\renewcommand{\labelenumi}{(\arabic{enumi})}


%

\newcommand{\forany}{\rm{for} \ {}^{\forall}}
\newcommand{\foranyeps}{
\rm{for} \ {}^{\forall}\varepsilon >0}
\newcommand{\foranyk}{
\rm{for} \ {}^{\forall}k}


\newcommand{\any}{{}^{\forall}}
\newcommand{\suchthat}{\, \rm{s.t.} \, \it{}}




\newcommand{\veps}{\varepsilon}
\newcommand{\paren}[1]{\mleft( #1\mright )}
\newcommand{\cbra}[1]{\mleft\{#1\mright\}}
\newcommand{\sbra}[1]{\mleft\lbrack#1\mright\rbrack}
\newcommand{\tbra}[1]{\mleft\langle#1\mright\rangle}
\newcommand{\abs}[1]{\left|#1\right|}
\newcommand{\norm}[1]{\left\|#1\right\|}
\newcommand{\lopen}[1]{\mleft(#1\mright\rbrack}
\newcommand{\ropen}[1]{\mleft\lbrack #1 \mright)}



%
\newcommand{\Rn}{\mathbb{R}^n}
\newcommand{\Cn}{\mathbb{C}^n}

\newcommand{\Rm}{\mathbb{R}^m}
\newcommand{\Cm}{\mathbb{C}^m}


\newcommand{\projs}[2]{\it{p}_{#1,\ldots,#2}}
\newcommand{\projproj}[2]{\it{p}_{#1,#2}}

\newcommand{\proj}[1]{p_{#1}}

%可測空間
\newcommand{\stdProbSp}{\paren{\Omega, \mathcal{F}, P}}

%微分作用素
\newcommand{\ddt}{\frac{d}{dt}}
\newcommand{\ddx}{\frac{d}{dx}}
\newcommand{\ddy}{\frac{d}{dy}}

\newcommand{\delt}{\frac{\partial}{\partial t}}
\newcommand{\delx}{\frac{\partial}{\partial x}}

%ハイフン
\newcommand{\hyphen}{\text{-}}

%displaystyle
\newcommand{\dstyle}{\displaystyle}

%⇔, ⇒, \UTF{21D0}%
\newcommand{\LR}{\Leftrightarrow}
\newcommand{\naraba}{\Rightarrow}
\newcommand{\gyaku}{\Leftarrow}

%理由
\newcommand{\naze}[1]{\paren{\because {\mathop{ #1 }}}}

%
\newcommand{\sankaku}{\hfill $\triangle$}

%
\newcommand{\push}{_{\#}}

%手抜き
\newcommand{\textif}{\textrm{if}\,\,\,}
\newcommand{\Ric}{\textrm{Ric}}
\newcommand{\tr}{\textrm{tr}}
\newcommand{\vol}{\textrm{vol}}
\newcommand{\diam}{\textrm{diam}}
\newcommand{\supp}{\textrm{supp}}
\newcommand{\Med}{\textrm{Med}}
\newcommand{\Leb}{\textrm{Leb}}
\newcommand{\Const}{\textrm{Const}}
\newcommand{\Avg}{\textrm{Avg}}
\newcommand{\id}{\textrm{id}}
\newcommand{\Ker}{\textrm{Ker}}
\newcommand{\im}{\textrm{Im}}
\newcommand{\dil}{\textrm{dil}}
\newcommand{\Ch}{\textrm{Ch}}
\newcommand{\Lip}{\textrm{Lip}}
\newcommand{\Ent}{\textrm{Ent}}
\newcommand{\grad}{\textrm{grad}}
\newcommand{\dom}{\textrm{dom}}
\newcommand{\diag}{\textrm{diag}}

\renewcommand{\;}{\, ; \,}
\renewcommand{\d}{\, {d}}

\newcommand{\gyouretsu}[1]{\begin{pmatrix} #1 \end{pmatrix} }

\renewcommand{\div}{\textrm{div}}


%%図式

\usepackage[dvipdfm,all]{xy}


\newenvironment{claim}[1]{\par\noindent\underline{step:}\space#1}{}
\newenvironment{claimproof}[1]{\par\noindent{($\because$)}\space#1}{\hfill $\blacktriangle $}


\newcommand{\pprime}{{\prime \prime}}

%%マグニチュード


\newcommand{\Mag}{\textrm{Mag}}

\usepackage{mathrsfs}


%%6.13
\def\chint#1{\mathchoice
{\XXint\displaystyle\textstyle{#1}}%
{\XXint\textstyle\scriptstyle{#1}}%
{\XXint\scriptstyle\scriptscriptstyle{#1}}%
{\XXint\scriptscriptstyle\scriptscriptstyle{#1}}%
\!\int}
\def\XXint#1#2#3{{\setbox0=\hbox{$#1{#2#3}{\int}$ }
\vcenter{\hbox{$#2#3$ }}\kern-.6\wd0}}
\def\ddashint{\chint=}
\def\dashint{\chint-}


%%7.13

\usepackage{here}

%7.15
\newcommand{\Span}{\textrm{Span}}

\newcommand{\Conv}{\textrm{Conv}}

%7.27

%9.4
\newcommand{\sing}{\textrm{sing}}

%
\newcommand{\C}[2]{{}_{#1}C_{#2} }


\title{領域上の分数階ソボレフ空間}
\date{}


\author{}


\begin{document}


\maketitle

\section{}

\subsection{領域上の非負分数階ソボレフ空間}

\begin{remark}いったん非負階のソボレフ空間を定義することにする. 別に最初から非負に限らないで定義することも可能である. 
色々な方法で様々なソボレフ空間を定義することができるが, それらの間の包含関係を把握しておきたい. 

\end{remark}


\begin{remark}$\Omega \subset \mathbb R^n$ 上のソボレフ空間$H^s (\Omega)$ を定義したい. 素朴に考えると, 

\begin{align*} H^s(\mathbb R^n) \coloneqq \cbra{u \in \mathcal S^\prime (\mathbb R^n) \mid \tbra{\cdot}^s Fu \in L^2(\mathbb R^n) }\end{align*}

なわけだから, 

\begin{align*} H^s(\Omega ) \coloneqq \cbra{ u \in \mathcal S^\prime (\Omega) \mid \tbra{\cdot}^s Fu \in L^2(\Omega) }\end{align*}

としたいところだが, そもそも, 全体で定義されていない関数$u $ のフーリエ変換を正当化する作業が面倒くさい. そこで, 次のような回避方法をとる. 

\end{remark}

\begin{dfn}$\Omega \subset \mathbb R^n$ とする. 

\begin{align*} H^s(\Omega) \coloneqq \cbra{ u |_\Omega : \Omega \rightarrow \mathbb R \mid u \in H^s(\mathbb R^n)}\end{align*}

と定義する. 

\end{dfn}

\begin{remark}
すると, 次にノルムをどう定義したらよいのかという問題が生じる. なぜなら, $u, v \in H^s(\mathbb R^n)$ で, $u \neq v$ なのに$u|_\Omega = v|_\Omega$ であるようなものはたくさんある. そこで, 次のような回避方法をとる. 
\end{remark}

\begin{dfn}(拡張可能ソボレフ空間). 

\begin{align*} \norm{u }_{H^s(\Omega)} \coloneqq \inf \cbra{ \norm{U}_{H^s(\mathbb R^n)} \mid U|_\Omega = u } \end{align*}

\end{dfn}

\begin{remark}$U|_\Omega, u$ はともに$L^2$ の元であるので, 各点の値は意味をもたない. あくまで同値類としての一致であるので, 適当に代表元を選んでやると, 
殆どいたるところ一致という意味である. 

\end{remark}

直後に使う事実を書いておく. 
\begin{prop}$(X, \mathcal F, \mu)$ を有限測度空間とする. $A \in \mathcal F, \mu(A) = \mu(X)$ ならば, 任意の$B \in \mathcal F$ に対して, 
$\mu(B \cap A) = \mu(B)$ 

\end{prop}
\begin{pf*}$\mu(X) - \mu (B^c) - \mu(A^c) \leq \mu(X) - \mu(B^c \cup A^c) \leq \mu(B \cap A) \leq  \mu(B)$ より明らか. 

\qed
\end{pf*}


\begin{prop}

\begin{align*} \inf \cbra{ \norm{U}_{H^s(\mathbb R^n)} \mid U|_\Omega = u } \end{align*}

は最小値を実現する. 

\end{prop}
\begin{pf*}
凸集合であることは明らか. 閉集合であることを示す. $U_n \rightarrow U $($H^m$収束)ならば, $U_n \rightarrow U$($L^2$) なので, 概収束部分列をとって添字を振り直して
$U_n \rightarrow U$(概収束)とできる. 

\begin{align*} &\mathcal L^n \cbra{x \in \Omega \mid Ux \neq ux } 
\\&= \mathcal L^n \cbra{x \in \Omega \mid Ux \neq ux , \lim U_n x = Ux}
\\&= \mathcal L^n \cbra{x \in \Omega \mid \lim U_n x \neq ux } 
\\&= \mathcal L^n \cbra{x \in \Omega \mid \lim U_n x \neq ux, U_n x = ux \quad (\any n)}
\\&= \mathcal L^n \cbra{x \in \Omega \mid ux \neq ux} 
\\&= 0
 \end{align*}
故に, 
\begin{align*} U|_\Omega = u \end{align*}

が成り立つ. 

\qed
\end{pf*}

\begin{remark}一瞬, 概収束の定義について, $f_n \rightarrow f\quad \textrm{a.e.}$ の定義が

\begin{align*} \mu (\cbra{x \mid \lim f_n(x) \neq f(x)} ) = 0\end{align*}

とされているので, $f$ や$f_n$ に各点の値を代入しているじゃないかとも思うが, 適当に$[f]$ から一つ代表元を とってきてという意味である. 殆ど至る所の概念は代表元の取り方によらない. つまり, 別の$f^\prime \in [f]$ にとりかえて

\begin{align*} \mu (\cbra{x \mid \lim f_n(x) \neq f^\prime (x)} ) = 0\end{align*}

としても$\mu([f \neq f^\prime]) = 0$ なので同じことである. 

\end{remark}

\begin{prop}$m \in \mathbb N_{\geq 0}$ とする. $u \in H^m(\Omega)$ に対して, $u_0$ を$u$ の$0$ 拡張とすると, 

\begin{align*} u_0 \in H^m \end{align*}

かつ, $\norm{u}_{H^m(\Omega)} = \norm{u_0}_{H^m}$ が成り立つ. 

\end{prop}
\begin{pf*}

\qed
\end{pf*}


\begin{dfn}(非負整数階微分ソボレフ空間). $m \in \mathbb N_{\geq 0}$ とする. $\Omega \subset \mathbb R^n$ を開集合とする. 

\begin{align*} & W^m(\mathbb R^n) \coloneqq \cbra{u \in L^2(\mathbb R^n) \mid \paren{ \sum_{\abs{\alpha} \leq m } \norm{ D^\alpha u}_{L^2(\mathbb R^n)} ^2 }^{1/2} < \infty  } 
\\ & W^m(\Omega ) \coloneqq \cbra{u \in L^2(\Omega ) \mid \paren{ \sum_{\abs{\alpha} \leq m } \norm{ D^\alpha u}_{L^2(\Omega )} ^2 }^{1/2} < \infty  } 
\end{align*}

と定め, 

\begin{align*} \norm{u}_{W^m(\Omega)}  \coloneqq \paren{ \sum_{\abs{\alpha} \leq m } \norm{ D^\alpha u}_{L^2(\Omega )} ^2 }^{1/2}       \end{align*}

によりノルムを定める.

\end{dfn}



\begin{prop}$\Omega \subset \mathbb R^n$ を空でない開集合とする. このとき, $u \in H^m(\Omega)$

\begin{align*} \norm{u}_{W^m(\Omega)} \lesssim \norm{u}_{H^m(\Omega)}\end{align*}

が成り立つ. 従って, 包含写像

\begin{align*} H^m(\Omega) \subset W^m(\Omega ) \end{align*}

は連続である. 

\end{prop}
\begin{pf*}

$u \in H^m(\Omega)$ であるので, $U \in H^m$ で$U|_{\Omega } = u$ かつ$\norm{U}_{H^m} = \norm{u}_{H^m(\Omega)}$ であるものがとれる. 

\begin{align*} \norm{u}_{W^m(\Omega)} \leq \norm{U}_{W^m} \lesssim \norm{U}_{H^m} = \norm{u}_{H^m(\Omega)} \end{align*} 
が成り立つ. 

\qed
\end{pf*}

\begin{prop}$m \in \mathbb N_{\geq 0 }$ とする. $\Omega \subset \mathbb R^n$ を空でない開集合とする. 拡張作用素
\begin{align*} E: W^m(\Omega) \rightarrow W^m(\mathbb R^n) \end{align*}
で連続なものが存在するならば, 

\begin{align*} W^m(\Omega) = H^m(\Omega ) \end{align*}

が成り立つ. 

\end{prop}
\begin{pf*}$u \in W^m(\Omega)$ をとる. 

\begin{align*} \norm u _{H^m(\Omega)} \leq \norm{Eu}_{H^m} = \norm{Eu}_{W^m} \lesssim \norm{u}_{W^m(\Omega)} \end{align*}

\qed
\end{pf*}




\begin{dfn}(近似可能ソボレフ空間). 

\begin{align*} H_0^s (\Omega ) \coloneqq \textrm{cl}(C_0^\infty(\Omega) ; \norm{\cdot}_{H^s(\Omega)} ) \end{align*}

\end{dfn}

\begin{notation}
ただし, $u \in \Omega \rightarrow \mathbb R$ に対して, $u_0: \mathbb R^n \rightarrow \mathbb R $ を全体へのゼロ拡張として定義する.  
\end{notation}

\begin{remark}$u \in L^2(\Omega )  \naraba u_0 \in L^2(\mathbb R^n)$ は成り立つが, $u \in H^s(\Omega ) \naraba u_0 \in H^s(\mathbb R^n)$ は成り立たない. つまり一般には, 

\begin{align*}  \cbra{U \in H^s(\mathbb R^n ) \mid U|_\Omega = u \in H^s(\Omega) }\end{align*}

が$u_0$ を含むとは限らないということ. 

\end{remark}




\begin{dfn}($0$拡張可能ソボレフ空間). 

\begin{align*} \tilde H^s (\Omega) \coloneqq \cbra{u \in H^s(\Omega ) \mid u_0 \in H^s(\Omega R^n) }\end{align*}



\end{dfn}

\subsection{領域上の非正分数階ソボレフ}

\begin{setting}
$s < 0$ とする. 
\end{setting}

\begin{remark}領域上の非正分数階ソボレフ空間は$0$拡張可能ソボレフ空間の双対として定義する. 


\end{remark}

\begin{dfn}

\begin{align*}  H^s(\Omega ) \coloneqq \paren{\tilde H^{-s}(\Omega)}^*   \end{align*}

\end{dfn}


\begin{dfn}(領域上の非正分数階$0$拡張可能ソボレフ空間). 

\begin{align*}  H^s(\Omega ) \coloneqq \paren{ H^{-s}(\Omega)}^* \end{align*}

\end{dfn}






\end{document}