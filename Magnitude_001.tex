\documentclass[10pt, fleqn, label-section=none]{bxjsarticle}

%\usepackage[driver=dvipdfm,hmargin=25truemm,vmargin=25truemm]{geometry}

\setpagelayout{driver=dvipdfm,hmargin=25truemm,vmargin=20truemm}


\usepackage{amsmath}
\usepackage{amssymb}
\usepackage{amsfonts}
\usepackage{amsthm}
\usepackage{mathtools}
\usepackage{mleftright}

\usepackage{ascmac}




\usepackage{otf}

\theoremstyle{definition}
\newtheorem{dfn}{定義}[section]
\newtheorem{ex}[dfn]{例}
\newtheorem{lem}[dfn]{補題}
\newtheorem{prop}[dfn]{命題}
\newtheorem{thm}[dfn]{定理}
\newtheorem{setting}[dfn]{設定}
\newtheorem{notation}[dfn]{記号}
\newtheorem{cor}[dfn]{系}
\newtheorem*{pf*}{証明}
\newtheorem{problem}[dfn]{問題}
\newtheorem*{problem*}{問題}
\newtheorem{remark}[dfn]{注意}
\newtheorem*{claim*}{\underline{claim}}



\newtheorem*{solution*}{解答}

%箇条書きの様式
\renewcommand{\labelenumi}{(\arabic{enumi})}


%

\newcommand{\forany}{\rm{for} \ {}^{\forall}}
\newcommand{\foranyeps}{
\rm{for} \ {}^{\forall}\varepsilon >0}
\newcommand{\foranyk}{
\rm{for} \ {}^{\forall}k}


\newcommand{\any}{{}^{\forall}}
\newcommand{\suchthat}{\, \rm{s.t.} \, \it{}}




\newcommand{\veps}{\varepsilon}
\newcommand{\paren}[1]{\mleft( #1\mright )}
\newcommand{\cbra}[1]{\mleft\{#1\mright\}}
\newcommand{\sbra}[1]{\mleft\lbrack#1\mright\rbrack}
\newcommand{\tbra}[1]{\mleft\langle#1\mright\rangle}
\newcommand{\abs}[1]{\left|#1\right|}
\newcommand{\norm}[1]{\left\|#1\right\|}
\newcommand{\lopen}[1]{\mleft(#1\mright\rbrack}
\newcommand{\ropen}[1]{\mleft\lbrack #1 \mright)}



%
\newcommand{\Rn}{\mathbb{R}^n}
\newcommand{\Cn}{\mathbb{C}^n}

\newcommand{\Rm}{\mathbb{R}^m}
\newcommand{\Cm}{\mathbb{C}^m}


\newcommand{\projs}[2]{\it{p}_{#1,\ldots,#2}}
\newcommand{\projproj}[2]{\it{p}_{#1,#2}}

\newcommand{\proj}[1]{p_{#1}}

%可測空間
\newcommand{\stdProbSp}{\paren{\Omega, \mathcal{F}, P}}

%微分作用素
\newcommand{\ddt}{\frac{d}{dt}}
\newcommand{\ddx}{\frac{d}{dx}}
\newcommand{\ddy}{\frac{d}{dy}}

\newcommand{\delt}{\frac{\partial}{\partial t}}
\newcommand{\delx}{\frac{\partial}{\partial x}}

%ハイフン
\newcommand{\hyphen}{\text{-}}

%displaystyle
\newcommand{\dstyle}{\displaystyle}

%⇔, ⇒, \UTF{21D0}%
\newcommand{\LR}{\Leftrightarrow}
\newcommand{\naraba}{\Rightarrow}
\newcommand{\gyaku}{\Leftarrow}

%理由
\newcommand{\naze}[1]{\paren{\because {\mathop{ #1 }}}}

%
\newcommand{\sankaku}{\hfill $\triangle$}

%
\newcommand{\push}{_{\#}}

%手抜き
\newcommand{\textif}{\textrm{if}\,\,\,}
\newcommand{\Ric}{\textrm{Ric}}
\newcommand{\tr}{\textrm{tr}}
\newcommand{\vol}{\textrm{vol}}
\newcommand{\diam}{\textrm{diam}}
\newcommand{\supp}{\textrm{supp}}
\newcommand{\Med}{\textrm{Med}}
\newcommand{\Leb}{\textrm{Leb}}
\newcommand{\Const}{\textrm{Const}}
\newcommand{\Avg}{\textrm{Avg}}
\newcommand{\id}{\textrm{id}}
\newcommand{\Ker}{\textrm{Ker}}
\newcommand{\im}{\textrm{Im}}
\newcommand{\dil}{\textrm{dil}}
\newcommand{\Ch}{\textrm{Ch}}
\newcommand{\Lip}{\textrm{Lip}}
\newcommand{\Ent}{\textrm{Ent}}
\newcommand{\grad}{\textrm{grad}}
\newcommand{\dom}{\textrm{dom}}
\newcommand{\diag}{\textrm{diag}}

\renewcommand{\;}{\, ; \,}
\renewcommand{\d}{\, {d}}

\newcommand{\gyouretsu}[1]{\begin{pmatrix} #1 \end{pmatrix} }

\renewcommand{\div}{\textrm{div}}


%%図式

\usepackage[dvipdfm,all]{xy}


\newenvironment{claim}[1]{\par\noindent\underline{step:}\space#1}{}
\newenvironment{claimproof}[1]{\par\noindent{($\because$)}\space#1}{\hfill $\blacktriangle $}


\newcommand{\pprime}{{\prime \prime}}

%%マグニチュード


\newcommand{\Mag}{\textrm{Mag}}

\usepackage{mathrsfs}


%%6.13
\def\chint#1{\mathchoice
{\XXint\displaystyle\textstyle{#1}}%
{\XXint\textstyle\scriptstyle{#1}}%
{\XXint\scriptstyle\scriptscriptstyle{#1}}%
{\XXint\scriptscriptstyle\scriptscriptstyle{#1}}%
\!\int}
\def\XXint#1#2#3{{\setbox0=\hbox{$#1{#2#3}{\int}$ }
\vcenter{\hbox{$#2#3$ }}\kern-.6\wd0}}
\def\ddashint{\chint=}
\def\dashint{\chint-}


%%7.13

\usepackage{here}

%7.15
\newcommand{\Span}{\textrm{Span}}

\newcommand{\Conv}{\textrm{Conv}}

%7.27

%9.4
\newcommand{\sing}{\textrm{sing}}

%
\newcommand{\C}[2]{{}_{#1}C_{#2} }


\title{Magnitude}
\date{}


\author{}


\begin{document}


\maketitle



\section{有限距離空間}


\begin{dfn}(類似度行列). $(X, d)$ を有限距離空間とする. $Z:X \times X \rightarrow \mathbb R$ を, 

\begin{align*} Z(x, y) \coloneqq e^{-d(x, y)} \end{align*}

により定め, これを類似度行列という. 

\end{dfn}

\begin{dfn}(ウェイト). $(X, d)$ を

\end{dfn}




\section{コンパクト距離空間}


\begin{prop}$K \subset l_2^d$ を凸体とする. このとき, 
\begin{align*} \Mag(K) \leq \sum_{k = 0}^d \frac{\omega_k}{4^k} V_k (K) \end{align*}

が成り立つ. また, $d = 1$ のとき, 等号が成立する. 

\end{prop}
\begin{pf*}

\qed
\end{pf*}




\begin{cor}$H$ をヒルベルト空間とし, $X \subset  H$ をコンパクト集合とする. $K$ を$X$ の閉凸包とする. $V_1 (K) < \infty $ ならば, $\Mag (X) < \infty$ が成り立つ. 

\end{cor}
\begin{pf*}

\qed
\end{pf*}


\begin{prop}

\end{prop}
\begin{pf*}

\qed
\end{pf*}


\begin{prop}(グロモフハウスドルフ距離に関する下半連続性). $X$ を正定値コンパクト距離空間とする. 任意の$\veps > 0$ に対して, $\delta > 0$ で, コンパクト正定値距離空間$Y$ が$d_H(X, Y) < \delta$ ならば, $\Mag X - \veps \leq \Mag Y$ となるものが存在する. 

\end{prop}
\begin{pf*}
任意に$\veps > 0$ をとる. 有限部分集合$X^\prime \subset X$ で$\Mag X - \veps \leq \Mag X ^\prime$ を満たすものをとる. $X^\prime$ のウェイトを$w$ で表すことにする. $\delta \coloneqq \frac{\veps}{ \norm{w}_1^2}$ ととる. コンパクト正定値距離空間$Y$ が$d(X, Y) < \delta$ をみたすとする. $f: X^\prime \rightarrow Y$ を, $x \in X^\prime$ に対して, $d(x, y) < \delta$ を満たす$y \in Y$ をとり, $f(x) \coloneqq y$ とすることで定める. $Y^\prime \coloneqq f(X^\prime)$ とすると, $Y^\prime$ は$Y$ の有限部分集合であり, $d(X^\prime , Y^\prime) < \delta $ を満たす. $Z_{X^\prime}, Z_{Y^\prime}$ により

\begin{align*} Z_f : X^\prime \times X^\prime \rightarrow \mathbb R\end{align*}
を, $Z_f (x, x^\prime ) \coloneqq Z_{Y^\prime}(f(x), f(x^\prime))$ により定め, 

\begin{align*} v: Y^\prime \rightarrow \mathbb R \end{align*}
を, $v(y) \coloneqq \sum_{x \in f^{-1}(y)} w(x)$ により定める. 

\begin{align*} \abs{d(f(x), f(x^\prime)) - d(x, x^\prime)} < \delta \quad (x, x^\prime \in X^\prime) \end{align*}

であるので, 

\begin{align*}\sup_{x, x^\prime \in X} \abs{Z_f(x, x^\prime) - Z_{X^\prime}(x, x^\prime)} 
&= \sup_{x, x^\prime \in X} \abs{e^{-d(f(x), f(x^\prime))} - e^{-d(x, x^\prime)} } 
\\&\leq \sup_{x, x^\prime \in X} \abs{d(f(x), f(x^\prime)) - d(x, x^\prime)} \\&< 2 \delta \end{align*}

が成り立つ. $v^t Z_{Y^\prime } v = w^t Z_{f} w$ が成り立つので, 

\begin{align*} \abs{\Mag X^\prime - v^t Z_{Y} v} 
&=  \abs{w^t Z_{X^\prime} w - w^t Z_{f} w} 
\\&=  \abs{w^t (Z_{X^\prime} w - Z_{f}) w} 
\\&= (\sum_x \abs{w(x)})^2 \sup \abs{Z_{X^\prime}  - Z_{f}} 
\\& < (\sum_x \abs{w(x)})^2 \cdot (2 \delta )
\\& < 2 \veps
 \end{align*}

従って, 

\begin{align*} \Mag Y 
&\geq \Mag Y^\prime 
\\&\geq \frac{(\sum v(y) )^2}{v^t Z_{Y^\prime} v} 
\\&= \frac{(\sum w(x) )^2}{w^t Z_{f} w}
\\&\geq  \frac{(\Mag X^\prime ) ^2 }{\Mag X^\prime + 2 \veps}
\\& \leq  \Mag X^\prime - 2 \veps 
\\& > \Mag X - \veps - 2 \veps \end{align*}

が成り立つ. 


\qed
\end{pf*}




















\end{document}